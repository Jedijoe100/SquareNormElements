% Copyright (c) 2015 Karim Belabas.
% Permission is granted to copy, distribute and/or modify this document
% under the terms of the GNU General Public License

% TeX macros for gp's reference card. Based on Stephen Gildea's multicolumn
% macro package, modified from his GNU emacs reference card. Based on an
% earlier version by Joseph H. Silverman who kindly let me use his original
% file.

\def\versionnumber{2.30}% Version of these reference cards
\def\PARIversion{2.9.0}% Version of PARI described on these reference cards
\def\year{2016}
\def\month{August}

\special{papersize=29.7cm,21cm}

% ignore parimacro.tex's \magnification setting
\let\oldmagnification\magnification
\catcode`@=11
\def\magnification{\count@}%
\catcode`@=12
\input parimacro.tex
\let\magnification\oldmagnification
\ifPDF
  \pdfpagewidth=11.69in
  \pdfpageheight=8.26in
\fi

\newcount\columnsperpage
% The final reference card has six columns, three on each side.
% This file can be used to produce it in any of three ways:
% 1 column per page
%    produces six separate pages, each of which needs to be reduced to 80%.
%    This gives the best resolution.
% 2 columns per page
%    produces three already-reduced pages.
%    You will still need to cut and paste.
% 3 columns per page
%    produces two pages which must be printed sideways to make a
%    ready-to-use 8.5 x 11 inch reference card.
%    For this you need a dvi device driver that can print sideways.
% [For 2 or 3 columns, you'll need 6 and 8 point fonts.]
% Which mode to use is controlled by setting \columnsperpage below:
\columnsperpage=3

% You shouldn't need to modify anything below this line.

\def\version{\month\ \year\ v\versionnumber}

\def\shortcopyrightnotice{\vskip .5ex plus 2 fill
  \centerline{\small \copyright\ \year\ Karim Belabas.
  Permissions on back.  v\versionnumber}}

\def\<#1>{$\langle${#1}$\rangle$}
\def\copyrightnotice{\vskip 1ex plus 2 fill
\begingroup\small
\centerline{Based on an earlier version by Joseph H. Silverman}
\centerline{\version. Copyright \copyright\ \year\ K. Belabas}

Permission is granted to make and distribute copies of this card provided the
copyright and this permission notice are preserved on all copies.

Send comments and corrections to \<Karim.Belabas@math.u-bordeaux.fr>
\endgroup}

% make \bye not \outer so that the \def\bye in the \else clause below
% can be scanned without complaint.
\def\bye{\par\vfill\supereject\end}

\newdimen\intercolumnskip
\newbox\columna
\newbox\columnb

\def\ncolumns{\the\columnsperpage}

\message{[\ncolumns\space
  column\if 1\ncolumns\else s\fi\space per page]}

\def\scaledmag#1{ scaled \magstep #1}

% This multi-way format was designed by Stephen Gildea
% October 1986.
\if 1\ncolumns
  \hsize 4in
  \vsize 10in
  \voffset -.7in
  \font\titlefont=\fontname\tenbf \scaledmag3
  \font\headingfont=\fontname\tenbf \scaledmag2
  \font\smallfont=\fontname\sevenrm
  \font\smallsy=\fontname\sevensy

  \footline{\hss\folio}
  \def\makefootline{\baselineskip10pt\hsize6.5in\line{\the\footline}}
\else
  \hsize 3.4in
  \vsize 7.90in
  \hoffset -.75in
  \voffset -.815in
  \font\titlefont=cmbx10 \scaledmag2
  \font\headingfont=cmbx10 %\scaledmag1
  \font\smallfont=cmr6
  \font\smallsy=cmsy6
  \font\eightrm=cmr8
  \font\eightbf=cmbx8
  \font\eightit=cmti8
  \font\eighttt=cmtt8
  \font\eightsy=cmsy8
  \font\eightsl=cmsl8
  \font\eighti=cmmi8
  \font\eightex=cmex10 at 8pt
  \textfont0=\eightrm
  \textfont1=\eighti
  \textfont2=\eightsy
  \textfont3=\eightex
  \def\rm{\fam0 \eightrm}
  \def\bf{\eightbf}
  \def\it{\eightit}
  \def\tt{\eighttt}
  \normalbaselineskip=.8\normalbaselineskip
  \normallineskip=.8\normallineskip
  \normallineskiplimit=.8\normallineskiplimit
  \normalbaselines\rm %make definitions take effect

  \if 2\ncolumns
    \let\maxcolumn=b
    \footline{\hss\rm\folio\hss}
    \def\makefootline{\vskip 2in \hsize=6.86in\line{\the\footline}}
  \else \if 3\ncolumns
    \let\maxcolumn=c
    \nopagenumbers
  \else
    \errhelp{You must set \columnsperpage equal to 1, 2, or 3.}
    \errmessage{Illegal number of columns per page}
  \fi\fi

  \intercolumnskip=.45in
  \def\abc{a}
  \output={%
      % This next line is useful when designing the layout.
      %\immediate\write16{Column \folio\abc\space starts with \firstmark}
      \if \maxcolumn\abc \multicolumnformat \global\def\abc{a}
      \else\if a\abc
        \global\setbox\columna\columnbox \global\def\abc{b}
        %% in case we never use \columnb (two-column mode)
        \global\setbox\columnb\hbox to -\intercolumnskip{}
      \else
        \global\setbox\columnb\columnbox \global\def\abc{c}\fi\fi}
  \def\multicolumnformat{\shipout\vbox{\makeheadline
      \hbox{\box\columna\hskip\intercolumnskip
        \box\columnb\hskip\intercolumnskip\columnbox}
      \makefootline}\advancepageno}
  \def\columnbox{\leftline{\pagebody}}

  \def\bye{\par\vfill\supereject
    \if a\abc \else\null\vfill\eject\fi
    \if a\abc \else\null\vfill\eject\fi
    \end}
\fi

% we won't be using math mode much, so redefine some of the characters
% we might want to talk about
%\catcode`\^=12
%\catcode`\_=12
%\catcode`\~=12

\chardef\\=`\\
\chardef\{=`\{
\chardef\}=`\}

\hyphenation{}

\parindent 0pt
\parskip 0pt

\def\small{\smallfont\textfont2=\smallsy\baselineskip=.8\baselineskip}

\outer\def\newcolumn{\vfill\eject}

\outer\def\title#1{{\titlefont\centerline{#1}}}

\outer\def\section#1{\par\filbreak
  \vskip 1ex plus .4ex minus .5ex
  {\headingfont #1}\mark{#1}%
  \vskip .5ex plus .3ex minus .5ex
}

\outer\def\subsec#1{\filbreak
  \vskip 0.07ex plus 0.05ex
  {\bf #1}
  \vskip 0.03ex plus 0.05ex
}

\newdimen\keyindent
\def\beginindentedkeys{\keyindent=1em}
\def\endindentedkeys{\keyindent=0em}
\endindentedkeys

\def\kbd#1{{\tt#1}\null} %\null so not an abbrev even if period follows

\newbox\libox
\setbox\libox\hbox{\kbd{M-x }}
\newdimen\liwidth
\liwidth=\wd\libox

\def\li#1#2{\leavevmode\hbox to \hsize{\hbox to .75\hsize
  {\hskip\keyindent\relax#1\hfil}%
  \hskip -\liwidth minus 1fil
  \kbd{#2}\hfil}}

\def\threecol#1#2#3{\hskip\keyindent\relax#1\hfil&\kbd{#2}\quad
  &\kbd{#3}\quad\cr}

\def\mod{\;\hbox{\rm mod}\;}
\def\expr{\hbox{\it expr}}
\def\seq{\hbox{\it seq}}
\def\args{\hbox{\it args}}
\def\file{\hbox{\it file}}
\def\QQ{\hbox{\bf Q}}
\def\MM{\hbox{\bf M}}
\def\ZZ{\hbox{\bf Z}}
\def\PP{\hbox{\bf P}}
\def\RR{\hbox{\bf R}}
\def\FF{\hbox{\bf F}}
\def\CC{\hbox{\bf C}}
\def\deg{\mathop{\rm deg}}
\def\bs{\char'134}
\def\pow{\^{}\hskip0pt}
\def\til{\raise-0.3em\hbox{\~{}}}
\def\typ#1{\kbd{t\_#1}}

%%%%%%%%%%%%%%%%%%%%%%%

\title{\TITLE}
\centerline{(PARI-GP version \PARIversion)}
