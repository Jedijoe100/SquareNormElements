\chapter{Specific Use of the GP Calculator}

Originally, \idx{GP} was designed as a debugging tool for the PARI system
library, and hence not much thought had been given to making it
user-friendly. The situation has now changed somewhat, and GP is very
useful as a stand-alone tool. The operations and functions available in
PARI and GP will be described in the next chapter. In the present one, we
describe the specific use of the GP programmable calculator.

For starting the calculator, the general commandline syntax is:

\kbd{gp [-s stacksize] [-p primelimit]}

\noindent
where items within brackets are optional\footnote{*}{On the Macintosh, even
after clicking on the gp icon, once in the MPW Shell, you still need to type
explicitly a command of the above form.}. These correspond to some internal
parameters of GP, or {\it defaults}. See \secref{se:defaults} below for a
list and explanation of all defaults, there are many more than just those
two. These defaults can be changed by adding parameters to the input line
as above, or interactively during a GP session or in a preferences file (also
known as \kbd{gprc}).

\unix Some new features were developed on UNIX platforms, and depend heavily
on the operating system in use. It is {\it possible\/} that some of these
will be ported to other operating systems (BeOS, MacOS, DOS, OS/2, Windows,
etc.) in future versions (most of them should be easy tasks for anybody
acquainted with those). As for now, most of them were not. So, whenever a
specific feature of the UNIX version is discussed in a paragraph, a UNIX sign
sticks out in the left margin, like here. Just skip these if you're stranded
on a different operating system: the core GP functions (i.e.~at least
everything which is even faintly mathematical in nature) will still be
available to you. It may also be possible (and then definitely advisable) to
install \idx{Linux} or \idx{FreeBSD} on your machine.

\misctitle{Note (added in version 2.0.12):} All the UNIX goodies are now
available for DOS, OS/2 and Windows 3.1, thanks to the \tet{EMX} runtime
package (\kbd{install} excluded under DOS, since DLLs are not supported by
the OS). They've been reported to be available under Windows 95/98 and NT
using the Cygwin package (untested by us, but supposedly supported by
\kbd{Configure}).

\emacs If you have GNU Emacs, you can work in a special Emacs shell (see
\secref{se:emacs}), which is started by typing \kbd{M-x gp} (where as
usual \kbd{M} is the \kbd{Meta} key) if you accept the default stack, prime
and buffer sizes, or \kbd{C-u M-x gp} which will ask you for the name of the
gp executable, the stack size, the prime limit and the buffer size. Specific
features of this Emacs shell will be indicated by an EMACS sign.\smallskip

If a \idx{preferences file} (or \kbd{gprc}, to be discussed in
\secref{se:gprc}) can be found, GP will then read it and execute the commands
it contains. This provides an easy way to customize GP without having to
delve into the code to hardwire it to your likings.

A copyright message then appears which includes the version number. Please
note this number, so as to be sure to have the most recent version if you
wish to have updates of PARI. The present manual is written for version
\vers, and has undergone major changes since the 1.39.xx versions.

After the copyright, the computer works for a few seconds (it is in fact
computing and storing a table of primes), writes the top-level help
information, some initial defaults, and then waits after printing its prompt
(initially: \kbd{?}).

Note that at any point the user can type \kbd{Ctrl-C} (that is press
simultaneously the \kbd{Control} and \kbd{C} keys): the current
computation will be interrupted and control given back to the user at the GP
prompt.

The top-level help information tells you that (as in many systems) to get
help, you should type a \kbd{?}. When you do this and hit return, a menu
appears, describing the eleven main categories of available functions and
what to do to get more detailed help. If you now type \kbd{?$n$} with $1\le
n\le11$, you will get the list of commands corresponding to category $n$
and simultaneously to Section $3.n$ of this manual.

If you type \kbd{?}\var{functionname} where \var{functionname} is the
name of a PARI function, you will get a short explanation of this
function.

\unix If extended help (see \secref{se:exthelp}) is available on your
system, you can double or triple the \kbd{?} sign to get much more:
respectively the complete description of the function (e.g.~\kbd{??~sqrt}),
or a list of GP functions relevant to your query (e.g.~ \kbd{???~"elliptic
curve"} or \kbd{???~"quadratic field"}).

If GP was compiled with the right options (see Appendix A), a line
editor will be available to correct the command line, get automatic
completions, and so on. See \secref{se:readline} for a short summary of
available commands. This might not be available for all architectures.

Whether extended on-line help and line editing are available or not is
indicated in the GP banner, between the version number and the copyright
message.

If you type \kbd{?\bs} you will get a short description of the metacommands
(keyboard shortcuts).

Finally, typing \kbd{?.} will return the list of available (pre-defined)
member functions. These are functions attached to specific kind of objects,
used to retrieve easily some information from complicated structures (you
can define your own but they won't be shown here). We will soon describe
these commands in more detail.

As a general rule, under GP, commands starting with \b\ or with some
other symbols like \kbd{?} or \kbd{\#}, are not computing commands, but are
metacommands which allow the user to exchange information with GP. The
available metacommands can be divided into default setting commands
(explained below) and simple commands (or keyboard shortcuts, to be dealt
with in \secref{se:meta}).

\section{Defaults and output formats}\sidx{defaults}\sidx{output formats}
\label{se:defaults}

\noindent
There are many internal variables in GP, defining how the system will behave
in certain situations, unless a specific override has been given. Most
of them are a matter of basic customization (colors, prompt) and will be set
once and for all in your \idx{preferences file} (see \secref{se:gprc}), but
some of them are useful interactively (set timer on, increase precision,
etc.).

The function used to manipulate these values is called \kbd{default}, which
is described in \secref{se:default}. The basic syntax is

\kbd{default(\var{def}, \var{value})},

\noindent
which sets the default \var{def} to \var{value}. In interactive
use, most of these can be abbreviated using historic GP metacommands (mostly,
starting with \b), which we shall describe in the next section.

Here we will only describe the available defaults and how they are used. Just
be aware that typing \kbd{default} by itself will list all of them, as well
as their current values (see \b{d}). Just after the default name, we give
between parentheses the initial value when GP starts (assuming you did not
tamper with it using command-line switches or a~\tet{gprc}).

\misctitle{(somewhat technical) Note:} As we will see in \secref{se:strings},
the second argument to default will be subject to string context expansion,
which means you can use run-time values. In other words, something like
\kbd{a = 3; default(logfile, "\var{some filename}" a ".log")} will work.

For the user's convenience, some defaults will be expanded further when
the values are used (after the above expansion has been performed):

$\bullet$ \idx{time expansion}: the string is sent through the library
function \tet{strftime}. This means that \kbd{\%}\var{char} combinations have
a special meaning, usually related to the time and date. For instance,
\kbd{\%H} = hour (24-hour clock) and \kbd{\%M} = minute [00,59] (on a Unix
system, you can try \kbd{man strftime} at your shell prompt to get a complete
list). This is applied to \kbd{prompt}, \kbd{psfile}, and \kbd{logfile}. For
instance,

\kbd{default(prompt,"(\%R) ? ")}

\noindent
will prepend the time of day, in the form \kbd{(\var{hh}:\var{mm})}
to GP's usual prompt.

\unix $\bullet$ \idx{environment expansion}: When the string contains a
sequence of the form \kbd{\${\it SOMEVAR}} (e.g.~\kbd{\$HOME}) the
environment is searched and if {\it SOMEVAR} is defined, the sequence is
replaced by the corresponding value. Also the \kbd{\til} symbol has the
same meaning as in the C and bash shells~--- \kbd{\til} by itself stands
for your home directory, and \kbd{\til{}user} is expanded to \kbd{user}'s
home directory. This is applied to all filenames\sidx{filename}.

\subsecidx{buffersize} (default \kbd{30000}): GP input is buffered, which means
only so many bytes of data can be read at a time before a command is
executed. This used to be a very important variable, to allow for very
large input files to be read into GP, for example large matrices, without it
complaining about ``unused characters''. Currently, \kbd{buffersize} is
automatically adjusted to the size of the data that are to be read. It will
never go down by itself though. Thus this option may come in handy to decrease
the buffer size after some unusually large \kbd{read}, when you don't need to
keep gigantic buffers around anymore.

\subsecidxunix{colors} (default \kbd{""}): this default is only usable if GP
\label{se:colors}
is running within certain color-capable terminals. For instance \kbd{rxvt},
\kbd{color\_xterm} and modern versions of \kbd{xterm} under X Windows, or
standard Linux/DOS text consoles. It causes GP to use a small palette of
colors for its output. With xterms, the colormap used corresponds to the
resources \kbd{Xterm*color$n$} where $n$ ranges from $0$ to $15$ (see the
file \kbd{misc/color.dft} for an example). Legal values for this default are
strings \kbd{"$a_1$,\dots,$a_k$"} where $k\le7$ and each $a_i$ is either

\noindent $\bullet$ the keyword \kbd{no} (use the default color, usually
black)

\noindent $\bullet$ an integer between 0 and 15 corresponding to the
aforementioned colormap

\noindent $\bullet$ a triple $[c_0,c_1,c_2]$ where $c_0$ stands for foreground
color, $c_1$ for background color, and $c_2$ for attributes (0 is default, 1 
is bold, 4 is underline).

The output objects thus affected are respectively error messages,
history numbers, prompt, input line, output, help messages, timer (that's
seven of them). If $k < 7$, the remaining $a_i$ are assumed to be $no$. For
instance
%
\bprog%
default(colors,"9, 5, no, no, 4")
\eprog
\noindent
typesets error messages in color $9$, history numbers in color $5$, output in
color $4$, and does not affect the rest.

\emacs{In the present version, this default is incompatible with Emacs.
Changing it will just fail silently (the alternative would be to display
escape sequences as is, since Emacs will refuse to interpret them). On the
other hand, you can customize highlighting in your \kbd{.emacs} so as to mimic
exactly this behaviour. See \kbd{emacs/pariemacs.txt}.}

If you use an old \kbd{readline} library (version number less than 2.0),
you should do as in the example above and leave $a_3$ and $a_4$ (prompt
and input line) strictly alone. Since old versions of \kbd{readline} did
not handle escape characters correctly (or more accurately, treated them
in the only sensible way since they did not care to check all your terminal
capabilities: it just ignored them), changing them would result in many
annoying display bugs.

The hacker's way to check if this is the case would be to look in the
\kbd{readline.h} include file (wherever your readline include files are) for
the string \kbd{RL\_PROMPT\_START\_IGNORE}. If it's there, you are safe.

A more sensible way is to make some experiments, and get a more recent
\kbd{readline} if yours doesn't work the way you'd like it to. See the file
\kbd{misc/gprc.dft} for some examples.

\subsecidx{compatible} (default \kbd{0}): The GP function names and syntax have
changed tremendously between versions 1.xx and 2.00. To help you cope with this
we provide some kind of backward compatibility, depending on the value of
this default:

\quad \kbd{compatible} = 0: no backward compatibility. In this mode, a very
handy function, to be described in \secref{se:whatnow}, is \kbd{whatnow},
which tells you what has become of your favourite functions, which GP
suddenly can't seem to remember.

\quad \kbd{compatible} = 1: warn when using obsolete functions, but otherwise
accept them. The output uses the new conventions though, and there may be
subtle incompatibilities between the behaviour of former and current
functions, even when they share the same name (the current function is used in
such cases, of course!). We thought of this one as a transitory help for GP
old-timers. Thus, to encourage switching to \kbd{compatible}=0, it is not
possible to disable the warning.

\quad \kbd{compatible} = 2: use only the old function naming scheme (as used up
to version 1.39.15), but {\it taking case into account}. Thus \kbd{I}
(${}=\sqrt{-1}$) is not the same as \kbd{i} (user variable, unbound by
default), and you won't get an error message using \kbd{i} as a loop index
as used to be the case.

\quad \kbd{compatible} = 3: try to mimic exactly the former behaviour. This is
not always possible when functions have changed in a fundamental way. But
these differences are usually for the better (they were meant to, anyway), and
will probably not be discovered by the casual user.

One adverse side effect is that any user functions and aliases that have been
defined {\it before\/} changing \kbd{compatible} will get erased if this
change modifies the function list, i.e.~if you move between groups $\{0,1\}$
and $\{2,3\}$ (variables are unaffected). We of course strongly encourage you
to try and get used to the setting \kbd{compatible}=0.

\subsecidx{debug} (default \kbd{0}): debugging level. If it is non-zero, some
extra messages may be printed (some of it in French), according to what is
going on (see~\b{g}).

\subsecidx{debugfiles} (default \kbd{0}): file usage debugging level. If it is
non-zero, GP will print information on file descriptors in use, from PARI's
point of view (see~\b{gf}).

\subsecidx{debugmem} (default \kbd{0}): memory debugging level. If it is
non-zero, GP will regularly print information on memory usage. If it's
greater than 2, it will indicate any important garbage collecting and the
function it is taking place in (see~\b{gm}).

\noindent {\bf Important Note:} As it noticeably slows down the performance
(and triggers bugs in a popular compiler), the first functionality (memory
usage) is disabled if you're not running a version compiled for debugging
(see Appendix~A).

\subsecidx{echo} (default \kbd{0}): this is a toggle, which can be either 1
(on) or 0 (off). When \kbd{echo} mode is on, each command is reprinted before
being executed. This can be useful when reading a file with the \b{r} or
\kbd{read} commands. For example, it is turned on at the beginning of the test
files used to check whether GP has been built correctly (see \b{e}).

\subsecidx{format} (default \kbd{"g0.28"} and \kbd{"g0.38"} on 32-bit and
64-bit machines, respectively): of the form x$m.n$, where x is a letter in
$\{\kbd{e},\kbd{f},\kbd{g}\}$, and $n$, $m$ are integers. If x is \kbd{f},
real numbers will be printed in \idx{fixed floating point format} with no
explicit exponent (e.g.~\kbd{0.000033}); if the letter is \kbd{e}, they will be
printed in \idx{scientific format}, always with an explicit exponent (e.g.
\kbd{3.3e-5}). If the letter is \kbd{g}, real numbers will be printed in
\kbd{f} format, except when their absolute value is less than $2^{-32}$,
in which case they are printed in \kbd{e} format. \label{se:format}

The number $n$ is the number of significant digits printed for real
numbers, except if $n<0$ where all the significant digits will be printed
(initial default 28, or 38 for 64-bit machines), and the number $m$ is the
number of characters to be used for printing integers, but is ignored if
equal to 0 (which is the default). This is a feeble attempt at formatting.

\subsecidxunix{help} (default: the location of the \kbd{gphelp} script): the
name of the external help program which will be used from within GP when
extended help is invoked, usually through a \kbd{??} or \kbd{???} request
(see \secref{se:exthelp}), or \kbd{M-H} under readline (see
\secref{se:readline}).

\subsecidx{histsize} (default \kbd{5000}): GP keeps a history of the last
\kbd{histsize} results computed so far, which you can recover using the
\kbd{\%} notation (see \secref{se:history}). When this number is exceeded,
the oldest values are erased. Tampering with this default is the only way to
get rid of the ones you don't need anymore.

\subsecidx{lines} (default \kbd{0}): if set to a positive value, GP prints at
most that many lines from each result, terminating the last line shown with
\kbd{[+++]} if further material has been suppressed. The various \kbd{print}
commands (see \secref{se:gp_program}) are unaffected, so you can always type
\kbd{print(\%)}, \b{a}, or \b{b} to view the full result. If the actual
screen width cannot be determined, a ``line'' is assumed to be 80 characters
long.

\subsecidx{log} (default \kbd{0}): this is a toggle, which can be either 1
(on) or 0 (off). When logging mode is turned on, GP opens a log file, whose
exact name is determined by the \kbd{logfile} default. Subsequently, all the
commands and results will be written to that file (see \b{l}). In case a file
with this precise name already existed, it will not be erased: your data will
be {\it appended\/} at the end.

\subsecidx{logfile} (default \kbd{"pari.log"}): name of the log file to be
used when the \kbd{log} toggle is on. Tilde and time expansion are performed.

\subsecidx{output} (default \kbd{1}): this can take any of the following three
values: 0  (=~{\it raw\/}), 1 (=~{\it prettymatrix\/}), or 2
(=~{\it prettyprint\/}). This means that, independently of the default
\kbd{format} for reals which we explained above, you can print results in
three ways: either in {\it raw\/}\sidx{raw format} format, i.e.~a format
which is equivalent to what you input, including explicit multiplication
signs, and everything typed on a line instead of two dimensional boxes. This
can have several advantages, for instance it allows you to pick the result
with a mouse or an editor, and to put it somewhere else.\label{se:output}

The second format is the {\it prettymatrix\/}\sidx{prettymatrix format} format.
The only difference to raw format is that matrices are printed as boxes
instead of horizontally. This is prettier, but takes more space and cannot be
used for input. Column vectors are still printed horizontally.

The third format is the {\it prettyprint\/}\sidx{prettyprint format} or
beautified format. In the present version \vers, this is not beautiful at
all.

Independently of the setting of this default, an object can be printed
in any of the three formats at any time using the commands \b{a}, \b{m}
and~\b{b} respectively (see below).

\subsecidx{parisize} (default, 1000000 bytes on the Mac, 4000000 otherwise):
GP, and in fact any program using the PARI library, needs a stack in which to
do its computations. \kbd{parisize} is the stack size, in bytes. It is
strongly recommended you increase this default (using the \kbd{-s}
command-line switch, or a \kbd{gprc}) if you can afford it. In case of
emergency, you can use the \tet{allocatemem} function to increase
\kbd{parisize}, once the session is started. GP will try to {\it double\/} the
stack size by itself when memory runs low during a computation, but
this very computation will then be lost, and you will have to type the
command again.

\subsecidx{path} (default \kbd{".:\til:\til/gp"} on UNIX systems,
\kbd{".;C:\bs;C:\bs GP} on DOS, OS/2 and Windows, and \kbd{"."} otherwise):
This is a list of directories, separated by colons ':' (semicolons ';' in the
DOS world, since colons are pre-empted for drive names). When asked to read a
file whose name does not contain \kbd{/} (i.e.~no explicit path was given),
GP will look for it in these directories, in the order they were written in
\kbd{path}. Here, as usual, '.' means the current directory, and '$.\,.$' its
immediate parent. Tilde expansion is performed.

\subsecidx{primelimit} (default \kbd{200000} on the Mac, and \kbd{500000}
otherwise): GP precomputes a list of all primes less than \kbd{primelimit} at
initialization time. These are used by many arithmetical functions.
If you don't plan to invoke any of them, you can just set this to 1.

\subsecidx{prompt} (default \kbd{"? "}): a string that will be printed as
prompt. Note that most usual escape sequences are available there: \b{e} for
Esc, \b{n} for Newline, \dots, \kbd{\bs\bs} for \kbd{\bs}. Time expansion is
performed.

This string is sent through the library function \kbd{\idx{strftime}} (on a
Unix system, you can try \kbd{man strftime} at your shell prompt). This means
that \kbd{\%} constructs have a special meaning, usually related to the time
and date. For instance, \kbd{\%H} = hour (24-hour clock) and \kbd{\%M} =
minute [00,59] (use \kbd{\%\%} to get a real \kbd{\%}).

If you use \kbd{readline}, escape sequences in your prompt will result in
display bugs. If you have a relatively recent \kbd{readline} (see the comment
at the end of \secref{se:colors}), you can brace them with special sequences
(\kbd{\bs[} and \kbd{\bs]}), and you will be safe. If these just result in
extra spaces in your prompt, then you'll have to get a more recent
\kbd{readline}. See the file \kbd{misc/gprc.dft} for an example.

\emacs {\bf Caution}: Emacs needs to know about the prompt pattern to
separate your input from previous GP results, without ambiguity. It's not a
trivial problem to adapt automatically this regular expression to an
arbitrary prompt (which can be self-modifying!). Thus, in this version \vers,
Emacs relies on the prompt being the default one. So, do not tamper with the
\kbd{prompt} variable {\it unless\/} you modify it simultaneously in your
\kbd{.emacs} file (see \kbd{emacs/pariemacs.txt} and \kbd{misc/gprc.dft} for
examples).

\subsecidx{psfile} (default \kbd{"pari.ps"}): name of the default file where
GP is to dump its PostScript drawings (these will always be appended, so that
no previous data are lost). Tilde and time expansion are performed.

\subsecidx{realprecision} (default \kbd{28} and \kbd{38} on 32-bit and 64-bit
machines respectively): the number of significant digits and, at the same
time, the number of printed digits of real numbers (see~\b{p}). Note that
PARI internal precision works on a word basis (32 or 64 bits), hence may not
coincide with the number of decimal digits you input. For instance to get 2
decimal digits you need one word of precision which, on a 32-bit machine,
actually gives you 9 digits ($9 < \log_{10}(2^{32}) < 10$):

\bprog%
? default(realprecision, 2)
\q    realprecision = 9 significant digits (2 digits displayed)
\eprog

\subsecidx{secure} (default \kbd{0}): this is a toggle which can be either 1
(on) or 0 (off). If on, the \tet{system} and \tet{extern} command are
disabled. These two commands are potentially dangerous when you execute
foreign scripts since they let GP execute arbitrary UNIX commands. GP will
ask for confirmation before letting you (or a script) unset this toggle.

\subsecidx{seriesprecision} (default \kbd{16}): precision of power series
(see~\b{ps}).

\subsecidx{simplify} (default \kbd{1}): this is a toggle which can be either
1 (on) or 0 (off). When the PARI library computes something, the type of the
result is not always the simplest possible. The only type conversions which
the PARI library does automatically are rational numbers to integers (when
they are of type \typ{FRAC} and equal to integers), and similarly rational
functions to polynomials (when they are of type \typ{RFRAC} and equal to
polynomials). This feature is useful in many cases, and saves time, but can
be annoying at times. Hence you can disable this and, whenever you feel like
it, use the function \kbd{simplify} (see Chapter 3) which allows you to
simplify objects to the simplest possible types recursively (see~\b{y}).
\sidx{automatic simplification}

\subsecidx{strictmatch} (default \kbd{1}): this is a toggle which can be
either 1 (on) or 0 (off). If on, unused characters after a sequence has been
processed will produce an error. Otherwise just a warning is printed. This
can be useful when you're not sure how many parentheses you have to close after
complicated nested loops.

\subsecidx{timer} (default \kbd{0}): this is a toggle which can be either 1
(on) or 0 (off). If on, every instruction sequence (anything ended by a
newline in your input) is timed, to some accuracy depending on the hardware
and operating system. The time measured is the user \idx{CPU time},
{\it not\/} including the time for printing the results (see \kbd{\#} and
\kbd{\#\#}).

\subsec{Note on output formats.}

\noindent
A zero real number is printed in \kbd{e} format as $0.Exx$ where $xx$ is
the (usually negative) {\it decimal\/} exponent of the number (cf.\ %
\secref{se:whatzero}). This allows the user to check the accuracy of the zero
in question (this could also be done using \b{x}, but that would be more
technical).

When the integer part of a real number $x$ is not known exactly because the
exponent of $x$ is greater than the internal precision, the real number is
printed in \kbd{e} format (note that in versions before 1.38.93, this was
instead printed with a $*$ at the end).

Note also that in beautified format, a number of type integer or real is
written without enclosing parentheses, while most other types have them.
Hence, if you see the expression $( 3.14 )$, it is not of type real, but
probably of type complex with zero imaginary part (if you want to be sure, type
\b{x} or use the function \kbd{type}).

\section{Simple metacommands}\label{se:meta}

\noindent
Simple metacommands are meant as shortcuts and should not be used in GP
scripts (see \secref{se:programming}). Beware that these, as all of GP input,
are now {\it case sensitive}. For example, \b{Q} is no longer identical to
\b{q}. In the following list, braces are used to denote optional arguments,
with their default values when applicable, e.g.~$\{n=0\}$ means that if $n$
is not there, it is assumed to be~$0$. Whitespace (or spaces) between the
metacommand and its arguments and within arguments is optional. (This can
cause problems only with \b{w}, when you insist on having a filename whose
first character is a digit, and with \b{r} or \b{w}, if the filename itself
contains a space. In such cases, just use the underlying \kbd{read} or
\kbd{write} function; see~\secref{se:write}).

\subseckbd{?} $\{{\it command}\}$: GP on-line help interface.
As already mentioned, if you type \kbd{?$n$} where $n$ is a number from 1
to 11, you will get the list of functions in Section $3.n$ of the manual
(the list of sections being obtained by simply typing \kbd{?}).
\label{se:exthelp}

These names are in general not informative enough. More details can be
obtained by typing \kbd{?{\it function}}, which gives a short explanation of
the function's calling convention and effects. Of course, to have complete
information, read Chapter 3 of this manual (the source code is at your
disposal as well, though a trifle less readable!). Much better help can be
obtained through the extended help system (see below).

 You then get the function description exactly as it stands
in Chapter 3. All operators (e.g.~\kbd{+}, \kbd{\&\&}, etc.) are accepted
by this extended help, as well as a few other keywords describing key GP
concepts, e.g.~\kbd{readline} (the line editor), \kbd{integer}, \kbd{nf}
(``number field'' as used in most algebraic number theory computations),
\kbd{ell} (elliptic curves), etc.

\unix If the line before the copyright message indicates that extended help
is available (this means \kbd{perl} is installed on your system, GP was
told about it at compile time, and the whole PARI distribution was
correctly installed), you can add more \kbd{?} signs for extended
functionalities:

\kbd{??~\var{keyword}} yields the functions description as it stands in this
manual, usually in Chapter~2 or~3. If you're not satisfied with the default
chapter chosen, you can impose a given chapter by ending the keyword with
\kbd{@} followed by the chapter number, e.g.~\kbd{??~Hello@2} will look in
Chapter~2 for section heading \kbd{Hello} (which doesn't exist, by the way).

\kbd{???~\var{pattern}} produces a list of sections in Chapter~3 of the
manual related to your query. As before, if \var{pattern} ends by \kbd{@}
ifollowed by a chapter number, that chapter is searched instead; you also
have the option to append a simple \kbd{@} (without a chapter number) to
browse through the whole manual.

If your query contains dangerous characters (e.g \kbd{?} or blanks) it is
advisable to enclose it within double quotes, as for GP strings (e.g
\kbd{???~"elliptic curve"}).

Note that extended help is much more powerful than the short help, since
it knows about operators as well: you can type \kbd{??~*} or
\kbd{??~\&\&}, whereas a single \kbd{?} would just yield a not too helpful

\kbd{*** unknown identifier.}

\noindent message. Also, you can ask for extended help on section
number~$n$ in Chapter~3, just by typing \kbd{??~$n$} (where \kbd{?$n$} would
yield merely a list of functions). Finally, a few key concepts in GP are
documented in this way: metacommands (e.g \kbd{??~"??"}), defaults (e.g
\kbd{??~psfile}) and type names (e.g \typ{INT} or \kbd{integer}), as well as
various miscellaneous keywords such as \kbd{edit} (short summary of line
editor commands), \kbd{operator}, \kbd{member}, \kbd{"user defined"},
\kbd{nf}, \kbd{ell}, \dots

Last but not least~: \kbd{??} without argument will open a \kbd{dvi}
previewer (\kbd{xdvi} by default, \kbd{\$GPXDVI} if it is defined in your
environment) containing the full user's manual. \kbd{??tutorial} and
\kbd{??refcard} do the same with the \idx{tutorial} and \idx{reference card}
respectively.

\misctitle{Technical note:} these functionalities are provided by an
external \kbd{perl} script that you are free to use outside any GP session
(and modify to your liking, if you are perl-knowledgeable). It is called
\kbd{\idx{gphelp}}, lies in the \kbd{doc} subdirectory of your distribution
(just make sure you run \kbd{Configure} first, see Appendix~A) and is
really two programs in one. The one which is used from within GP is
\kbd{gphelp} which runs \TeX\ on a selected part of this manual, then opens
a previewer. \kbd{gphelp -detex} is a text mode equivalent, which looks
often nicer especially on a colour-capable terminal (see
\kbd{misc/gprc.dft} for examples). The default \kbd{help} selects which
help program will be used from within GP. You are welcome to improve this
help script, or write new ones (and we really would like to know about it
so that we may include them in future distributions). By the way, outside
of GP you can give more than one keyword as argument to \kbd{gphelp}.

\subseckbd{/*...*/}: comment. Everything between the stars is ignored by
GP. These comments can span any number of lines.

\subseckbd{\bs\bs}: one-line comment. The rest of the line
is ignored by GP.

\subsec{\b{a}} $\{n\}$: prints the object number $n$ ($\%n$)
in raw format. If the number $n$ is omitted, print the latest computed object
($\%$). \label{se:history}

\subsec{\b{b}} $\{n\}$: Same as \b{a}, in prettyprint (i.e.~beautified)
format.

\subsec{\b{c}}:\sidx{available commands} prints the list of all available
hardcoded functions under GP, not including operators written as special
symbols (see \secref{se:operators}). More information can be obtained using
the \kbd{?} metacommand (see above). For user-defined functions / member
functions, see \b{u} and \b{um}.

\subsec{\b{d}}: prints the \idx{defaults} as described in the
previous section (shortcut for \kbd{default()}, see \secref{se:default}).

\subsec{\b{e}} $\{n\}$: switches the \kbd{echo} mode on (1) or off (0). If
$n$ is explicitly given, set echo to $n$.

\subsec{\b{g}} $\{n\}$: sets the debugging level \kbd{debug} to the
non-negative integer $n$.

\subsec{\b{gf}} $\{n\}$: sets the file usage debugging level \kbd{debufiles}
to the non-negative integer $n$.

\subsec{\b{gm}} $\{n\}$: sets the memory debugging level \kbd{debugmem}
to the non-negative integer $n$.

\subsec{\b{h}} $\{m$\kbd{-}$n\}$: outputs some debugging info about the
hashtable. If the argument is a number $n$, outputs the contents of cell
$n$. Ranges can be given in the form $m$\kbd{-}$n$ (from cell $m$ to cell
$n$, \$ = last cell). If a function name is given instead of a number or
range, outputs info on the internal structure of the hash cell this
function occupies (a \kbd{struct entree} in C). If the range is reduced to
a dash ('\kbd{-}'), outputs statistics about hash cell usage.

\subsec{\b{l}} $\{${\it logfile/}$\}$: switches \kbd{log} mode on and off.
If a {\it logfile} argument is given, change the default logfile name to 
{\it logfile} and switch log mode on.

\subsec{\b{m}}: as \b{a}, but using prettymatrix format.

\subsec{\b{p}} $\{n\}$: sets \kbd{realprecision} to $n$ decimal
digits. Prints its current value if $n$ is omitted.

\subsec{\b{ps}} $\{n\}$: sets \kbd{seriesprecision} to $n$ significant terms.
Prints its current value if $n$ is omitted.

\subsec{\b{q}}: \idx{quit}s the GP session and returns to the system.
Shortcut for the function \kbd{quit} (see \secref{se:quit}).

\subsec{\b{r}} $\{${\it filename\/}$\}$: \idx{read}s into GP all the commands
contained in the named file as if they had been typed from the keyboard, one
line after the other. Can be used in combination with the \b{w} command (see
below). Related but not equivalent to the function \kbd{read} (see
\secref{se:read}); in particular, if the file contains more than one line of
input, there will be one history entry for each of them, whereas \kbd{read}
would only record the last one. If {\it filename\/} is omitted, re-read the
previously used input file (fails if no file has ever been successfully read
in the current session).

\unix This command accepts compressed files in \idx{compress}ed (\kbd{.Z})
or \idx{gzip}ped (\kbd{.gz} or \kbd{.z}) format. They will be uncompressed on
the fly as GP reads them, without changing the files themselves.

\subsec{\b{s}}: prints the state of the PARI \idx{stack} and \idx{heap}.
This is used primarily as a debugging device for PARI, and is not intended
for the casual user.

\subsec{\b{t}}: prints the \idx{internal longword format} of all the PARI
types. The detailed bit or byte format of the initial codeword(s) is
explained in Chapter~4, but its knowledge is not necessary for a GP user.

\subsec{\b{u}}: prints the definitions of all user-defined functions.

\subsec{\b{um}}: prints the definitions of all user-defined member functions.

\subsec{\b{v}}: prints the \idx{version number} and implementation architecture
(680x0, Sparc, Alpha, other) of the GP executable you are using. In library
mode, you can use instead the two character strings \kbd{PARIVERSION} and
\kbd{PARIINFO}, which correspond to the first two lines printed by GP just
before the Copyright message.

\subsec{\b{w}} $\{n\}$ $\{${\it filename\/}$\}$: \idx{write}s the object number
$n$ ( $\%n$ ) into the named file, in raw format. If the number $n$ is
omitted, writes the latest computed object ( $\%$ ). If {\it filename\/} is
omitted, appends to \kbd{logfile} (the GP function \kbd{write} is a trifle more
powerful, as you can have filenames whose first character is a digit).

\subsec{\b{x}}: prints the complete tree with addresses and contents (in
hexadecimal) of the \idx{internal representation} of the latest computed
object in GP. As for \b{s}, this is used primarily as a debugging device for
PARI. However, used on a PARI integer, it can be used as a
decimal$\rightarrow$hexadecimal converter.

\subsec{\b{y}} $\{n\}$: switches \kbd{simplify} on (1) or off (0). If $n$
is explicitly given, set simplify to $n$.

\subseckbd{\#}: switches the \kbd{timer} on or off.

\subseckbd{\#\#}: prints the time taken by the latest computation.
Useful when you forgot to turn on the \kbd{timer}.

\section{Input formats for the PARI types}

\noindent
Before describing more sophisticated functions in the next section, let us
see here how to input values of the different data types known to PARI.
Recall that blanks are ignored in any expression which is not a string (see
below).

\subsec{Integers} \sidx{integer}
(type \typ{INT}\idxtyp{INT}): type the integer (with an initial
\kbd{+} or \kbd{-}, if desired) with no decimal point.

\subsec{Real numbers} \sidx{real number}
(type \typ{REAL}\idxtyp{REAL}): type the number with a decimal
point. The internal precision of the real number will be the supremum of the
input precision and the default precision. For example, if the default
precision is 28 digits, typing \kbd{2.} will give a number with internal
precision 28, but typing a 45 significant digit real number will give a
number with internal precision at least 45 (although less may be printed).

You can also use scientific notation with the letter \kbd{E} or
\kbd{e} (like \kbd{6.02 E 23} or \kbd{1e-5}).

\subsec{Integermods}\sidx{integermod}
(type \typ{INTMOD}\idxtyp{INTMOD}): to enter $n \mod m$, type
\kbd{Mod(n,m)}, {\it not\/} \kbd{n\%m} (see Chapter~3).

\subsec{Rational numbers}\sidx{rational number}
(types \typ{FRAC}\idxtyp{FRAC} and
\typ{FRACN}\idxtyp{FRACN}): under GP, all fractions are
automatically reduced to lowest terms, so it is in principle impossible to
work with reducible fractions (of type \typ{FRACN}), although of course in
library mode this is easy. To enter $n/m$ just type it as written. As
explained in \secref{se:gdiv}, division will {\it not\/} be performed, only
reduction to lowest terms.\label{se:FRAC}

If you really want a reducible fraction under GP, you must use the \kbd{type}
function (see \secref{se:gptype}), by typing \kbd{type(x,FRACN)}. Be warned
however that this function must be used with extreme care.

\subsec{Complex numbers}\sidx{complex number}
(type \typ{COMPLEX}\idxtyp{COMPLEX}): to enter $x+iy$, type \kbd{x + I*y}
({\it not\/} \kbd{x+i*y}). The letter \kbd{\idx{I}} stands for
$\sqrt{-1}$. Recall from Chapter 1 that $x$ and $y$ can be of type
\typ{INT}, \typ{REAL}, \typ{INTMOD}, \typ{FRAC}/\typ{FRACN}, or
\typ{PADIC}.

\subsec{$p$-adic numbers}\sidx{p-adic number}\label{se:padic}
(type \typ{PADIC}\idxtyp{PADIC}): to enter a $p$-adic number, simply write a
rational or integer expression and add to it \kbd{O($p$\pow $k$)}, where $p$
and $k$ are integers. This last expression indicates three things to GP:
first that it is dealing with a \typ{PADIC} type (the fact that $p$ is an
integer, and not a polynomial, which would be used to enter a series, see
\secref{se:series}), secondly the ``prime'' $p$ (note that it is not
checked whether $p$ is indeed prime; you can work on 10-adics if you want, but
beware of disasters as soon as you do something non-trivial like taking a
square root), and finally the number of significant $p$-adic digits $k$.
Note that \kbd{O(25)} is not the same as \kbd{O(5\pow 2)}; you probably
want the latter!

For example, you can type in the $7$-adic number

\kbd{2*7\pow(-1) + 3 + 4*7 + 2*7\pow 2 + O(7\pow3)}

\noindent
exactly as shown, or equivalently as
\kbd{905/7 + O(7\pow3)}.

\subsec{Quadratic numbers}\sidx{quadratic number}
(type \typ{QUAD}\idxtyp{QUAD}): first, you must define the
default quadratic order or field in which you want to work. This is done
using the \kbd{\idx{quadgen}} function, in the following way. Write something
like

\kbd{w = quadgen(d)}

\noindent
where \kbd{d} is the {\it discriminant\/} of the quadratic order in
which you want to work (hence $d$ is congruent to $0$ or $1$ modulo $4$). The
name \kbd{w} is of course just a suggestion, but corresponds to traditional
usage. You can of course use any variable name that you like. However,
quadratic numbers are always printed with a \kbd{w}, regardless of the
discriminant. So beware, two numbers can be printed in the same way and not
be equal. However GP will refuse to add or multiply them for example.

Now $(1,w)$ will be the ``canonical'' integral basis of the quadratic order
(i.e.~$w=\sqrt{d}/2$ if $d\equiv 0 \mod 4$, and $w=(1+\sqrt{d})/2$ if
$d\equiv 1 \mod 4$, where $d$ is the discriminant), and to enter $x+yw$ you
just type \kbd{x + y*w}.

\subsec{Polmods}\sidx{polmod} (type \typ{POLMOD}\idxtyp{POLMOD}): exactly as
for integermods, to enter $x \mod y$ (where $x$ and $y$ are polynomials),
type \kbd{Mod(x,y)}, not \kbd{x\%y} (see \secref{se:Mod}). Note that when $y$
is an irreducible polynomial in one variable, polmods whose modulus is $y$
are simply algebraic numbers in the finite extension defined by the
polynomial $y$. This allows us to work easily in \idx{number field}s, finite
extensions of the $p$-adic field $\Q_p$, or \idx{finite field}s.

\label{se:rempolmod}
\misctitle{Important remark.} Since the variables\sidx{variable} occurring
in a polmod are not free variables, it is essential in order to avoid
inconsistencies that polmods use the same variable in internal operations
(i.e.~between polmods) and variables of lower priority (which have been
introduced later in the GP session) for external operations (typically
between a polynomial and a polmod). For example, PARI will not recognize
that \kbd{Mod(y, y\pow2 + 1)} is the same as \kbd{Mod(x, x\pow2 + 1)}.
Hopefully, this problem will pass away when type ``element of a number
field'' is eventually introduced.

On the other hand, \kbd{Mod(x, x\pow2 + 1) + Mod(x, x\pow2 + 1)}
(which gives \kbd{Mod(2*x, x\pow2 + 1)}) and \kbd{x + Mod(y, y\pow2 + 1)}
(which gives a result mathematically equivalent to $\kbd{x} + i$ with
$i^2=-1$) are completely correct, while \kbd{y + Mod(x, x\pow2 + 1)}
gives \kbd{Mod(x + y, x\pow2 + 1)}, which may not be what you want (\kbd{y}
is treated here as a numerical parameter, not as a polynomial variable).

\misctitle{Note (added in version 2.0.16)} As long as the main variables
are the same, it is allowed to mix \typ{POL} and \typ{POLMOD}s. The result
will be the expected \typ{POLMOD}. For instance \kbd{x + Mod(x, x\pow2 +
1)} is equal to \kbd{Mod(2*x, x\pow2 + 1)}. This wasn't the case prior to
version 2.0.16: it returned a polynomial in \kbd{x} equivalent to $\kbd{x}
+ i$, which was in fact an invalid object (you couldn't \kbd{lift} it).

\subsec{Polynomials}\sidx{polynomial}\label{se:pol}
(type \typ{POL}\idxtyp{POL}): type the polynomial in a natural way, not
forgetting to put a ``$*$'' between a coefficient and a formal variable
(this $*$ does not appear in beautified output). Any \idx{variable} name
can be used except for the reserved names \kbd{I} (used exclusively for the
square root of $-1$), \kbd{Pi} ($3.14\dots$), \kbd{Euler} (Euler's
constant), and all the function names: predefined functions, as described
in Chapter~3 (use \b{c} to get the complete list of them) and user-defined
functions, which you ought to know about (use \b{u} if you are subject to
memory lapses). The total number of different variable names is limited to
$16384$ and $65536$ on 32-bit and 64-bit machines respectively, which
should be enough. If you ever need hundreds of variables, you should
probably be using vectors instead.

\subsec{Power series}\sidx{power series}\label{se:series}
(type \typ{SER}\idxtyp{SER}): type a rational function or
polynomial expression and add to it \hbox{\kbd{O({\it expr\/} \pow $k$)}},
where {\it expr\/} is an expression which has non-zero valuation (it can be a
polynomial, power series, or a rational function; the most common case being
simply a variable name).
This indicates to GP that it is dealing with a power series, and the desired
precision is $k$ times the valuation of {\it expr\/} with respect to the
main variable of {\it expr\/} (to check the ordering of the variables, or
to modify it, use the function \kbd{reorder}; see~\secref{se:reorder}).

\subsec{Rational functions}\sidx{rational function}
(types \typ{RFRAC}\idxtyp{RFRAC} and
\typ{RFRACN}\idxtyp{RFRACN}): as for fractions, all rational
functions are automatically reduced to lowest terms under GP. All that was
said about fractions in \secref{se:FRAC} remains valid here.

\subsec{Binary quadratic forms of positive or negative discriminant}%
\sidx{binary quadratic forms}
(type \typ{QFR}\idxtyp{QFR} and \typ{QFI}\idxtyp{QFI}):
these are input using the function \kbd{Qfb} (see Chapter~3). For example
\kbd{Qfb(1,2,3)} will create the binary form $x^2+2xy+3y^2$. It will be
imaginary (of internal type \typ{QFI}) since $2^2 - 4*3 = -8$ is negative.

In the case of forms with positive discriminant (type \typ{QFR}), you
may add an optional fourth component (related to the regulator, more
precisely to Shanks and Lenstra's ``distance''), which must be a real number.
See also the function \kbd{qfbprimeform} which directly creates a prime form
of given discriminant (see Chapter~3).

\subsec{Row and column vectors}\sidx{row vector}\sidx{column vector} (types
\typ{VEC}\idxtyp{VEC} and \typ{COL}\idxtyp{COL}): to
enter a row vector, type the components separated by commas ``\kbd{,}'', and
enclosed between brackets ``\kbd{[}$\,$'' and ``$\,$\kbd{]}'', e.g.~%
\kbd{[1,2,3]}. To enter a column vector, type the vector horizontally, and
add a tilde ``\til'' to transpose. \kbd{[ ]} yields the empty (row) vector.
The function \tet{Vec} can be used to transform any object into a vector (see
Chapter~3).

\subsec{Matrices} (type \typ{MAT}\idxtyp{MAT}):\sidx{matrix} to
enter a matrix, type the components line by line, the components being
separated by commas ``\kbd{,}'', the lines by semicolons ``\kbd{;}'', and
everything enclosed in brackets ``\kbd{[}$\,$'' and ``$\,$\kbd{]}'', e.g.
\kbd{[x,y; z,t; u,v]}. \kbd{[ ; ]} yields the empty (0x0) matrix. The
function \tet{Mat} can be used to transform any object into a matrix (see
Chapter 3).

Note that although the internal representation is essentially the same (only
the type number is different), a row vector of column vectors is {\it not\/}
a matrix; for example, multiplication will not work in the same way.

Note also that it is possible to create matrices (by conversion of empty
column vectors and concatenation, or using the \kbd{matrix} function) with a
given positive number of columns, each of which has zero rows. It is not
possible to create or represent matrices with zero columns and a nonzero
number of rows.

\subsec{Lists} (type \typ{LIST})\idxtyp{LIST}:\sidx{list}
lists cannot be input directly; you have to use the function
\kbd{listcreate} first, then \kbd{listput} each time you want to append a
new element (but you can access the elements directly as with the
vector types described above). The function \kbd{List} can be used to
transform (row or column) vectors into lists (see Chapter~3).

\subsec{Strings} (type \typ{STR})\idxtyp{STR}:\sidx{string}%
\sidx{character string} to enter a string, just enclose it between double
quotes \kbd{"}, like this: \kbd{"this is a string"}. The function \kbd{Str}
can be used to transform any object into a string (see Chapter~3).

\section{GP operators}\label{se:operators}

\noindent
Loosely speaking, an \idx{operator} is a function (usually associated to
basic arithmetic operations) whose name contains only non-alphanumeric
characters. In practice, most of these are simple functions, which take
arguments, and return a value; assignment operators also have side effects.
Each of these has some fixed and unchangeable priority, which means that,
in a given expression, the operations with the highest priority will be
performed first. Operations at the same priority level will always be
performed in the order they were written, i.e.~from left to right. Anything
enclosed between parenthesis is considered a complete subexpression, and
will be resolved independently of the surrounding context. For instance,
assuming that {\it op}$_1$, {\it op}$_2$, {\it op}$_3$ are standard binary
operators with increasing priorities (think of \kbd{+}, \kbd{*}, \kbd{\pow}
for instance),
$$ x~\hbox{\it op}_1~y~\hbox{\it op}_2~z~\hbox{\it op}_2~x~\hbox{\it op}_3~y $$
is equivalent to
$$ x~\hbox{\it op}_1~((y~\hbox{\it op}_2~z)~\hbox{\it op}_2~
 (x~\hbox{\it op}_3~y)).$$

GP knows quite a lot of different operators, some of them unary (having
only one argument), some binary. Unary operators are defined for either
prefix (preceding their single argument: {\it op\/}~$x$) or postfix (following
the argument: $x$~{\it op\/}) position, never both
(some are syntactically correct in both positions, but with different
meanings). Binary operators all use the syntax $x$~{\it op\/}~$y$. Most of
them are well known, some are borrowed from C~syntax, and a few are specific
to GP. Beware that some GP operators may differ slightly from their C
counterparts. For instance, GP's postfix \kbd{++} returns the {\it new\/}
value, like the prefix \kbd{++} of~C, and the binary shifts \kbd{<<},
\kbd{>>} have a priority which is different from (higher than) that of
their C counterparts.
When in doubt, just surround everything by parentheses (besides, your code
will probably be more legible).

\noindent Here is the complete list (in order of decreasing priority, binary
unless mentioned otherwise):

\sidx{priority}
\def\point#1{\noindent $\bullet$ #1\hfill\break\indent\strut}
\point{Priority 9}
%
\kbd{++} and \kbd{--} (unary, postfix): \kbd{$x$++} assigns the value $x+1$ to
$x$, then returns the new value of $x$. This corresponds to the C
statement \kbd{++$x$} (there is no prefix \kbd{++} operator in GP).
\kbd{$x$--} does the same with $x-1$.

\point{Priority 8}
%
\kbd{{\it op\/}=}, where {\it op\/} is any simple binary operator
(i.e.~a binary operator with no side effects, i.e.~one of those defined below)
which is not a boolean operator (comparison or logical).
\kbd{x~{\it op\/}=~$y$} assigns $(\kbd{x}~{\it op\/}~y)$ to~\kbd{x},
and returns the new value of~\kbd{x}, {\it not\/} a reference to the
\idx{variable}~\kbd{x}. (Thus an assignment cannot occur on the lefthand
side of another assignment.)

\point{Priority 7}
%
\kbd{=} is the assignment operator. The result of \kbd{x~=~$y$} is the value
of the expression~$y$, which is also assigned to the variable~\kbd{x}. This
is {\it not\/} the equality test operator. Beware that a statement like
\kbd{x~=~1} is always true (i.e.~non-zero), and sets \kbd{x} to~1.

\point{Priority 6}
%
\kbd{!} (unary, prefix): logical {\it not}. \kbd{!$x$} return $1$ if $x$ is
equal to $0$ (specifically, if \kbd{gcmp0($x$)==1}), and $0$ otherwise.

\kbd{'} (unary, prefix): quote its argument without evaluating it.
\bprog%
? a = x + 1; x = 1;
? subst(a,x,1)
\ \ ***\ \ \ variable name expected: subst(a,x,1)
\ \ \ \ \ \ \ \ \ \ \ \ \ \ \ \ \ \ \ \ \ \ \ \ \ \ \ \ \ \ \ \ \ \ \ \ \ \ \ \ \pow---
? subst(a,'x,1)
\%1 = 2
\eprog

\point{Priority 5}
%
\kbd{\pow}: powering.

\kbd{'} (unary, postfix): derivative with respect to the main variable.

\strut\kbd{\til} (unary, postfix): vector/matrix transpose.

\kbd{!} (unary, postfix): factorial. $x\kbd{!}=x(x-1)\cdots 1$.

\kbd{.}: \kbd{$x$.$b$} extracts member $b$ from structure $x$.

\point{Priority 4}
%
\kbd{+}, \kbd{-} (unary, prefix): \kbd{-} toggles the sign of its argument,
\kbd{+} has no effect whatsoever.

\point{Priority 3}
%
\kbd{*}: multiplication.

\kbd{/}: exact division (\kbd{3/2}=$3/2$, not $1.5$).

\kbd{\bs}, \kbd{\%}: euclidean quotient and remainder, i.e.~if $x =
qy + r$, with $0\le r < y$ (if $x$ and $y$ are polynomials, assume instead
that $\deg r< \deg y$ and that the leading terms of $r$ and $x$ have the
same sign), then $\kbd{x \b{ } y} = q$, $\kbd{x\%y} = r$.

\kbd{\bs/}: rounded euclidean quotient for integers (rounded towards
$+\infty$ when the exact quotient would be a half-integer).

\kbd{<<}, \kbd{>>}: left and right binary shift: \kbd{x<<n}$~=~x * 2^n$
if $n>0$, and $x \b{/} 2^{-n}$ otherwise; and
\kbd{x>>n}$~=~$\kbd{x<<(-n)}.

\point{Priority 2}
%
\kbd{+}, \kbd{-}: addition/subtraction.

\point{Priority 1}
%
\kbd{<}, \kbd{>}, \kbd{<=}, \kbd{>=}: the usual comparison operators,
returning 1 for \kbd{true} and 0 for \kbd{false}. For instance,
\kbd{x<=1} returns $1$ if $x\le 1$ and $0$ otherwise.

\kbd{<>}, \kbd{!=}: test for (exact) inequality.

\kbd{==}: test for (exact) equality.

\point{Priority 0}
%
\kbd{\&}, \kbd{\&\&}: logical {\it and}.

\kbd{|}, \kbd{||}: logical (inclusive) {\it or}. Any sequence of logical
{\it or\/} and {\it and\/} operations is evaluated from left to right,
and aborted as soon as the final truth value is known. Thus, for instance,
\kbd{(x \&\& 1/x)} or \kbd{(type(p) == "t\_INT" \&\& isprime(p))} will never
produce an error since the second argument need not (and will not) be processed
when the first is already zero (false).

\misctitle{Remark:} For the sake of efficiency, you should use the
\kbd{++}, \kbd{--} and {\it op\/}\kbd{=} operators whenever possible:

\bprog%
? a=200000;
? i=0; while(i<a, i=i+1)
time = 4,919 ms.
? i=0; while(i<a, i+=1)
time = 4,478 ms.
? i=0; while(i<a, i++)
time = 3,639 ms.
\eprog

\noindent The shift operators should be preferred to multiplication
whenever possible:

\bprog%
? a=1<<20000;
? i=1; while(i<a, i=i*2);
time = 5,255 ms.
? i=1; while(i<a, i<<=1);
time = 988 ms.
\eprog

\section{The general GP input line}
\subsec{Generalities}. User interaction with a GP session proceeds as
follows: a sequence of characters is typed by the user at the GP prompt. This
can be either a \b~command, a function definition, an expression, or a
sequence of expressions (i.e.~a program). In the latter two cases, after the
last expression has been computed its result is put into an internal
(``history'') array, and printed. The successive elements of this array are
called \kbd{\%1}, \kbd{\%2}, \dots As a shortcut, the latest computed
expression can also be called \kbd{\%}, the previous one \kbd{\%`}, the one
before that \kbd{\%``} and so on.

If you want to suppress the printing of the result, for example because it
is a long unimportant intermediate result, end the expression with a
\kbd{;} sign. This same sign is used as an instruction separator when several
instructions are written on the same line (note that for the pleasure of BASIC
addicts, the \kbd{:} sign can also be used, but we will try to stick to
C-style conventions in this manual). The final expression computed, even
if not printed, will still be assigned to the history array, so you may have
to pay close attention when you intend to refer back to it by number since
this number does not appear explicitly. Of course, if you just want to use
it on the next line, use \kbd{\%} as usual.

Any legal expression can be typed in, and is evaluated using the
conventions about operator priorities and left to right associativity (see
the previous section), using the available operator symbols, function names
(including user-defined functions and member functions see
\secref{se:user_defined}), and special variables. Please note that, from
version $1.900$ on, there\sidx{case distinction} {\it is\/} a distinction
between lowercase and uppercase. Also, note that, outside of constant
strings, blanks are completely ignored in the input to GP.

The special variable\idx{variable (special)} names known to GP are
\kbd{\idx{Euler}} (Euler's constant $\gamma=0.577\dots$), \kbd{\idx{I}}
(the square root of $-1$), \kbd{\idx{Pi}} (3.14\dots)~--- which could be
thought of as functions with no arguments, and which may therefore be
invoked without parentheses~---, and \kbd{\idx{O}} which obeys the
following syntax:

\kbd{O({\it expr\/}\pow k)}

\noindent
When {\it expr\/} is an integer or a rational number, this creates an
{\it expr}-adic number (zero in fact) of precision \kbd{k}. When {\it expr\/}
is a polynomial, a power series or a rational function whose main variable is
$X$, say, this creates a power series (also zero) of precision $v*\kbd{k}$
where $v$ is the $X$-adic valuation of {\it expr\/} (see \ref{se:padic}
and~\ref{se:pol}).

\subsec{Special editing characters}.\sidx{editing characters} A GP program
can of course have more than one line. Since GP executes your commands as
soon as you have finished typing them, there must be a way to tell it to
wait for the next line or lines of input before doing anything. There are
three ways of doing this.

The first one is simply to use the \idx{backslash character} \kbd{\bs} at the
end of the line that you are typing, just before hitting \kbd{<Return>}. This
tells GP that what you will write on the next line is the physical
continuation of what you have just written. In other words, it makes GP
forget your newline character. For example if you use this while defining a
function, and if you ask for the definition of the function using
\kbd{?name}, you will see that your backslash has disappeared and that
everything is on the same line. You can type a \kbd{\bs} anywhere. It will be
interpreted as above only if (apart from ignored whitespace characters) it is
immediately followed by a newline. For example, you can type

\bprog%
? 3 + \bs
4%
\eprog

\noindent instead of typing \kbd{3 + 4}.

The second one is a slight variation on the first, and is mostly useful when
defining a user function (see \secref{se:user_defined}): since an equal sign
can never end a valid expression, GP will disregard a newline immediately
following an \kbd{=}.

\bprog%
? a =
123
\%1 = 123
\eprog

The third one cannot be used everywhere, but is in general much more useful.
It is the use of braces \kbd{\obr} and  \kbd{\cbr}.\sidx{brace characters}
When GP sees an opening brace (\kbd{\obr}) {\it at the beginning of a line}
(modulo spaces as usual), it understands that you are typing a multi-line
command, and newlines will be ignored until you type a closing brace
\kbd{\cbr}. However, there is an important (but easily obeyed) restriction:
inside an open brace-close brace pair, all your input lines will be
concatenated, suppressing any newlines. Thus, all newlines should occur after
a semicolon (\kbd{;}), a comma (\kbd{,}) or an operator (for clarity's sake,
we don't recommend splitting an identifier over two lines in this way). For
instance, the following program

\bprog
\obr
\q a = b
\q b = c
\cbr
\eprog

\noindent would silently produce garbage, since what GP will really see is
\kbd{a=bb=c} which will assign the value of \kbd{c} to both \kbd{bb} and
\kbd{a} (if this really is what you intended, you're a hopeless case).

\section{The GP/PARI programming language}

The GP calculator uses a purely interpreted language. The structure of this
language is reminiscent of LISP with a functional notation, \kbd{f(x,y)}
rather than \kbd{(f x y)}: all \idx{programming} constructs, such as
\kbd{if}, \kbd{while,} etc... are functions \footnote{*}{Not exactly, since
not all their arguments need be evaluated. For instance it would be stupid
to evaluate both branches of an \kbd{if} statement: since only one will
apply, GP only expands this one.} (see \secref{se:programming} for a
complete list), and the main loop does not really execute, but rather
evaluates (sequences of) expressions. Of course, it is by no means a true
LISP.

\subsec{Variables and symbolic expressions}.\sidx{variable}

  In GP you can use up to 16383 variable names (up to 65535 on 64-bit
machines). These names can be any standard identifier names, i.e.~they must
start with a letter and contain only valid keyword characters: \kbd{\_} or
alphanumeric characters ([\kbd{\_A-Za-z0-9}]). To avoid confusion with other
symbols, you must not use other non-alphanumeric symbols like \kbd{\$}, or
'\kbd{.}'. In addition to the function names which you must not use (see the
list with \b{c}), there are exactly three special variable names which you
are not allowed to use: \kbd{Pi} and \tet{Euler}, which represent well known
constants, and $\kbd{I}=\sqrt{-1}$.

Note that GP names are case sensitive since version 1.900. This means for
instance that the symbol \kbd{i} is perfectly safe to use, and will not be
mistaken for $\sqrt{-1}$, and that \kbd{o} is not synonymous anymore to
\kbd{O}. If you grew addicted to the previous behaviour, you can have it back
by setting the default \kbd{compatible} to $3$.

  Now the main thing to understand is that PARI/GP is {\it not\/} a symbolic
manipulation package, although it shares some of the functionalities. One of
the main consequences of this fact is that all expressions are evaluated as
soon as they are written, they never stay in a purely abstract form%
\footnote{**}{An obvious but important exception are character strings which
are evaluated\dots\ essentially to themselves (type \typ{STR}). Not exactly
so though, since we do some work to treat the quoted characters correctly
(those preceded by a \b{)}.}.
%
As an important example, consider what happens when you use a variable name
{\it before\/} assigning a value into it. This is perfectly acceptable to GP,
which considers this variable in fact as a polynomial of degree 1, with
coefficients 1 in degree 1, 0 in degree 0, whose variable is the variable
name you used.

  If later you assign a value to that variable, the objects which you have
created before will still be considered as polynomials. If you want to obtain
their value, use the function \kbd{eval} (see \secref{se:eval}).

  Finally, note that if the variable $x$ contains a vector or list, you can
assign a result to $x[m]$ (i.e.~write something like $x[k]=\var{expr}$). If
$x$ is a matrix, you can assign a result to $x[m,n]$, but {\it not\/} to
$x[m]$. If you want to assign an expression to the $m$-th column of a matrix
$x$, use $x[,m]=\var{expr}$ instead. Similarly, use $x[m,]=\var{expr}$ to
assign an expression to the $m$-th row of $x$. This process is recursive, so
if $x$ is a matrix of matrices of \dots, an expression such as
$x[1,1][,3][4]=1$ would be perfectly valid (assuming of course that all
matrices along the way have the correct dimensions).

\misctitle{Note:} We'll see in \secref{se:user_defined} that it is possible
to restrict the use of a given variable by declaring it to be \tet{global} or
\tet{local}. This can be useful to enforce clean programming style, but is in
no way mandatory.

\misctitle{(Technical) Note:} Variables are numbered in the order that they
appear since the beginning of the session, and the main variable of an
expression is always the lowest numbered variable. Hence if you are working
with expressions involving several variables and want to have them ordered in
a specific manner {\it in the internal representation}, the simplest is just
to write down the variables one after the other under GP before starting any
real computations. If you already have started working and want to change the
names of the variables in an object, use the function \tet{changevar}. If you
only want to have them ordered when the result is printed, you can also use
the function \tet{reorder}, but this won't change anything to the internal
representation.

\misctitle{(Very technical) Note:}
Each variable has a stack of values, implemented as a linked list. When a new
scope is entered (during a function call which uses it as a parameter, or if
the variable is used as a loop index, see \secref{se:user_defined} and
\secref{se:programming}), the value of the actual parameter is pushed on the
stack. If the parameter is not supplied, a special $0$ value called
\teb{gnil} is pushed on the stack (this value is not printed if it is
returned as the result of a GP expression sequence). Upon exit, the stack
decreases. You can \kbd{kill} a variable, decreasing the stack yourself. This
should be used only at the top level of GP, to undo the effect of an
assignment, not from a function. However, the stack has a bottom: the value
of a variable is the monomial of degree 1 in this variable, as is natural for
a mathematician.

\subsec{Expressions and expression sequences}.

  An \idx{expression}\sidx{expression sequence} is formed by combining the
GP operators, functions (including user-defined functions, see below) and
control statements. It may be preceded by an assignment statement '$=$'
into a variable. It always has a value, which can be any PARI object.

  Several expressions can be combined on a single line by separating them
with semicolons (';') and also with colons (':') for those who are used to
BASIC. Such an expression sequence will be called simply a \var{seq}. A
\var{seq} also has a value, which is the value of the last non-empty
expression in the sequence. Under GP, the value of the \var{seq}, and only
this last value, is always put on the stack (i.e. it will become the next
object $\%n$). The values of the other expressions in the \var{seq} are
discarded after the execution of the \var{seq} is complete, except of
course if they were assigned into variables. In addition, the value of
the \var{seq} (or of course of an expression if there is only one) is
printed if the line does not end with a semicolon (';').

\subsec{User defined functions}.\sidx{user defined functions}
\label{se:user_defined}

It is very easy to define a new function under GP, which can then be used
like any other function. The syntax is as follows:

\kbd{name(list of formal variables) = local(list of local variables); \var{seq}}

\noindent which looks better written on consecutive lines:
\bprog% name($x_0$, $x_1$, \dots) = 
\obr
\q local($t_0$, $t_1$, \dots);
\q local(\dots);
\q
\q \dots
\cbr
\eprog
\noindent (note that the first newline is disregarded due to the preceding
\kbd{=} sign, and the others because of the enclosing braces). Both lists
of variables are comma-separated and allowed to be empty. The \tet{local}
statements can be omitted; as usual \var{seq} is any expression sequence.

\kbd{name} is the name given to the function and is subject to the same
restrictions as variable names. In addition, variable names are not valid
function names, you have to \kbd{kill} the variable first (the converse is
true: function names can't be used as variables, see \secref{se:kill}).
Previously used function names can be recycled: you are just redefining the
function (the previous definition is lost of course).

\kbd{list of formal variables} is the list of variables corresponding to
those which you will actually use when calling your function. The number of
actual parameters supplied when calling the function has to be less than the
number of formal variables.

Uninitialized formal variables will be given a default value. An equal
(\kbd{=}) sign following a variable name in the function definition,
followed by any expression, gives the variable a default value. The
expression gets evaluated the moment the function is defined, and is fixed
afterward. A variable for which you supply no default value will be
initialized to zero.

\kbd{list of local variables} is the list of the additional local variables
which are used in the function body. Note that if you omit some or all of
these local variable declarations, the non-declared variables will become
global, hence known outside of the function, and this may have undesirable
side-effects. On the other hand, in some cases it may also be what you want.
Local variables can be given a default value as the formal variables.

\misctitle{Example:} For instance \kbd{foo(x=1,y=2,z=3) = print(x ":" y ":"
z)}, defines a function which prints its arguments (at most three of them),
separated by colons. This then follows the rules of default arguments
generation as explained at the beginning of \secref{se:functions}.

\bprog%
? foo(6,7)
6:7:3
? foo(,5)
1:5:3
? foo
1:2:3
\eprog

  Once the function is defined using the above syntax, you can use it like
any other function. In addition, you can also recall its definition exactly
as you do for predefined functions, that is by writing \kbd{?\var{name}}.
This will print the list of arguments, as well as their default values,
the text of \var{seq}, and a short help text if one was provided using
the \kbd{addhelp} function (see \secref{se:addhelp}). One small difference
to predefined functions is that you can never redefine the built-in
functions, while you can redefine a user-defined function as many times
as you want.

Typing \b{u} will output the list of user-defined functions.

An amusing example of a user-defined function is the following. It is
intended to illustrate both the use of user-defined functions and the power
of the \kbd{sumalt} function. Although the \idx{Riemann zeta-function} is
included in the standard functions, let us assume that this is not the case
(or that we want another implementation). One way to define it, which is
probably the simplest (but certainly not the most efficient), is as
follows:

\sidx{zeta function}
\bprog%
zet(s) =
\obr
\q  local(j); /* not needed, and possibly confusing (see below) */
\q  sumalt(j=1, (-1)\pow(j-1)*j\pow(-s)) / (1 - 2\pow(1-s))
\cbr
\eprog

\noindent This gives reasonably good accuracy and speed as long as you are
not too far from the domain of convergence. Try it for $s$ integral between
$-5$ and $5$, say, or for $s=0.5+i*t$ where $t=14.134\dots$

The iterative constructs which use a variable name (\kbd{for$xxx$},
\kbd{prod$xxx$}, \kbd{sum$xxx$}, \kbd{vector}, \kbd{matrix}, \kbd{plot},
etc.) also consider the given variable to be local to the construct. A value
is pushed on entry and pulled on exit. So, it is not necessary for a function
using such a construct to declare the variable as a dummy formal parameter.

In particular, since loop variables are not visible outside their loops,
the variable \kbd{j} need not be declared in the protoype of our \kbd{zet}
function above.

\kbd{zet(s) = sumalt(j=1, (-1)\pow(j-1)*j\pow(-s)) / (1 - 2\pow(1-s))}

\noindent would be a perfectly sensible (and in fact better) definition.
Since local/global scope is a very tricky point, here's one more example.
What's wrong with the following definition?
\bprog%
? first\_prime\_div(x) = 
\obr
\q local(p);
\q forprime(p=2, x, if (x\%p == 0, break));
\q p
\cbr
? first\_prime\_div(10)
\%1 = 0
\eprog

\misctitle{Answer:} the index $p$ in the \kbd{forprime} loop is local to
the loop and is not visible to the outside world. Hence, it doesn't survive
the \kbd{break} statement. More precisely, at this point the loop index is
restored to its preceding value, which is 0 (local variables are
initialized to 0 by default). To sum up, the routine returns the $p$
declared local to it, not the one which was local to \kbd{forprime} and ran
through consecutive prime numbers. Here's a corrected version:

\bprog%
? first\_prime\_div(x) = forprime(p=2, x, if (x\%p == 0, return(p)))
\eprog

Again, it is strongly recommended to declare all other local variables that
are used inside a function: if a function accesses a variable which is not
one of its formal parameters, the value used will be the one which happens to
be on top of the stack at the time of the call. This could be a ``global''
value, or a local value belonging to any function higher in the call chain.
So, be warned.

There's no problem with recursive functions as long as one pays proper
attention to variable scope. Here's a last example, used to retrieve the
coefficient array of a multivariate polynomial (a non-trivial task due to
PARI's unsophisticated representation for those objects)~:
\sidx{multivariate polynomial}

\bprog%
coeffs(P, nbvar) =
\obr
\q local(v);
\h
\q if (type(P) != "t\_POL",
\q\q   for (i=0, nbvar-1, P = [P]);
\q\q   return (P)
\q );
\q v = vector(poldegree(P)+1, i, polcoeff(P,i-1));
\q vector(length(v), i, coeffs(v[i], nbvar-1))
\cbr
\eprog

\noindent If $P$ is a polynomial in $k$ variables, show that after the
assignment {\tt v = coeffs(P,k)}, the coefficient of $x_1^{n_1}\dots
x_k^{n_k}$ in P is given by {\tt v[$n_1$+1][\dots][$n_k$+1]}. What would
happen if the declaration {\tt local(v)} had been omitted ?

\misctitle{Function which take functions as parameters ?} This is easy
in GP using the following trick (neat example due to Bill Daly):

\bprog%
calc(f, x) = eval(Str( f "(x)"))
\eprog

\noindent If you call this with \kbd{calc("sin", 1)}, it will
return $\sin(1)$ (evaluated!).

\misctitle{Restrictions on variable use:} it is not allowed to use the same
variable name for different parameters of your function. Or to use a given
variable both as a formal parameter and a local variable in a given function.
Hence
\bprog%
? f(x,x) = 1
\q ***   user function f: variable x declared twice.
\eprog

Also, the statement \kbd{\idx{global}(x, y, z, t)} (see \secref{se:global})
declares the corresponding variables to be global. It is then forbidden to
use them as formal parameters or loop indexes as described above, since the
parameter would ``shadow'' the variable.

\misctitle{Implementation note.} For the curious reader, here is how these
stacks are handled: a \idx{hashing function} is computed from the identifier,
and used as an index in \tet{hashtable}, a table of pointers. Each of
these pointers begins a linked list of structures (type \tet{entree}).
The linked list is searched linearly for the identifier (each list will
typically have less than 7 components or so). When the correct \kbd{entree}
is found, it points to the top of the stack of values for that identifier if
it is a variable, to the function itself if it is a predefined function, and
to a copy of the text of the function if it is a user-defined function. When
an error occurs, all of this maze (rather a tree, in fact) is searched and
(hopefully) restored to the state preceding the last call of the main
evaluator.

\misctitle{Note:} The above syntax (using the \tet{local} keyword) was
introduced in version 2.0.13. The old syntax

\kbd{name(list of true formal variables, list of local variables) = \var{seq}}

\noindent is still recognized but is deprecated since genuine arguments and
local variables become undistinguishable.

\subsec{Member functions}.\sidx{member functions}

Member functions use the `dot' notation to retrieve information from
complicated structures (by default: types \tet{ell}, \tet{nf}, \tet{bnf},
\tet{bnr} and prime ideals). The syntax \kbd{structure.member} is taken to
mean: retrieve \kbd{member} from \kbd{structure}, e.g.~\kbd{ell.j} returns
the $j$-invariant of the elliptic curve \kbd{ell} (or outputs an error
message if \kbd{ell} doesn't have the correct type).

You can define your own member functions using the syntax:

\bprog%
  structure.member = {\it function text}
\eprog
\noindent where {\it function text\/} is written as the {\it seq\/} in a
standard user function (without local variables), whose only argument would
be \kbd{structure}. For instance, the current implementation of the \kbd{ell}
type is simply an horizontal vector, the $j$-invariant being the thirteenth
component. This could be implemented as

\bprog
x.j =
\obr
\q if (type(x) != "t\_VEC" || length(x) < 14,
\q\q error("this is not a proper elliptic curve: " x)
\q );
\q x[13]
\cbr
\eprog

You can redefine one of your own member functions simply by typing a new
definition for it. On the other hand, as a safety measure, you can't redefine
the built-in member functions, so typing the above text would in fact produce
an error (you'd have to call it e.g.~\kbd{x.j2} in order for GP to accept it).

Typing \b{um} will output the list of user-defined member functions.

\misctitle{Note:} Member functions were not meant to be too complicated or to
depend on any data that wouldn't be global. Hence they do no have parameters
(besides the implicit \kbd{structure}) or local variables. Of course, if you
need some preprocessing work in there, there's nothing to prevent you from
calling your own functions (using freely their local variables) from a member
function. For instance, one could implement (a dreadful idea as far as
efficiency goes):

\bprog
correct\_ell\_if\_needed(x) = 
\obr
\q local(tmp);
\q if (type(x) != "t\_VEC", tmp = ellinit(x))
\q \bs\bs {\it some further checks}
\q tmp
\cbr
x.j = correct\_ell\_if\_needed(x)[13];
\eprog

\subsec{Strings and Keywords}\sidx{string}\sidx{keyword}
\label{se:strings}

\noindent
GP variables can now hold values of type character string
(internal type \typ{STR}).
This section describes how they are actually used, as well as some convenient
tricks (automatic concatenation and expansion, keywords) valid in string
context.

As explained above, the general way to input a string is to enclose characters
between quotes~\kbd{"}. This is the only input construct where whitespace
characters are significant: the string will contain the exact number
of spaces you typed in. Besides, you can ``escape'' characters by putting a
\kbd{\bs} just before them; this has the following effects:

{
\def\q{\quad}
\obeylines
\q \b{e}: the \kbd{<Escape>} character.
\q \b{n}: the \kbd{<Newline>} character.
\q \b{t}: the \kbd{<Tab>} character.
\q \b{any-other-char}: the \kbd{any-other-char} character.
}
In particular, the only way to put a \kbd{"} into a string is to escape it.
Thus, for instance, \kbd{"\bs"a\bs""} would produce the
string whose content is ``a''. This is definitely {\it not\/} the same thing as
typing \kbd{"a"}, whose content is merely the one-letter string a.

You can concatenate two strings using the \tet{concat} function. If either
argument is a string, the other is automatically converted to a string if
necessary (it will be evaluated first).

\bprog%
? concat("ex", 1+1)
\%1 = "ex2"
? a = 2; b = "ex"; concat(b, a)
\%2 = "ex2"
? concat(a, b)
\%3 = "2ex"
\eprog

Some functions expect strings for some of their arguments: \tet{print} would
be an obvious example, \tet{Str} is a less obvious but very useful one (see
the end of this section for a complete list). While typing in such an
argument, you will be said to be in {\it \idx{string context}}. The rest of
this section is devoted to special syntactical tricks which can be used with
such arguments (and only here; you will get an error message if you try these
outside of string context):

$\bullet$ Writing two strings alongside one another will just concatenate
them, producing a longer string. Thus it is equivalent to type in
\kbd{"a " "b"} or \kbd{"a b"}. A little tricky point in the first expression:
the first whitespace is enclosed between quotes, and so is part of a string;
while the second (before the \kbd{"b"}) is completely optional and GP
actually suppresses it, as it would with any number of whitespace characters
at this point (i.e.~outside of any string).

$\bullet$ If you insert an expression without quotes when GP expects a
string, it gets ``expanded'': it is evaluated as a standard GP expression,
and the final result (as would have been printed if you had typed it by
itself) is then converted to a string, as if you had typed it directly. For
instance \kbd{"a" 1+1 "b"} is equivalent to \kbd{"a2b"}: three strings get
created, the middle one being the expansion of \kbd{1+1}, and these are then
concatenated according to the rule described above. Another tricky point
here: assume you did not assign a value to \kbd{aaa} in a GP expression
before. Then typing \kbd{aaa} by itself in a string context will actually
produce the correct output (i.e.~the string whose content is aaa), but in a
fortuitous way. This \kbd{aaa} gets expanded to the monomial of degree one in
the variable \kbd{aaa}, which is of course printed as \kbd{aaa}, and thus
will expand to the three letters you were expecting. But you will have
defined a variable as a side effect.

$\bullet$ Since there are cases where expansion is not really desirable, we
now distinguish between ``Keywords'' and ``Strings''. String is what has been
described so far. Keywords are special relatives of Strings which are
automatically assumed to be quoted, whether you actually type in the quotes
or not. Thus expansion is never performed on them. They get concatenated,
though. The analyzer supplies automatically the quotes you have ``forgotten''
and treats Keywords just as normal strings otherwise. For instance, if you
type \kbd{"a"b+b} in Keyword context, you will get the string whose contents
are ab+b. In String context, on the other hand, you would get a2\kbd{*}b
(and you would have created the variable \kbd{b} in the process if it didn't
exist before, but not the variable~\kbd{a}).

All GP functions have prototypes (described in Chapter~3 below) which specify
the types of arguments they expect: either generic PARI objects (GEN),
or strings, or keywords, or unevaluated expression sequences.
In the keyword case, only a very small set of words
will actually be meaningful (the \kbd{default} function is a prominent
example).

Let's now try some not-so-stupid exercises to get the hang of it. Try to
guess the results of the following commands without actually typing them,
assuming that the \kbd{print} command evaluates and prints its (string)
arguments in left-to-right order, ending with a newline (and returns 0
as an unprinted result):

\bprog%
\q print()
\q print(1+3"a,3" ,4)
\q print(a=3, (1 + ((a-3)==print())) (a = (a == 5\bs/2)))
\eprog

\noindent To round this up, here is a less artificial example, used to create
generic matrices\sidx{generic matrix}:

\bprog%
? genmat(u,v,s="x") = \idx{matrix}(u,v,i,j, eval(Str(s "" i "" j)))
? genmat(2,3) + genmat(2,3,m)
\%1 =
[x11 + m11 x12 + m12 x13 + m13]
[x21 + m21 x22 + m22 x23 + m23]
\eprog

\noindent
Note that the argument of \kbd{Str} is evaluated in string context, and
really consists of 5 pieces (exercise: why are the empty strings necessary?).
This part could also have been written as \kbd{concat(concat(Str(s), i), j)}
(but {\it not\/} as \kbd{concat(Str(s), concat(i,j))}!). In practice,
\kbd{Str} will often be easier to use than \kbd{concat}, if slightly more
cryptic.

\noindent The arguments of the following functions are processed in string
context:
\bprog%
\idx{Str}
\idx{addhelp} {\rm (second argument)}
\idx{default} {\rm (second argument)}
\idx{error}
\idx{extern}
\idx{plotstring} {\rm (second argument)}
\idx{plotterm} {\rm (first argument)}
{\rm all the \kbd{\idx{print}{\it xxx\/}} functions}
\idx{read}
\idx{system}
{\rm all the \kbd{\idx{write}{\it xxx\/}} functions}
\eprog

\noindent The arguments of the following functions are processed as keywords:
\bprog%
\idx{alias}
\idx{default} {\rm (first argument)}
\idx{install} {\rm (all arguments but the last)}
\idx{type} {\rm (second argument)}
\idx{whatnow}
\eprog

\section{Interfacing GP with other languages}
\noindent
The PARI library was meant to be interfaced with C programs. This specific
use will be dealt with extensively in Chapter~4. GP itself provides a
convenient, if simple-minded, interpreter, which enables you to execute
rather intricate scripts (see \secref{se:programming}).

Scripts, when properly written, tend to be shorter and much clearer than C
programs, and are certainly easier to write, maintain or debug. You don't
need to deal with memory management, garbage collection, pointers,
declarations, and so on. Because of their intrinsic simplicity, they are more
robust as well. They are unfortunately somewhat slower. Thus their use will
remain complementary: it is suggested that you test and debug your algorithms
using scripts, before actually coding them in C for the sake of speed.

\unix{Note that the \kbd{install} command enables you to concentrate on
critical parts of your programs only (which can of course be written with the
help of other mathematical libraries than PARI!), and to easily and
efficiently import foreign functions for use under GP
(see~\secref{se:install}).}

We are aware of three PARI-related public domain libraries. {\it We neither
endorse nor support any of them}. You might want to give them a try if you
are familiar with the languages they are based on. First, there are
\tet{PariPerl}%
\footnote{*}{
see \kbd{%
 http://nswt.tuwien.ac.at:8000/htdocs/internet/unix/perl/math-pari.html}},
%
written by Ilya Zakharevich (\kbd{ilya@math.ohio-state.edu}),
and \tet{PariPython}%
\footnote{**}{
see \kbd{http://www.math.jussieu.fr/\til{}fermigie/PariPython/readme.html}},
%
by St\'efane Fermigier (\kbd{fermigie@math.jussieu.fr}). Finaly, Michael Stoll
(\kbd{Michael\_Stoll@math.uni-bonn.de}) has integrated PARI into \tet{CLISP},
which is a Common Lisp implementation by Bruno Haible, Marcus Daniels and
others. These provide interfaces to GP functions for use in \kbd{perl},
\kbd{python} or \kbd{Lisp} programs.\sidx{Perl}\sidx{Python}\sidx{Lisp}
To our knowledge, only the \kbd{python} and \kbd{perl} interfaces have been
upgraded to version 2.0 of PARI, the \kbd{CLISP} one being still based on
version 1.39.$xx$.

\section{The preferences file}\sidx{startup}\sidx{gprc}\sidx{preferences file}
\label{se:gprc}

\noindent
When GP is started, it looks for a customization file, or \kbd{gprc} in the
following places (in this order, only the first one found will be read):

\noindent$\bullet$ On the Macintosh (only), GP looks in the directory which
contains the GP executable itself for a file called \kbd{gprc}. No other places
are examined.

\noindent$\bullet$ If the operating system supports environment variables
(essentially, anything but MacOS), GP checks whether the environment variable
\tet{GPRC} is set. Under DOS, you can set it in \kbd{AUTOEXEC.BAT}.
On Unix, this can be done with something like:
\smallskip

\settabs\+\indent&\kbd{GPRC=/my/dir/anyname; export GPRC}\quad&\cr

\+&\kbd{GPRC=/my/dir/anyname; export GPRC}\quad&in \kbd{sh} syntax
(for instance in your \kbd{.profile}),\cr

\+&\kbd{setenv GPRC /my/dir/anyname} &in \kbd{csh} syntax
(in your \kbd{.login} or \kbd{.cshrc} file).\cr

\noindent If so, the file named by \kbd{\$GPRC} is the \kbd{gprc}.

\noindent$\bullet$ If \kbd{GPRC} is not set, and if the environment variable
\kbd{HOME} is defined, GP then tries

\kbd{\$HOME/.gprc} on a Unix system

\kbd{\$HOME\bs\_$\,$gprc} on a DOS, OS/2, or Windows system.

\noindent$\bullet$ If \kbd{HOME} also leaves us clueless, we try

\strut\kbd{\til/.gprc} on a Unix system (where as usual \kbd{\til} stands for
your home directory), or

\kbd{\b{\_}$\,$gprc} on a DOS, OS/2, or Windows system.

\noindent$\bullet$ Finally, if no gprc was found among the user files
mentioned above we look for \kbd{/etc/gprc} (\kbd{\bs etc\bs gprc})
for a system-wide gprc file (you'll need root privileges to set up such a
file yourself).

Note that on Unix systems, the \kbd{gprc}'s default name starts with a '.' and
thus is hidden to regular \kbd{ls} commands; you need to type \kbd{ls -a} to
see whether it's already there without your knowing about it.

In any case, GP will open the corresponding file and process the commands in
there, {\it before\/} doing anything else, e.g.~creating the PARI stack. If
the file doesn't exist or cannot be read, GP will proceed to the
initialization phase at once, eventually emitting a prompt. If any explicit
commandline switches are given, they will override the values read from the
\kbd{gprc} file.

The syntax in this file (and valid in this file only, at this very precise
moment!) is simple-minded, but should be sufficient for most purposes. It
is read line by line, white space being optional as usual (unless surrounded
by quotes). Two types of lines are first dealt with by a preprocessor:

$\bullet$ comments are removed. This applies to all text surrounded by
\kbd{/*~\dots~*/} as well as everything following \kbd{\bs\bs} on a given
line.

$\bullet$ lines starting with \kbd{\#if} \var{keyword} are treated as
comments if \var{keyword} is not defined, and read normally otherwise. The
condition can be negated using either \kbd{\#if not} (or \kbd{\#if !}). Only
two keywords are recognized:

\kbd{EMACS}: defined if GP is running in an Emacs shell (see
\secref{se:emacs}).

\kbd{READL}: defined if GP is compiled with \kbd{readline} support (see
\secref{se:readline}).

\noindent For instance you could set your prompt in the following portable
way:
\bprog%
\b{\bs} self modifying prompt looking like \hbox{{\rm (18:03) \key{gp}} >}
prompt   = "(\%R) \b{e}[1mgp\b{e}[m > "
\h
\b{\bs}  readline wants non-printing characters to be braced between \pow A/\pow B pairs
\#if READL  prompt = "(\%R) \pow A\b{e}[1m\pow Bgp\pow A\b{e}[m\pow B > "
\h
\b{\bs} escape sequences not supported under emacs
\#if EMACS  prompt = "(\%R) gp > "
\eprog

\noindent After the preprocessing there remain two types of lines:

$\bullet$ lines of the form \var{default} \kbd{=} \var{value}, where
\var{default} is one of the available defaults (see \secref{se:defaults}),
which will be set to \var{value} on actual startup. Don't forget the
quotes around strings (e.g.~for \kbd{prompt} or \kbd{help}).

$\bullet$ lines of the form \kbd{read "\var{some\_GP\_file}"} where
\kbd{\var{some\_GP\_file}} is a regular GP script this time, which will
be read just before GP prompts you for commands, but after initializing the
defaults. This is the right place to input files containing \kbd{alias}
commands, or your favorite macros.

A sample \kbd{gprc} file called \kbd{gprc.dft} is provided in the
standard distribution (in directory \kbd{lib}). It's a good idea to have a
look at it and customize it to your needs.

\section{Using GP under GNU Emacs}
\label{se:emacs}

Thanks to the initial help of Annette Hoffman from the University of
Saarbr\"ucken, and David Carlisle from the University of Manchester, it is
possible to use GP as a subprocess of GNU \idx{Emacs}. (Of course, you need
GNU Emacs to be installed on your machine!). To use this, you should
include in your \kbd{.emacs} file the following lines:
\bprog%
(autoload 'gp-mode "\miscdir/pari" nil t)
(autoload 'gp-script-mode "\miscdir/pari" nil t)
(autoload 'gp "\miscdir/pari" nil t)
(autoload 'gpman "\miscdir/pari" nil t)
(setq auto-mode-alist
\q (cons '("\bs\bs.gp\$" . gp-script-mode) auto-mode-alist))
\eprog

where \kbd{pari.el} is the name of the file that will have to be loaded by
GNU Emacs (if you have changed the name, or if you have the file in a
different directory, you must of course supply the correct name). This file
is included in the PARI distribution and probably has been installed at the
same time as GP.

Once this is done, under GNU Emacs if you type \kbd{M-x gp} (where as usual
\kbd{M} is the \kbd{Meta} key, i.e.~Escape, or on SUN keyboards, the Left
key), a special shell will be started, which in particular launches GP with
the default stack size, prime limit and input buffer size. If you type
instead \kbd{C-u M-x gp}, you will be asked for the name of the GP
executable, the stack size, the prime limit and the input buffer size before
the execution of GP begins. If for any of these you simply type return, the
default value will be used. On UNIX machines it will be the place you told
\kbd{Configure} (usually \kbd{/usr/local/bin/gp}) for the executable, 4000000
for the stack, 500000 for the prime limit and 30000 for the buffer size.

\smallskip
  You can then work as usual under GP, but with two notable advantages (which
don't really matter if readline is available to you, see below). First and
foremost, you have at your disposal all the facilities of a text editor like
Emacs, in particular for correcting or copying blocks. Second, you can have
an on-line help which is much more complete than what you obtain by typing
\kbd{?name}. This is done by typing \kbd{M-?}. In the minibuffer, Emacs asks
what function you want to describe, and after your reply you obtain the
description which is in the users manual, including the description of
functions (such as \kbd{\bs}, \kbd{\%}) which use special symbols.

  This help system can also be menu-driven, by using the command
\kbd{M-\char`\\ c} which opens a help menu window which enables you to choose
the category of commands for which you want an explanation.

  Nevertheless, if extended help is available on your system (see
\secref{se:exthelp}), you should use it instead of the above, since it's
nicer (it ran through \TeX) and understands many more keywords.

  Finally you can use command completion in the following way. After the
prompt, type the first few letters of the command, then \kbd{<TAB>} where
\kbd{<TAB>} is the TAB key. If there exists a unique command starting with
the letters you have typed, the command name will be completed. If not,
either the list of commands starting with the letters you typed will be
displayed in a separate window (which you can then kill by typing as usual
\kbd{C-x 1} or by typing in more letters), or ``no match found'' will be
displayed in the Emacs command line. If your GP was linked with the readline
library, read the section on completion in the section below (the paragraph
on online help is not relevant).

Note that if for some reason the session crashes (due to a bug in your
program or in the PARI system), you will usually stay under Emacs, but the GP
buffer will be killed. To recover it, simply type again \kbd{M-x gp} (or
\kbd{C-u M-x gp}), and a new session of GP will be started after the old one,
so you can recover what you have typed. Note that this will of course
{\it not} work if for some reason you exited Emacs before coming back (except
for the \kbd{C-z} temporary stopping command).

\smallskip
You also have at your disposal a few other commands and many possible
customizations (colours, prompt). Read the file \kbd{emacs/pariemacs.txt} in
standard distribution for details.


\section{Using GP with readline}
\sidx{line editor}\sidx{completion}

  Thanks to the initial help of Ilya Zakharevich, there is a possibility of
line editing and command name completion outside of an Emacs buffer {\it
if} you have compiled GP with the GNU \idx{readline} library. If you don't
have Emacs available, or can't stand using it, we really advise you to make
sure you get this very useful library before configuring or compiling GP.
In fact, with \kbd{readline}, even line editing becomes {\it more} powerful
outside an Emacs buffer!

\subsec{A (too) short introduction to readline}:
\label{se:readline}
  The basics are as follows (read the readline user manual~!), assume that
\kbd{C-} stands for ``the \kbd{Control} key combined with another'' and the
same for \kbd{M-} with the \kbd{Meta} key (generally \kbd{C-} combinations
act on characters, while the \kbd{M-} ones operate on words). The \kbd{Meta}
key might be called \kbd{Alt} on some keyboards, will display a black diamond
on most others, and can safely be replaced by \kbd{Esc} in any case. Typing
any ordinary key inserts text where the cursor stands, the arrow keys
enabling you to move in the line. There are many more movement commands,
which will be familiar to the Emacs user, for instance \kbd{C-a}/\kbd{C-e}
will take you to the start/end of the line, \kbd{M-b}/\kbd{M-f} move the
cursor backward/forward by a word, etc. Just press the \kbd{Return} key at
any point to send your command to GP.

  All the commands you type in are stored in a history (with multiline
commands being saved as single concatenated lines). The Up and Down arrows (or
\kbd{C-p}/\kbd{C-n}) will move you through it, \kbd{M-<}/\kbd{M->} sending
you to the start/end of the history. \kbd{C-r}/\kbd{C-s} will start an
incremental backward/forward search. You can kill text (\kbd{C-k} kills till
the end of line, \kbd{M-d} to the end of current word) which you can then
yank back using the \kbd{C-y} key (\kbd{M-y} will rotate the kill-ring).
\kbd{C-\_} will undo your last changes incrementally (\kbd{M-r} undoes all
changes made to the current line). \kbd{C-t} and \kbd{M-t} will transpose
the character (word) preceding the cursor and the one under the cursor.

  Keeping the \kbd{M-} key down while you enter an integer (a minus sign
meaning reverse behaviour) gives an argument to your next readline command
(for instance \kbd{M-- C-k} will kill text back to the start of line). If you
prefer \idx{Vi}--style editing, \kbd{M-C-j} will toggle you to Vi mode.

  Of course you can change all these default bindings. For that you need to
create a file named \kbd{.inputrc} in your home directory. For instance
(notice the embedding conditional in case you would want specific bindings
for GP):
%
\bprog%
\$if Pari-GP
\q set show-all-if-ambiguous
\q "\b{C}-h": backward-delete-char
\q "\b{e}\b{C}-h": backward-kill-word
\q "\b{C}-xd": dump-functions
\q (: "\b{C}-v()\b{C}-b" \qquad \# can be annoying when copy-pasting~!
\q [: "\b{C}-v[]\b{C}-b"
\$endif%
\eprog
%
\noindent\kbd{C-x C-r} will re-read this init file, incorporating any
changes made to it during the current session.

\misctitle{Note:} By default, \kbd{(} and \kbd{[} are bound to the function
\kbd{pari-matched-insert} which, if ``electric parentheses'' are enabled
(default: off) will automatically insert the matching closure (respectively
\kbd{)} and \kbd{]}). This behaviour can be toggled on and off by giving
the numeric argument $-2$ to \kbd{(} (\kbd{M--2(}), which is useful if you
want, e.g to copy-paste some text into the calculator. If you don't want a
toggle, you can use \kbd{M--0} / \kbd{M--1} to specifically switch it on or
off).

\misctitle{Note:} In recent versions of readline (2.1 for instance), the
\kbd{Alt} or \kbd{Meta} key can give funny results (output 8-bit accented
characters for instance). If you don't want to fall back to the \kbd{Esc}
combination, put the following two lines in your \kbd{.inputrc}:
%
\bprog%
  set convert-meta on
  set output-meta off%
\eprog

% don't remove this leading space (needed by gphelp)
 \subsec{Command completion and online help}

  As in the Emacs shell, \kbd{<TAB>} will complete words for you. But, under
readline, this mechanism will be context-dependent: GP will strive to only
give you meaningful completions in a given context (it will fail sometimes,
but only under rare and restricted conditions).

  For instance, shortly after a \kbd{\til}, we expect a user name, then a
path to some file. Directly after \kbd{default(} has been typed, we would
expect one of the \kbd{default} keywords. After \kbd{whatnow(} , we expect
the name of an old function, which may well have disappeared from this
version. After a '.', we expect a member keyword. And generally of course, we
expect any GP symbol which may be found in the hashing lists: functions (both
yours and GP's), and variables.

  If, at any time, only one completion is meaningful, GP will provide it
together with

$\bullet$ an ending comma if we're completing a default,

$\bullet$ a pair of parentheses if we're completing a function name. In
that case hitting \kbd{<TAB>} again will provide the argument list as given
by the online help\footnote{*}{recall that you can always undo the effect
of the preceding keys by hitting \kbd{C-\_}}.

Otherwise, hitting \kbd{<TAB>} once more will give you the list of possible
completions. Just experiment with this mechanism as often as possible,
you'll probably find it very convenient. For instance, you can obtain
\kbd{default(seriesprecision,10)}, just by hitting \kbd{def<TAB>se<TAB>10},
which saves 18 keystrokes (out of 27).

  Hitting \kbd{M-h} will give you the usual short online help concerning the
word directly beneath the cursor, \kbd{M-H} will yield the extended help
corresponding to the \kbd{help} default program (usually opens a \idx{dvi}
previewer, or runs a primitive tex-to-ASCII program). None of these disturb
the line you were editing.
\vfill\eject
