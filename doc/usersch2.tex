% Copyright (c) 2000  The PARI Group
%
% This file is part of the PARI/GP documentation
%
% Permission is granted to copy, distribute and/or modify this document
% under the terms of the GNU General Public License
\chapter{The gp Calculator}

\section{Introduction}

Originally, \tet{gp} was designed as a debugging device for the PARI system
library. Over the years, it has become a powerful user-friendly stand-alone
calculator. The mathematical functions available in PARI and \kbd{gp} are
described in the next chapter. In the present one, we describe the specific
use of the \kbd{gp} programmable calculator.

\emacs If you have GNU Emacs and use the PariEmacs package, you can work in a
special Emacs shell, described in \secref{se:emacs}. Specific features of
this Emacs shell are indicated by an EMACS sign in the left margin.

We briefly mention at this point GNU
TeXmacs (\url{http://www.texmacs.org/}), a free wysiwyg editing
platform that allows to embed an entire gp session in a document, and
provides a nice alternative to PariEmacs.

\subsec{Startup}

To start the calculator, the general command line syntax is:

\kbd{gp [-D \var{key}=\var{val}] [\var{files}]}

\noindent
where items within brackets are optional. The [\var{files}] argument is a
list of files written in the GP scripting language, which will be loaded on
startup. There can be any number of arguments of the form
\kbd{-D \var{key}=\var{val}}, setting some internal parameters of \kbd{gp},
or \var{defaults}: each sets the default \var{key} to the value \var{val}. See
\secref{se:defaults} below for a list and explanation of all defaults. These
defaults can be changed by adding parameters to the input line as above, or
interactively during a \kbd{gp} session, or in a preferences file also known
as \tet{gprc}.

If a \idx{preferences file} (to be discussed in \secref{se:gprc}) is
found, \kbd{gp} then reads it and executes the commands it contains. This
provides an easy way to customize \kbd{gp}. The \var{files} argument is
processed right after the \kbd{gprc}.

A copyright banner then appears which includes the version number, and a lot
of useful technical information. After the copyright, the computer writes the
top-level help information, some initial defaults, and then waits after
printing its prompt, which is '\kbd{?~}' by default . Whether extended
on-line help and line editing are available or not is indicated in this
\kbd{gp} banner, between the version number and the copyright message.
Consider investigating the matter with the person who installed \kbd{gp} if
they are not. Do this as well if there is no mention of the GMP kernel.

\subsec{Getting help}

To get help, type a \kbd{?} and hit return. A menu appears, describing the
main categories of available functions and how to get more detailed
help. If you now type \kbd{?$n$} with $n = 1, 2, \dots$, you get the list of
commands corresponding to category $n$ and simultaneously to Section $3.n$ of
this manual. If you type \kbd{?}\var{functionname} where \var{functionname}
is the name of a PARI function, you will get a short explanation of this
function.

If extended help (see \secref{se:exthelp}) is available on your system,
you can double or triple the \kbd{?} sign to get much more: respectively the
complete description of the function (e.g.~\kbd{??sqrt}), or a list of
\kbd{gp} functions relevant to your query (e.g.~ \kbd{???"elliptic curve"}
or \kbd{???"quadratic field"}).

If \kbd{gp} was properly installed (see Appendix~A), a line editor is
available to correct the command line, get automatic completions, and so on.
See \secref{se:readline} or \kbd{??readline} for a short summary of the line
editor's commands.

If you type \kbd{?\bs} you will get a short description of the metacommands
(keyboard shortcuts).

Finally, typing \kbd{?.} will return the list of available (pre-defined)
member functions. These are functions attached to specific kind of objects,
used to retrieve easily some information from complicated structures (you can
define your own but they won't be shown here). We will soon describe these
commands in more detail.

More generally, commands starting with the symbols \b\ or \kbd{?}, are not
computing commands, but are metacommands which allow you to exchange
information with \kbd{gp}. The available metacommands can be divided into
default setting commands (explained below) and simple commands (or keyboard
shortcuts, to be dealt with in \secref{se:meta}).

\subsec{Input}

Just type in an instruction, e.g. \kbd{1 + 1}, or \kbd{Pi}. No action is
undertaken until you hit the \kbd{<Return>} key. Then computation starts, and
a result is eventually printed. To suppress printing of the result, end the
expression with a \kbd{;} sign. Note that many systems use \kbd{;} to
indicate end of input. Not so in \kbd{gp}: a final semicolon means the
result should not be printed. (Which is certainly useful if it occupies
several screens.)

\subsec{Interrupt, Quit}

Typing \kbd{quit} at the prompt ends the session and exits \kbd{gp}. At any
point you can type \kbd{Ctrl-C} (that is press simultaneously the
\kbd{Control} and \kbd{C} keys): the current computation is interrupted and
control given back to you at the \kbd{gp} prompt, together with a message
like
\bprog
  ***   at top-level: gcd(a,b)
  ***                 ^--------
  *** gcd: user interrupt after 236 ms.
@eprog\noindent
telling you how much time elapsed since the last command was typed in and
in which GP function the computation was aborted. It does not mean that that
much time was spent in the function, only that the evaluator was busy
processing that specific function when you stopped it.

\section{The general gp input line}

The \kbd{gp} calculator uses a purely interpreted language GP. The structure
of this language is reminiscent of LISP with a functional notation,
\kbd{f(x,y)} rather than \kbd{(f x y)}: all programming constructs,
such as \kbd{if}, \kbd{while,} etc\dots are functions\footnote{*}{Not exactly,
since not all their arguments need be evaluated. For instance it would be
stupid to evaluate both branches of an \kbd{if} statement: since only one
will apply, only this one is evaluated.}, and the main loop does not really
execute, but rather evaluates (sequences of) expressions. Of course, it is by
no means a true LISP, and has been strongly influenced by C and Perl since
then.

\subsec{Introduction} User interaction with a \kbd{gp} session proceeds as
follows. First, one types a sequence of characters at the \kbd{gp} prompt;
see \secref{se:readline} for a description of the line editor. When you hit
the \kbd{<Return>} key, \kbd{gp} gets your input, evaluates it, then prints
the result and assigns it to an ``history'' array.

More precisely, the input is case-sensitive and, outside of character
strings, blanks are completely ignored. Inputs are either metacommands or
sequences of expressions. Metacommands are shortcuts designed to alter gp's
internal state, such as the working precision or general verbosity level; we
shall describe them in \secref{se:meta}, and ignore them for the time being.

The evaluation of a sequence of instructions proceeds in two phases: your
input is first digested (byte-compiled) to a bytecode suitable for fast
evaluation, in particular loop bodies are compiled only once but a priori
evaluated many times; then the bytecode is evaluated.

An \idx{expression}\sidx{expression sequence} is formed by combining
constants, variables, operator symbols, functions and control statements.
It is evaluated using the conventions about operator priorities and left to
right associativity. An expression always has a value, which can be any PARI
object:
\bprog
? 1 + 1
%1 = 2          \\@com an ordinary integer
? x
%2 = x          \\@com a polynomial of degree 1 in the unknown \kbd{x}
? print("Hello")
Hello           \\@com \kbd{void} return value, 'Hello' printed as side effect
? f(x) = x^2
%4 = (x)->x^2   \\@com a user function
@eprog
\noindent In the third example, \kbd{Hello} is printed as a side effect, but
is not the return value. The \kbd{print} command is a \emph{procedure},
which conceptually returns nothing. But in fact procedures return a special
\kbd{void} object, meant to be ignored (but which evaluates
to $0$ in a numeric context, and stored as $0$ in the history or results).
The final example assigns to the variable \kbd{f} the function $x\mapsto
x^2$, the alternative form \kbd{f = x->x\pow2} achieving the same effect; the
return value of a function definition is, unsurprisingly, a function object
(of type \typ{CLOSURE}).

Several expressions are combined on a single line by separating them with
semicolons ('\kbd{;}'). Such an expression sequence will be called a
\var{seq}. A \var{seq} also has a value, which is the value of the last
expression in the sequence. Under \kbd{gp}, the value of the \var{seq}, and
only this last value, becomes an history entry. The values of the other
expressions in the \var{seq} are discarded after the execution of the
\var{seq} is complete, except of course if they were assigned into variables.
In addition, the value of the \var{seq} is printed if the line does not end
with a semicolon \kbd{;}.

\subsec{The gp history of results}

This is not to be confused with the history of your \emph{commands},
maintained by \kbd{readline}. The \kbd{gp} history contains the \emph{results}
they produced, in sequence.

The successive elements of the history array are called \kbd{\%1}, \kbd{\%2},
\dots As a shortcut, the latest computed expression can also be
called \kbd{\%}, the previous one \kbd{\%`},
the one before that \kbd{\%``} and so on.

When you suppress the printing of the result with a semicolon, it is still
stored in the history, but its history number will not appear either. It is a
better idea to assign it to a variable for later use than to mentally
recompute what its number is. Of course, on the next line, you may just use
\kbd{\%}.

The time used to compute that history entry is also stored as part of the entry
and can be recovered using the \kbd{\%\#} operator: \kbd{\%\#1}, \kbd{\%\#2},
\kbd{\%\#`}; \kbd{\%\#} by itself returns the time needed to compute the last
result (the one returned by \kbd{\%}).

\misctitle{Remark}
The history ``array'' is in fact better thought of as a queue: its size is
limited to 5000 entries by default, after which \kbd{gp} starts forgetting
the initial entries. So \kbd{\%1} becomes unavailable as \kbd{gp} prints
\kbd{\%5001}. You can modify the history size using \tet{histsize}.

\subsec{Special editing characters}\sidx{editing characters} A GP program
can of course have more than one line. Since your commands are executed as
soon as you have finished typing them, there must be a way to tell \kbd{gp}
to wait for the next line or lines of input before doing anything. There are
three ways of doing this.

The first one is to use the \idx{backslash character} \kbd{\bs} at the end of
the line that you are typing, just before hitting \kbd{<Return>}. This tells
\kbd{gp} that what you will write on the next line is the physical
continuation of what you have just written. In other words, it makes \kbd{gp}
forget your newline character. You can type a \kbd{\bs} anywhere. It is
interpreted as above only if (apart from ignored whitespace characters) it is
immediately followed by a newline. For example, you can type
\bprog
? 3 + \
4
@eprog
\noindent instead of typing \kbd{3 + 4}.

The second one is a variation on the first, and is mostly useful when
defining a user function (see \secref{se:user_defined}): since an equal sign
can never end a valid expression, \kbd{gp} disregards a newline immediately
following an \kbd{=}.
\bprog
? a =
123
%1 = 123
@eprog

The third one is in general much more useful, and uses braces \kbd{\obr} and
\kbd{\cbr}.\sidx{brace characters} An opening brace \kbd{\obr} signals that
you are typing a multi-line command, and newlines are ignored until you type
a closing brace \kbd{\cbr}. There are two important, but easily obeyed,
restrictions: first, braces do not nest; second, inside an open brace-close
brace pair, all input lines are concatenated, suppressing any newlines. Thus,
all newlines should occur after a semicolon (\kbd{;}), a comma (\kbd{,}) or
an operator (for clarity's sake, never split an identifier over two lines in
this way). For instance, the following program
\bprog
{
  a = b
  b = c
}
@eprog

\noindent would silently produce garbage, since this is interpreted as
\kbd{a=bb=c} which assigns the value of \kbd{c} to both \kbd{bb} and
\kbd{a}. It should have been written
\bprog
{
  a = b;
  b = c;
}
@eprog

\section{The PARI types}

\noindent
We see here how to input values of the different data types known to PARI.
Recall that blanks are ignored in any expression which is not a string (see
below).

\misctitle{A note on efficiency}
The following types are provided for convenience, not for speed:
\typ{INTMOD}, \typ{FRAC}, \typ{PADIC}, \typ{QUAD}, \typ{POLMOD},
\typ{RFRAC}. Indeed, they always perform a reduction of some kind after
each basic operation, even though it is usually more efficient to perform
a single reduction at the end of some complex computation. For instance,
in a convolution product $\sum_{i+j = n} x_i y_j$ in $\Z/N\Z$ --- common
when multiplying polynomials! ---, it is quite wasteful to perform $n$
reductions modulo $N$. In short, basic individual operations on these types
are fast, but recursive objects with such components could be handled more
efficiently: programming with libpari will save large constant factors here,
compared to GP.

\subsec{Integers (\typ{INT})}%
\sidx{integer}\kbdsidx{t_INT}
After an (optional) leading \kbd{+} or \kbd{-}, type in the
decimal digits of your integer. No decimal point!
\bprog
? 1234567
%1 = 1234567
? -3
%2 = -3
? 1.         \\@com oops, not an integer
%3 = 1.000000000000000000000000000
@eprog

Integers can be input in hexadecimal notation by prefixing them with
\kbd{0x}; hexadecimal digits ($a, \dots, f$) can be input either in lowercase
or in uppercase:
\bprog
? 0xF
%4 = 15
? 0x1abcd
%5 = 109517
@eprog

Integers can also be input in binary by prefixing them with
\kbd{0b}:
\bprog
? 0b010101
%6 = 21
@eprog

\subsec{Real numbers (\typ{REAL})}%
\sidx{real number}\kbdsidx{t_REAL}

Real numbers are represented (approximately) in a floating point system,
internally in base 2, but converted to base 10 for input / output purposes.
A \typ{REAL} object has a given \emph{accuracy} (or \emph{precision}) $\ell
\geq 0$; it comprises

\item a sign $s$: $+1$, $-1$ or $0$;

\item a mantissa $m$: a multiprecision integer, $0\leq m < 10^\ell$;

\item an exponent $e$: a small integer in $[-E,E]$, where $E \approx 2^B
  \log_{10} 2$, and $B = 32$ on a 32-bit machine and 64 otherwise.

This data may represent any real number $x$ such that
$$|x - s m 10^e| < 10^{e-\ell}.$$
We consider that a \typ{REAL} with sign $s = 0$ has accuracy $\ell = 0$, so
that its mantissa is useless, but it still has an exponent $e$ and acts like
a machine epsilon for all accuracies $< e$.

After an (optional) leading \kbd{+} or \kbd{-}, type a number with a decimal
point. Leading zeroes may be omitted, up to the decimal point, but trailing
zeroes are important: your \typ{REAL} is assigned an internal precision,
which is the supremum of the input precision, one more than the number of
decimal digits input, and the default \tet{realprecision}. For example, if
the default precision is 28 digits, typing \kbd{2.} yields a precision of 28
digits, but \kbd{2.0\dots0} with 45 zeros gives a number with internal
precision at least 45, although less may be printed.

You can also use scientific notation with the letter \kbd{E} or
\kbd{e}. As usual, \kbd{e$n$} is interpreted as $\times 10^n$ for all
integers $n$. Since the result is converted to a \typ{REAL}, you may often
omit the decimal point in this case: \kbd{6.02 E 23} or \kbd{1e-5} are fine,
but \kbd{e10} is not.

By definition, \kbd{0.E $n$} returns a real $0$ of exponent $n$, whereas
\kbd{0.} returns a real 0 ``of default precision'' (of exponent
$-\tet{realprecision}$), see \secref{se:whatzero}, behaving like the machine
epsilon for the current default accuracy: any float of smaller absolute value
is indistinguishable from $0$.

\misctitle{Note on output formats} A zero real number is printed in \kbd{e}
format as $0.Exx$ where $xx$ is the (usually negative) \emph{decimal}
exponent of the number (cf.~\secref{se:whatzero}). This allows the user to
check the accuracy of that particular zero.

When the integer part of a real number $x$ is not known exactly because the
exponent of $x$ is greater than the internal precision, the real number is
printed in \kbd{e} format.

\misctitle{Technical note} The internal \emph{precision} is actually
expressed in bits and can be viewed and manipulated globally in interactive
use via \kbd{realprecision} (decimal digits, as explained above; shortcut
\kbd{\bs p}) or \kbd{realbitprecision} (bits; shortcut \kbd{\bs ps}), the
latter allowing finer granularity. See \secref{se:trans} for details. In
programs we advise to leave this global variable alone and adapt precision
locally for a given sequence of computations using \tet{localbitprec}.

\subsec{Intmods (\typ{INTMOD})}%
\sidx{intmod}\kbdsidx{t_INTMOD}
To create the image of the integer $a$ in $\Z/b\Z$ (for some non-zero
integer $b$), type \kbd{Mod(a,b)}; \emph{not} \kbd{a\%b}.
Internally, all operations are done on integer representatives belonging to
$[0,b-1]$.

Note that this type is available for convenience, not for speed: each
elementary operation involves a reduction modulo $b$.

If $x$ is a \typ{INTMOD} \kbd{Mod(a,b)}, the following member function is
defined:

\kbd{x.mod}: return the modulus \kbd{b}.

\subsec{Rational numbers (\typ{FRAC})}%
\sidx{rational number}\kbdsidx{t_FRAC}
All fractions are automatically reduced to lowest
terms, so it is impossible to work with reducible fractions. To enter $n/m$
just type it as written. As explained in \secref{se:gdiv}, floating point
division is \emph{not} performed, only reduction to lowest
terms.\label{se:FRAC}

Note that rational computation are almost never the fastest method to proceed:
in the PARI implementation, each elementary operation involves computing a gcd.
It is generally a little more efficient to cancel denominators and work with
integers only:
\bprog
? P = Pol( vector(10^3,i, 1/i) ); \\@com big polynomial with small rational coeffs
? P^2
time = 1,392 ms.
? c = content(P); c^2 * (P/c)^2;  \\@com same computation in integers
time = 1,116 ms.
@eprog\noindent
And much more efficient (but harder to setup) to use homomorphic imaging
schemes and modular computations. As the simple example below indicates, if you
only need modular information, it is very worthwhile to work with
\typ{INTMOD}s directly, rather than deal with \typ{FRAC}s all the way through:
\bprog
? p = nextprime(10^7);
? sum(i=1, 10^5, 1/i) % p
time = 13,288 ms.
%1 = 2759492
? sum(i=1, 10^5, Mod(1/i, p))
time = 60 ms.
%2 = Mod(2759492, 10000019)
@eprog\noindent

\subsec{Finite field elements (\typ{FFELT})}%
\sidx{finite field element}\kbdsidx{t_FFELT}
Let $T\in\F_p[X]$ be a monic irreducible polynomial defining your
finite field over $\F_p$, for instance obtained using \tet{ffinit}. Then the
\tet{ffgen} function creates a generator of the finite field as an
$\F_p$-algebra, namely the class of $X$ in $\F_p[X]/(T)$, from which you can
build all other elements. For instance, to create the field $\F_{2^8}$, we
write
\bprog
? T = ffinit(2, 8);
? y = ffgen(T, 'y);
? y^0    \\ the unit element in the field
%3 = 1
? y^8
%4 = y^6 + y^5 + y^4 + y^3 + y + 1
@eprog\noindent
The second (optional) parameter to \tet{ffgen} is only used to display
the result; it is customary to use the name of the variable we assign the
generator to. If \kbd{g} is a \typ{FFELT}, the following member functions are
defined:

\kbd{g.pol}: the polynomial (with reduced integer coefficients)
expressing \kbd{g} in term of the field generator.

\kbd{g.p}: the characteristic of the finite field.

\kbd{g.f}: the dimension of the definition field over its prime field; the
cardinality of the definition field is thus $p^f$.

\kbd{g.mod}: the minimal polynomial (with reduced integer
coefficients) of the field generator.

\subsec{Complex numbers (\typ{COMPLEX})}%
\sidx{complex number}\kbdsidx{t_COMPLEX}
To enter $x+iy$, type \kbd{x + I*y}. (That's \kbd{I}, \emph{not} \kbd{i}!)
The letter \tet{I} stands for $\sqrt{-1}$. The ``real'' and ``imaginary''
parts $x$ and $y$ can be of type \typ{INT}, \typ{REAL}, \typ{INTMOD},
\typ{FRAC}, or \typ{PADIC}.

\subsec{$p$-adic numbers (\typ{PADIC}):}%
\sidx{p-adic number}\label{se:padic}\kbdsidx{t_PADIC}
Typing \kbd{O($p$\pow $k$)}, where $p$ and $k$ are integers, yields a
$p$-adic $0$ of accuracy~$k$, representing any $p$-adic number whose
valuation is $\geq k$. To input a general non-0 $p$-adic number, write
a suitably precise rational or integer approximation and add \kbd{O($p$\pow
$k$)} to it.

Note that it is not checked whether $p$ is indeed prime but results are
undefined if this is not the case: you can work on $10$-adics if you want,
but disasters will happen as soon as you do something non-trivial like
taking a square root. Note that \kbd{O(25)} is not the same as
\kbd{O(5\pow 2)}; you want the latter!

For example, you can type in the $7$-adic number

\kbd{2*7\pow(-1) + 3 + 4*7 + 2*7\pow 2 + O(7\pow3)}

\noindent exactly as shown, or equivalently as \kbd{905/7 + O(7\pow3)}.

If $a$ is a \typ{PADIC}, the following member functions are defined:

\kbd{a.mod}: returns the modulus $p^k$.

\kbd{a.p}: returns $p$.

Note that this type is available for convenience, not for speed:
internally, \typ{PADIC}s are stored as $p$-adic units modulo some $p^k$.
Each elementary operation involves updating $p^k$ (multiplying or
dividing by powers of $p$) and a reduction mod $p^k$. In particular,
additions are slow.
\bprog
    ? n = 1+O(2^20);   for (i=1,10^6, n++)
    time = 841 ms.
    ? n = Mod(1,2^20); for (i=1,10^6, n++)
    time = 441 ms.
    ? n = 1;           for (i=1,10^6, n++)
    time = 328 ms.
@eprog\noindent The penalty attached to maintaining $p^k$ decreases
steeply as $p$ increases (and updates become rare). But \typ{INTMOD}s
remain at least 25\% more efficient. (On the other hand, they do not allow
denominators!)
% n = 1+O(1009^2);   for (i=1,10^6, n++)
% n = Mod(1,1009^2); for (i=1,10^6, n++)

\subsec{Quadratic numbers (\typ{QUAD})}%
\sidx{quadratic number}\kbdsidx{t_QUAD}
This type is used to work in the quadratic order of \emph{discriminant}
\kbd{d}, where \kbd{d} is a non-square integer congruent to $0$ or $1$
(modulo $4$). The command
\bprog
    w = quadgen(d)
@eprog\noindent
assigns to \kbd{w} the ``canonical'' generator for the integer basis
of the order of discriminant $d$, i.e.~$w=\sqrt{d}/2$ if $d\equiv 0 \mod 4$,
and $w=(1+\sqrt{d})/2$ if $d\equiv 1 \mod 4$. The name \kbd{w} is of course
just a suggestion, but corresponds to traditional usage. You can use any
variable name that you like, but \kbd{quadgen(d)} is always printed as
\kbd{w}, regardless of the discriminant. So beware, two \typ{QUAD}s can be
printed in the same way and not be equal; however, \kbd{gp} will refuse to add
or multiply them for example.

Since the order is $\Z + \kbd{w}\Z$, any other element can be input
as \kbd{z = $x$+$y$*w} for some integers $x$ and $y$. In fact, you may work in
its fraction field $\Q(\sqrt{d})$ and use \typ{FRAC} values for $x$ and $y$.

The member function \kbd{z.disc} retrieves the discriminant $d$; $x$ and $y$
are obtained via \kbd{real(z)} and \kbd{imag(z)} respectively.

\subsec{Polmods (\typ{POLMOD})}%
\sidx{polmod}\kbdsidx{t_POLMOD}
Exactly as for intmods, to enter $x \mod y$ (where $x$ and $y$ are
polynomials), type \kbd{Mod(x,y)}, not \kbd{x\%y}. Note that when $y$ is an
irreducible polynomial in one variable, polmods whose modulus is $y$ are simply
algebraic numbers in the finite extension defined by the polynomial $y$.
This allows us to work easily in \idx{number field}s, finite extensions of
the $p$-adic field $\Q_p$, or \idx{finite field}s.

Note that this type is available for convenience, not for speed: each
elementary operation involves a reduction modulo $y$.
If $p$ is a \typ{POLMOD}, the following member functions are defined:

\kbd{p.pol}: return a representative of the polynomial class of minimal degree.

\kbd{p.mod}: return the modulus.

\label{se:rempolmod}
\misctitle{Important remark}\sidx{variable (priority)}
Mathematically, the variables\sidx{variable} occurring in a polmod are not
free variables. But internally, a congruence class in $R[t]/(y)$ is
represented by its representative  of lowest degree, which is a \typ{POL} in
$R[t]$, and computations occur with polynomials in the variable $t$. PARI
will not recognize that \kbd{Mod(y, y\pow2 + 1)} is ``the same'' as
\kbd{Mod(x, x\pow2 + 1)}, since \kbd{x} and \kbd{y} are different variables.

To avoid inconsistencies, polmods must use the same variable in internal
operations (i.e.~between polmods) and variables of lower priority for
external operations, typically between a polynomial and a polmod. See
\secref{se:priority} for a definition of ``priority'' and a discussion of
(PARI's idea of) multivariate polynomial arithmetic.
For instance:
\bprog
    ? Mod(x, x^2+ 1) + Mod(x, x^2 + 1)
    %1 = Mod(2*x, x^2 + 1)    \\@com $2i$ (or $-2i$), with $i^2=-1$
    ? x + Mod(y, y^2 + 1)
    %2 = x + Mod(y, y^2 + 1)  \\@com in $\Q(i)[x]$
    ? y + Mod(x, x^2 + 1)
    %3 = Mod(x + y, x^2 + 1)  \\@com in $\Q(y)[i]$
@eprog\noindent
The first two are straightforward, but the last one may not be what you
want: \kbd{y} is treated here as a numerical parameter, not as a polynomial
variable.

If the main variables are the same, it is allowed to mix \typ{POL} and
\typ{POLMOD}s. The result is the expected \typ{POLMOD}. For instance
\bprog
    ? x + Mod(x, x^2 + 1)
    %1 = Mod(2*x, x^2 + 1)
@eprog

\subsec{Polynomials (\typ{POL})}%
\sidx{polynomial}\label{se:pol}\kbdsidx{t_POL}
Type the polynomial in a natural way, not forgetting to put a ``$*$'' between
a coefficient and a formal variable;
\bprog
? 1 + 2*x + 3*x^2
%1 = 3*x^2 + 2*x + 1
@eprog\noindent
This assumes that \kbd{x} is still a ''free variable''.
\bprog
? x = 1; 1 + 2*x + 3*x^2
%2 = 6
@eprog\noindent
generates an integer, not a polynomial! It is good practice to never assign
values to polynomial variables to avoid the above problem, but a foolproof
construction is available using \kbd{'x} instead of~\kbd{x}: \kbd{'x}
is a constant evaluating to the free variable with name \kbd{x},
independently of the current value of~\kbd{x}.
\bprog
? x = 1; 1 + 2*'x + 3*'x^2
%3 = 1 + 2*x + 3*x^2
? x = 'x; 1 + 2*x + 3*x^2
%4 = 1 + 2*x + 3*x^2
@eprog\noindent
You may also use the functions \kbd{Pol} or \kbd{Polrev}:
\bprog
? Pol([1,2,3])       \\@com \kbd{Pol} creates a polynomial in \kbd{x} by default
%1 = x^2 + 2*x + 3
? Polrev([1,2,3])
%2 = 3*x^2 + 2*x + 1
? Pol([1,2,3], 'y)   \\@com we use \kbd{'y}, safer than \kbd{y}
%3 = y^2 + 2*y + 3
@eprog\noindent The latter two are much more efficient constructors than an
explicit summation (the latter is quadratic in the degree, the former linear):
\bprog
? for (i=1, 10^4, Polrev( vector(100, i,i) ) )
time = 124ms

? for (i=1, 10^4, sum(i = 1, 100, (i+1) * 'x^i) )
time = 3,985ms
@eprog

Polynomials are always printed as \emph{univariate} polynomials, with
monomials sorted by decreasing degree:
\bprog
? (x+y+1)^2
%1 = x^2 + (2*y + 2)*x + (y^2 + 2*y + 1)
@eprog\noindent
(Univariate polynomial in \kbd{x} whose coefficients are polynomials in
\kbd{y}.) See \secref{se:varsymb} for valid variable names, and a discussion
of multivariate polynomial rings.

\subsec{Power series (\typ{SER})}%
\sidx{power series}\kbdsidx{t_SER}\label{se:series}
Typing \kbd{O(X\pow $k$)}, where $k$ is an integer, yields an $X$-adic $0$ of
accuracy~$k$, representing any power series in \kbd{X} whose valuation is
$\geq k$. Of course, \kbd{X} can be replaced by any other variable name! To
input a general non-0 power series, type in a polynomial or rational
function (in \kbd{X}, say), and add \kbd{O(X\pow $k$)} to it. The discussion
in the \typ{POL} section about variables remains valid; a constructor
\tet{Ser} replaces \tet{Pol} and \tet{Polrev}.

\misctitle{Caveat} Power series with inexact coefficients sometimes have a
non-intuitive behavior: if $k$ significant terms are requested, an inexact
zero is counted as significant, even if it is the coefficient of lowest
degree. This means that useful higher order terms may be disregarded.

If a series with a zero leading coefficient must be inverted, then as a
desperation measure that coefficient is discarded, and a warning is issued:
\bprog
? C = 0. + y + O(y^2);
? 1/C
  *** _/_: Warning: normalizing a series with 0 leading term.
%2 = y^-1 + O(1)
@eprog\noindent
The last output could be construed as a bug since it is a priori impossible
to deduce such a result from the input ($0.$ represents any sufficiently
small real number). But it was thought more useful to try and go on with an
approximate computation than to raise an early exception.

If the series precision is insufficient, errors may occur (mostly division
by $0$), which could have been avoided by a better global understanding of
the computation:
\bprog
? A = 1/(y + 0.); B = 1. + O(y);
? B * denominator(A)
%2 = 0.E-28 + O(y)
? A/B
  *** _/_: Warning: normalizing a series with 0 leading term.
%3 = 1.000000000000000000000000000*y^-1 + O(1)
? A*B
  *** _*_: Warning: normalizing a series with 0 leading term.
%4 = 1.000000000000000000000000000*y^-1 + O(1)
@eprog

\subsec{Rational functions (\typ{RFRAC})}%
\sidx{rational function}\kbdsidx{t_RFRAC}
As for fractions, all rational functions are automatically reduced to lowest
terms. All that was said about fractions in \secref{se:FRAC} remains valid
here.

\subsec{Binary quadratic forms of positive or negative discriminant
(\typ{QFR} and \typ{QFI})}%
\sidx{binary quadratic form}\kbdsidx{t_QFR}\kbdsidx{t_QFI}
These are input using the function \kbd{Qfb}. For example \kbd{Qfb(1,2,3)}
creates the binary form $q = x^2+2xy+3y^2$. It is imaginary (of internal type
\typ{QFI}) since its discriminant $2^2 - 4\times 3 = -8$ is negative.
Although imaginary forms could be positive or negative definite, only
positive definite forms are implemented.

The discriminant can be retrieved via \kbd{poldisc} or \kbd{q.disc}. The
individual components are obtained via either of
\bprog
  [a,b,c] = Vec(q);
  a = component(q,1);
  b = component(q,2);
  c = component(q,3);
@eprog

In the case of forms with positive discriminant (\typ{QFR}), you may add an
optional fourth component (related to the regulator, more precisely to Shanks
and Lenstra's ``distance''), which must be a real number. See also the
function \kbd{qfbprimeform} which directly creates a prime form of given
discriminant.

\subsec{Row and column vectors (\typ{VEC} and \typ{COL})}%
\sidx{row vector}\sidx{column vector}\kbdsidx{t_VEC}\kbdsidx{t_COL})
To enter a row vector, type the components separated by commas ``\kbd{,}'',
and enclosed between brackets ``\kbd{[}$\,$'' and ``$\,$\kbd{]}'',
e.g.~\kbd{[1,2,3]}. To enter a column vector, type the vector horizontally,
and add a tilde ``\til'' to transpose. \kbd{[ ]} yields the empty (row)
vector. The function \tet{Vec} can be used to transform any object into a
vector (see Chapter~3). The construction $[i..j]$, where $i \leq j$ are two
integers returns the vector $[i, i+1,\dots, j-1, j]$
\bprog
? [1,2,3]
%1 = [1, 2, 3]
? [-2..3]
%2 = [-2, -1, 0, 1, 2, 3]
@eprog

Let the variable $v$ contain a (row or column) vector:

\item \kbd{v[m]} refers to its $m$-th entry; you can assign any value to
\kbd{v[m]},  i.e.~write something like $v[m]=\var{expr}$.

\item \kbd{v[i..j]}, where $i \leq j$, returns the vector slice containing
elements $v[i],\dots, v[j]$; you can \emph{not} assign a result to
\kbd{v[i..j]}.

\item \kbd{v[\pow i]} returns the vector whose $i$-th entry has been removed;
you can \emph{not} assign a result to \kbd{v[\pow i]}.

\noindent In the last two constructions \kbd{v[i..j]} and \kbd{v[\pow i]},
$i$ and $j$ are allowed to be negative integers, in which case, we start
counting from the end of the vector: e.g., $-1$ is the index of the last
element.
\bprog
? v = [1,2,3,4];
? v[2..4]
%2 = [2, 3, 4]
? v[^3]
%3 = [1, 2, 4]
? v[^-1]
%3 = [1, 2, 3]
? v[-3..-1]
%4 = [2, 3, 4]
@eprog

\misctitle{Remark} \tet{vector} is the standard constructor for row vectors
 whose $i$-th entry is given by a simple function of $i$; \tet{vectorv}
 is similar for column vectors:
 \bprog
 ? vector(10, i, i^2+1)
 %1 = [2, 5, 10, 17, 26, 37, 50, 65, 82, 101]
 @eprog
 The functions \tet{Vec} and \tet{Col} convert objects to row and column
vectors respectively (as well as \tet{Vecrev} and \tet{Colrev}, which revert
the indexing):
 \bprog
 ? T = poltchebi(5)   \\ 5-th Chebyshev polynomial
 %1 = 16*x^5 - 20*x^3 + 5*x
 ? Vec(T)
 %2 = [16, 0, -20, 0, 5, 0]  \\ coefficients of T
 ? Vecrev(T)
 %3 = [0, 5, 0, -20, 0, 16]  \\ ... in reverse order
 @eprog

\misctitle{Remark} For $v$ a \typ{VEC}, \typ{COL}, \typ{LIST} or \typ{MAT},
 the alternative set-notations
 \bprog
 [g(x) | x <- v, f(x)]
 [x | x <- v, f(x)]
 [g(x) | x <- v]
 @eprog\noindent
 are available as shortcuts for
 \bprog
 apply(g, select(f, Vec(v)))
 select(f, Vec(v))
 apply(g, Vec(v))
 @eprog\noindent respectively, and may serve as \typ{VEC} constructors:
 \bprog
 ? [ p | p <- primes(10), isprime(p+2) ]
 %2 = [3, 5, 11, 17, 29]
 @eprog\noindent returns the primes $p$ (among the first 10 primes) such
 that $(p, p+2)$ is a twin pair;
 \bprog
 ? [ p^2 | p <- primes(10), p % 4 == 1 ]
 %1 = [25, 169, 289, 841]
 @eprog\noindent returns the squares of the primes congruent to $1$ modulo $4$,
 where $p$ runs among the first 10 primes.

\subsec{Matrices (\typ{MAT})}%
\sidx{matrix}\kbdsidx{t_MAT}
To enter a matrix, type the components row by row, the components being
separated by commas ``\kbd{,}'', the rows by semicolons ``\kbd{;}'', and
everything enclosed in brackets ``\kbd{[}$\,$'' and ``$\,$\kbd{]}'', e.g.
\kbd{[x,y; z,t; u,v]}. \kbd{[;]} yields an empty ($0 \times 0$) matrix. The
function \tet{Mat} transforms any object into a matrix, and \tet{matrix}
creates matrices whose $(i,j)$-th entry is described by a function $f(i,j)$:
\bprog
? Mat(1)
%1 =
[1]
? matrix(2,2, i,j, 2*i+j)
%2 =
[3 4]

[5 6]
@eprog

\noindent Let the variable $M$ contain a matrix, and let $i,j,k,l$ denote
four
integers:

\item \kbd{M[i,j]} refers to its $(i,j)$-th entry; you can assign any result
to \kbd{M[i,j]}.

\item \kbd{M[i,]} refers to its $i$-th row; you can assign a \typ{VEC}
of the right dimension to \kbd{M[i,]}.

\item \kbd{M[,j]} refers to its $j$-th column; you can assign a \typ{COL}
of the right dimension to \kbd{M[,j]}.

\noindent \emph{But} \kbd{M[i]} is meaningless and triggers an error. The
``range'' $i..j$ and ``caret'' \kbd{\pow}$c$ notations are available as for
vectors; you can not \emph{assign} to any of these:

\item \kbd{M[i..j, k..l]}, $i\leq j$, $k\leq l$, returns the submatrix built
from the rows $i$ to $j$ and columns $k$ to $l$ of $M$.

\item \kbd{M[i..j,]} returns the submatrix built from the rows $i$ to
$j$ of $M$.

\item \kbd{M[,i..j]} returns the submatrix built from the columns $i$ to
$j$ of $M$.

\item \kbd{M[i..j, \pow k]}, $i\leq j$, returns the submatrix built
from the rows $i$ to $j$ and column $k$ removed.

\item \kbd{M[\pow k,]} returns the submatrix with row $k$ removed.

\item \kbd{M[,\pow k]} returns the submatrix with column $k$ removed.

\noindent Finally,

\item \kbd{M[i..j, k]} returns the \typ{COL} built from the $k$-th column
  (entries $i$ to $j$).

\item \kbd{M[\pow i, k]} returns the \typ{COL} built from the $k$-th column
  (entry $i$ removed).

\item \kbd{M[k, i..j]} returns the \typ{VEC} built from the $k$-th row
  (entries $i$ to $j$).

\item \kbd{M[k, \pow i]} returns the \typ{VEC} built from the $k$-th row
  (entry $i$ removed).

\bprog
? M = [1,2,3;4,5,6;7,8,9];
? M[1..2, 2..3]
%2 =
[2 3]

[5 6]
? M[1..2,]
%3 =
[1 2 3]

[4 5 6]
? M[,2..3]
%4 =
[2 3]

[5 6]

[8 9]
@eprog

All this is recursive, so if \kbd{M} is a matrix of matrices of \dots, an
expression such as \kbd{M[1,1][,3][4] = 1} is perfectly valid (and actually
identical to \kbd{M[1,1][4,3] = 1}), assuming that all matrices along the way
have compatible dimensions.

\misctitle{Technical note (design flaw)} Matrices are internally represented
as a vector of columns. All matrices with $0$ columns are thus represented
by the same object (internally, an empty vector), and there is no way to
distinguish between them. Thus it is not possible to create or represent
matrices with zero columns and an actual nonzero number of rows.
The empty matrix \kbd{[;]} is handled as though it had an arbitrary number of
rows, exactly as many as needed for the current computation to make sense:
\bprog
? [1,2,3; 4,5,6] * [;]
%1 = [;]
@eprog\noindent
The empty matrix on the first line is understood as a $3\times 0$ matrix, and
the result as a $2\times 0$ matrix. On the other hand, it is possible to
create matrices with a given positive number of columns, each of which has
zero rows, e.g.~using \kbd{Mat} as above or using the \kbd{matrix} function.

Note that although the internal representation is essentially the same, a row
vector of column vectors is \emph{not} a matrix; for example, multiplication
will not work in the same way. It is easy to go from one representation to
the other using \tet{Vec} / \tet{Mat}, though:
\bprog
? [1,2,3;4,5,6]
%1 =
[1 2 3]

[4 5 6]

? Vec(%)
%2 = [[1, 4]~, [2, 5]~, [3, 6]~]
? Mat(%)
%3 =
[1 2 3]

[4 5 6]
@eprog

\subsec{Lists (\typ{LIST})}%
\sidx{list}\kbdsidx{t_LIST}
Lists can be input directly, as in \kbd{List([1,2,3,4])}; but in most cases,
one creates an empty list, then appends elements using \kbd{listput}:
\bprog
  ? a = List(); listput(a,1); listput(a,2);
  ? a
  %2 = List([1, 2])
@eprog\noindent
Elements can be accessed directly as with the vector types described above.

\subsec{Strings (\typ{STR})}%
\sidx{string}\sidx{character string}\kbdsidx{t_STR}
To enter a string, enclose it between double quotes \kbd{"}, like this:
\kbd{"this is a string"}. The function \kbd{Str} can be used to transform any
object into a string.

\subsec{Small vectors (\typ{VECSMALL})}%
\kbdsidx{t_VECSMALL}
This is an internal type, used to code in an efficient way vectors containing
only small integers, such as permutations. Most \kbd{gp} functions will
refuse to operate on these objects.

\subsec{Functions (\typ{CLOSURE})}%
\kbdsidx{t_CLOSURE}
We will explain this at length in \secref{se:user_defined}. For the time
being, suffice it to say that functions can be assigned to variables, as any
other object, and the following equivalent basic forms are available to
create new ones
\bprog
  f = (x,y) -> x^2 + y^2

  f(x,y) = x^2 + y^2
@eprog

\subsec{Error contexts (\typ{ERROR})}%
\kbdsidx{t_ERROR}
An object of this type is created whenever an error occurs: it contains some
information about the error and the error context. Usually, an appropriate
error is printed immediately, the computation is aborted, and GP enters the
``break loop'':
\bprog
  ? 1/0; 1 + 1
    ***   at top-level: 1/0;1+1
    ***                  ^------
    *** _/_: division by a non-invertible object
    ***   Break loop: type 'break' to go back to the GP prompt

@eprog\noindent Here the computation is aborted as soon as we try to evaluate
$1/0$, and $1 + 1$ is never executed. Exceptions can be trapped
using \tet{iferr}, however: we can evaluate some expression and either recover
an ordinary result (no error occurred), or an exception (an error did occur).
\bprog
  ? i = Mod(6,12); iferr(1/i, E, print(E)); 1 + 1
  error("impossible inverse modulo: Mod(6, 12).")
  %1 = 2
@eprog\noindent One can ignore the exception, print it as above, or extract
non trivial information from the error context:
\bprog
  ? i = Mod(6,12); iferr(1/i, E, print(component(E,1)));
  Mod(6, 12)
@eprog\noindent We can also rethrow the exception: \kbd{error(E)}.

\subsec{Infinity (\typ{INFINITY})}%
\kbdsidx{t_INFINITY}

There are only two objects of this type \kbd{+oo} and \kbd{-oo}, representing
$\pm\infty$. This type only contain only two elements \kbd{oo} and \kbd{-oo},
They are used in functions sur as \kbd{intnum} or \kbd{polrootsreal}, to
encode infinite real intervals. These objects can only be negated and
compared to real numbers (\typ{INT}, \typ{REAL}, \typ{FRAC}), but not
included in any computation, i.e.~\kbd{1+oo} is an error, not kbd{oo} again.

\section{GP operators}\label{se:operators}

\noindent Loosely speaking, an \idx{operator} is a function, usually
attached to basic arithmetic operations, whose name contains only
non-alphanumeric characters. For instance \kbd{+} or \kbd{-}, but also
\kbd{=} or \kbd{+=}, or even \kbd{[ ]} (the selection operator). As all
functions, operators take arguments, and return a value; \emph{assignment}
operators also have side effects: besides returning a value, they change the
value of some variable.

Each operator has a fixed and unchangeable priority, which means that, in
a given expression, the operations with the highest priority is performed
first. Unless mentioned otherwise, operators at the same priority level are
left-associative (performed from left to right), unless they are assignments,
in which case they are right-associative. Anything enclosed between
parenthesis is considered a complete subexpression, and is resolved
recursively, independently of the surrounding context. For instance,
\bprog
  a + b + c    -->   (a + b) + c     \\@com left-associative
  a = b = c    -->   a = (b = c)     \\@com right-associative
@eprog\noindent
Assuming that \var{op}$_1$, \var{op}$_2$, \var{op}$_3$ are
binary operators with increasing priorities (think of \kbd{+},
\kbd{*}, \kbd{\pow}),
$$ x~\var{op}_1~y~\var{op}_2~z~\var{op}_2~x~\var{op}_3~y $$ is
equivalent to $$ x~\var{op}_1~((y~\var{op}_2~z)~\var{op}_2~
(x~\var{op}_3~y)).$$

GP contains many different operators, either unary (having only
one argument) or binary, plus a few special selection operators. Unary
operators are defined as either \emph{prefix}  or \emph{postfix}, meaning
that they respectively precede (\var{op}~$x$) and follow ($x$~\var{op}) their
single argument. Some symbols are syntactically correct in both positions,
like \kbd{!}, but then represent different operators: the \kbd{!} symbol
represents the negation and factorial operators when in prefix and postfix
position respectively. Binary operators all use the (infix) syntax
$x$~\var{op}~$y$.

Most operators are standard (\kbd{+}, \kbd{\%}, \kbd{=}), some are
borrowed from the C language (\kbd{++}, \kbd{<<}), and a few are
specific to GP (\kbd{\bs}, \kbd{\#}). Beware that some GP operators differ
slightly from their C counterparts. For instance, GP's postfix \kbd{++}
returns the \emph{new} value, like the prefix \kbd{++} of~C, and the binary
shifts \kbd{<<}, \kbd{>>} have a priority which is different from (higher
than) that of their C counterparts. When in doubt, just surround everything
by parentheses; besides, your code will be more legible.

\noindent Here is the list of available operators, ordered by decreasing
\idx{priority}, binary and left-associative unless mentioned otherwise. An
expression is an \tev{lvalue} if something can be assigned to it. (The name
comes from left-value, to the left of a \kbd{=} operator; e.g.
\kbd{x}, or \kbd{v[1]} are lvalues, but \kbd{x + 1} is not.)

\def\point#1{\noindent\item #1\hfill\break\indent\strut}
\point{Priority 14}
%
\kbd{:} as in \kbd{x:small}, is used to indicate to the GP2C compiler that the
variable on the left-hand side always contains objects of the type specified
on the right hand-side (here, a small integer) in order to produce more
efficient or more readable C code. This is ignored by GP.

%
\point{Priority 13}
\kbd{( )} is the function call operator. If $f$ is a closure and \var{args}
is a comma-separated list of arguments (possibly empty),
$f\kbd{(\var{args})}$ evaluates $f$ on those arguments.

\point{Priority 12}
%
\kbd{++} and \kbd{--} (unary, postfix): if $x$ is an \tet{lvalue},
\kbd{$x$++} assigns the value $x+1$ to $x$, then returns the new value of
$x$. This corresponds to the C statement \kbd{++$x$}: there is no prefix
\kbd{++} operator in GP. \kbd{$x$--} does the same with $x-1$. These
operators are not associative, i.e. \kbd{x++++} is invalid, since
\kbd{x++} is not an lvalue.

\point{Priority 11}
%
\kbd{.}\var{member} (unary, postfix): \kbd{$x$.\var{member}} extracts
\var{member} from structure $x$ (see~\secref{se:member}).

\kbd{[ ]} is the selection operator. \kbd{$x$[$i$]} returns the $i$-th
component of vector $x$; \kbd{$x$[$i$,$j$]}, \kbd{$x$[,$j$]} and
\kbd{$x$[$i$,]} respectively return the entry of coordinates $(i,j)$, the
$j$-th column, and the $i$-th row of matrix $x$. If the assignment operator
(\kbd{=}) immediately follows a sequence of selections, it assigns its right
hand side to the selected component. E.g \kbd{x[1][1] = 0} is valid; but
beware that \kbd{(x[1])[1] = 0} is not (because the parentheses force the
complete evaluation of \kbd{x[1]}, and the result is not modifiable).

\point{Priority 10}
%
\kbd{'} (unary, postfix): derivative with respect to the main variable.
If $f$ is a function (\typ{CLOSURE}), $f'$ is allowed and defines a new
function, which will perform \idx{numerical derivation} when evaluated
at a scalar $x$; this is defined as $(f(x+\varepsilon) - f(x-\varepsilon)) /
2\varepsilon$ for a suitably small epsilon depending on current precision.
\bprog
? (x^2 + y*x + y^2)'  \\@com derive with respect to main variable \kbd{x}
%1 = 2*x + y
? SIN = cos'
%2 = cos'
? SIN(Pi/6)         \\@com numerical derivation
%3 = -0.5000000000000000000000000000
? cos'(Pi/6)        \\@com works directly: no need for intermediate \kbd{SIN}
%4 = -0.5000000000000000000000000000
@eprog

\strut\kbd{\til} (unary, postfix): vector/matrix transpose.

\kbd{!} (unary, postfix): factorial. $x\kbd{!}=x(x-1)\cdots 1$.

\kbd{!} (unary, prefix): logical \var{not}. \kbd{!$x$} returns $1$ if $x$ is
equal to $0$ (specifically, if \kbd{gequal0($x$)==1}), and $0$ otherwise.

\point{Priority 9}
%
\kbd{\#} (unary, prefix): cardinality; \kbd{\#$x$} returns \kbd{length($x$)}.

\point{Priority 8}
%
\kbd{\pow}: powering. This operator is right associative:
\kbd{2 \pow 3\pow 4} is understood as \kbd{2 \pow (3\pow 4)}.

\point{Priority 7}
%
\kbd{+}, \kbd{-} (unary, prefix): \kbd{-} toggles the sign of its argument,
\kbd{+} has no effect whatsoever.

\point{Priority 6}
%
\kbd{*}: multiplication.

\kbd{/}: exact division (\kbd{3/2} yields $3/2$, not $1.5$).

\kbd{\bs}, \kbd{\%}: Euclidean quotient and remainder, i.e.~if $x =
qy + r$, then $\kbd{x \b{ } y} = q$, $\kbd{x\%y} = r$. If $x$ and $y$
are scalars, then $q$ is an integer and $r$ satisfies $0\le r < |y|$; if $x$
and $y$ are polynomials, then $q$ and $r$ are polynomials such that $\deg r<
\deg y$ and the leading terms of $r$ and $x$ have the same sign.

\kbd{\bs/}: rounded Euclidean quotient for integers (rounded towards
$+\infty$ when the exact quotient would be a half-integer).

\kbd{<<}, \kbd{>>}: left and right binary shift. By definition,
\kbd{x<<n}$~=~x * 2^n$ if $n>0$, and $\kbd{truncate}(x 2^{-n})$ otherwise.
Right shift is defined by \kbd{x>>n}$~=~$\kbd{x<<(-n)}.

\point{Priority 5}
%
\kbd{+}, \kbd{-}: addition/subtraction.

\point{Priority 4}
%
\kbd{<}, \kbd{>}, \kbd{<=}, \kbd{>=}: the usual comparison operators,
returning 1 for \kbd{true} and 0 for \kbd{false}. For instance,
\kbd{x<=1} returns $1$ if $x\le 1$ and $0$ otherwise.

\kbd{<>}, \kbd{!=}: test for (exact) inequality.

\kbd{==}: test for (exact) equality. \typ{QFR} having the same coefficients
but a different distance component are tested as equal.

\kbd{===}: test whether two objects are identical component-wise. This is
stricter than \kbd{==}: for instance, the integer 0, a 0 polynomial or a
vector with 0 entries, are all tested equal by \kbd{==}, but they are not
identical.

\point{Priority 3}
%
\kbd{\&\&}: logical \var{and}.

\kbd{||}: logical (inclusive) \var{or}. Any sequence of logical
\var{or} and \var{and} operations is evaluated from left to right,
and aborted as soon as the final truth value is known. Thus, for instance,
\bprog
  x == 0 || test(1/x)
@eprog\noindent
will never produce an error since \kbd{test(1/x)} is not even evaluated
when the first test is true (hence the final truth value is true). Similarly
\bprog
  type(p) == "t_INT" && isprime(p)
@eprog\noindent
does not evaluate \kbd{isprime(p)} if \kbd{p} is not an integer.

\point{Priority 2}
%
\kbd{=} (assignment, \var{lvalue} \kbd{=} \var{expr}). The result of
\kbd{x~=~$y$} is the value of the expression~$y$, which is also assigned to
the variable~\kbd{x}. This assignment operator is right-associative. This is
\emph{not} the equality test operator; a statement like \kbd{x~=~1} is always
true (i.e.~non-zero), and sets \kbd{x} to~1; the equality test would be
\kbd{x == 1}. The right hand side of the assignment operator is evaluated
before the left hand side.

It is crucial that the left hand-side be an \var{lvalue} there, it avoids
ambiguities in expressions like \kbd{1 + x = 1}. The latter evaluates as
\kbd{1 + (x = 1)}, not as \kbd{(1 + x) = 1}, even though the priority of
\kbd{=} is lower than the priority of \kbd{+}: \kbd{1 + x} is not an lvalue.

If the expression cannot be parsed in a way where the left hand side is an
lvalue, raise an error.
\bprog
? x + 1 = 1
  ***   syntax error, unexpected '=', expecting $end or ';': x+1=1
  ***                                                           ^--
@eprog
\leavevmode
\kbd{\var{op}=}, where \var{op} is any binary operator
among \kbd{+}, \kbd{-}, \kbd{*}, \kbd{\%}, \kbd{/}, \kbd{\bs}, \kbd{\bs/},
\kbd{<<}, or
\kbd{>>} (composed assignment \var{lvalue} \var{op}\kbd{=} \var{expr}).
The expression \kbd{x~\var{op}=~$y$} assigns $(\kbd{x}~\var{op}~y)$
to~\kbd{x}, and returns the new value of~\kbd{x}. The result is \emph{not}
an \tev{lvalue}; thus
\bprog
  (x += 2) = 3
@eprog\noindent
is invalid. These assignment operators are right-associative:
\bprog
  ? x = 'x; x += x *= 2
  %1 = 3*x
@eprog

\point{Priority 1}
\kbd{->} (function definition): \kbd{(\var{vars})->\var{expr}} returns a
function object, of type \typ{CLOSURE}.

\misctitle{Remark} Use the \var{op}\kbd{=} operators as often as possible
since they make complex assignments more legible: one needs not parse
complicated expressions twice to make sure they are indeed identical. Compare
\bprog
v[i+j-1] = v[i+j-1] + 1    -->    v[i+j-1]++

M[i,i+j] = M[i,i+j] * 2    -->    M[i,i+j] *= 2
@eprog

\misctitle{Remark} Less important but still interesting. The
\kbd{++}, \kbd{--} and \var{op}\kbd{=} operators are slightly more efficient:
\bprog
? a = 10^6;
? i = 0; while(i<a, i=i+1)
time = 365 ms.
? i = 0; while(i<a, i++)
time = 352 ms.
@eprog
\noindent For the same reason, the shift operators should be preferred to
multiplication:
\bprog
? a = 1<<(10^5);
? i = 1; while(i<a, i=i*2);
time = 1,052 ms.
? i = 1; while(i<a, i<<=1);
time = 617 ms.
@eprog

\section{Variables and symbolic expressions}\sidx{variable}\label{se:varsymb}
In this section we use \emph{variable} in the standard mathematical
sense, symbols representing algebraically independent elements used to build
rings of polynomials and power series, and explain the all-important concept
of \emph{variable priority}. In the next \secref{se:scope}, we shall no
longer consider only free variables, but adopt the viewpoint of computer
programming and assign values to these symbols: (bound) variables are names
attached to values in a given scope.

\subsec{Variable names}\label{se:varname} A valid name starts with a letter,
followed by any number of keyword characters: \kbd{\_} or alphanumeric
characters ([\kbd{A-Za-z0-9}]). The built-in function names are reserved and
cannot be used; see the list with \b{c}, including the constants \kbd{Pi},
\kbd{Euler}, \kbd{Catalan}, $\kbd{I}=\sqrt{-1}$ and $\kbd{oo} = \infty$.

GP names are case sensitive. For instance, the symbol \kbd{i} is perfectly
safe to use, and will not be mistaken for $\kbd{I} = \sqrt{-1}$; analogously,
\kbd{o} is not synonymous to \kbd{O}.

In GP you can use up to 16383 variable names (up to 65535 on 64-bit
machines). If you ever need thousands of variables and this becomes a serious
limitation, you should probably be using vectors instead: e.g. instead of
variables \kbd{X1}, \kbd{X2}, \kbd{X3}, \dots, you might equally well store
their values in \kbd{X[1]}, \kbd{X[2]}, \kbd{X[3]}, \dots

\subsec{Variables and polynomials}\sidx{free variable}
The quote operator \kbd{'t} registers a new \emph{free variable} with the
interpreter, which will be written as \kbd{t}, and evaluates to a monomial
of degree $1$ in the said variable.

\misctitle{Caveat}
For reasons of backward compatibility, there is no such thing as an
``unbound'' (uninitialized) variable in GP. If you use a valid variable name
in an expression, \kbd{t} say, for the first time \emph{before} assigning a
value into it, it is interpreted as \kbd{'t} rather than raising an
exception. One should not rely on this feature in serious programs, which
would otherwise break if some unexpected assignment (e.g. \kbd{t = 1})
occurs: use \kbd{'t} directly or \kbd{t = 't} first, then \kbd{t}.
A statement like \kbd{t = 't} in effect restores \kbd{t} as a free variable.
%
\bprog
? t = 't; t^2 + 1
%1 = t^2 + 1
? t = 2; t^2 + 1
%2 = 5
? %1
%3 = t^2 + 1
? eval(%1)
%4 = 5
@eprog\noindent
In the above, we initialize \kbd{t} to a monomial, then bind it to $2$.
Assigning a value to a polynomial variable does not affect previous
expressions involving it; to take into account the new variable's value, one
must force a new evaluation, using the function \kbd{eval} (see
\secref{se:eval}).

\misctitle{Caveat2}
The use of an explicit quote operator avoids the following kind of problems:
\bprog
? t = 't; p = t^2 + 1; subst(p, t, 2)
%1 = 5
? t = 2;
? subst(p, t, 3)    \\@com \kbd{t} is no longer free: it evaluates to 2
  ***   at top-level: subst(p,t,3)
  ***                         ^----
  ***   variable name expected.
? subst(p, 't, 3)   \\ OK
%3 = 10
@eprog

\subsec{Variable priorities, multivariate objects}\sidx{variable (priority)}\label{se:priority}
A multivariate polynomial in PARI is just a polynomial (in one variable),
whose coefficients are themselves polynomials, arbitrary but for the fact
that they do not involve the main variable. (PARI currently has no sparse
representation for polynomials, listing only non-zero monomials.) All
computations are then done formally on the coefficients as if the
polynomial was univariate.

This is not symmetrical. So if I enter \kbd{'x + 'y} in a clean session,
what happens? This is understood as
$$ x^1 + (y^1 + 0*y^0)*x^0 \in (\Z[y])[x] $$
but how do we know that $x$ is ``more important'' than $y$ ? Why not $y^1 +
x*y^0$, which is the same mathematical entity after all?

The answer is that variables are ordered implicitly by the interpreter:
when a new identifier (e.g~$x$, or $y$ as above) is input, the corresponding
variable is registered as having a strictly lower priority than any variable in
use at this point\footnote{*}{This is not strictly true:
the variables $x$ and $y$ are predefined, and satisfy $x > y$.
Variables of higher priority than $x$ can be created
using \tet{varhigher}.}%
. To see the ordering used by \kbd{gp} at any given time, type
\kbd{variable()}.\kbdsidx{variable}

Given such an ordering, multivariate polynomials are stored so that the
variable with the highest priority is the main variable. And so on,
recursively, until all variables are exhausted. A different storage pattern
(which could only be obtained via \kbd{libpari} programming and low-level
constructors) would produce an invalid object, and eventually a disaster.

In any case, if you are working with expressions involving several variables
and want to have them ordered in a specific manner in the internal
representation just described, the simplest is just to write down the
variables one after the other under \kbd{gp} before starting any real
computations. You may also define variables from your \tet{gprc} to have a
consistent ordering of common variable names in all your \kbd{gp} sessions,
e.g read in a file \kbd{variables.gp} containing
\bprog
'x; 'y; 'z; 't; 'a;
@eprog\noindent There is no way to change the priority of existing variables,
but you may always create new ones with well-defined priorities using
\kbd{varhigher} or \kbd{varlower}.

\misctitle{Important note} PARI allows Euclidean division of multivariate
polynomials, but assumes that the computation takes place in the fraction
field of the coefficient ring (if it is not an integral domain, the result
will a priori not make sense). This can become tricky. For instance
assume $x$ has highest priority, then $y$:
\bprog
? x % y
%1 = 0
? y % x
%2 = y             \\@com these two take place in $\Q(y)[x]$
? x * Mod(1,y)
%3 = Mod(1, y)*x   \\@com in $(\Q(y)/y\Q(y))[x] \sim \Q[x]$
? Mod(x,y)
%4 = 0
@eprog
\noindent In the last example, the division by $y$ takes place in
$\Q(y)[x]$,
hence the \kbd{Mod} object is a coset in $(\Q(y)[x]) / (y\Q(y)[x])$, which
is the null ring since $y$ is invertible! So be very wary of variable
ordering when your computations involve implicit divisions and many
variables. This also affects functions like \tet{numerator}/\tet{denominator}
or \tet{content}:
\bprog
? denominator(x / y)
%1 = 1
? denominator(y / x)
%2 = x
? content(x / y)
%3 = 1/y
? content(y / x)
%4 = y
? content(2 / x)
%5 = 2
@eprog
\noindent Can you see why? Hint: $x/y = (1/y) * x$ is in $\Q(y)[x]$ and
denominator is taken with respect to $\Q(y)(x)$; $y/x = (y*x^0) / x$ is in
$\Q(y)(x)$ so $y$ is invertible in the coefficient ring. On the other hand,
$2/x$ involves a single variable and the coefficient ring is simply $\Z$.

These problems arise because the variable ordering defines an \emph{implicit}
variable with respect to which division takes place. This is
the price to pay to allow \kbd{\%} and \kbd{/} operators on polynomials
instead of requiring a more cumbersome \kbd{divrem($x$, $y$, \var{var})}
(which also exists). Unfortunately, in some functions like \tet{content} and
\tet{denominator}, there is no way to set explicitly a main variable like in
\tet{divrem} and remove the dependence on implicit orderings. This will
hopefully be corrected in future versions.

\subsec{Multivariate power series}
Just like multivariate polynomials, power series are fundamentally
single-variable objects. It is awkward to handle many variables at once,
since PARI's implementation cannot handle multivariate error terms like
$O(x^i y^j)$. (It can handle the polynomial $O(y^j) \times x^i$ which is
a very different thing, see below.)

The basic assumption in our model is that if variable $x$ has higher
priority than $y$, then $y$ does not depend on $x$: setting $y$ to a
function of $x$ after some computations with bivariate power series does
not make sense a priori. This is because implicit constants in
expressions like $O(x^i)$ depend on $y$ (whereas in $O(y^j)$ they can not
depend on $x$). For instance
\bprog
  ? O(x) * y
  %1 = O(x)
  ? O(y) * x
  %2 = O(y)*x
@eprog\noindent
Here is a more involved example:
\bprog
  ? A = 1/x^2 + 1 + O(x); B = 1/x + 1 + O(x^3);
  ? subst(z*A, z, B)
  %2 = x^-3 + x^-2 + x^-1 + 1 + O(x)
  ? B * A
  %3 = x^-3 + x^-2 + x^-1 + O(1)
  ? z * A
  %4 = z*x^-2 + z + O(x)
@eprog\noindent
The discrepancy between \kbd{\%2} and \kbd{\%3} is surprising. Why does
\kbd{\%2} contain a spurious constant term, which cannot be
deduced from the input? Well, we ignored the rule that forbids to
substitute an expression involving high-priority variables
to a low-priority variable. The result \kbd{\%4} is correct according to
our rules since the implicit constant in $O(x)$ may depend on $z$. It is
obviously wrong if $z$ is allowed to have negative valuation in $x$. Of
course, the correct error term should be $O(xz)$, but this is not
possible in PARI.

\section{Variables and Scope}\sidx{variable scope}\label{se:scope}
This section is rather technical, and strives to explain potentially
confusing concepts. Skip to the last subsection for practical advice, if the
next discussion does not make sense to you. After learning about user
functions, study the example in \secref{se:bewarescope} then come back.

\misctitle{Definitions} % rather not make it a section to allow for ??my

A \emph{scope} is an enclosing context where names and values are attached.
A user's function body, the body of a loop, an individual command line, all
define scopes; the whole program defines the \emph{global} scope. The
argument of \tet{eval} is evaluated in the enclosing scope.

Variables are bound to values within a given scope. This is traditionally
implemented in two different ways:

\item\sidx{lexical scoping} lexical (or static) scoping: the binding makes
sense within a given block of program text. The value is private to the block
and may not be accessed from outside. Where to find the value is determined
at compile time.

\item\sidx{dynamic scoping} dynamic scoping: introducing a local variable,
say \kbd{x}, pushes a new value on a stack attached to the name \kbd{x}
(possibly empty at this point), which is popped out when the control flow
leaves the scope. Evaluating \kbd{x} in any context, possibly outside of the
given block, always yields the top value on this dynamic stack.

GP implements both lexical and dynamic scoping, using the keywords%
\footnote{*}{The names are borrowed from the \tet{Perl} scripting language.}
\tet{my} (lexical) and \tet{local} (dynamic):
\bprog
  x = 0;
  f() = x
  g() =    my(x = 1); f()
  h() = local(x = 1); f()
@eprog\noindent
The function \kbd{g} returns 0 since the global \kbd{x} binding
is unaffected by the introduction of a private variable of the same name in
\kbd{g}. On the other hand, \kbd{h} returns 1; when it calls \kbd{f()}, the
binding stack for the \kbd{x} identifier contains two items: the global
binding to 0, and the binding to 1 introduced in \kbd{h}, which is still
present on the stack since the control flow has not left \kbd{h} yet.

\subsec{Scoping rules}

Named parameters in a function definition, as well as all loop
indices\footnote{**}{
More generally, in all iterative constructs which use a variable name
(\kbd{for}, \kbd{prod}, \kbd{sum}, \kbd{vector}, \kbd{matrix},
\kbd{plot}, etc.) the given variable is lexically scoped to the construct's
body.},
have lexical scope within the function body and the loop body respectively.
\bprog
p = 0;
forprime (p = 2, 11, print(p)); p   \\ prints 0 at the end

x = 0;
f(x) = x++;
f(1)  \\ returns 2, and leave global x unaffected (= 0)
@eprog\noindent
If you exit the loop prematurely, e.g.~using the \kbd{break} statement, you
must save the loop index in another variable since its value prior the loop
will be restored upon exit. For instance
\bprog
  for(i = 1, n,
    if (ok(i), break);
  );
  if (i > n, return(failure));
@eprog\noindent
is incorrect, since the value of $i$ tested by the $(i > n)$ is quite
unrelated to the loop index. One ugly workaround is
\bprog
  for(i = 1, n,
    if (ok(i), isave = i; break);
  );
  if (isave > n, return(failure));
@eprog\noindent
But it is usually more natural to wrap the loop in a user function
and use \kbd{return} instead of \kbd{break}:
\bprog
try() =
{
  for(i = 1, n,
    if (ok(i), return (i));
  );
  0 \\ failure
}
@eprog

A list of variables can be lexically or dynamically scoped (to the block
between the declaration and the end of the innermost enclosing scope) using a
\kbd{my} or \kbd{local} declaration:
\bprog
for (i = 1, 10,
  my(x, y, z, i2 = i^2); \\ temps needed within the loop body
  ...
)
@eprog\noindent
Note how the declaration can include (optional) initial values, \kbd{i2 =
i\pow 2} in the above. Variables for which no explicit default value is given
in the declaration are initialized to $0$. It would be more natural to
initialize them to free variables, but this would break backward
compatibility. To obtain this behavior, you may explicitly use the quoting
operator:
\bprog
my(x = 'x, y = 'y, z = 'z);
@eprog\noindent
A more complicated example:
\bprog
for (i = 1, 3,
  print("main loop");
  my(x = i);          \\ local to the outermost loop
  for (j = 1, 3,
    my (y = x^2);     \\ local to the innermost loop
    print (y + y^2);
    x++;
  )
)
@eprog\noindent
When we leave the loops, the values of \kbd{x}, \kbd{y}, \kbd{i}, \kbd{j}
are the same as before they were started.

Note that \tet{eval} is evaluated in the given scope, and can access values
of lexical variables:
\bprog
? x = 1;
? my(x = 0); eval("x")
%2 = 0    \\@com we see the local \kbd{x} scoped to this command line, not the global one
@eprog

Variables dynamically scoped using \kbd{local} should more appropriately be
called \emph{temporary values} since they are in fact local to the function
declaring them \emph{and} any subroutine called from within. In practice, you
almost certainly want true private variables, hence should use almost
exclusively \kbd{my}.

We strongly recommended to explicitly scope (lexically) all variables to the
smallest possible block. Should you forget this, in expressions involving such
``rogue'' variables, the value used will be the one which happens to be on
top of the value stack at the time of the call; which depends on the whole
calling context in a non-trivial way. This is in general \emph{not} what you
want.

\section{User defined functions}\sidx{user defined functions}
\label{se:user_defined}

The most important thing to understand about user-defined functions is
that they are ordinary GP objects, bound to variables just like any
other object. Those variables are subject to scoping rules as any other:
while you can define all your functions in global scope, it is usually
possible and cleaner to lexically scope your private helper functions to the
block of text where they will be needed.

Whenever gp meets a construction of the form \kbd{expr(\var{argument list})}
and the expression \kbd{expr} evaluates to a function (an object of type
\typ{CLOSURE}), the function is called with the proper arguments. For
instance, constructions like \kbd{funcs[i](x)} are perfectly valid,
assuming \kbd{funcs} is an array of functions.

\subsec{Defining a function}\label{se:userfundef}

A user function is defined as follows:

  \kbd{(\var{list of formal variables}) -> \var{seq}}.

\noindent The list of formal variables is a comma-separated list of
\emph{distinct} variable names and allowed to be empty. It there is a single
formal variable, the parentheses are optional. This list corresponds to the
list of parameters you will supply to your function when calling it.

In most cases you want to assign a function to a variable immediately, as in
\bprog
R = (x,y) -> sqrt( x^2+y^2 );
sq = x -> x^2;  \\@com or equivalently \kbd{(x) -> x\pow2}
@eprog\noindent
but it is quite possible to define (a priori short-lived) anonymous functions.
The trailing semicolon is not part of the definition, but as usual prevents
\kbd{gp} from printing the result of the evaluation, i.e. the function
object. The construction

  \kbd{f(\var{list of formal variables}) = \var{seq}}

\noindent is available as an alias for

  \kbd{f = (\var{list of formal variables}) -> \var{seq}}

\noindent Using that syntax, it is not possible to define anonymous functions
(obviously), and the above two examples become:
\bprog
R(x,y) = sqrt( x^2+y^2 );
sq(x) = x^2;
@eprog\noindent
The semicolon serves the same purpose as above: preventing the printing
of the resulting function object; compare
\bprog
? sq(x) = x^2;  \\@com no output
? sq(x) = x^2   \\@com print the result: a function object
%2 = (x)->x^2
@eprog\noindent Of course, the sequence \var{seq} can be arbitrarily
complicated, in which case it will look better written on consecutive lines,
with properly scoped variables:
\bprog
{
f(x0, x1, @dots) =
  my(t0, t1, @dots); \\@com variables lexically scoped to the function body
  @dots
}
@eprog \noindent Note that the following variant would also work:
\bprog
f(x0, x1, @dots) =
{
  my(t0, t1, @dots); \\@com variables lexically scoped to the function body
  @dots
}
@eprog \noindent
(the first newline is disregarded due to the preceding \kbd{=} sign, and the
others because of the enclosing braces). The \tet{my} statements can actually
occur anywhere within the function body, scoping the variables to more
restricted blocks than the whole function body.

Arguments are passed by value, not as variables: modifying a function's
argument in the function body is allowed, but does not modify its value in the
calling scope. In fact, a \emph{copy} of the actual parameter is assigned to
the formal parameter when the function is called. Formal parameters are
lexically scoped to the function body. It is not allowed to use the same
variable name for different parameters of your function:
\bprog
? f(x,x) = 1
  ***   variable declared twice: f(x,x)=1
  ***                                ^----
@eprog

\misctitle{Functions taking an unlimited number of arguments}

A function taking an unlimited number of arguments is called \emph{variadic}.
To create such a function, use the syntax

  \kbd{(\var{list of formal variables}, \var{var}[..]) -> \var{seq}}

\noindent The parameter \var{var} is replaced by a vector containing all the
remaining arguments.

\bprog
? f(c[..]) = sum(i=1,#c,c[i]);
? f(1,2,3)
%1 = 6
? sep(s,v[..]) = for(i=1,#v-1,print1(v[i],s)); if (#v, print(v[#v]));
? sep(":", 1, 2, 3)
1:2:3
@eprog

\misctitle{Finishing touch}
You can add a specific help message for your function using \kbd{addhelp},
but the online help system already handles it. By default \kbd{?\var{name}}
will print the definition of the function \var{name}: the list of arguments,
as well as their default values, the text of \var{seq} as you input it.
Just as \b{c} prints the list of all built-in commands, \b{u} outputs the
list of all user-defined functions.

\misctitle{Backward compatibility (lexical scope)} Lexically scoped
variables were introduced in version~2.4.2. Before that, the formal
parameters were dynamically scoped. If your script depends on this behavior,
you may use the following trick: replace the initial \kbd{f(x) =} \ by
\bprog
f(x_orig) = local(x = x_orig)
@eprog
\misctitle{Backward compatibility (disjoint namespaces)} Before version
2.4.2, variables and functions lived in disjoint namespaces and it was not
possible to have a variable and a function share the same name. Hence the
need for a \kbd{kill} function allowing to reuse symbols. This is no longer
the case.

There is now no distinction between variable and function names: we
have PARI objects (functions of type \typ{CLOSURE}, or more mundane
mathematical entities, like \typ{INT}, etc.) and variables bound to them.
There is nothing wrong with the following sequence of assignments:
\bprog
? f = 1       \\@com assigns the integer 1 to \kbd{f}
%1 = 1;
? f() = 1     \\@com a function with a constant value
%2 = ()->1
? f = x^2     \\@com \kbd{f} now holds a polynomial
%3 = x^2
? f(x) = x^2  \\@com \dots and now a polynomial function
%4 = (x)->x^2
? g(fun) = fun(Pi);\\@com a function taking a function as argument
? g(cos)
%6 = -1.000000000000000000000000000

@eprog\noindent
Previously used names can be recycled as above: you are just redefining the
variable. The previous definition is lost of course.

\misctitle{Important technical note} Built-in functions are a special case
since they are read-only (you cannot overwrite their default meaning),
and they use features not available to user functions, in particular pointer
arguments. In the present version \vers{}, it is possible to assign a built-in
function to a variable, or to use a built-in function name to create an
anonymous function, but some special argument combinations may not be
available:
\bprog
? issquare(9, &e)
%1 = 1
? e
%2 = 3
? g = issquare;
? g(9)
%4 = 1
? g(9, &e)  \\@com pointers are not implemented for user functions
  ***   unexpected &: g(9,&e)
  ***                     ^---
@eprog

\subsec{Function call, Default arguments}

You may now call your function, as in \kbd{f(1,2)}, supplying values
for the formal variables. The number of parameters actually supplied may be
\emph{less} than the number of formal variables in the function definition.
An uninitialized formal variable is given an implicit default value of (the
integer)~0, i.e. after the definition
\bprog
f(x, y) = ...
@eprog\noindent
you may call \kbd{f(1, 2)}, supplying values for the two formal
parameters, or for example
\settabs\+\indent&xxxxxxxxx& equivalent to xxxx &\cr
\+& \kbd{f(2)}  & equivalent to &\kbd{f(2,0)},\cr
\+& \kbd{f()}   & & \kbd{f(0,0)},\cr
\+& \kbd{f(,3)} & &\kbd{f(0,3)}.  (``Empty argument'' trick)\cr
\noindent This \emph{implicit} default value of $0$, is actually deprecated
and setting
\bprog
  default(strictargs, 1)
@eprog\noindent allows to disable it (see \secref{se:def,strictargs}).\kbdsidx{strictargs}

The recommended practice is to \emph{explicitly} set a default value:
in the function definition, you can append \kbd{=}\var{expr} to a formal
parameter, to give that variable a default value. The expression
gets evaluated the moment the function is called, and may involve the
preceding function parameters: a default value for $x_i$ may involve $x_j$
for $j < i$. For instance, after
\bprog
f(x = 1, y = 2, z = y+1) = ....
@eprog\noindent
typing in \kbd{f(3,4)} would give you \kbd{f(3,4,5)}. In the rare case when
you want to set some far away argument, and leave the defaults in between as
they stand, use the ``empty argument'' trick: \kbd{f(6,,1)} would yield
\kbd{f(6,2,1)}. Of course, \kbd{f()} by itself yields \kbd{f(1,2,3)} as was
to be expected.

In short, the argument list is filled with user supplied values, in
order. A comma or closing parenthesis, where a value should have been,
signals we must use a default value. When no input arguments are left, the
defaults are used instead to fill in remaining formal parameters.
A final example:
\bprog
f(x, y=2, z=3) = print(x, ":", y, ":", z);
@eprog
\noindent defines a function which prints its arguments (at most three of
them), separated by colons.
\bprog
? f(6,7)
6:7:3
? f(,5)
0:5:3
? f()
0:2:3
@eprog\noindent If \kbd{strictargs} is set (recommended), $x$ is now a
mandatory argument, and the above becomes:
\bprog
? default(strictargs,1)
? f(6,7)
6:7:3
? f(,5)
  ***   at top-level: f(,5)
  ***                 ^-----
  ***   in function f: x,y=2,z=3
  ***                  ^---------
  ***   missing mandatory argument 'x' in user function.
@eprog

\misctitle{Example} We conclude with an amusing example, intended to
illustrate both user-defined functions and the power of the \kbd{sumalt}
function. Although the \idx{Riemann zeta-function} is included (as
\kbd{zeta}) among the standard functions, let us assume that we want to check
other implementations. Since we are highly interested in the critical strip,
we use the classical formula
$$ (2^{1-s} - 1)\zeta(s) = \sum_{n\geq 1} (-1)^n n^{-s},
  \qquad\Re s > 0.$$
The implementation is obvious:\sidx{zeta function}
\bprog
ZETA(s) = sumalt(n=1, (-1)^n*n^(-s)) / (2^(1-s) - 1)
@eprog
\noindent
Note that \kbd{n} is automatically lexically scoped to the \kbd{sumalt}
``loop'', so that it is unnecessary to add a \kbd{my(n)} declaration to the
function body. Surprisingly, this gives very good accuracy in a larger region
than expected:
\bprog
? check = z -> ZETA(z) / zeta(z);
? check(2)
%1 = 1.000000000000000000000000000
? check(200)
%2 = 1.000000000000000000000000000
? check(0)
%3 = 0.9999999999999999999999999994
? check(-5)
%4 = 1.00000000000000007549266557
? check(-11)
%5 = 0.9999752641047824902660847745
? check(1/2+14.134*I)  \\@com very close to a non-trivial zero
%6 = 1.000000000000000000003747432 + 7.62329066 E-21*I
? check(-1+10*I)
%7 = 1.000000000000000000000002511 + 2.989950968 E-24*I
@eprog\noindent Now wait a minute; not only are we summing a series which is
certainly no longer alternating (it has complex coefficients), but we are
also way outside of the region of convergence, and still get decent results! No
programming mistake this time: \kbd{sumalt} is a
``magic'' function\footnote{*}{\kbd{sumalt} is heuristic, but its use can be
rigorously justified for a given function, in particular our $\zeta(s)$
formula. Indeed, Peter Borwein (\emph{An efficient algorithm for the Riemann
zeta function}, CMS Conf.~Proc.~{\bf 27} (2000), pp.~29--34) proved that the
formula used in \kbd{sumalt} with $n$ terms computes $(1-2^{1-s})\zeta(s)$
with a relative error of the order of $(3+\sqrt{8})^{-n}|\Gamma(s)|^{-1}$.},
providing very good convergence acceleration; in effect, we are computing
the analytic continuation of our original function. To convince ourselves
that \kbd{sumalt} is a non-trivial implementation, let us try a simpler
example:
\bprog
? sum(n=1, 10^7, (-1)^n/n, 0.) / (-log(2)) \\@com approximates the well-known formula
time = 7,417 ms.
%1 = 0.9999999278652515622893405457
? sumalt(n=1, (-1)^n/n) / (-log(2))        \\@com accurate and fast
time = 0 ms.
%2 = 1.000000000000000000000000000
@eprog\noindent No, we are not using a powerful simplification tool here,
only numerical computations. Remember, PARI is not a computer algebra system!


\subsec{Beware scopes}\label{se:bewarescope}
Be extra careful with the scopes of variables. What is wrong with the
following definition?
\bprog
FirstPrimeDiv(x) =
{ my(p);
  forprime(p=2, x, if (x%p == 0, break));
  p
}
? FirstPrimeDiv(10)
%1 = 0
@eprog\noindent \misctitle{Hint} The function body is equivalent to
\bprog
{ my(newp = 0);
  forprime(p=2, x, if (x%p == 0, break));
  newp
}
@eprog\noindent
\misctitle{Detailed explanation} The index \kbd{p} in the \kbd{forprime}
loop is lexically scoped to the loop and is not visible to the outside world.
Hence, it will not survive the \kbd{break} statement. More precisely, at this
point the loop index is restored to its preceding value. The initial
\kbd{my(p)}, although well-meant, adds to the confusion: it indeed scopes
\kbd{p} to the function body, with initial value $0$, but the \kbd{forprime}
loop introduces \emph{another} variable, unfortunately also called \kbd{p},
scoped to the loop body, which shadows the one we wanted. So we always return
$0$, since the value of the \kbd{p} scoped to the function body never changes
and is initially $0$.

To sum up, the routine returns the \kbd{p} declared local to
it, not the one which was local to \kbd{forprime} and ran through consecutive
prime numbers. Here is a corrected version:
\bprog
? FirstPrimeDiv(x) = forprime(p=2, x, if (x%p == 0, return(p)))
@eprog

\subsec{Recursive functions} Recursive functions\sidx{recursion} can easily
be written as long as one pays proper attention to variable scope. Here is an
example, used to retrieve the coefficient array of a multivariate polynomial
(a non-trivial task due to PARI's unsophisticated representation for those
objects): \sidx{multivariate polynomial}
\bprog
coeffs(P, nbvar) =
{
  if (type(P) != "t_POL",
    for (i=1, nbvar, P = [P]);
    return (P)
  );
  vector(poldegree(P)+1, i, coeffs(polcoeff(P, i-1), nbvar-1))
}
@eprog

\noindent If $P$ is a polynomial in $k$ variables, show that after the
assignment {\tt v = coeffs(P,k)}, the coefficient of $x_1^{n_1}\dots
x_k^{n_k}$ in P is given by {\tt v[$n_1$+1][\dots][$n_k$+1]}.

The operating system automatically limits the \idx{recursion depth}:
\sidx{deep recursion}
\bprog
? dive(n) = dive(n+1)
? dive(0);
  ***   [...] at: dive(n+1)
  ***             ^---------
  ***   in function dive: dive(n+1)
  ***                     ^---------
  \\@com (last 2 lines repeated 19 times)
  ***   deep recursion.
@eprog\noindent
There is no way to increase the recursion limit (which may be different on
your machine) from within \kbd{gp}. To increase it before launching \kbd{gp},
you can use \tet{ulimit} or \tet{limit}, depending on your shell, and raise
the process available stack space (increase \tet{stacksize}).

\subsec{Function which take functions as parameters} This is done as follows:
\bprog
? calc(f, x) = f(x)
? calc(sin, Pi)
%2 = -5.04870979 E-29
? g(x) = x^2;
? calc(g, 3)
%4 = 9
@eprog
\noindent If we do not need \kbd{g} elsewhere, we should use an anonymous
function here, \kbd{calc(x->x\pow 2, 3)}. Here is a variation:
\bprog
? funs = [cos, sin, tan, x->x^3+1]; \\@com an array of functions
? call(i, x) = funs[i](x)
@eprog\noindent
evaluates the appropriate function on argument \kbd{x},
provided $1\leq i\leq 4$. Finally, a more useful example:
\bprog
APPLY(f, v) = vector(#v, i, f(v[i]))
@eprog\noindent
applies the function \kbd{f} to every element in the vector \kbd{v}.
(The built-in function \kbd{apply} is more powerful since it also applies to
lists and matrices.)

\subsec{Defining functions within a function}
Defining a single function is easy:
\bprog
init(x) = (add = y -> x+y);
@eprog\noindent Basically, we are defining a global variable \kbd{add}
whose value is the function \kbd{y->x+y}. The parentheses were added for
clarity and are not mandatory.
\bprog
? init(5);
? add(2)
%2 = 7
@eprog\noindent
A more refined approach is to
avoid global variables and \emph{return} the function:
\bprog
init(x) = y -> x+y
add = init(5)
@eprog\noindent Then \kbd{add(2)} still returns 7, as expected! Of course,
if \kbd{add} is in global scope, there is no gain, but we can
lexically scope it to the place where it is useful:
\bprog
  my ( add = init(5) );
@eprog

How about multiple functions then? We can use the last idea and return a
vector of functions, but if we insist on global variables?
The first idea
\bprog
init(x) = add(y) = x+y; mul(y) = x*y;
@eprog
\noindent does not work since in the construction \kbd{f() = }\var{seq}, the
function body contains everything until the end of the expression. Hence
executing \kbd{init} defines the wrong function \kbd{add} (itself defining
a function \kbd{mul}). The way out is to
use parentheses for grouping, so that enclosed subexpressions will be
evaluated independently:
\bprog
? init(x) = ( add(y) = x+y ); ( mul(y) = x*y );
? init(5);
? add(2)
%3 = 7
? mul(3)
%4 = 15
@eprog\noindent This defines two global functions which have access to the
lexical variables private to \kbd{init}! The following would work in exactly
the same way:
\bprog
? init5() = my(x = 5); ( add(y) = x+y ); ( mul(y) = x*y );
@eprog

\subsec{Closures as Objects} Contrary to what you might think after the
preceding examples, GP's closures may not be used to simulate true
``objects'', with private and public parts and methods to access and
manipulate them. In fact, closures indeed incorporate an existing context
(they may access lexical variables that existed at the time of their
definition), but then may not change it. More precisely, they access a copy,
which they are welcome to change, but a further function call still accesses
the original context, as it existed at the time the function was defined:
\bprog
init() =
{ my(count = 0);
  inc()=count++;
  dec()=count--;
}
? inc()
%1 = 1
? inc()
%2 = 1
? inc()
%3 = 1
@eprog

\section{Member functions}\sidx{member functions} \label{se:member}

Member functions use the `dot' notation to retrieve information from
complicated structures. The built-in structures are \tev{bid}, \tev{ell},
\tev{galois}, \tev{ff}, \tev{nf}, \tev{bnf}, \tev{bnr} and \tev{prid}, which
will be described at length in Chapter~3. The syntax \kbd{structure.member}
is taken to mean: retrieve \kbd{member} from \kbd{structure},
e.g.~\kbd{E.j} returns the $j$-invariant of the elliptic curve \kbd{E},
or outputs an error message if \kbd{E} is not a proper \tev{ell} structure.
To define your own member functions, use the syntax

\ \kbd{\var{var}.\var{member} = \var{seq}},

\noindent where the formal variable \var{var} is scoped to the function
body \var{seq}. This is of course reminiscent of a user function with a
single formal variable \var{var}. For instance, the current implementation of
the \kbd{ell} type is a vector, the $j$-invariant being the thirteenth
component. It could be implemented as

\bprog
x.j =
{
  if (type(x) != "t_VEC" || #x < 14, error("not an elliptic curve: " x));
  x[13]
}
@eprog\noindent As for user functions, you can redefine your member functions
simply by typing new definitions. On the other hand, as a safety measure, you
cannot redefine the built-in member functions, so attempting to redefine
\kbd{x.j} as above would in fact produce an error; you would have to call it
e.g.~\kbd{x.myj} in order for \kbd{gp} to accept it.

\misctitle{Rationale} In most cases, member functions are simple accessors
of the form
\bprog
  x.a = x[1];
  x.b = x[2];
  x.c = x[3];
@eprog\noindent
where \kbd{x} is a vector containing relevant data. There are at least
three alternative approaches to the above member functions: 1) hardcode
\kbd{x[1]}, etc. in the program text, 2) define constant global variables
\kbd{AINDEX = 1}, \kbd{BINDEX = 2}  and hardcode \kbd{x[AINDEX]}, 3)
user functions \kbd{a(x) = x[1]} and so on.

Even if 2) improves on 1), these solutions are neither elegant nor flexible,
and they scale badly. 3) is a genuine possibility, but the main advantage of
member functions is that their namespace is independent from the variables
(and functions) namespace, hence we can use very short identifiers without
risk. The $j$-invariant is a good example: it would clearly not be a good
idea to define \kbd{j(E) = E[13]}, because clashes with loop indices are
likely.

\misctitle{Note} Typing \b{um} will output all user-defined member functions.

\misctitle{Member function names} A valid name starts with a letter followed
by any number of keyword characters: \kbd{\_} or alphanumeric characters
([\kbd{A-Za-z0-9}]). The built-in member function names are reserved and
cannot be used (see the list with \kbd{?.}). Finally, names starting with
\kbd{e} or \kbd{E} followed by a digit are forbidden, due to a clash with
the floating point exponent notation: we understand \kbd{1.e2} as
$100.000\dots$, not as extracting member \kbd{e2} of object \kbd{1}.

\section{Strings and Keywords}\sidx{string}\sidx{keyword}
\label{se:strings}

\subsec{Strings} GP variables can hold values of type character string
(internal type \typ{STR}). This section describes how they are actually used,
as well as some convenient tricks (automatic concatenation and expansion,
keywords) valid in string context.

As explained above, the general way to input a string is to enclose
characters between quotes~\kbd{"}. This is the only input construct where
whitespace characters are significant: the string will contain the exact
number of spaces you typed in. Besides, you can ``escape'' characters by
putting a \kbd{\bs} just before them; the translation is as follows
\bprog
   \e: <Escape>
   \n: <Newline>
   \t: <Tab>
@eprog
For any other character $x$, \b{$x$} is expanded to $x$. In particular, the
only way to put a \kbd{"} into a string is to escape it. Thus, for
instance, \kbd{"\bs"a\bs""} would produce the string whose content is
``a''. This is definitely \emph{not} the same thing as typing \kbd{"a"},
whose content is merely the one-letter string a.

You can concatenate two strings using the \tet{concat} function. If either
argument is a string, the other is automatically converted to a string if
necessary (it will be evaluated first).

\bprog
? concat("ex", 1+1)
%1 = "ex2"
? a = 2; b = "ex"; concat(b, a)
%2 = "ex2"
? concat(a, b)
%3 = "2ex"
@eprog

Some functions expect strings for some of their arguments: \tet{print} would
be an obvious example, \tet{Str} is a less obvious but useful one (see the
end of this section for a complete list). While typing in such an argument,
you will be said to be in \tev{string context}. The rest of this section is
devoted to special syntactical tricks which can be used with such arguments
(and only here; you will get an error message if you try these outside of
string context):

\item Writing two strings alongside one another will just concatenate
them, producing a longer string. Thus it is equivalent to type in
\kbd{"a " "b"} or \kbd{"a b"}. A little tricky point in the first expression:
the first whitespace is enclosed between quotes, and so is part of a string;
while the second (before the \kbd{"b"}) is completely optional and \kbd{gp}
actually suppresses it, as it would with any number of whitespace characters
at this point (i.e.~outside of any string).

\item If you insert any expression when a string is expected, it gets
``expanded'': it is evaluated as a standard GP expression, and the final
result (as would have been printed if you had typed it by itself) is then
converted to a string, as if you had typed it directly. For instance \kbd{"a"
1+1 "b"} is equivalent to \kbd{"a2b"}: three strings get created, the middle
one being the expansion of \kbd{1+1}, and these are then concatenated
according to the rule described above. Another tricky point here: assume you
did not assign a value to \kbd{aaa} in a GP expression before. Then typing
\kbd{aaa} by itself in a string context will actually produce the correct
output (i.e.~the string whose content is aaa), but in a fortuitous way. This
\kbd{aaa} gets expanded to the monomial of degree one in the variable
\kbd{aaa}, which is of course printed as \kbd{aaa}, and thus will expand to
the three letters you were expecting.

\misctitle{Warning} Expression involving strings are not handled in a
special way; even in string context, the largest possible expression is
evaluated, hence \kbd{print("a"[1])} is incorrect since \kbd{"a"} is not an
object whose first component can be extracted. On the other hand
\kbd{print("a", [1])} is correct (two distinct argument, each converted to a
string), and so is \kbd{print("a" 1)} (since \kbd{"a"1} is not a valid
expression, only \kbd{"a"} gets expanded, then \kbd{1}, and the result is
concatenated as explained above).

\subsec{Keywords} Since there are cases where expansion is not desirable, we
now distinguish between ``Keywords'' and ``Strings''. String is what has been
described so far. Keywords are special relatives of Strings which are
automatically assumed to be quoted, whether you actually type in the quotes
or not. Thus expansion is never performed on them. They get concatenated,
though. The analyzer supplies automatically the quotes you have ``forgotten''
and treats Keywords just as normal strings otherwise. For instance, if you
type \kbd{"a"b+b} in Keyword context, you will get the string whose contents
are ab+b. In String context, on the other hand, you would get a2\kbd{*}b.

All GP functions have prototypes (described in Chapter~3 below) which
specify the types of arguments they expect: either generic PARI objects
(GEN), or strings, or keywords, or unevaluated expression sequences. In the
keyword case, only a very small set of words will actually be meaningful
(the \kbd{default} function is a prominent example).

\misctitle{Reference} The arguments of the following functions are processed
in string context:

\settabs\+\indent&\cr
\+&\tet{Str}\cr
\+&\tet{addhelp} (second argument)\cr
\+&\tet{default} (second argument)\cr
\+&\tet{error}\cr
\+&\tet{extern}\cr
\+&\tet{plotstring} (second argument)\cr
\+&\tet{plotterm} (first argument)\cr
\+&\tet{read} and \tet{readvec}\cr
\+&\tet{system}\cr
\+&all the \tet{print}\var{xxx} functions\cr
\+&all the \tet{write}\var{xxx} functions\cr

\noindent The arguments of the following functions are processed as keywords:

\+&\tet{alias}\cr
\+&\tet{default} (first argument)\cr
\+&\tet{install} (all arguments but the last)\cr
\+&\tet{trap} (first argument)\cr
\+&\tet{whatnow}\cr

\subsec{Useful example} The function \kbd{Str} converts its arguments into
strings and concatenate them. Coupled with \tet{eval}, it is very powerful.
The following example creates generic matrices\sidx{generic
matrix}\sidx{matrix}:
\bprog
? genmat(u,v,s="x") = matrix(u,v,i,j, eval( Str(s,i,j) ))
? genmat(2,3) + genmat(2,3,"m")
%1 =
[x11 + m11 x12 + m12 x13 + m13]
[x21 + m21 x22 + m22 x23 + m23]
@eprog

\section{Errors and error recovery}

\subsec{Errors} Your input program is first compiled to a more efficient
bytecode; then the latter is evaluated, calling appropriate functions from
the PARI library. Accordingly, there are two kind of errors: syntax errors
produced by the compiler, and runtime errors produced by the PARI library
either by the evaluator itself, or in a mathematical function.
Both kinds are fatal to your computation: \kbd{gp} will report the error
and perform some cleanup (restore variables modified while evaluating the
erroneous command, close open files, reclaim unused memory, etc.).

At this point, the default is to return to the usual prompt, but if the
\kbd{recover} option (\secref{se:def,recover}) is off then \kbd{gp} exits
immediately.  This can be useful for batch-mode operation to make untrapped
errors fatal.

When reporting a \emph{syntax error}, \kbd{gp} gives meaningful
context by copying (part of) the expression it was trying to compile,
indicating where the error occurred with a caret \kbd{\pow-}, as in
\bprog
? factor()
  ***   too few arguments: factor()
  ***                             ^-
? 1+
  ***   syntax error, unexpected $end: 1+
  ***                                   ^-
@eprog\noindent
possibly enlarged to a full arrow given enough trailing context
\bprog
? if (isprime(1+, do_something())
  ***   syntax error, unexpected ',': if(isprime(1+,do_something()))
  ***                                              ^----------------
@eprog\noindent
These error messages may be mysterious, because \kbd{gp} cannot guess what
you were trying to do, and the error may occur once \kbd{gp} has been
sidetracked. The first error is straightforward: \kbd{factor} has one
mandatory argument, which is missing.

The other two are simple typos involving an ill-formed addition
\kbd{1 + } missing its second operand. The error messages differ
because the parsing context is slightly different: in the first
case we reach the end of input (\kbd{\$end}) while still expecting a token,
and in the second one, we received an unexpected token (the comma).

Here is a more complicated one:
\bprog
? factor(x
  ***   syntax error, unexpected $end, expecting )-> or ',' or ')': factor(x
  ***                                                                      ^-
@eprog\noindent
The error is a missing parenthesis, but from \kbd{gp}'s point of view,
you might as well have intended to give further arguments to \kbd{factor}
(this is possible and useful, see the description of the function). In fact
\kbd{gp} expected either a closing parenthesis, or a second argument
separated from the first by a comma. And this is essentially what the error
message says: we reached the end  of the input (\kbd{\$end}) while expecting
a \kbd{')'} or a \kbd{','}.

Actually, a third possibility is mentioned in the error message \kbd{)->},
which could never be valid in the above context, but a subexpression like
\kbd{(x)->sin(x)}, defining an inline closure would be valid, and the parser is
not clever enough to rule that out, so we get the same message as in
\bprog
? (x
  ***   syntax error, unexpected $end, expecting )-> or ',' or ')': (x
  ***                                                                ^-
@eprog\noindent
where all three proposed continuations would be valid.

\emph{Runtime errors} from the evaluator are nicer because they answer a
correctly worded query, otherwise the bytecode compiler would have protested
first; here is a slightly pathological case:
\bprog
? if (siN(x) < eps, do_something())
  ***   at top-level: if(siN(x)<eps,do_someth
  ***                    ^--------------------
  ***   not a function in function call
@eprog\noindent (no arrow!) The code is syntactically correct and compiled
correctly, even though the \kbd{siN} function, a typo for \kbd{sin}, was not
defined at this point. When trying to evaluate the bytecode, however, it
turned out that \kbd{siN} is still undefined so we cannot evaluate the
function call \kbd{siN(x)}.

\emph{Library runtime errors} are even nicer because they have more
mathematical content, which is easier to grasp than a parser's logic:
\bprog
? 1/Mod(2,4)
  ***   at top-level: 1/Mod(2,4)
  ***                  ^---------
  *** _/_: impossible inverse in Fp_inv: Mod(2, 4).
@eprog\noindent telling us that a runtime error occurred while evaluating the
binary \kbd{/} operator (the \kbd{\_} surrounding the operator are
placeholders), more precisely the \kbd{Fp\_inv} library function was fed
the argument \kbd{Mod(2,4)} and could not invert it. More context is provided
if the error occurs deep in the call chain:
\bprog
? f(x) = 1/x;
? g(N) = for(i = -N, N, f(i + O(5)));
? g(10)
  ***   at top-level: g(10)
  ***                 ^-----
  ***   in function g: for(i=-N,N,f(i))
  ***                             ^-----
  ***   in function f: 1/x
  ***                   ^--
  *** _/_: impossible inverse in ginv: O(5).
@eprog\noindent In this example, the debugger reports (at least) 3 enclosed
frames: last (innermost) is the body of user function $f$, the body of $g$,
and the top-level (global scope). In fact, the \kbd{for} loop in $g$'s body
defines an extra frame, since there exist variables scoped to the loop body.

\subsec{Error recovery}\sidx{error recovery}\label{se:errorrec}

It is annoying to wait for a program to finish and find out the hard
way that there was a mistake in it (like the division by 0 above), sending
you back to the prompt. First you may lose some valuable intermediate data.
Also, correcting the error may not be obvious; you might have to change your
program, adding a number of extra statements and tests to narrow down
the problem.

A different situation, still related to error recovery, is when you
actually foresee that some error may occur, are unable to prevent it, but
quite capable of recovering from it, given the chance. Examples include lazy
factorization, where you knowingly use a pseudo prime $N$ as if it were
prime; you may then encounter an ``impossible'' situation, but this would
usually exhibit a factor of $N$, enabling you to refine the factorization and
go on. Or you might run an expensive computation at low precision to guess
the size of the output, hence the right precision to use. You can then
encounter errors like ``precision loss in truncation'', e.g when trying to
convert \kbd{1E1000}, known to $28$ digits of accuracy, to an integer; or
``division by 0'', e.g inverting \kbd{0E1000} when all accuracy has been
lost, and no significant digit remains. It would be enough to restart part of
the computation at a slightly higher precision.

We now describe \tev{error trapping}, a useful mechanism which alleviates
much of the pain in the first situation (the break loop debugger), and
provides satisfactory ways out of the second one (the \tet{iferr} exception
handler).

\subsec{Break loop}\label{se:break_loop}

A \tev{break loop} is a special debugging mode that you enter whenever a
user interrupt (\kbd{Control-C}) or runtime error occurs, freezing the
\kbd{gp} state, and preventing cleanup until you get out of the loop. By
runtime error, we mean an error from the evaluator, the library or a user error
(from \tet{error}), \emph{not} syntax errors. When a break loop starts, a
prompt is issued (\kbd{break>}). You can type in a \kbd{gp} command, which is
evaluated when you hit the \kbd{<Return>} key, and the result is printed as
during the main \kbd{gp} loop, except that no history of results is kept. Then
the break loop prompt reappears and you can type further commands as long as
you do not exit the loop. If you are using readline, the history of commands is
kept, and line editing is available as usual. If you type in a command that
results in an error, you are sent back to the break loop prompt: errors do
\var{not} terminate the loop.

To get out of a break loop, you can use \tet{next}, \tet{break}, \tet{return},
or type \kbd{C-d} (\kbd{EOF}), any of which will let \kbd{gp} perform its
usual cleanup, and send you back to the \kbd{gp} prompt. Note that \kbd{C-d}
is slightly dangerous, since typing it \emph{twice} will not only send you
back to the \kbd{gp} prompt, but to your shell prompt! (Since \kbd{C-d} at
the \kbd{gp} prompt exits the gp session.)

If the break loop was started by a user interrupt \kbd{Control-C}, and not by
an error, inputting an empty line, i.e hitting the \kbd{<Return>} key at the
\kbd{break>} prompt, resumes the temporarily interrupted computation. A single
empty line has no effect in case of a fatal error, to avoid getting get out of
the loop prematurely, thereby losing valuable debugging data. Any of
\tet{next}, \tet{break}, \tet{return}, or \kbd{C-d} will abort the computation
and send you back to  the \kbd{gp} prompt as above.

Break loops are useful as a debugging tool. You may inspect the values of
\kbd{gp} variables to understand why an error occurred, or change
\kbd{gp}'s state in the middle of a computation (increase debugging level,
start storing results in a log file, set variables to different values\dots):
hit \kbd{C-c}, type in your modifications, then let the computation go on as
explained above. A break loop looks like this:
\bprog
? v = 0; 1/v
  ***   at top-level: v=0;1/v
  ***                      ^--
  *** _/_: impossible inverse in gdiv: 0.
  ***   Break loop (type 'break' to go back to the GP prompt)
break>
@eprog
\noindent So the standard error message is printed first. The
\kbd{break>} at the bottom is a prompt, and hitting \kbd{v} then
\kbd{<Return>}, we see:
\bprog
break> v
0
@eprog\noindent explaining the problem. We could have typed any \kbd{gp}
command, not only the name of a variable, of course. Lexically-scoped
variables are accessible to the evaluator during the break loop:
\bprog
? for(v = -2, 2, print(1/v))
-1/2
-1
  ***   at top-level: for(v=-2,2,print(1/v))
  ***                                   ^----
  *** _/_: impossible inverse in gdiv: 0.
  ***   Break loop (type 'break' to go back to the GP prompt)
break> v
0
@eprog\noindent
Even though loop indices are automatically lexically scoped and no longer
exist when the break loop is run, enough debugging information is retained in
the bytecode to reconstruct the evaluation context. Of course, when the error
occurs in a nested chain of user function calls, lexically scoped variables are
available only in the corresponding frame:
\bprog
? f(x) = 1/x;
? g(x) = for(i = 1, 10, f(x+i));
? for(j = -5,5, g(j))
  ***   at top-level: for(j=-5,5,g(j))
  ***                            ^-----
  ***   in function g: for(i=1,10,f(x+i))
  ***                             ^-------
  ***   in function f: 1/x
  ***                   ^--
  *** _/_: impossible inverse in gdiv: 0.
  ***   Break loop: type 'break' to go back to GP prompt
break> [i,j,x]     \\ @com the $x$ in $f$'s body.
[i, j, 0]
break> dbg_up      \\ @com go up one frame
  ***   at top-level: for(j=-5,5,g(j))
  ***                            ^-----
  ***   in function g: for(i=1,10,f(x+i))
  ***                             ^-------
break> [i,j,x]      \\ @com the $x$ in $g$'s body, $i$ in the for loop.
[5, j, -5]
@eprog
The following GP commands are available during a break loop to help debugging:

\tet{dbg_up}$(n)$: go up $n$ frames, as seen above.

\tet{dbg_down}$(n)$: go down $n$ frames, cancelling previous \kbd{dbg\_up}'s.

\tet{dbg_x}$(t)$: examine $t$, as \kbd{\bs x} but more flexible.

\tet{dbg_err}$()$: returns the current error context \typ{ERROR}. The error
 components often provide useful additional information:
\bprog
  ? O(2) + O(3)
    ***   at top-level: O(2)+O(3)
    ***                     ^-----
    *** _+_: inconsistent addition t_PADIC + t_PADIC.
    ***   Break loop: type 'break' to go back to GP prompt
  break> E = dbg_err()
  error("inconsistent addition t_PADIC + t_PADIC.")
  break> Vec(E)
  ["e_OP", "+", O(2), O(3)]
@eprog

\misctitle{Note} The debugger is enabled by default, and fires up as soon as
a runtime error occurs. If you do not like this behavior, you may disable it by
setting the default \tet{breakloop} to 0 in for \kbd{gprc}. A runtime error
will send you back to the prompt. Note that the break loop is automatically
disabled when running \kbd{gp} in non interactive mode, i.e.~when the program's
standard input is not attached to a terminal.

\misctitle{Technical Note} When you enter a break loop due to a PARI stack
overflow, the PARI stack is reset so that you can run commands. Otherwise the
stack would immediately overflow again! Still, as explained above, you do not
lose the value of any \kbd{gp} variable in the process.

\subsec{Protecting code}
The expression

  \kbd{iferr(\var{statements}, ERR, \var{recovery})}

\noindent evaluates and returns the value of \var{statements}, unless an
error occurs during the evaluation in which case the value of \var{recovery}
is returned. As in an if/else clause, with the difference that
\var{statements} has been partially evaluated, with possible side effects.
We shall give a lot more details about the \kbd{ERR} argument shortly; it is
the name of a variable, lexically scoped to the \var{recovery} expression
sequence, whose value is set by the exception handler to help the recovery
code decide what to do about the error.

For instance one can define a fault tolerant inversion function as follows:
\bprog
? inv(x) = iferr(1/x, ERR, "oo")    \\ ERR is unused...
? for (i=-1,1, print(inv(i)))
-1
oo
1
@eprog\noindent Protected codes can be nested without adverse effect.
Let's now see how \kbd{ERR} can be used; as written, \kbd{inv} is too
tolerant:
\bprog
? inv("blah")
%2 = "oo"
@eprog Let's improve it by checking that we caught a ``division by 0''
exception, and not an unrelated one like the type error \kbd{1 / "blah"}.
\bprog
? inv2(x) = {
  iferr(1/x,
        ERR, if (errname(ERR) != "e_INV", error(ERR)); "oo")
}
? inv2(0)
%3 = "oo"  \\ as before
? inv2("blah")
  ***   at top-level: inv2("blah")
  ***                 ^------------
  ***   in function inv2: ...f(errname(ERR)!="e_INV",error(ERR));"oo")
  ***                                                 ^-----------------
  *** error: forbidden division t_INT / t_STR.
  @eprog\noindent In the \kbd{inv2("blah")} example, the error type was not
expected, so we rethrow the exception: \kbd{error(ERR)} triggers the original
error that we mistakenly trapped. Since the recovery code should always check
whether the error is the one expected, this construction is very common and
can be simplified to
\bprog
? inv3(x) = iferr(1/x,
                  ERR, "oo",
                  errname(ERR) == "e_INV")
@eprog\noindent More generally

  \kbd{iferr(\var{statements}, ERR, \var{recovery}, \var{predicate})}

\noindent only catches the exception if \var{predicate} (allowed to check
various things about \kbd{ERR}, not only its name) is non-zero.

Rather than trapping everything, then rethrowing whatever we do not like, we
advise to only trap errors of a specific kind, as above. Of course,
sometimes, one just want to trap \emph{everything} because we do not know
what to expect. The following function check whether \tet{install} works
correctly in your \kbd{gp}:
\bprog
broken_install() =
{ \\ can we install?
  iferr(install(addii,GG),
        ERR, return ("OS"));
  \\ can we use the installed function?
  iferr(if (addii(1,1) != 2, return("BROKEN")),
        ERR, return("USE"));
  return (0);
}
@eprog
\noindent The function returns
\kbd{OS} if the operating system does not support \kbd{install},
\kbd{USE} if using an installed function triggers an error,
\kbd{BROKEN} if the installed function did not behave as expected,
and 0 if everything works.

The \kbd{ERR} formal parameter contains more useful data than just the error
name, which we recovered using \kbd{errname(ERR)}. In fact, a \typ{ERROR}
object usually has extra components, which can be accessed as
\kbd{component(ERR,1)}, \kbd{component(ERR,2)}, and so on. Or globally by
casting the error to a \typ{VEC}: \kbd{Vec(ERR)} returns the vector
of all components at once. See \secref{se:iferr} for the list of all
exception types, and the corresponding contents of \kbd{ERR}.

\section{Interfacing GP with other languages}
\noindent
The PARI library was meant to be interfaced with C programs. This specific
use is dealt with extensively in the \emph{User's guide to the PARI library}.
Of course, \kbd{gp} itself provides a convenient  interpreter to execute
rather intricate scripts (see \secref{se:programming}).

Scripts, when properly written, tend to be shorter and clearer than C
programs, and are certainly easier to write, maintain or debug. You don't
need to deal with memory management, garbage collection, pointers,
declarations, and so on. Because of their intrinsic simplicity, they are more
robust as well. They are unfortunately somewhat slower. Thus their use will
remain complementary: it is suggested that you test and debug your algorithms
using scripts, before actually coding them in C if speed is paramount.
The GP2C compiler often eases this part.

The \kbd{install} command (see~\secref{se:install}) efficiently imports
foreign functions for use under \kbd{gp}, which can of course be written
using other libraries than PARI. Thus you may code only critical parts
of your program in C, and still maintain most of the program as a GP script.

We are aware of three PARI-related Free Software packages to embed PARI in
other languages. We \emph{neither endorse nor support} any of them, but you
may want to give them a try if you are familiar with the languages they are
based on. The first is the Python-based SAGE
system (\url{http://sagemath.org/}). The second is the \tet{Math::Pari} Perl
module (see any CPAN mirror), written by Ilya Zakharevich.
Finally, Michael Stoll and Sam Steingold have integrated PARI into \tet{CLISP}
(\url{http://clisp.cons.org/}), a Common Lisp implementation.

These provide interfaces to \kbd{gp} functions for use in
\kbd{python}, \kbd{perl}, or \kbd{Lisp}\sidx{Perl}\sidx{Python}\sidx{Lisp}
programs, respectively.

\section{Defaults}\sidx{defaults}
\label{se:defaults}

\noindent There are many internal variables in \kbd{gp}, defining how the
system will behave in certain situations, unless a specific override has been
given. Most of them are a matter of basic customization (colors, prompt) and
will be set once and for all in your \idx{preferences file} (see
\secref{se:gprc}), but some of them are useful interactively (set timer on,
increase precision, etc.).

The function used to manipulate these values is called \kbd{default}, which
is described in \secref{se:default}. The basic syntax is

\kbd{default(\var{def}, \var{value})},

\noindent which sets the default \var{def} to \var{value}. In interactive
use, most of these can be abbreviated using \kbd{gp} metacommands
(mostly, starting with \b), which we shall describe in the next section.

Available defaults are described in the reference guide,
\secref{se:gp_defaults}, the most important one being \tet{parisizemax}.
Just be aware that typing \kbd{default} by itself will list all of them, as
well as their current values (see \b{d}).

\misctitle{Note} The suffixes \kbd{k}, \kbd{M} or \kbd{G} can be appended to
a \var{value} which is a numeric argument, with the effect of multiplying it
by $10^3$, $10^6$ and $10^9$ respectively. Case is not taken into account
there, so for instance \kbd{30k} and \kbd{30K} both stand for $30000$. This
is mostly useful to modify or set the defaults \kbd{parisize} and
\kbd{parisizemax} which typically involve a lot of trailing zeroes.

\misctitle{(somewhat technical) Note} As we saw in \secref{se:strings},
the second argument to \kbd{default} is subject to string context
expansion, which means you can use run-time values. In other words, something
like
\bprog
  a = 3;
  default(logfile, "file" a ".log")
@eprog
logs the output in \kbd{file3.log}.

Some special defaults, corresponding to file names and prompts, expand further
the resulting value at the time they are set. Two kinds of expansions may be
performed:

\item \teb{time expansion}: the string is sent through the library
function \tet{strftime}. This means that \kbd{\%}\var{char} combinations have
a special meaning, usually related to the time and date. For instance,
\kbd{\%H} = hour (24-hour clock) and \kbd{\%M} = minute [00,59] (on a Unix
system, you can try \kbd{man strftime} at your shell prompt to get a complete
list). This is applied to \kbd{prompt}, \kbd{psfile}, and \kbd{logfile}. For
instance,

\kbd{default(prompt,"(\%H:\%M) ? ")}

\noindent
will prepend the time of day, in the form \kbd{(\var{hh}:\var{mm})}
to \kbd{gp}'s usual prompt.

\item \teb{environment expansion}: When the string contains a sequence of
the form \kbd{\$\var{SOMEVAR}}, e.g.~\kbd{\$HOME}, the environment is
searched and if \var{SOMEVAR} is defined, the sequence is replaced by the
corresponding value. Also the \kbd{\til} symbol has the same meaning as in
many shells~--- \kbd{\til} by itself stands for your home directory, and
\kbd{\til{}user} is expanded to \kbd{user}'s home directory. This is applied
to all file names\sidx{filename}. \label{se:envir}

\section{Simple metacommands}\label{se:meta}

\noindent
Simple metacommands are meant as shortcuts and should not be used in GP
scripts (see \secref{se:programming}). Beware that these, as all of \kbd{gp}
input, are \emph{case sensitive}. For example, \b{Q} is not identical to
\b{q}. In the following list, braces are used to denote optional arguments,
with their default values when applicable, e.g.~$\{n=0\}$ means that if $n$
is not there, it is assumed to be~$0$. Whitespace (or spaces) between the
metacommand and its arguments and within arguments is optional. (This can
cause problems only with \b{w}, when you insist on having a file name whose
first character is a digit, and with \b{r} or \b{w}, if the file name itself
contains a space. In such cases, just use the underlying \tet{read} or
\tet{write} function; see~\secref{se:write}).

\subseckbd{?$\{\var{command}\}$} The \kbd{gp} on-line help interface.
If you type \kbd{?$n$} where $n$ is a number from 1 to 11, you will get the
list of functions in Section $3.n$ of the manual (the list of sections being
obtained by simply typing \kbd{?}). \label{se:exthelp}

These names are in general not informative enough. More details can be
obtained by typing \kbd{?\var{function}}, which gives a short explanation of
the function's calling convention and effects. Of course, to have complete
information, read Chapter 3 of this manual (the source code is at your
disposal as well, though a trifle less readable).

If the line before the copyright message indicates that extended help is
available (this means \kbd{perl} is present on your system and the PARI
distribution was correctly installed), you can add more \kbd{?} signs for
extended functionality:

\kbd{??~\var{keyword}} yields the function description as it stands in this
manual, usually in Chapter~2 or~3. If you're not satisfied with the default
chapter chosen, you can impose a given chapter by ending the keyword with
\kbd{@} followed by the chapter number, e.g.~\kbd{??~Hello@2} will look in
Chapter~2 for section heading \kbd{Hello} (which doesn't exist, by the way).

All operators (e.g.~\kbd{+}, \kbd{\&\&}, etc.) are accepted by this
extended help, as well as a few other keywords describing key \kbd{gp} concepts,
e.g.~\kbd{readline} (the line editor), \kbd{integer}, \kbd{nf} (``number
field'' as used in most algebraic number theory computations), \kbd{ell}
(elliptic curves), etc.

In case of conflicts between \emph{function} and \emph{default} names
(e.g \tet{log},
\tet{simplify}), the function has higher priority. To get the \emph{default}
help, use
\bprog
  ?? default(log)
  ?? default(simplify)
@eprog

\kbd{???~\var{pattern}} produces a list of sections in Chapter~3 of the
manual related to your query. As before, if \var{pattern} ends by \kbd{@}
followed by a chapter number, that chapter is searched instead; you also
have the option to append a simple \kbd{@} (without a chapter number) to
browse through the whole manual.

If your query contains dangerous characters (e.g \kbd{?} or blanks) it is
advisable to enclose it within double quotes, as for GP strings (e.g
\kbd{???~"elliptic curve"}).

Note that extended help is much more powerful than the short help, since
it knows about operators as well: you can type \kbd{??~*} or
\kbd{??~\&\&}, whereas a single \kbd{?} would just yield a not too helpful
\bprog
&&: unknown identifier.}
@eprog\noindent message. Also, you can ask for extended help on section
number~$n$ in Chapter~3, just by typing \kbd{??~$n$} (where \kbd{?$n$} would
yield merely a list of functions). Finally, a few key concepts in \kbd{gp} are
documented in this way: metacommands (e.g \kbd{??~"??"}), defaults (e.g
\kbd{??~psfile}) and type names (e.g \typ{INT} or \kbd{integer}), as well as
various miscellaneous keywords such as \kbd{edit} (short summary of line
editor commands), \kbd{operator}, \kbd{member}, \kbd{"user defined"},
\kbd{nf}, \kbd{ell}, \dots

Last but not least: \kbd{??} without argument will open a \kbd{dvi}
previewer (\kbd{xdvi} by default, \kbd{\$GPXDVI} if it is defined in your
environment) containing the full user's manual. \kbd{??tutorial} and
\kbd{??refcard} do the same with the \idx{tutorial} and \idx{reference card}
respectively.

\misctitle{Technical note} This functionality is provided by an
external \kbd{perl} script that you are free to use outside any \kbd{gp} session
(and modify to your liking, if you are perl-knowledgeable). It is called
\tet{gphelp}, lies in the \kbd{doc} subdirectory of your distribution
(just make sure you run \kbd{Configure} first, see Appendix~A) and is
really two programs in one. The one which is used from within \kbd{gp} is
\kbd{gphelp} which runs \TeX\ on a selected part of this manual, then opens
a previewer. \kbd{gphelp -detex} is a text mode equivalent, which looks
often nicer especially on a colour-capable terminal (see
\kbd{misc/gprc.dft} for examples). The default \kbd{help} selects which
help program will be used from within \kbd{gp}. You are welcome to improve this
help script, or write new ones (and we would like to know about it
so that we may include them in future distributions). By the way, outside
of \kbd{gp} you can give more than one keyword as argument to \kbd{gphelp}.

\subseckbd{/*...*/} A comment. Everything between the stars is ignored by
\kbd{gp}. These comments can span any number of lines.

\subseckbd{\bs\bs} A one-line comment. The rest of the line
is ignored by \kbd{gp}.

\subsec{\b{a} $\{n\}$} Prints the object number $n$ ($\%n$)
in raw format. If the number $n$ is omitted, print the latest computed object
($\%$). \label{se:history}

\subsec{\b{c}}\sidx{available commands} Prints the list of all available
hardcoded functions under \kbd{gp}, not including operators written as special
symbols (see \secref{se:operators}). More information can be obtained using
the \kbd{?} metacommand (see above). For user-defined functions / member
functions, see \b{u} and \b{um}.

\subsec{\b{d}} Prints the \idx{defaults} as described in the
previous section (shortcut for \kbd{default()}, see \secref{se:default}).

\subsec{\b{e} $\{n\}$} Switches the \tet{echo} mode on (1) or off (0). If
$n$ is explicitly given, set echo to $n$.

\subsec{\b{g} $\{n\}$} Sets the debugging level \tet{debug} to the
non-negative integer $n$.

\subsec{\b{gf} $\{n\}$} Sets the file usage debugging level \tet{debugfiles}
to the non-negative integer $n$.

\subsec{\b{gm} $\{n\}$} Sets the memory debugging level \tet{debugmem}
to the non-negative integer $n$.

\subsec{\b{h} $\{m$\kbd{-}$n\}$} Outputs some debugging info about the
hashtable. If the argument is a number $n$, outputs the contents of cell
$n$. Ranges can be given in the form $m$\kbd{-}$n$ (from cell $m$ to cell
$n$, \$ = last cell). If a function name is given instead of a number or
range, outputs info on the internal structure of the hash cell this
function occupies (a \kbd{struct entree} in C). If the range is reduced to
a dash ('\kbd{-}'), outputs statistics about hash cell usage.

\subsec{\b{l} $\{$\var{logfile}$\}$} Switches \tet{log} mode on and off.
If a \var{logfile} argument is given, change the default logfile name to
\var{logfile} and switch log mode on.

\subsec{\b{m}} As \b{a}, but using prettymatrix format.

\subsec{\b{o} $\{n\}$} Sets \tet{output} mode to $n$ ($0$: raw, $1$:
prettymatrix, $3$: external prettyprint).

\subsec{\b{p} $\{n\}$} Sets \tet{realprecision} to $n$ decimal digits.
Prints its current value if $n$ is omitted.

\subsec{\b{pb} $\{n\}$} Sets \tet{realbitprecision} to $n$ bits.
Prints its current value if $n$ is omitted.

\subsec{\b{ps} $\{n\}$} Sets \tet{seriesprecision} to $n$ significant terms.
Prints its current value if $n$ is omitted.

\subsec{\b{q}} Quits the \kbd{gp} session and returns to the system.
Shortcut for \tet{quit}\kbd{()} (see \secref{se:quit}).

\subsec{\b{r} $\{$\var{filename}$\}$} Reads into \kbd{gp} all the
commands contained in the named file as if they had been typed from the
keyboard, one line after the other. Can be used in combination with the \b{w}
command (see below). Related but not equivalent to the function \kbd{read}
(see \secref{se:read}); in particular, if the file contains more than one
line of input, there will be one history entry for each of them, whereas
\kbd{read} would only record the last one. If \var{filename} is omitted,
re-read the previously used input file (fails if no file has ever been
successfully read in the current session). If a \kbd{gp} \tet{binary file}
(see \secref{se:writebin}) is read using this command, it is silently loaded,
without cluttering the history.

Assuming \kbd{gp} figures how to decompress files on your machine, this
command accepts compressed files in \tet{compress}ed (\kbd{.Z}) or
\tet{gzip}ped (\kbd{.gz} or \kbd{.z}) format. They will be uncompressed on
the fly as \kbd{gp} reads them, without changing the files themselves.

\subsec{\b{s}} Prints the state of the PARI \tev{stack} and \tev{heap}.
This is used primarily as a debugging device for PARI.

\subsec{\b{t}} Prints the \idx{internal longword format} of all the PARI
types. The detailed bit or byte format of the initial codeword(s) is
explained in Chapter~4, but its knowledge is not necessary for a \kbd{gp}
user.

\subsec{\b{u}} Prints the definitions of all user-defined functions.

\subsec{\b{um}} Prints the definitions of all user-defined member functions.

\subsec{\b{v}} Prints the \idx{version number} and implementation architecture
(680x0, Sparc, Alpha, other) of the \kbd{gp} executable you are using.

\subsec{\b{w} $\{n\}$ $\{$\var{filename}$\}$} Writes the object number
$n$ ( $\%n$ ) into the named file, in raw format. If the number $n$ is
omitted, writes the latest computed object ( $\%$ ). If \var{filename} is
omitted, appends to \kbd{logfile} (the GP function \tet{write} is a trifle more
powerful, as you can have arbitrary file names).

\subsec{\b{x} $\{n\}$} Prints the complete tree with addresses and contents
(in hexadecimal) of the \idx{internal representation} of the object number
$n$ ( $\%n$ ). If the number $n$ is omitted, uses the latest computed object
in \kbd{gp}. As for \b{s}, this is used primarily as a debugging device for
PARI, and the format should be self-explanatory. The underlying GP
function \tet{dbg_x} is more versatile, since it can be applied to other
objects than history entries.

\subsec{\b{y} $\{n\}$} Switches \kbd{simplify} on (1) or off (0). If $n$
is explicitly given, set simplify to $n$.

\subseckbd{\#} Switches the \kbd{timer} on or off.

\subseckbd{\#\#} Prints the time taken by the latest computation.
Useful when you forgot to turn on the \kbd{timer}.

\section{The preferences file}\sidx{startup}\sidx{preferences file}
\label{se:gprc}

This file, called \tet{gprc} in the sequel, is used to modify or extend
\kbd{gp} default behavior, in all \kbd{gp} sessions: e.g customize
\kbd{default} values or load common user functions and aliases. \kbd{gp}
opens the \kbd{gprc} file and processes the commands in there, \emph{before}
doing anything else, e.g.~creating the PARI stack. If the file does not exist
or cannot be read, \kbd{gp} will proceed to the initialization phase at once,
eventually emitting a prompt. If any explicit command line switches are
given, they override the values read from the preferences file.

\subsec{Syntax} The syntax in the \kbd{gprc} file (and valid in this file
only) is simple-minded, but should be sufficient for most purposes. The file
is read line by line; as usual, white space is ignored unless surrounded by
quotes and the standard multiline constructions using braces, \kbd{\bs}, or
\kbd{=} are available (multiline comments between \kbd{/*~\dots~*/} are also
recognized).

\subsubsec{Preprocessor:}
Two types of lines are first dealt with by a preprocessor:

\item comments are removed. This applies to all text surrounded by
\kbd{/*~\dots~*/} as well as to everything following \kbd{\bs\bs} on a given
line.

\item lines starting with \kbd{\#if} \var{boolean} are treated as
comments if \var{boolean} evaluates to \kbd{false}, and read normally
otherwise. The condition can be negated using either \kbd{\#if not} (or
\kbd{\#if !}). If the rest of the current line is empty, the test applies to
the next line (same behavior as \kbd{=} under \kbd{gp}). The following
tests can be performed:

\kbd{EMACS}: \kbd{true} if \kbd{gp} is running in an Emacs or TeXmacs shell (see
\secref{se:emacs}).

\kbd{READL}: \kbd{true} if \kbd{gp} is compiled with \kbd{readline} support (see
\secref{se:readline}).

\kbd{VERSION} \var{op} \var{number}: where \var{op} is in the set
$\{ \kbd{>}, \kbd{<}, \kbd{<=}, \kbd{>=} \}$, and \var{number} is a PARI
version number of the form \var{Major}.\var{Minor}.\var{patch}, where the
last two components can be omitted (i.e.~$1$ is understood as version $1.0.0$).
This is \kbd{true} if \kbd{gp}'s version number satisfies the required
inequality.

\kbd{BITS\_IN\_LONG} \kbd{==} \var{number}: \var{number} is $32$ (resp.~$64$).
This is \kbd{true} if \kbd{gp} was built for a 32-bit (resp.~64-bit)
architecture.

\subsubsec{Commands:}
After preprocessing, the remaining lines are executed as
sequence of expressions (as usual, separated by \kbd{;} if necessary). Only
two kinds of expressions are recognized:

\item \var{default} \kbd{=} \var{value}, where \var{default} is one of
the available defaults (see \secref{se:defaults}), which will be set to
\var{value} on actual startup. Don't forget the quotes around strings
(e.g.~for \kbd{prompt} or \kbd{help}).

\item \kbd{read "\var{some\_GP\_file}"} where \kbd{\var{some\_GP\_file}}
is a regular GP script this time, which will be read just before \kbd{gp}
prompts you for commands, but after initializing the defaults. In particular,
file input is delayed until the \kbd{gprc} has been fully loaded. This is the
right place to input files containing \kbd{alias} commands, or your favorite
macros.

\noindent For instance you could set your prompt in the following portable way:
\bprog
\\ self modifying prompt looking like @com\hbox{\rm(18:03) \key{gp}\kbd{ >}}
prompt   = "(%H:%M) \e[1mgp\e[m > "

\\ readline wants non-printing characters to be braced between ^A/^B pairs
#if READL prompt = "(%H:%M) ^A\e[1m^Bgp^A\e[m^B > "

\\ escape sequences not supported under emacs
#if EMACS prompt = "(%H:%M) gp > "
@eprog

\noindent Note that any of the last two lines could be broken in the
following way
\bprog
#if EMACS
  prompt = "(%H:%M) gp > "
@eprog
\noindent since the preprocessor directive applies to the next line if the
current one is empty.

A sample \kbd{gprc} file called \kbd{misc/gprc.dft} is provided in the
standard distribution. It is a good idea to have a look at it and customize
it to your needs. Since this file does not use multiline constructs, here is
one (note the terminating \kbd{;} to separate the expressions):
\bprog
#if VERSION > 2.2.3
{
  read "my_scripts";     \\ syntax errors in older versions
  new_galois_format = 1; \\ default introduced in 2.2.4
}
#if ! EMACS
{
  colors = "9, 5, no, no, 4, 1, 2";
  help   = "gphelp -detex -ch 4 -cb 0 -cu 2";
}
@eprog

\subsec{The gprc location}
When \kbd{gp} is started, it looks for a customization file, or \kbd{gprc} in
the following places (in this order, only the first one found will be
loaded):

\noindent\item \kbd{gp} checks whether the environment variable
\tet{GPRC} is set. On Unix, this can be done with something like: \smallskip

\settabs\+\indent&\kbd{GPRC=/my/dir/anyname; export GPRC}\quad&\cr

\+&\kbd{GPRC=/my/dir/anyname; export GPRC}\quad&in \kbd{sh} syntax
(for instance in your \kbd{.profile}),\cr

\+&\kbd{setenv GPRC /my/dir/anyname} &in \kbd{csh} syntax
(in your \kbd{.login} or \kbd{.cshrc} file).\cr

\+&\kbd{env GPRC=/my/dir/anyname gp} &on the command line launching \kbd{gp}.\cr

\noindent If so, the file named by \kbd{\$GPRC} is the \kbd{gprc}.

\noindent\item If \kbd{GPRC} is not set, and if the environment variable
\kbd{HOME} is defined, \kbd{gp} then tries

\kbd{\$HOME/.gprc} on a Unix system

\kbd{\$HOME\bs gprc.txt} on a DOS, OS/2, or Windows system.

\noindent\item If no gprc was found among the user files mentioned above
we look for \kbd{/etc/gprc} for a system-wide gprc file (you will need root
privileges to set up such a file yourself).

\noindent\item Finally, we look in pari's \kbd{datadir} for a file named

\kbd{.gprc} on a Unix system

\kbd{gprc.txt} on a DOS, OS/2, or Windows system. If you are using our
Windows installer, this is where the default preferences file is written.

\noindent Note that on Unix systems, the \kbd{gprc}'s default name starts
with a '.' and thus is hidden to regular \kbd{ls} commands; you need to type
\kbd{ls -a} to list it.


\section{Using readline} \sidx{line editor}\sidx{completion}

This very useful library provides line editing and contextual completion
to \kbd{gp}. You are encouraged to read the \kbd{readline} user manual,
but we describe basic usage here.

\misctitle{A (too) short introduction to readline}\label{se:readline}
In the following, \kbd{C-} stands for ``the \kbd{Control} key combined with
another'' and the same for \kbd{M-} with the \kbd{Meta} key; generally
\kbd{C-} combinations act on characters, while the \kbd{M-} ones operate on
words. The \kbd{Meta} key might be called \kbd{Alt} on some keyboards, will
display a black diamond on most others, and can safely be replaced by
\kbd{Esc} in any case.

Typing any ordinary key inserts text where the cursor stands, the arrow keys
enabling you to move in the line. There are many more movement commands,
which will be familiar to the Emacs user, for instance \kbd{C-a}/\kbd{C-e}
will take you to the start/end of the line, \kbd{M-b}/\kbd{M-f} move the
cursor backward/forward by a word, etc. Just press the \kbd{<Return>} key at
any point to send your command to \kbd{gp}.

  All the commands you type at the \kbd{gp} prompt are stored in a history,
a multiline command being saved as a single concatenated line. The Up and Down
arrows (or \kbd{C-p}/\kbd{C-n}) will move you through the history,
\kbd{M-<}/\kbd{M->} sending you to the start/end of the history.
\kbd{C-r}/\kbd{C-s} will start an incremental backward/forward search. You
can kill text (\kbd{C-k} kills till the end of line, \kbd{M-d} to the end of
current word) which you can then yank back using the \kbd{C-y} key (\kbd{M-y}
will rotate the kill-ring). \kbd{C-\_} will undo your last changes
incrementally (\kbd{M-r} undoes all changes made to the current line).
\kbd{C-t} and \kbd{M-t} will transpose the character (word) preceding the
cursor and the one under the cursor.

  Keeping the \kbd{M-} key down while you enter an integer (a minus sign
meaning reverse behavior) gives an argument to your next readline command
(for instance \kbd{M-- C-k} will kill text back to the start of line). If you
prefer \idx{Vi}--style editing, \kbd{M-C-j} will toggle you to Vi mode.

  Of course you can change all these default bindings. For that you need to
create a file named \kbd{.inputrc} in your home directory. For instance
(notice the embedding conditional in case you would want specific bindings
for \kbd{gp}):
%
\bprog
$if Pari-GP
  set show-all-if-ambiguous
  "\C-h": backward-delete-char
  "\e\C-h": backward-kill-word
  "\C-xd": dump-functions
  (: "\C-v()\C-b"       #@com can be annoying when copy-pasting!
  [: "\C-v[]\C-b"
$endif
@eprog
\noindent\kbd{C-x C-r} will re-read this init file, incorporating any
changes made to it during the current session.

\misctitle{Note} By default, \kbd{(} and \kbd{[} are bound to the function
\kbd{pari-matched-insert} which, if ``electric parentheses'' are enabled
(default: off) will automatically insert the matching closure (respectively
\kbd{)} and \kbd{]}). This behavior can be toggled on and off by giving
the numeric argument $-2$ to \kbd{(} (\kbd{M--2(}), which is useful if you
want, e.g to copy-paste some text into the calculator. If you do not want a
toggle, you can use \kbd{M--0} / \kbd{M--1} to specifically switch it on or
off).

\misctitle{Note} In some versions of readline (2.1 for instance), the
\kbd{Alt} or \kbd{Meta} key can give funny results (output 8-bit accented
characters for instance). If you do not want to fall back to the \kbd{Esc}
combination, put the following two lines in your \kbd{.inputrc}:
%
\bprog
  set convert-meta on
  set output-meta off
@eprog

\misctitle{Command completion and online help} Hitting
\kbd{<TAB>} will complete words for you. This mechanism is context-dependent:
\kbd{gp} will strive to only give you meaningful completions in a given
context (it will fail sometimes, but only under rare and restricted
conditions).

  For instance, shortly after a \kbd{\til}, we expect a user name, then a
path to some file. Directly after \kbd{default(} has been typed, we would
expect one of the \kbd{default} keywords. After a '.', we expect a member
keyword. And generally of course, we expect any GP symbol which may be found
in the hashing lists: functions (both yours and GP's), and variables.

  If, at any time, only one completion is meaningful, \kbd{gp} will provide it
together with

\item an ending comma if we are completing a default,

\item a pair of parentheses if we are completing a function name. In
that case hitting \kbd{<TAB>} again will provide the argument list as given
by the online help. (Recall that you can always undo the effect
of the preceding keys by hitting \kbd{C-\_}; this applies here.)

Otherwise, hitting \kbd{<TAB>} once more will give you the list of possible
completions. Just experiment with this mechanism as often as possible,
you will probably find it very convenient. For instance, you can obtain
\kbd{default(seriesprecision,10)}, just by hitting \kbd{def<TAB>se<TAB>10},
which saves 18 keystrokes (out of 27).

  Hitting \kbd{M-h} will give you the usual short online help concerning the
word directly beneath the cursor, \kbd{M-H} will yield the extended help
corresponding to the \kbd{help} default program (usually opens a \idx{dvi}
previewer, or runs a primitive tex-to-ASCII program). None of these disturb
the line you were editing.

\section{GNU Emacs and PariEmacs}
\label{se:emacs}

If you install the PariEmacs package (see Appendix A), you may use \kbd{gp}
as a subprocess in \idx{Emacs}. You then need to include in your \kbd{.emacs}
file the following lines:
\bprog
  (autoload 'gp-mode "pari" nil t)
  (autoload 'gp-script-mode "pari" nil t)
  (autoload 'gp "pari" nil t)
  (autoload 'gpman "pari" nil t)

  (setq auto-mode-alist
    (cons '("\\.gp$" . gp-script-mode) auto-mode-alist))
@eprog
\noindent which autoloads functions from the PariEmacs package and ensures
that file with the \kbd{.gp} suffix are edited in gp-script mode.

Once this is done, under GNU Emacs if you type \kbd{M-x gp} (where as usual
\kbd{M} is the \kbd{Meta} key), a special shell will be started launching
\kbd{gp} with the default stack size and prime limit. You can then work as
usual under \kbd{gp}, but with all the facilities of an advanced text editor.
See the PariEmacs documentation for customizations, menus, etc.

\newpage
