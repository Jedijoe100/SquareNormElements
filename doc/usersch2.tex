% $Id$
% Copyright (c) 2000  The PARI Group
%
% This file is part of the PARI/GP documentation
%
% Permission is granted to copy, distribute and/or modify this document
% under the terms of the GNU General Public License
\chapter{The gp Calculator}

\section{Introduction}

Originally, \tet{gp} was designed as a debugging device for the PARI system
library. Over the years, it has become a powerful user-friendly stand-alone
calculator. The mathematical functions available in PARI and \kbd{gp} are
described in the next chapter. In the present one, we describe the specific
use of the \kbd{gp} programmable calculator.

\emacs If you have GNU Emacs and use the PariEmacs package, you can work in a
special Emacs shell, described in \secref{se:emacs}. Specific features of
this Emacs shell are indicated by an EMACS sign in the left margin.

\subsec{Startup}

To start the calculator, the general command line syntax is:

\kbd{gp [-s \var{stacksize}] [-p \var{primelimit}] [\var{files}]}

\noindent
where items within brackets are optional. The [\var{files}] argument is a
list of files written in the GP scripting language, which will be loaded on
startup. The ones starting with a minus sign are \emph{flags}, setting some
internal parameters of \kbd{gp}, or \var{defaults}. See \secref{se:defaults}
below for a list and explanation of all defaults, there are many more than
just those two. These defaults can be changed by adding parameters to the
input line as above, or interactively during a \kbd{gp} session, or in a
preferences file also known as \tet{gprc}.

If a \idx{preferences file} (to be discussed in \secref{se:gprc}) is
found, \kbd{gp} then reads it and executes the commands it contains. This
provides an easy way to customize \kbd{gp}. The \var{files} argument is
processed right after the \kbd{gprc}.

A copyright banner then appears which includes the version number, and a lot
of useful technical information. After the copyright, the computer writes the
top-level help information, some initial defaults, and then waits after
printing its prompt, which is '\kbd{?~}' by default . Whether extended
on-line help and line editing are available or not is indicated in this
\kbd{gp} banner, between the version number and the copyright message.
Consider investigating the matter with the person who installed \kbd{gp} if
they are not. Do this as well if there is no mention of the GMP kernel.

\subsec{Getting help}

To get help, type a \kbd{?} and hit return. A menu appears, describing the
eleven main categories of available functions and how to get more detailed
help. If you now type \kbd{?$n$} with $1\le n\le11$, you get the list of
commands corresponding to category $n$ and simultaneously to Section $3.n$ of
this manual. If you type \kbd{?}\var{functionname} where \var{functionname}
is the name of a PARI function, you will get a short explanation of this
function.

If extended help (see \secref{se:exthelp}) is available on your system,
you can double or triple the \kbd{?} sign to get much more: respectively the
complete description of the function (e.g.~\kbd{??sqrt}), or a list of
\kbd{gp} functions relevant to your query (e.g.~ \kbd{???"elliptic curve"}
or \kbd{???"quadratic field"}).

If \kbd{gp} was properly installed (see Appendix~A), a line editor is
available to correct the command line, get automatic completions, and so on.
See \secref{se:readline} or \kbd{??readline} for a short summary of the line
editor's commands.

If you type \kbd{?\bs} you will get a short description of the metacommands
(keyboard shortcuts).

Finally, typing \kbd{?.} will return the list of available (pre-defined)
member functions. These are functions attached to specific kind of objects,
used to retrieve easily some information from complicated structures (you can
define your own but they won't be shown here). We will soon describe these
commands in more detail.

More generally, commands starting with the symbols \b\ or \kbd{?}, are not
computing commands, but are metacommands which allow you to exchange
information with \kbd{gp}. The available metacommands can be divided into
default setting commands (explained below) and simple commands (or keyboard
shortcuts, to be dealt with in \secref{se:meta}).

\subsec{Input}

Just type in an instruction, e.g. \kbd{1 + 1}, or \kbd{Pi}. No action is
undertaken until you hit the \kbd{<Return>} key. Then computation starts, and
a result is eventually printed. To suppress printing of the result, end the
expression with a \kbd{;} sign. Note that many systems use \kbd{;} to
indicate end of input. Not so in \kbd{gp}: a final semicolon means the
result should not be printed. (Which is certainly useful if it occupies
several screens.)

\subsec{Interrupt, Quit}

Typing \kbd{quit} at the prompt ends the session and exits \kbd{gp}. At any
point you can type \kbd{Ctrl-C} (that is press simultaneously the
\kbd{Control} and \kbd{C} keys): the current computation is interrupted and
control given back to you at the \kbd{gp} prompt, together with a message
like
\bprog
  ***   at top-level: gcd(a,b)
  ***                 ^--------
  *** gcd: user interrupt after 236 ms.
@eprog\noindent
telling you how much time ellapsed since the last command was typed in and
in which GP function the computation was aborted. It does not mean that that
much time was spent in the function, only that the evaluator was busy
processing that specific function when you stopped it.

\section{The general gp input line}

The \kbd{gp} calculator uses a purely interpreted language GP. The structure
of this language is reminiscent of LISP with a functional notation,
\kbd{f(x,y)} rather than \kbd{(f x y)}: all programming constructs,
such as \kbd{if}, \kbd{while,} etc\dots are functions\footnote{*}{Not exactly,
since not all their arguments need be evaluated. For instance it would be
stupid to evaluate both branches of an \kbd{if} statement: since only one
will apply, only this one is evaluated.}, and the main loop does not really
execute, but rather evaluates (sequences of) expressions. Of course, it is by
no means a true LISP, and has been strongly influenced by C and Perl since
then.

\subsec{Introduction}. User interaction with a \kbd{gp} session proceeds as
follows. First, one types a sequence of characters at the \kbd{gp} prompt;
see \secref{se:readline} for a description of the line editor. When you hit
the \kbd{<Return>} key, \kbd{gp} gets your input, evaluates it, then prints
the result and assigns it to an ``history'' array.

More precisely, the input is case-sensitive and, outside of character
strings, blanks are completely ignored. Inputs are either metacommands or
sequences of expressions. Metacommands are shortcuts designed to alter gp's
internal state, such as the working precision or general verbosity level; we
shall describe them in \secref{se:meta}, and ignore them for the time being.

The evaluation of a sequence of instructions proceeds in two phases: your
input is first digested (byte-compiled) to a bytecode suitable for fast
evaluation, in particular loop bodies are compiled only once but a priori
evaluated many times; then the bytecode is evaluated.

An \idx{expression}\sidx{expression sequence} is formed by combining
constants, variables, operator symbols, functions and control statements.
It is evaluated using the conventions about operator priorities and left to
right associativity. An expression always has a value, which can be any PARI
object:
\bprog
? 1 + 1
%1 = 2          \\@com an ordinary integer
? x
%2 = x          \\@com a polynomial of degree 1 in the unknown \kbd{x}
? print("Hello")
Hello           \\@com \kbd{void} return value
? f(x) = x^2
%3 = (x)->x^2   \\@com a user function
@eprog
\noindent In the third example, \kbd{Hello} is printed as a side effect, but
is not the return value. The \kbd{print} command is a \emph{procedure},
which conceptually returns nothing. But in fact procedures return a special
\kbd{void} object, meant to be ignored; in particular, it does not clutter
the history (but evaluates to $0$ in a numeric context). The final example
assigns to the variable \kbd{f} the function $x\mapsto x^2$, the alternative
form \kbd{f = x->x\pow2} achieving the same effect; the return value of a
function definition is, unsurprisingly, a function object (of type
\typ{CLOSURE}).

Several expressions are combined on a single line by separating them with
semicolons ('\kbd{;}'). Such an expression sequence will be called a
\var{seq}. A \var{seq} also has a value, which is the value of the last
expression in the sequence. Under \kbd{gp}, the value of the \var{seq}, and
only this last value, becomes an history entry. The values of the other
expressions in the \var{seq} are discarded after the execution of the
\var{seq} is complete, except of course if they were assigned into variables.
In addition, the value of the \var{seq} is printed if the line does not end
with a semicolon \kbd{;}.

\subsec{The gp history of results}.

This is not to be confused with the history of your \emph{commands},
maintained by \kbd{readline}. The \kbd{gp} history contains the \emph{results}
they produced, in sequence. More precisely, several inputs act through side
effects and produce a \kbd{void} result, for instance a \kbd{print} statement
or a \kbd{for} loop. The \kbd{gp} history consists exactly of the
non-\kbd{void} results.

The successive elements of the history array are called \kbd{\%1}, \kbd{\%2},
\dots As a shortcut, the latest computed expression can also be
called \kbd{\%}, the previous one \kbd{\%`},
the one before that \kbd{\%``} and so on. The total number of history entries
is \kbd{\%\#}.

When you suppress the printing of the result with a semicolon, it is still
stored in the history, but its history number will not appear either. It is a
better idea to assign it to a variable for later use than to mentally
recompute what its number is. Of course, on the next line, you may just use
\kbd{\%}.

This history ``array'' is in fact better thought of as a queue: its size is
limited to 5000 entries by default, after which \kbd{gp} starts forgetting
the initial entries. So \kbd{\%1} becomes unavailable as \kbd{gp} prints
\kbd{\%5001}. You can modify the history size using \tet{histsize}.

\subsec{Special editing characters}.\sidx{editing characters} A GP program
can of course have more than one line. Since your commands are executed as
soon as you have finished typing them, there must be a way to tell \kbd{gp}
to wait for the next line or lines of input before doing anything. There are
three ways of doing this.

The first one is to use the \idx{backslash character} \kbd{\bs} at the end of
the line that you are typing, just before hitting \kbd{<Return>}. This tells
\kbd{gp} that what you will write on the next line is the physical
continuation of what you have just written. In other words, it makes \kbd{gp}
forget your newline character. You can type a \kbd{\bs} anywhere. It is
interpreted as above only if (apart from ignored whitespace characters) it is
immediately followed by a newline. For example, you can type
\bprog
? 3 + \
4
@eprog
\noindent instead of typing \kbd{3 + 4}.

The second one is a variation on the first, and is mostly useful when
defining a user function (see \secref{se:user_defined}): since an equal sign
can never end a valid expression, \kbd{gp} disregards a newline immediately
following an \kbd{=}.
\bprog
? a =
123
%1 = 123
@eprog

The third one is in general much more useful, and uses braces \kbd{\obr} and
\kbd{\cbr}.\sidx{brace characters} An opening brace \kbd{\obr} signals that
you are typing a multi-line command, and newlines are ignored until you type
a closing brace \kbd{\cbr}. There are two important, but easily obeyed,
restrictions: first, braces do not nest; second, inside an open brace-close
brace pair, all input lines are concatenated, suppressing any newlines. Thus,
all newlines should occur after a semicolon (\kbd{;}), a comma (\kbd{,}) or
an operator (for clarity's sake, never split an identifier over two lines in
this way). For instance, the following program
\bprog
{
  a = b
  b = c
}
@eprog

\noindent would silently produce garbage, since this is interpreted as
\kbd{a=bb=c} which assigns the value of \kbd{c} to both \kbd{bb} and
\kbd{a}. It should have been written
\bprog
{
  a = b;
  b = c;
}
@eprog

\section{The PARI types}

\noindent
We see here how to input values of the different data types known to PARI.
Recall that blanks are ignored in any expression which is not a string (see
below).

\misctitle{A note on efficiency.}
The following types are provided for convenience, not for speed:
\typ{INTMOD}, \typ{FRAC}, \typ{PADIC}, \typ{QUAD}, \typ{POLMOD},
\typ{RFRAC}. Indeed, they always perform a reduction of some kind after
each basic operation, even though it is usually more efficient to perform
a single reduction at the end of some complex computation. For instance,
in a convolution product $\sum_{i+j = n} x_i y_j$ in $\Z/N\Z$ --- common
when multiplying polynomials! ---, it is quite wasteful to perform $n$
reductions modulo $N$. In short, basic individual operations on these types
are fast, but recursive objects with such components could be handled more
efficiently: programming with libpari will save large constant factors here,
compared to GP.

\subsec{Integers} \sidx{integer}
(\tet{t_INT}): after an (optional) leading \kbd{+} or \kbd{-}, type in the
decimal digits of your integer. No decimal point!
\bprog
? 1234567
%1 = 1234567
? -3
%2 = -3
? 1.         \\@com oops, not an integer
%3 = 1.000000000000000000000000000
@eprog

\subsec{Real numbers} \sidx{real number}
(\tet{t_REAL}): after an (optional) leading \kbd{+} or \kbd{-},
type a number with a decimal point. Leading zeroes may be omitted, up to the
decimal point, but trailing zeroes are important: your \typ{REAL}
is assigned an internal precision, which is the supremum of the input
precision and the default precision, expressed in decimal digits. For
example, if the default precision is 28 digits, typing \kbd{2.} yields a
precision of 28 digits, but \kbd{2.0\dots0} with 45 zeros gives a number with
internal precision at least 45, although less may be printed.

You can also use scientific notation with the letter \kbd{E} or
\kbd{e}. As usual, \kbd{e$n$} is interpreted as $\times 10^n$ for all
integers $n$. Since the result is converted to a \typ{REAL}, you may often
omit the decimal point in this case: \kbd{6.02 E 23} or \kbd{1e-5} are fine,
but \kbd{e10} is not.

By definition, \kbd{0.E $n$} returns a real $0$ of exponent $n$, whereas
\kbd{0.} returns a real 0 ``of default precision'' (of exponent
$-\tet{realprecision}$), see \secref{se:whatzero}, behaving like the machine
epsilon for the current default accuracy: any float of smaller absolute value
is undistinguishable from $0$.


\misctitle{Note on output formats.} A zero real number is printed in \kbd{e}
format as $0.Exx$ where $xx$ is the (usually negative) \emph{decimal}
exponent of the number (cf.~\secref{se:whatzero}). This allows the user to
check the accuracy of that particular zero.

When the integer part of a real number $x$ is not known exactly because the
exponent of $x$ is greater than the internal precision, the real number is
printed in \kbd{e} format.

\subsec{Intmods}\sidx{intmod}
(\tet{t_INTMOD}): to create the image of the integer $a$ in $\Z/b\Z$ (for
some non-zero integer $b$), type \kbd{Mod(a,b)}; \emph{not} \kbd{a\%b}.
Internally, all operations are done on integer representatives belonging to
$[0,b-1]$.

Note that this type is available for convenience, not for speed: each
elementary operation involves a reduction modulo $b$.

If $x$ is a \typ{INTMOD} \kbd{Mod(a,b)}, the following member function is
defined:

\kbd{x.mod}: return the modulus \kbd{b}.

\subsec{Rational numbers}\sidx{rational number}
(\tet{t_FRAC}): all fractions are automatically reduced to lowest
terms, so it is impossible to work with reducible fractions. To enter $n/m$
just type it as written. As explained in \secref{se:gdiv}, floating point
division is \emph{not} performed, only reduction to lowest
terms.\label{se:FRAC}

Note that rational computation are almost never the fastest method to proceed:
in the PARI implementation, each elementary operation involves computing a gcd.
It is generally a little more efficient to cancel denominators and work with
integers only:
\bprog
? P = Pol( vector(10^3,i, 1/i) ); \\@com big polynomial with small rational coeffs
? P^2
time = 1,392 ms.
? c = content(P); c^2 * (P/c)^2;  \\@com same computation in integers
time = 1,116 ms.
@eprog\noindent
And much more efficient (but harder to setup) to use homomorphic imaging
schemes and modular computations. As the simple exemple below indicates, if you
only need modular information, it is very much worthwile to work with
\typ{INTMOD}s directly, rather than deal with \typ{RFRAC}s all the way through:
\bprog
? p = nextprime(10^7);
? sum(i=1, 10^5, 1/i) % p
time = 13,288 ms.
%1 = 2759492
? sum(i=1, 10^5, Mod(1/i, p))
time = 60 ms.
%2 = Mod(2759492, 10000019)
@eprog\noindent

\subsec{Finite field elements}\sidx{finite field element} (\tet{t_FFELT}):
let $T\in\F_p[X]$ be a monic irreducible polynomial defining your
finite field over $\F_p$, for instance obtained using \tet{ffinit}. Then the
\tet{ffgen} function creates a generator of the finite field as an
$\F_p$-algebra, namely the class of $X$ in $\F_p[X]/(T)$, from which you can
build all other elements. For instance, to create the field $\F_{2^8}$, we
write
\bprog
? T = ffinit(2, 8);
? y = ffgen(T, 'y);
? y^0    \\ the unit element in the field
%3 = 1
? y^8
%4 = y^6 + y^5 + y^4 + y^3 + y + 1
@eprog\noindent
The second (optional) parameter to \tet{ffgen} is only used to display
the result; it is customary to use the name of the variable we assign the
generator to. If \kbd{f} is a \typ{FFELT}, the following member functions are
defined:

\kbd{f.pol}: the polynomial (with reduced integer coefficients)
expressing \kbd{f} in term of the field generator.

\kbd{f.p}: the characteristic of the finite field.

\kbd{f.mod}: the minimal polynomial (with reduced integer
coefficients) of the field generator.

\subsec{Complex numbers}\sidx{complex number} (\tet{t_COMPLEX}): to
enter $x+iy$, type \kbd{x + I*y}. (That's \kbd{I}, \emph{not} \kbd{i}!) The
letter \tet{I} stands for $\sqrt{-1}$. The ``real'' and ``imaginary''
parts $x$ and $y$ can be of type \typ{INT}, \typ{REAL}, \typ{INTMOD},
\typ{FRAC}, or \typ{PADIC}.

\subsec{$p$-adic numbers}\sidx{p-adic number}\label{se:padic}
(\tet{t_PADIC}):
Typing \kbd{O($p$\pow $k$)}, where $p$ and $k$ are integers, yields a
$p$-adic $0$ of accuracy~$k$, representing any $p$-adic number whose
valuation is $\geq k$. To input a general non-0 $p$-adic number, write
a suitably precise rational or integer approximation and add \kbd{O($p$\pow
$k$)} to it.

Note that it is not checked whether $p$ is indeed prime but results are
undefined if this is not the case: you can work on $10$-adics if you want,
but disasters will happen as soon as you do something non-trivial like
taking a square root. Note that \kbd{O(25)} is not the same as
\kbd{O(5\pow 2)}; you want the latter!

For example, you can type in the $7$-adic number

\kbd{2*7\pow(-1) + 3 + 4*7 + 2*7\pow 2 + O(7\pow3)}

\noindent exactly as shown, or equivalently as \kbd{905/7 + O(7\pow3)}.

If $a$ is a \typ{PADIC}, the following member functions are defined:

\kbd{a.mod}: returns the modulus $p^k$.

\kbd{a.p}: returns $p$.

Note that this type is available for convenience, not for speed:
internally, \typ{PADIC}s are stored as $p$-adic units modulo some $p^k$.
Each elementary operation involves updating $p^k$ (multiplying or
dividing by powers of $p$) and a reduction mod $p^k$. In particular,
additions are slow.
\bprog
    ? n = 1+O(2^20);   for (i=1,10^6, n++)
    time = 841 ms.
    ? n = Mod(1,2^20); for (i=1,10^6, n++)
    time = 441 ms.
    ? n = 1;           for (i=1,10^6, n++)
    time = 328 ms.
@eprog\noindent The penalty associated with maintaining $p^k$ decreases
steeply as $p$ increases (and updates become very rare). But \typ{INTMOD}s
remain at least 25\% more efficient. (But they do not have denominators!)
% n = 1+O(1009^2);   for (i=1,10^6, n++)
% n = Mod(1,1009^2); for (i=1,10^6, n++)
\subsec{Quadratic numbers}\sidx{quadratic number} (\tet{t_QUAD}):
This type is used to work in the quadratic order of \emph{discriminant}
\kbd{d}, where \kbd{d} is a non-square integer congruent to $0$ or $1$
(modulo $4$). The command
\bprog
    w = quadgen(d)
@eprog\noindent
assigns to \kbd{w} the ``canonical'' generator for the integer basis
of the order of discriminant $d$, i.e.~$w=\sqrt{d}/2$ if $d\equiv 0 \mod 4$,
and $w=(1+\sqrt{d})/2$ if $d\equiv 1 \mod 4$. The name \kbd{w} is of course
just a suggestion, but corresponds to traditional usage. You can use any
variable name that you like, but \kbd{quadgen(d)} is always printed as
\kbd{w}, regardless of the discriminant. So beware, two \typ{QUAD}s can be
printed in the same way and not be equal; however, \kbd{gp} will refuse to add
or multiply them for example.

Since the order is $\Z + \kbd{w}\Z$, any other element can be input
as \kbd{$x$+$y$*w} for some integers $x$ and $y$. In fact, you may work in
its fraction field $\Q(\sqrt{d})$ and use \typ{FRAC} values for $x$ and $y$.

\subsec{Polmods}\sidx{polmod} (\tet{t_POLMOD}): exactly as
for intmods, to enter $x \mod y$ (where $x$ and $y$ are polynomials),
type \kbd{Mod(x,y)}, not \kbd{x\%y}. Note that when $y$ is an irreducible
polynomial in one variable, polmods whose modulus is $y$ are simply
algebraic numbers in the finite extension defined by the polynomial $y$.
This allows us to work easily in \idx{number field}s, finite extensions of
the $p$-adic field $\Q_p$, or \idx{finite field}s.

Note that this type is available for convenience, not for speed: each
elementary operation involves a reduction modulo $y$.
If $p$ is a \typ{POLMOD}, the following members functions are defined:

\kbd{p.pol}: return a representative of the polynomial class of minimal degree.

\kbd{p.mod}: return the modulus.

\label{se:rempolmod}
\misctitle{Important remark.}\sidx{variable (priority)}
Mathematically, the variables\sidx{variable} occurring in a polmod are not
free variables. But internally, a congruence class in $R[t]/(y)$ is
represented by its representative  of lowest degree, which is a \typ{POL} in
$R[t]$, and computations occur with polynomials in the variable $t$. PARI
will not recognize that \kbd{Mod(y, y\pow2 + 1)} is ``the same'' as
\kbd{Mod(x, x\pow2 + 1)}, since \kbd{x} and \kbd{y} are different variables.

To avoid inconsistencies, polmods must use the same variable in internal
operations (i.e.~between polmods) and variables of lower priority for
external operations, typically between a polynomial and a polmod. See
\secref{se:priority} for a definition of ``priority'' and a discussion of
(PARI's idea of) multivariate polynomial arithmetic.
For instance:
\bprog
    ? Mod(x, x^2+ 1) + Mod(x, x^2 + 1)
    %1 = Mod(2*x, x^2 + 1)    \\@com $2i$ (or $-2i$), with $i^2=-1$
    ? x + Mod(y, y^2 + 1)
    %2 = x + Mod(y, y^2 + 1)  \\@com in $\Q(i)[x]$
    ? y + Mod(x, x^2 + 1)
    %3 = Mod(x + y, x^2 + 1)  \\@com in $\Q(y)[i]$
@eprog\noindent
The first two are straightforward, but the last one may not be what you
want: \kbd{y} is treated here as a numerical parameter, not as a polynomial
variable.

If the main variables are the same, it is allowed to mix \typ{POL} and
\typ{POLMOD}s. The result is the expected \typ{POLMOD}. For instance
\bprog
    ? x + Mod(x, x^2 + 1)
    %1 = Mod(2*x, x^2 + 1)
@eprog

\subsec{Polynomials}\sidx{polynomial}\label{se:pol}
(\tet{t_POL}): type the polynomial in a natural way, not
forgetting to put a ``$*$'' between a coefficient and a formal variable;
\bprog
? 1 + 2*x + 3*x^2
%1 = 3*x^2 + 2*x + 1
@eprog\noindent
This assumes that \kbd{x} is still a ''free variable''.
\bprog
? x = 1; 1 + 2*x + 3*x^2
%2 = 6
@eprog\noindent
generates an integer, not a polynomial! It is good practice to never assign
values to polynomial variables to avoid the above problem, but a foolproof
construction is available using \kbd{'x} instead of~\kbd{x}: \kbd{'x}
is a constant evaluating to the free variable with name \kbd{x},
independently of the current value of~\kbd{x}.
\bprog
? x = 1; 1 + 2*'x + 3*'x^2
%3 = 1 + 2*x + 3*x^2
? x = 'x; 1 + 2*x + 3*x^2
%4 = 1 + 2*x + 3*x^2
@eprog\noindent
You may also use the functions \kbd{Pol} or \kbd{Polrev}:
\bprog
? Pol([1,2,3])         \\@com \kbd{Pol} creates a polynomial in \kbd{x} by default
%1 = x^2 + 2*x + 3
? Polrev([1,2,3])
%2 = 3*x^2 + 2*x + 1
? Pol([1,2,3], 'y)     \\@com we use \kbd{'y}, safer than \kbd{y}
%3 = y^2 + 2*y + 3
@eprog\noindent The latter two are much more efficient constructors than an
explicit summation (the latter is quadratic in the degree, the former linear):
\bprog
? for (i=1, 10^4, Polrev( vector(100, i,i) ) )
time = 124ms

? for (i=1, 10^4, sum(i = 1, 100, (i+1) * 'x^i) )
time = 3,985ms
@eprog

Polynomials are always printed as \emph{univariate} polynomials, with
monomials sorted by degreasing degree:
\bprog
? (x+y+1)^2
%1 = x^2 + (2*y + 2)*x + (y^2 + 2*y + 1)
@eprog\noindent
(Univariate polynomial in \kbd{x} whose coefficients are polynomials in
\kbd{y}.) See \secref{se:varsymb} for valid variable names, and a discussion
of multivariate polynomial rings.

\subsec{Power series}\sidx{power series}\label{se:series}
(\tet{t_SER}):
Typing \kbd{O(X\pow $k$)}, where $k$ is an integer, yields an $X$-adic $0$ of
accuracy~$k$, representing any power series in \kbd{X} whose valuation is
$\geq k$. Of course, \kbd{X} can be replaced by any other variable name! To
input a general non-0 power series, type in a polynomial or rational
function (in \kbd{X}, say), and add \kbd{O(X\pow $k$)} to it. The discussion
in the \typ{POL} section about variables remains valid; a constructor
\tet{Ser} replaces \tet{Pol} and \tet{Polrev}.

\misctitle{Caveat.} Power series with inexact coefficients sometimes have a
non-intuitive behaviour: if $k$ significant terms are requested, an inexact
zero is counted as significant, even if it is the coefficient of lowest
degree. This means that useful higher order terms may be disregarded.

If a series with a zero leading coefficient must be inverted, then as a
desperation measure that coefficient is discarded, and a warning is issued:
\bprog
? C = 0. + y + O(y^2);
? 1/C
  *** _/_: Warning: normalizing a series with 0 leading term.
%2 = y^-1 + O(1)
@eprog\noindent
The last output could be construed as a bug since it is a priori impossible
to deduce such a result from the input ($0.$ represents any sufficiently
small real number). But it was thought more useful to try and go on with an
approximate computation than to raise an early exception.

If the series precision is insufficient, errors may occur (mostly division
by $0$), which could have been avoided by a better global understanding of
the computation:
\bprog
? A = 1/(y + 0.); B = 1. + O(y);
? B * denominator(A)
%2 = 0.E-28 + O(y)
? A/B
  *** _/_: Warning: normalizing a series with 0 leading term.
%3 = 1.000000000000000000000000000*y^-1 + O(1)
? A*B
  *** _*_: Warning: normalizing a series with 0 leading term.
%4 = 1.000000000000000000000000000*y^-1 + O(1)
@eprog

\subsec{Rational functions}\sidx{rational function}
(\tet{t_RFRAC}): as for fractions, all rational
functions are automatically reduced to lowest terms. All that was
said about fractions in \secref{se:FRAC} remains valid here.

\subsec{Binary quadratic forms of positive or negative discriminant}%
\sidx{binary quadratic form}
(\tet{t_QFR} and \tet{t_QFI}):
these are input using the function \kbd{Qfb} (see Chapter~3). For example
\kbd{Qfb(1,2,3)} creates the binary form $x^2+2xy+3y^2$. It is imaginary (of
internal type \typ{QFI}) since $2^2 - 4*3 = -8$ is negative. Although
imaginary forms could be positive or negative definite, only positive
definite forms are implemented.

In the case of forms with positive discriminant (\typ{QFR}), you may add
an optional fourth component (related to the regulator, more precisely to
Shanks and Lenstra's ``distance''), which must be a real number. See also the
function \kbd{qfbprimeform} which directly creates a prime form of given
discriminant (see Chapter~3).

\subsec{Row and column vectors}\sidx{row vector}\sidx{column vector}
(\tet{t_VEC} and \tet{t_COL}): to enter a row vector, type the components
separated by commas ``\kbd{,}'', and enclosed between brackets
``\kbd{[}$\,$'' and ``$\,$\kbd{]}'', e.g.~\kbd{[1,2,3]}. To enter a column
vector, type the vector horizontally, and add a tilde ``\til'' to transpose.
\kbd{[ ]} yields the empty (row) vector. The function \tet{Vec} can be used
to transform any object into a vector (see Chapter~3).

If the variable $x$ contains a (row or column) vector, $x[m]$ refers to its
$m$-th entry. You can assign a result to $x[m]$ (i.e.~write something like
$x[k]=\var{expr}$).

\subsec{Matrices} (\tet{t_MAT}):\sidx{matrix} to enter a matrix, type
the components line by line, the components being separated by commas
``\kbd{,}'', the lines by semicolons ``\kbd{;}'', and everything enclosed in
brackets ``\kbd{[}$\,$'' and ``$\,$\kbd{]}'', e.g. \kbd{[x,y; z,t; u,v]}.
\kbd{[;]} yields an empty ($0 \times 0$) matrix. The function \tet{Mat}
transforms any object into a matrix, and \tet{matrix} creates matrices
whose $(i,j)$-th entry is described by a function $f(i,j)$:
\bprog
? Mat(1)
%1 =
[1]
? matrix(2,2, i,j, 2*i+j)
%2 =
[3 4]

[5 6]
@eprog

If \kbd{M} is a matrix, \kbd{M[i,j]} refers to its $(i,j)$-th entry, 
\kbd{M[i,]} to its $i$-th row, and
\kbd{M[,j]} to its $j$-th column; but \kbd{M[i]} is meaningless and triggers
an error. You can assign a result to \kbd{M[i,j]}, \kbd{M[i,]} or
\kbd{M[,j]}; provided that the expression is a row or column vector of the
right dimension in the later two cases. This process is recursive, so if
\kbd{M}
is a matrix of matrices of \dots, an expression such as \kbd{M[1,1][,3][4] =
1} is perfectly valid (and actually identical to \kbd{M[1,1][4,3] = 1}),
assuming that all matrices along the way have compatible dimensions.

\misctitle{Technical note (design flaw).} Matrices are internally represented
as a vector of columns. All matrices with $0$ columns are thus represented
by the same object (internally, an empty vector), and there is no way to
distinguish between them. Thus it is not possible to create or represent
matrices with zero columns and an actual nonzero number of rows.
The empty matrix \kbd{[;]} is handled as though it had an arbitrary number of
rows, exactly as many as needed for the current computation to make sense:
\bprog
? [1,2,3; 4,5,6] * [;]
%1 = [;]
@eprog\noindent
The empty matrix on the first line is understood as a $3\times 0$ matrix, and
the result as a $2\times 0$ matrix. On the other hand, it is possible to
create matrices with a given positive number of columns, each of which has
zero rows, e.g.~using \kbd{Mat} as above or using the \kbd{matrix} function.

Note that although the internal representation is essentially the same, a row
vector of column vectors is \emph{not} a matrix; for example, multiplication
will not work in the same way. It is easy to go from one representation to
the other using \tet{Vec} / \tet{Mat}, though:
\bprog
? [1,2,3;4,5,6]
%1 =
[1 2 3]

[4 5 6]

? Vec(%)
%2 = [[1, 4]~, [2, 5]~, [3, 6]~]
? Mat(%)
%3 =
[1 2 3]

[4 5 6]
@eprog

\subsec{Lists} (\tet{t_LIST}):\sidx{list} lists can be input
directly, as in \kbd{List([1,2,3,4])}; but in most cases, one creates
an empty list, then appends elements using \kbd{listput}:
\bprog
  ? a = List(); listput(a,1); listput(a,2);
  ? a
  %2 = List([1, 2])
@eprog\noindent
Elements can be accessed directly as with the vector types described above.

\subsec{Strings} (\tet{t_STR}):\sidx{string}\sidx{character string} to
enter a string, just enclose it between double quotes \kbd{"}, like this:
\kbd{"this is a string"}. The function \kbd{Str} can be used to transform any
object into a string.

\subsec{Small vectors} (\tet{t_VECSMALL}): this is an internal type,
used to code in an efficient way vectors containing only small integers, such
as permutations. Most \kbd{gp} functions will refuse to operate on these
objects.

\subsec{Functions} (\tet{t_CLOSURE}): we will explain this at length in
\secref{se:user_defined}. For the time being, suffices it to say that
functions can be assigned to variables, as any other object, and the
following equivalent basic forms are available to create new ones
\bprog
  f = (x,y) -> x^2 + y^2

  f(x,y) = x^2 + y^2
@eprog

\section{GP operators}\label{se:operators}

\noindent Loosely speaking, an \idx{operator} is a function, usually
associated to basic arithmetic operations, whose name contains only
non-alphanumeric characters. For instance \kbd{+} or \kbd{-}, but also
\kbd{=} or \kbd{+=}, or even \kbd{[ ]} (the selection operator). As all
functions, operators take arguments, and return a value; \emph{assignment}
operators also have side effects: besides returning a value, they change the
value of some variable.

Each operator has a fixed and unchangeable priority, which means that, in
a given expression, the operations with the highest priority is performed
first. Unless mentionned otherwise, operators at the same priority level are
left-associative (performed from left to right), unless they are assignments,
in which case they are right-associative. Anything enclosed between
parenthesis is considered a complete subexpression, and is resolved
recursively, independently of the surrounding context. For instance,
\bprog
  a + b + c    -->   (a + b) + c     \\@com left-associative
  a = b = c    -->   a = (b = c)     \\@com right-associative
@eprog\noindent
Assuming that \var{op}$_1$, \var{op}$_2$, \var{op}$_3$ are
binary operators with increasing priorities (think of \kbd{+},
\kbd{*}, \kbd{\pow}),
$$ x~\var{op}_1~y~\var{op}_2~z~\var{op}_2~x~\var{op}_3~y $$ is
equivalent to $$ x~\var{op}_1~((y~\var{op}_2~z)~\var{op}_2~
(x~\var{op}_3~y)).$$

GP contains many different operators, either unary (having only
one argument) or binary, plus a few special selection operators. Unary
operators are defined as either \emph{prefix}  or \emph{postfix}, meaning
that they respectively precede (\var{op}~$x$) and follow ($x$~\var{op}) their
single argument. Some symbols are syntactically correct in both positions,
like \kbd{!}, but then represent different operators: the \kbd{!} symbol
represents the negation and factorial operators when in prefix and postfix
position respectively. Binary operators all use the (infix) syntax
$x$~\var{op}~$y$.

Most operators are standard (\kbd{+}, \kbd{\%}, \kbd{=}), some are
borrowed from the C language (\kbd{++}, \kbd{<<}), and a few are
specific to GP (\kbd{\bs}, \kbd{\#}). Beware that some GP operators differ
slightly from their C counterparts. For instance, GP's postfix \kbd{++}
returns the \emph{new} value, like the prefix \kbd{++} of~C, and the binary
shifts \kbd{<<}, \kbd{>>} have a priority which is different from (higher
than) that of their C counterparts. When in doubt, just surround everything
by parentheses; besides, your code will be more legible.

\noindent Here is the list of available operators, ordered by decreasing
\idx{priority}, binary and left-associative unless mentioned otherwise. An
expression is an \tev{lvalue} if something can be assigned to it. (The name
comes from left-value, to the left of a \kbd{=} operator; e.g.
\kbd{x}, or \kbd{v[1]} are lvalues, but \kbd{x + 1} is not.)

\def\point#1{\noindent $\bullet$ #1\hfill\break\indent\strut}
\point{Priority 14}
%
\kbd{:} as in \kbd{x:small}, is used to indicate to the GP2C compiler that the
variable on the left-hand side always contains objects of the type specified
on the right hand-side (here, a small integer) in order to produce more
efficient or more readable C code. This is ignored by GP.

%
\point{Priority 13}
\kbd{( )} is the function call operator. If $f$ is a closure and \var{args}
is a comma-separated list of arguments (possibly empty),
$f\kbd{(\var{args})}$ evaluates $f$ on those arguments.

\point{Priority 12}
%
\kbd{++} and \kbd{--} (unary, postfix): if $x$ is an \tet{lvalue},
\kbd{$x$++} assigns the value $x+1$ to $x$, then returns the new value of
$x$. This corresponds to the C statement \kbd{++$x$}: there is no prefix
\kbd{++} operator in GP. \kbd{$x$--} does the same with $x-1$. These
operators are not associative, i.e. \kbd{x++++} is invalid, since
\kbd{x++} is not an lvalue.

\point{Priority 11}
%
\kbd{.}\var{member} (unary, postfix): \kbd{$x$.\var{member}} extracts
\var{member} from structure $x$ (see~\secref{se:member}).

\kbd{[ ]} is the selection operator. \kbd{$x$[$i$]} returns the $i$-th
component of vector $x$; \kbd{$x$[$i$,$j$]}, \kbd{$x$[,$j$]} and
\kbd{$x$[$i$,]} respectively return the entry of coordinates $(i,j)$, the
$j$-th column, and the $i$-th row of matrix $x$. If the assignment operator
(\kbd{=}) immediately follows a sequence of selections, it assigns its right
hand side to the selected component. E.g \kbd{x[1][1] = 0} is valid; but
beware that \kbd{(x[1])[1] = 0} is not (because the parentheses force the
complete evaluation of \kbd{x[1]}, and the result is not modifiable).

\point{Priority 10}
%
\kbd{'} (unary, postfix): derivative with respect to the main variable.
If $f$ is a function (\typ{CLOSURE}), $f'$ is allowed and defines a new
function, which will perform \idx{numerical derivation} when evaluated
at a scalar $x$; this is defined as $(f(x+\varepsilon) - f(x-\varepsilon)) /
2\varepsilon$ for a suitably small epsilon depending on current precision.
\bprog
? (x^2 + y*x + y^2)'  \\@com derivation with respect to main variable \kbd{x}
%1 = 2*x + y
? SIN = cos'
%2 = cos'
? SIN(Pi/6)         \\@com numerical derivation
%3 = -0.5000000000000000000000000000
? cos'(Pi/6)        \\@com works directly: no need for the intermediate \kbd{SIN}
%4 = -0.5000000000000000000000000000
@eprog

\strut\kbd{\til} (unary, postfix): vector/matrix transpose.

\kbd{!} (unary, postfix): factorial. $x\kbd{!}=x(x-1)\cdots 1$.

\kbd{!} (unary, prefix): logical \var{not}. \kbd{!$x$} return $1$ if $x$ is
equal to $0$ (specifically, if \kbd{gcmp0($x$)==1}), and $0$ otherwise.

\point{Priority 9}
%
\kbd{\#} (unary, prefix): cardinality; \kbd{\#$x$} returns \kbd{length($x$)}.

\point{Priority 8}
%
\kbd{\pow}: powering. This operator is right associative:
\kbd{2 \pow 3\pow 4} is understood as \kbd{2 \pow (3\pow 4)}.

\point{Priority 7}
%
\kbd{+}, \kbd{-} (unary, prefix): \kbd{-} toggles the sign of its argument,
\kbd{+} has no effect whatsoever.

\point{Priority 6}
%
\kbd{*}: multiplication.

\kbd{/}: exact division (\kbd{3/2} yields $3/2$, not $1.5$).

\kbd{\bs}, \kbd{\%}: Euclidean quotient and remainder, i.e.~if $x =
qy + r$, then $\kbd{x \b{ } y} = q$, $\kbd{x\%y} = r$. If $x$ and $y$
are scalars, $q$ is an integer and $r$ satisfies $0\le r < y$; if $x$ and $y$
are polynomials, $q$ and $r$ are polynomials such that $\deg r< \deg y$ and
the leading terms of $r$ and $x$ have the same sign.

\kbd{\bs/}: rounded Euclidean quotient for integers (rounded towards
$+\infty$ when the exact quotient would be a half-integer).

\kbd{<<}, \kbd{>>}: left and right binary shift. By definition,
\kbd{x<<n}$~=~x * 2^n$ if $n>0$, and $\kbd{truncate}(x 2^{-n})$ otherwise.
Right shift is defined by \kbd{x>>n}$~=~$\kbd{x<<(-n)}.

\point{Priority 5}
%
\kbd{+}, \kbd{-}: addition/subtraction.

\point{Priority 4}
%
\kbd{<}, \kbd{>}, \kbd{<=}, \kbd{>=}: the usual comparison operators,
returning 1 for \kbd{true} and 0 for \kbd{false}. For instance,
\kbd{x<=1} returns $1$ if $x\le 1$ and $0$ otherwise.

\kbd{<>}, \kbd{!=}: test for (exact) inequality.

\kbd{==}: test for (exact) equality.

\point{Priority 3}
%
\kbd{\&}, \kbd{\&\&}: logical \var{and}.

\kbd{|}, \kbd{||}: logical (inclusive) \var{or}. Any sequence of logical
\var{or} and \var{and} operations is evaluated from left to right,
and aborted as soon as the final truth value is known. Thus, for instance,
\bprog
  x == 0 || test(1/x)
@eprog\noindent
will never produce an error since \kbd{test(1/x)} is not even evaluated
when the first test is true (hence the final truth value is true). Similarly
\bprog
  type(p) == "t_INT" && isprime(p)
@eprog\noindent
does not evaluate \kbd{isprime(p)} if \kbd{p} is not an integer.

\point{Priority 2}
%
\kbd{=} (assignment, \var{lvalue} \kbd{=} \var{expr}). The result of
\kbd{x~=~$y$} is the value of the expression~$y$, which is also assigned to
the variable~\kbd{x}. This assignment operator is right-associative. This is
\emph{not} the equality test operator; a statement like \kbd{x~=~1} is always
true (i.e.~non-zero), and sets \kbd{x} to~1; the equality test would be
\kbd{x == 1}. The right hand side of the assignment operator is evaluated
before the left hand side.

It is crucial that the left hand-side be an \var{lvalue} there, it avoids
ambiguities in expressions like \kbd{1 + x = 1}. The latter evaluates as
\kbd{1 + (x = 1)}, not as \kbd{(1 + x) = 1}, even though the priority of
\kbd{=} is lower than the priority of \kbd{+} : \kbd{1 + x} is not an lvalue.

If the expression cannot be parsed in a way where the left hand side is an
lvalue, raise an error.
\bprog
? x + 1 = 1
  ***   unused characters: x+1=1
  ***                         ^--
@eprog
\leavevmode
\kbd{\var{op}=}, where \var{op} is any binary operator
among \kbd{+}, \kbd{-}, \kbd{*}, \kbd{\%}, \kbd{/}, \kbd{\bs}, \kbd{\bs/},
\kbd{<<}, or
\kbd{>>} (composed assignment \var{lvalue} \var{op}\kbd{=} \var{expr}).
The expression \kbd{x~\var{op}=~$y$} assigns $(\kbd{x}~\var{op}~y)$
to~\kbd{x}, and returns the new value of~\kbd{x}. The result is \emph{not}
an \tev{lvalue}; thus
\bprog
  (x += 2) = 3
@eprog\noindent
is invalid. These assignment operators are right-associative:
\bprog
  ? x = 'x; x += x *= 2
  %1 = 3*x
@eprog

\point{Priority 0}
\kbd{->} (function definition): \kbd{(\var{vars})->\var{expr}} returns a
function object, of type \typ{CLOSURE}.

\misctitle{Remark:} Use the \var{op}\kbd{=} operators as often as possible
since they make complex assignments more legible: one needs not parse
complicated expressions twice to make sure they are indeed identical. Compare
\bprog
v[i+j-1] = v[i+j-1] + 1    -->    v[i+j-1]++

M[i,i+j] = M[i,i+j] * 2    -->    M[i,i+j] *= 2
@eprog

\misctitle{Remark:} Less important but still interesting. The
\kbd{++}, \kbd{--} and \var{op}\kbd{=} operators are slightly more efficient:
\bprog
? a = 10^6;
? i = 0; while(i<a, i=i+1)
time = 365 ms.
? i = 0; while(i<a, i++)
time = 352ms.
@eprog
\noindent For the same reason, the shift operators should be preferred to
multiplication:
\bprog
? a = 1<<(10^5);
? i = 1; while(i<a, i=i*2);
time = 1,052 ms.
? i = 1; while(i<a, i<<=1);
time = 617 ms.
@eprog

\section{Variables and symbolic expressions}\sidx{variable}\label{se:varsymb}
In this section we use \emph{variable} in the standard mathematical
sense, symbols representing algebraically independent elements used to build
rings of polynomials and power series, and explain the all-important concept
of \emph{variable priority}. In the next \secref{se:scope}, we shall no
longer consider only free variables, but adopt the viewpoint of computer
programming and assign values to these symbols: (bound) variables are names
associated to values in a given scope.

\subsec{Variable names}\label{se:varname}. A valid name starts with a letter,
followed by any number of keyword characters: \kbd{\_} or alphanumeric
characters ([\kbd{A-Za-z0-9}]). The built-in function names are reserved and
cannot be used; see the list with \b{c}, including the constants \kbd{Pi},
\kbd{Euler} and $\kbd{I}=\sqrt{-1}$.

GP names are case sensitive. For instance, the symbol \kbd{i} is perfectly
safe to use, and will not be mistaken for $\kbd{I} = \sqrt{-1}$; analogously,
\kbd{o} is not synonymous to \kbd{O}.

In GP you can use up to 16383 variable names (up to 65535 on 64-bit
machines). If you ever need thousands of variables and this becomes a serious
limitation, you should probably be using vectors instead: e.g. instead of
variables \kbd{X1}, \kbd{X2}, \kbd{X3}, \dots, you might equally well store
their values in \kbd{X[1]}, \kbd{X[2]}, \kbd{X[3]}, \dots

\subsec{Variables and polynomials}\sidx{free variable}
What happens when you use a valid variable name,\kbd{t} say, for the first
time \emph{before} assigning a value into it? This registers a new
\emph{free variable} with the interpreter (which will be written as \kbd{t}),
and evaluates to a monomial of degree $1$ in the said variable \kbd{t}. It is
important to understand that PARI/GP is \emph{not} a symbolic manipulation
package: even free variables already have default values%
\footnote{*}{More generally, any expression has a value, and is
\emph{replaced} by its value as soon as it is read; it never stays in an
abstract form.}%
, there is no such thing as an ``unbound'' variable in GP.
You have access to this default value using the quote operator: \kbd{'t}
always evaluates to the above monomial of degree~$1$, independently of
assignments made since then (e.g. \kbd{t = 1}).
%
\bprog
? t^2 + 1
%1 = t^2 + 1
? t = 2; t^2 + 1
%2 = 5
? %1
%3 = t^2 + 1
? eval(%1)
%4 = 5
@eprog\noindent
In the above, \kbd{t} is initially a free variable, later bound to $2$. We
see that assigning a value to a variable does not affect previous expressions
involving it; to take into account the new variable's value, one must force a
new evaluation, using the function \kbd{eval} (see \secref{se:eval}). It is
preferable to leave alone your ``polynomial variables'', never assigning
values to them, and to use \kbd{subst} and its more powerful variants rather
than \kbd{eval}. You will avoid the following kind of problems:
\bprog
? p = t^2 + 1; subst(p, t, 2)
%1 = 5
? t = 2;
? subst(p, t, 3)    \\@com \kbd{t} is no longer free: it evaluates to 2
  ***   at top-level: subst(p,t,3)
  ***                         ^----
  ***   variable name expected.
? subst(p, 't, 3)   \\ OK
%3 = 10
@eprog\noindent
A statement like \kbd{x = 'x} in effect restores \kbd{x} as a free variable.

\subsec{Variable priorities, multivariate objects}.\sidx{variable (priority)}\label{se:priority}
A multivariate polynomial in PARI is just a polynomial (in one variable),
whose coefficients are themselves polynomials, arbitrary but for the fact
that they do not involve the main variable. (PARI currently has no sparse
representation for polynomials, listing only non-zero monomials.) All
computations are then done formally on the coefficients as if the
polynomial was univariate.

This is not symmetrical. So if I enter \kbd{x + y} in a clean session,
what happens ? This is understood as
$$ x^1 + (y^1 + 0*y^0)*x^0 \in (\Z[y])[x] $$
but how do we know that $x$ is ``more important'' than $y$ ? Why not $y^1 +
x*y^0$, which is the same mathematical entity after all ?

The answer is that variables are ordered implicitly by the interpreter:
when a new identifier (e.g~$x$, or $y$ as above) is input, the corresponding
variable is registered as having a strictly lower priority than any variable in
use at this point\footnote{*}{This is not strictly true:
the variable $x$ is predefined and always has the highest possible priority.}%
. To see the ordering used by \kbd{gp} at any given time, type
\kbd{variable()}.\kbdsidx{variable}

Given such an ordering, multivariate polynomials are stored so that the
variable with the highest priority is the main variable. And so on,
recursively, until all variables are exhausted. A different storage pattern
(which could only be obtained via \kbd{libpari} programming and low-level
constructors) would produce an invalid object, and eventually a disaster.

In any case, if you are working with expressions involving several variables
and want to have them ordered in a specific manner in the internal
representation just described, the simplest is just to write down the
variables one after the other under \kbd{gp} before starting any real computations.
You could also define variables from your \tet{gprc} to have a consistent
ordering of common variable names in all your \kbd{gp} sessions, e.g read in a file
\kbd{variables.gp} containing
\bprog
x;y;z;t;a;b;c;d;
@eprog

\misctitle{Important note:} PARI allows Euclidean division of multivariate
polynomials, but assumes that the computation takes place in the fraction
field of the coefficient ring (if it is not an integral domain, the result
will a priori not make sense). This can become tricky; for instance
assume $x$ has highest priority (which is always the case), then
$y$:
\bprog
? x % y
%1 = 0
? y % x
%2 = y             \\@com these two take place in $\Q(y)[x]$
? x * Mod(1,y)
%3 = Mod(1, y)*x   \\@com in $(\Q(y)/y\Q(y))[x] \sim \Q[x]$
? Mod(x,y)
%4 = 0
@eprog
\noindent In the last example, the division by $y$ takes place in
$\Q(y)[x]$,
hence the \kbd{Mod} object is a coset in $(\Q(y)[x]) / (y\Q(y)[x])$, which
is the null ring since $y$ is invertible! So be very wary of variable
ordering when your computations involve implicit divisions and many
variables. This also affects functions like \tet{numerator}/\tet{denominator}
or \tet{content}:
\bprog
? denominator(x / y)
%1 = 1
? denominator(y / x)
%2 = x
? content(x / y)
%3 = 1/y
? content(y / x)
%4 = y
? content(2 / x)
%5 = 2
@eprog
\noindent Can you see why ? Hint: $x/y = (1/y) * x$ is in $\Q(y)[x]$ and
denominator is taken with respect to $\Q(y)(x)$; $y/x = (y*x^0) / x$ is in
$\Q(y)(x)$ so $y$ is invertible in the coefficient ring. On the other hand,
$2/x$ involves a single variable and the coefficient ring is simply $\Z$.

These problems arise because the variable ordering defines an \emph{implicit}
variable with respect to which division takes place. This is
the price to pay to allow \kbd{\%} and \kbd{/} operators on polynomials
instead of requiring a more cumbersome \kbd{divrem($x$, $y$, \var{var})}
(which also exists). Unfortunately, in some functions like \tet{content} and
\tet{denominator}, there is no way to set explicitly a main variable like in
\tet{divrem} and remove the dependence on implicit orderings. This will
hopefully be corrected in future versions.

\subsec{Multivariate power series}.
Just like multivariate polynomials, power series are fundamentally
single-variable objects. It is awkward to handle many variables at once,
since PARI's implementation cannot handle multivariate error terms like
$O(x^i y^j)$. (It can handle the polynomial $O(y^j) \times x^i$ which is
a very different thing, see below.)

The basic assumption in our model is that if variable $x$ has higher
priority than $y$, then $y$ does not depend on $x$: setting $y$ to a
function of $x$ after some computations with bivariate power series does
not make sense a priori. This is because implicit constants in
expressions like $O(x^i)$ depend on $y$ (whereas in $O(y^j)$ they can not
depend on $x$). For instance
\bprog
  ? O(x) * y
  %1 = O(x)
  ? O(y) * x
  %2 = O(y)*x
@eprog\noindent
Here is a more involved example:
\bprog
  ? A = 1/x^2 + 1 + O(x); B = 1/x + 1 + O(x^3);
  ? subst(z*A, z, B)
  %2 = x^-3 + x^-2 + x^-1 + 1 + O(x)
  ? B * A
  %3 = x^-3 + x^-2 + x^-1 + O(1)
  ? z * A
  %4 = z*x^-2 + z + O(x)
@eprog\noindent
The discrepancy between \kbd{\%2} and \kbd{\%3} is surprising. Why does
\kbd{\%2} contain a spurious constant term, which cannot be
deduced from the input ? Well, we ignored the rule that forbids to
substitute an expression involving high-priority variables
to a low-priority variable. The result \kbd{\%4} is correct according to
our rules since the implicit constant in $O(x)$ may depend on $z$. It is
obviously wrong if $z$ is allowed to have negative valuation in $x$. Of
course, the correct error term should be $O(xz)$, but this is not
possible in PARI.

\section{Variables and Scope}\sidx{variable scope}\label{se:scope}
This section is rather technical, and strives to explain potentially
confusing concepts. Skip to the last subsection for practical advice, if the
next discussion does not make sense to you. After learning about user
functions, study the example in \secref{se:bewarescope} then come back.

\misctitle{Definitions}. % rather not make it a section to allow for ??my

A \emph{scope} is an enclosing context where names and values are associated.
A user's function body, the body of a loop, an individual command line, all
define scopes; the whole program defines the \emph{global} scope. The
argument of \tet{eval} is evaluated in the enclosing scope.

Variables are bound to values within a given scope. This is traditionnally
implemented in two different ways:

\item\sidx{lexical scoping} lexical (or static) scoping: the binding makes
sense within a given block of program text. The value is private to the block
and may not be accessed from outside. Where to find the value is determined
at compile time.

\item\sidx{dynamic scoping} dynamic scoping: introducing a local variable,
say \kbd{x}, pushes a new value on a stack associated to the name \kbd{x}
(possibly empty at this point), which is popped out when the control flow
leaves the scope. Evaluating \kbd{x} in any context, possibly outside of the
given block, always yields the top value on this dynamic stack.

GP implements both lexical and dynamic scoping, using the keywords%
\footnote{*}{The names are borrowed from the \tet{Perl} scripting language.}
\tet{my} (lexical) and \tet{local} (dynamic):
\bprog
  x = 0;
  f() = x
  g() =    my(x = 1); f()
  h() = local(x = 1); f()
@eprog\noindent
The function \kbd{g} returns 0 since the global \kbd{x} binding
is unaffected by the introduction of a private variable of the same name in
\kbd{g}. On the other hand, \kbd{h} returns 1; when it calls \kbd{f()}, the
binding stack for the \kbd{x} identifier contains two items: the global
binding to 0, and the binding to 1 introduced in \kbd{h}, which is still
present on the stack since the control flow has not left \kbd{h} yet.

\subsec{Scoping rules}.

Named parameters in a function definition, as well as all loop
indices\footnote{*}{
More generally, in all iterative constructs which use a variable name
(\kbd{for}, \kbd{prod}, \kbd{sum}, \kbd{vector}, \kbd{matrix},
\kbd{plot}, etc.) the given variable is lexically scoped to the construct's
body.},
have lexical scope within the function body and the loop body respectively.
\bprog
p = 0;
forprime (p = 2, 11, print(p)); p   \\ prints 0 at the end

x = 0;
f(x) = x++;
f(1)  \\ returns 2, and leave global x unaffected (= 0)
@eprog\noindent
If you exit the loop prematurely, e.g.~using the \kbd{break} statement, you
must save the loop index in another variable since its value prior the loop
will be restored upon exit. For instance
\bprog
  for(i = 1, n,
    if (ok(i), break);
  );
  if (i > n, return(failure));
@eprog\noindent
is incorrect, since the value of $i$ tested by the $(i > n)$ is quite
unrelated to the loop index. One ugly workaround is
\bprog
  for(i = 1, n,
    if (ok(i), isave = i; break);
  );
  if (isave > n, return(failure));
@eprog\noindent
But it is usually more natural to wrap the loop in a user function
and use \kbd{return} instead of \kbd{break}:
\bprog
try() =
{
  for(i = 1, n,
    if (ok(i), return (i));
  );
  0 \\ failure
}
@eprog

A list of variables can be lexically or dynamically scoped (to the block
between the declaration and the end of the innermost enclosing scope) using a
\kbd{my} or \kbd{local} declaration:
\bprog
for (i = 1, 10,
  my(x, y, z, i2 = i^2); \\ temps needed within the loop body
  ...
)
@eprog\noindent
Note how the declaration can include (optional) initial values, \kbd{i2 =
i\pow 2} in the above. Variables for which no explicit default value is given
in the declaration are initialized to $0$. It would be more natural to
initialize them to free variables, but this would break backward
compatibility. To obtain this behaviour, you may explicitly use the quoting
operator:
\bprog
my(x = 'x, y = 'y, z = 'z);
@eprog\noindent
A more complicated example:
\bprog
for (i = 1, 3,
  print("main loop");
  my(x = i);          \\ local to the outermost loop
  for (j = 1, 3,
    my (y = x^2);     \\ local to the innermost loop
    print (y + y^2);
    x++;
  )
)
@eprog\noindent
When we leave the loops, the values of \kbd{x}, \kbd{y}, \kbd{i}, \kbd{j}
are the same as before they were started.

Note that \tet{eval} is evaluated in the given scope, and can access values
of lexical variables:
\bprog
? x = 1;
? my(x = 0); eval("x")
%2 = 0    \\@com we see the local \kbd{x} scoped to this command line, not the global one
@eprog

Variables dynamically scoped using \kbd{local} should more appropriately be
called \emph{temporary values} since they are in fact local to the function
declaring them \emph{and} any subroutine called from within. In practice, you
almost certainly want true private variables, hence should use almost
exclusively \kbd{my}.

We strongly recommended to explicitly scope (lexically) all variables to the
smallest possible block. Should you forget this, in expressions involving such
``rogue'' variables, the value used will be the one which happens to be on
top of the value stack at the time of the call; which depends on the whole
calling context in a non-trivial way. This is in general \emph{not} what you
want.

\section{User defined functions}\sidx{user defined functions}
\label{se:user_defined}

The most important thing to understand about user-defined functions is
that they are ordinary GP objects, bound to variables just like any
other object. Those variables are subject to scoping rules as any other:
while you can define all your functions in global scope, it is usually
possible and cleaner to lexically scope your private helper functions to the
block of text where they will be needed.

Whenever gp meets a construction of the form \kbd{expr(\var{argument list})}
and the expression \kbd{expr} evaluates to a function (an object of type
\typ{CLOSURE}), the function is called with the proper arguments. For
instance, constructions like \kbd{funcs[i](x)} are perfectly valid,
assuming \kbd{funcs} is an array of functions.

\subsec{Defining a function}.\label{se:userfundef}
A user function is defined as follows:

  \kbd{(\var{list of formal variables}) -> \var{seq}}.

\noindent The list of formal variables is a comma-separated list of
\emph{distinct} variable names and allowed to be empty. It there is a single
formal variable, the parentheses are optional. This list corresponds to the
list of parameters you will supply to your function when calling it.

In most cases you want to assign a function to a variable immediately, as in
\bprog
R = (x,y) -> sqrt( x^2+y^2 );
sq = x -> x^2;  \\@com or equivalently \kbd{(x) -> x\pow2}
@eprog\noindent
but it is quite possible to define (a priori short-lived) anonymous functions.
The trailing semicolon is not part of the definition, but as usual prevents
\kbd{gp} from printing the result of the evaluation, i.e. the function
object. The construction

  \kbd{f(\var{list of formal variables}) = \var{seq}}

\noindent is available as an alias for

  \kbd{f = (\var{list of formal variables}) -> \var{seq}}

\noindent Using that syntax, it is not possible to define anonymous functions
(obviously), and the above two examples become:
\bprog
R(x,y) = sqrt( x^2+y^2 );
sq(x) = x^2;
@eprog\noindent
The semicolon serves the same purpose as above: preventing the printing
of the resulting function object; compare
\bprog
? sq(x) = x^2;  \\@com no output
? sq(x) = x^2   \\@com print the result: a function object
%2 = (x)->x^2
@eprog\noindent Of course, the sequence \var{seq} can be arbitrarily
complicated, in which case it will look better written on consecutive lines,
with properly scoped variables:
\bprogpart
{
f($x_0$, $x_1$, @dots) =
  my($t_0$, $t_1$, @dots); \\@com variables lexically scoped to the function body
  @dots
}
@eprog \noindent Note that the following variant would also work:
\bprogpart
f($x_0$, $x_1$, @dots) =
{
  my($t_0$, $t_1$, @dots); \\@com variables lexically scoped to the function body
  @dots
}
@eprog \noindent
(the first newline is disregarded due to the preceding \kbd{=} sign, and the
others because of the enclosing braces). The \tet{my} statements can actually
occur anywhere within the function body, scoping the variables to more
restricted blocks than the whole function body.

Arguments are passed by value, not as variables: modifying a function's
argument in the function body is allowed, but does not modify its value in the
calling scope. In fact, a \emph{copy} of the actual parameter is assigned to
the formal parameter when the function is called. Formal parameters are
lexically scoped to the function body. It is not allowed to use the same
variable name for different parameters of your function:
\bprog
? f(x,x) = 1
  ***   variable declared twice: f(x,x)=1
  ***                                ^----
@eprog

\misctitle{Finishing touch}.
You can add a specific help message for your function using \kbd{addhelp},
but the online help system already handles it. By default \kbd{?\var{name}}
will print the definition of the function \var{name}: the list of arguments,
as well as their default values, the text of \var{seq} as you input it.
Just as \b{c} prints the list of all built-in commands, \b{u} outputs the
list of all user-defined functions.

\misctitle{Backward compatibility (lexical scope)}. Lexically scoped
variables were introduced in version~2.4.2. Before that, the formal
parameters were dynamically scoped. If your script depends on this behaviour,
you may use the following trick: replace the initial \kbd{f(x) =} \ by
\bprog
f(x_orig) = local(x = x_orig)
@eprog
\misctitle{Backward compatibility (disjoint namespaces)}. Before version
2.4.2, variables and functions lived in disjoint namespaces and it was not
possible to have a variable and a function share the same name. Hence the
need for a \kbd{kill} function allowing to reuse symbols. This is no longer
the case.

There is now no distinction between variable and function names: we
have PARI objects (functions of type \typ{CLOSURE}, or more mundane
mathematical entities, like \typ{INT}, etc.) and variables bound to them.
There is nothing wrong with the following sequence of assignments:
\bprog
? f = 1       \\@com assigns the integer 1 to \kbd{f}
%1 = 1;
? f() = 1     \\@com a function with a constant value
%2 = ()->1
? f = x^2     \\@com \kbd{f} now holds a polynomial
%3 = x^2
? f(x) = x^2  \\@com \dots and now a polynomial function
%4 = (x)->x^2
? g(fun) = fun(Pi);\\@com a function taking a function as argument
? g(cos)
%6 = -1.000000000000000000000000000

@eprog\noindent
Previously used names can be recycled as above: you are just redefining the
variable. The previous definition is lost of course.

\misctitle{Important technical note.} Built-in functions are a special case
since they are read-only (you cannot overwrite their default meaning),
and they use features not available to user functions, in particular pointer
arguments. In the present version \vers{}, it is possible to assign a built-in
function to a variable, or to use a built-in function name to create an
anonymous function, but some special argument combinations may not be
available:
\bprog
? issquare(9, &e)
%1 = 1
? e
%2 = 3
? g = issquare;
? g(9)
%4 = 1
? g(9, &e)  \\@com pointers are not implemented for user functions
  ***   unexpected &: g(9,&e)
  ***                     ^---
@eprog

\subsec{Function call, Default arguments}

You may now call your function, as in \kbd{f(1,2)}, supplying values
for the formal variables. The number of parameters actually supplied may be
\emph{less} than the number of formal variables in the function definition.
An uninitialized formal variable is given an implicit default value of (the
integer)~0, i.e. after the definition
\bprog
f(x, y) = ...
@eprog\noindent
you may call \kbd{f(1, 2)}, supplying values for the two formal
parameters, or for example
\settabs\+\indent&xxxxxxxxx& equivalent to xxxx &\cr
\+& \kbd{f(2)}  & equivalent to &\kbd{f(2,0)},\cr
\+& \kbd{f()}   & & \kbd{f(0,0)},\cr
\+& \kbd{f(,3)} & &\kbd{f(0,3)}.  (``Empty argument'' trick)\cr
\noindent
More generally, the argument list is filled with user supplied values, in
order. A comma or closing parenthesis, where a value should have been,
signals we must use a default value. When no input arguments are left, the
defaults are used instead to fill in remaining formal parameters.

Of course, you can change these default values to something more useful than
$0$. In the function definition, you can append \kbd{=}\var{expr} to a formal
parameter, to give that variable an explicit default value. The expression
gets evaluated the moment the function is called, and may involve the
preceding function parameters: a default value for $x_i$ may involve $x_j$
for $j < i$. For instance, after
\bprog
f(x = 1, y = 2, z = y+1) = ....
@eprog\noindent
typing in \kbd{f(3,4)} would give you \kbd{f(3,4,5)}. In the rare case when
you want to set some far away argument, and leave the defaults in between as
they stand, use the ``empty argument'' trick: \kbd{f(6,,1)} would yield
\kbd{f(6,2,1)}. Of course, \kbd{f()} by itself yields \kbd{f(1,2,3)} as was
to be expected.

More specifically,
\bprog
f(x, y=2, z=3) = print(x, ":", y, ":", z);
@eprog
\noindent defines a function which prints its arguments (at most three of
them), separated by colons.
\bprog
? f(6,7)
6:7:3
? f(,5)
0:5:3
? f()
0:2:3
@eprog

\misctitle{Example}. We conclude with an amusing example, intended to
illustrate both user-defined functions and the power of the \kbd{sumalt}
function. Although the \idx{Riemann zeta-function} is included (as
\kbd{zeta}) among the standard functions, let us assume that we want to check
other implementations. Since we are highly interested in the critical strip,
we use the classical formula
$$ (2^{1-s} - 1)\zeta(s) = \sum_{n\geq 1} (-1)^n n^{-s},
  \qquad\Re s > 0.$$
The implementation is obvious:\sidx{zeta function}
\bprog
ZETA(s) = sumalt(n=1, (-1)^n*n^(-s)) / (2^(1-s) - 1)
@eprog
\noindent
Note that \kbd{n} is automatically lexically scoped to the \kbd{sumalt}
``loop'', so that it is unnecessary to add a \kbd{my(n)} declaration to the
function body. Surprisingly, this gives very good accuracy in a larger region
than expected:
\bprog
? check = z -> ZETA(z) / zeta(z);
? check(2)
%1 = 1.000000000000000000000000000
? check(200)
%2 = 1.000000000000000000000000000
? check(0)
%3 = 0.9999999999999999999999999994
? check(-5)
%4 = 1.00000000000000007549266557
? check(-11)
%5 = 0.9999752641047824902660847745
? check(1/2+14.134*I)  \\@com very close to a non-trivial zero
%6 = 1.000000000000000000003747432 + 7.62329066 E-21*I
? check(-1+10*I)
%7 = 1.000000000000000000000002511 + 2.989950968 E-24*I
@eprog\noindent Now wait a minute; not only are we summing a series which is
certainly no longer alternating (it has complex coefficients), but we are
also way outside of the region of convergence, and still get decent results! No
programming mistake this time: \kbd{sumalt} is a
``magic'' function\footnote{*}{\kbd{sumalt} is heuristic, but its use can be
rigorously justified for a given function, in particular our $\zeta(s)$
formula. Indeed, Peter Borwein (\emph{An efficient algorithm for the Riemann
zeta function}, CMS Conf.~Proc.~{\bf 27} (2000), pp.~29--34) proved that the
formula used in \kbd{sumalt} with $n$ terms computes $(1-2^{1-s})\zeta(s)$
with a relative error of the order of $(3+\sqrt{8})^{-n}|\Gamma(s)|^{-1}$.},
providing very good convergence acceleration; in effect, we are computating
the analytic continuation of our original function. To convince ourselves
that \kbd{sumalt} is a non-trivial implementation, let us try a simpler
exemple:
\bprog
? sum(n=1, 10^7, (-1)^n/n, 0.) / (-log(2)) \\@com approximates the well-known formula
time = 7,417 ms.
%1 = 0.9999999278652515622893405457
? sumalt(n=1, (-1)^n/n) / (-log(2))        \\@com accurate and fast
time = 0 ms.
%2 = 1.000000000000000000000000000
@eprog\noindent No, we are not using a powerful simplification tool here,
only numerical computations. Remember, PARI is not a computer algebra system!


\subsec{Beware scopes!}\label{se:bewarescope}
Be extra careful with the scopes of variables. What is wrong with the
following definition?
\bprog
FirstPrimeDiv(x) =
{ my(p);
  forprime(p=2, x, if (x%p == 0, break));
  p
}
? FirstPrimeDiv(10)
%1 = 0
@eprog\noindent \misctitle{Hint:} the function body is equivalent to
\bprog
{ my(newp = 0);
  forprime(p=2, x, if (x%p == 0, break));
  newp
}
@eprog\noindent
\misctitle{Detailed explanation:} The index \kbd{p} in the \kbd{forprime}
loop is lexically scoped to the loop and is not visible to the outside world.
Hence, it will not survive the \kbd{break} statement. More precisely, at this
point the loop index is restored to its preceding value. The initial
\kbd{my(p)}, although well-meant, adds to the confusion: it indeed scopes
\kbd{p} to the function body, with initial value $0$, but the \kbd{forprime}
loop introduces \emph{another} variable, unfortunately also called \kbd{p},
scoped to the loop body, which shadows the one we wanted. So we always return
$0$, since the value of the \kbd{p} scoped to the function body never changes
and is initially $0$.

To sum up, the routine returns the \kbd{p} declared local to
it, not the one which was local to \kbd{forprime} and ran through consecutive
prime numbers. Here is a corrected version:
\bprog
? FirstPrimeDiv(x) = forprime(p=2, x, if (x%p == 0, return(p)))
@eprog

\subsec{Recursive functions}. Recursive functions\sidx{recursion} can easily
be written as long as one pays proper attention to variable scope. Here is an
example, used to retrieve the coefficient array of a multivariate polynomial
(a non-trivial task due to PARI's unsophisticated representation for those
objects): \sidx{multivariate polynomial}
\bprog
coeffs(P, nbvar) =
{
  if (type(P) != "t_POL",
    for (i=1, nbvar, P = [P]);
    return (P)
  );
  vector(poldegree(P)+1, i, coeffs(polcoeff(P, i-1), nbvar-1))
}
@eprog

\noindent If $P$ is a polynomial in $k$ variables, show that after the
assignment {\tt v = coeffs(P,k)}, the coefficient of $x_1^{n_1}\dots
x_k^{n_k}$ in P is given by {\tt v[$n_1$+1][\dots][$n_k$+1]}.

The operating system automatically limits the \idx{recursion depth}:
\sidx{deep recursion}
\bprog
? dive(n) = dive(n+1)
? dive(0);
  ***   [...] at: dive(n+1)
  ***             ^---------
  ***   in function dive: dive(n+1)
  ***                     ^---------
  \\@com (last 2 lines repeated 19 times)
  ***   deep recursion.
@eprog\noindent
There is no way to increase the recursion limit (which may be different on
your machine) from within \kbd{gp}. To increase it before launching \kbd{gp},
you can use \tet{ulimit} or \tet{limit}, depending on your shell, and raise
the process available stack space (increase \tet{stacksize}).

\subsec{Function which take functions as parameters ?} Very easy:
\bprog
? calc(f, x) = f(x)
? calc(sin, Pi)
%2 = -5.04870979 E-29
? g(x) = x^2;
? calc(g, 3)
%4 = 9
@eprog
\noindent If we do not need \kbd{g} elsewhere, we should use an anonymous
function here, \kbd{calc(x->x\pow 2, 3)}. Here is a variation:
\bprog
? funs = [cos, sin, tan, x->x^3+1]; \\@com an array of functions
? call(i, x) = funs[i](x)
@eprog\noindent
evaluates the appropriate function on argument \kbd{x},
provided $1\leq i\leq 4$. Finally, a more useful example:
\bprog
APPLY(f, v) = vector(#v, i, f(v[i]))
@eprog\noindent
applies the function \kbd{f} to every element in the vector \kbd{v}.
(The built-in function \kbd{apply} is more powerful since it also applies to
lists and matrices.)

\subsec{Defining functions within a function ?}
Defining a single function is easy:
\bprog
init(x) = (add = y -> x+y);
@eprog\noindent Basically, we are defining a global variable \kbd{add}
whose value is the function \kbd{y->x+y}. The parentheses were added for
clarity and are not mandatory.
\bprog
? init(5);
? add(2)
%2 = 7
@eprog\noindent
A more refined approach is to
avoid global variables and \emph{return} the function:
\bprog
init(x) = y -> x+y
add = init(5)
@eprog\noindent Then \kbd{add(2)} still returns 7, as expected! Of course,
if \kbd{add} is in global scope, there is no gain, but we can
lexically scope it to the place where it is useful:
\bprog
  my ( add = init(5) );
@eprog

How about multiple functions then ? We can use the last idea and return a
vector of functions, but if we insist on global variables ?
The first idea
\bprog
init(x) = add(y) = x+y; mul(y) = x*y;
@eprog
\noindent does not work since in the construction \kbd{f() = }\var{seq}, the
function body contains everything until the end of the expression. Hence
executing \kbd{init} defines the wrong function \kbd{add} (itself defining
a function \kbd{mul}). The way out is to
use parentheses for grouping, to that enclosed subexpressions be
evaluated independently:
\bprog
? init(x) = ( add(y) = x+y ); ( mul(y) = x*y );
? init(5);
? add(2)
%3 = 7
? mul(3)
%4 = 15
@eprog\noindent This defines two global functions which have access to the
lexical variables private to \kbd{init}! The following would work in exactly
the same way:
\bprog
? init5() = my(x = 5); ( add(y) = x+y ); ( mul(y) = x*y );
@eprog

\subsec{Closures as Objects?} Contrary to what you might think after the
preceding examples, GP's closures may not be used to simulate true
``objects'', with private and public parts and methods to access and
manipulate them. In fact, closures indeed incorporate an existing context
(they may access lexical variables that existed at the time of their
definition), but then may not change it. More precisely, they access a copy,
which they are welcome to change, but a further function call still accesses
the original context, as it existed at the time the function was defined:
\bprog
init() =
{ my(count = 0);
  inc()=count++;
  dec()=count--;
}
? inc()
%1 = 1
? inc()
%2 = 1
? inc()
%3 = 1
@eprog

\section{Member functions}\sidx{member functions} \label{se:member}

Member functions use the `dot' notation to retrieve information from
complicated structures. The built-in structures are \tev{bid}, \tev{ell},
\tev{galois}, \tev{ff}, \tev{nf}, \tev{bnf}, \tev{bnr} and \tev{prid}, which
will be described at length in Chapter~3. The syntax \kbd{structure.member}
is taken to mean: retrieve \kbd{member} from \kbd{structure},
e.g.~\kbd{E.j} returns the $j$-invariant of the elliptic curve \kbd{E},
or outputs an error message if \kbd{E} is not a proper \tev{ell} structure.
To define your own member functions, use the syntax

\ \kbd{\var{var}.\var{member} = \var{seq}},

\noindent where the formal variable \var{var} is scoped to the function
body \var{seq}. This is of course reminiscent of a user function with a
single formal variable \var{var}. For instance, the current implementation of
the \kbd{ell} type is a vector, the $j$-invariant being the thirteenth
component. It could be implemented as

\bprog
x.j =
{
  if (type(x) != "t_VEC" || #x < 14, error("not an elliptic curve: " x));
  x[13]
}
@eprog\noindent As for user functions, you can redefine your member functions
simply by typing new definitions. On the other hand, as a safety measure, you
cannot redefine the built-in member functions, so attempting to redefine
\kbd{x.j} as above would in fact produce an error; you would have to call it
e.g.~\kbd{x.myj} in order for \kbd{gp} to accept it.

\misctitle{Rationale}. In most cases, member functions are simple accessors
of the form
\bprog
  x.a = x[1];
  x.b = x[2];
  x.c = x[3];
@eprog\noindent
where \kbd{x} is a vector containing relevant data. There are at least
three alternative approaches to the above member functions: 1) hardcode
\kbd{x[1]}, etc. in the program text, 2) define constant global variables
\kbd{AINDEX = 1}, \kbd{BINDEX = 2}  and hardcode \kbd{x[AINDEX]}, 3)
user functions \kbd{a(x) = x[1]} and so on.

Even if 2) improves on 1), these solutions are neither elegant nor flexible,
and they scale badly. 3) is a genuine possibility, but the main advantage of
member functions is that their namespace is independent from the variables
(and functions) namespace, hence we can use very short identifiers without
risk. The $j$-invariant is a good example: it would clearly not be a good
idea to define \kbd{j(E) = E[13]}, because clashes with loop indices are
likely.

\misctitle{Note}. Typing \b{um} will output all user-defined member functions.

\misctitle{Technical warning}. Do not apply a member whose name starts with
\kbd{e} or \kbd{E} to an integer constant, where it would be confused with
the usual floating point exponent. E.g compare
\bprog
  ? x.e2 = x+1
  ? 1.e2
  %1 = 100.000000000 \\@com taken to mean $1.0E2$.
  ? (1).e2
  %2 = 2
  ? 1.0.e2
  %3 = 2.00000000000
@eprog

\section{Strings and Keywords}\sidx{string}\sidx{keyword}
\label{se:strings}

\subsec{Strings}. GP variables can hold values of type character string
(internal type \typ{STR}). This section describes how they are actually used,
as well as some convenient tricks (automatic concatenation and expansion,
keywords) valid in string context.

As explained above, the general way to input a string is to enclose
characters between quotes~\kbd{"}. This is the only input construct where
whitespace characters are significant: the string will contain the exact
number of spaces you typed in. Besides, you can ``escape'' characters by
putting a \kbd{\bs} just before them; the translation is as follows
\bprog
   \e: <Escape>
   \n: <Newline>
   \t: <Tab>
@eprog
For any other character $x$, \b{$x$} is expanded to $x$. In particular, the
only way to put a \kbd{"} into a string is to escape it. Thus, for
instance, \kbd{"\bs"a\bs""} would produce the string whose content is
``a''. This is definitely \emph{not} the same thing as typing \kbd{"a"},
whose content is merely the one-letter string a.

You can concatenate two strings using the \tet{concat} function. If either
argument is a string, the other is automatically converted to a string if
necessary (it will be evaluated first).

\bprog
? concat("ex", 1+1)
%1 = "ex2"
? a = 2; b = "ex"; concat(b, a)
%2 = "ex2"
? concat(a, b)
%3 = "2ex"
@eprog

Some functions expect strings for some of their arguments: \tet{print} would
be an obvious example, \tet{Str} is a less obvious but useful one (see the
end of this section for a complete list). While typing in such an argument,
you will be said to be in \tev{string context}. The rest of this section is
devoted to special syntactical tricks which can be used with such arguments
(and only here; you will get an error message if you try these outside of
string context):

$\bullet$ Writing two strings alongside one another will just concatenate
them, producing a longer string. Thus it is equivalent to type in
\kbd{"a " "b"} or \kbd{"a b"}. A little tricky point in the first expression:
the first whitespace is enclosed between quotes, and so is part of a string;
while the second (before the \kbd{"b"}) is completely optional and \kbd{gp}
actually suppresses it, as it would with any number of whitespace characters
at this point (i.e.~outside of any string).

$\bullet$ If you insert any expression when a string is expected, it gets
``expanded'': it is evaluated as a standard GP expression, and the final
result (as would have been printed if you had typed it by itself) is then
converted to a string, as if you had typed it directly. For instance \kbd{"a"
1+1 "b"} is equivalent to \kbd{"a2b"}: three strings get created, the middle
one being the expansion of \kbd{1+1}, and these are then concatenated
according to the rule described above. Another tricky point here: assume you
did not assign a value to \kbd{aaa} in a GP expression before. Then typing
\kbd{aaa} by itself in a string context will actually produce the correct
output (i.e.~the string whose content is aaa), but in a fortuitous way. This
\kbd{aaa} gets expanded to the monomial of degree one in the variable
\kbd{aaa}, which is of course printed as \kbd{aaa}, and thus will expand to
the three letters you were expecting.

\misctitle{Warning:} expression involving strings are not handled in a
special way; even in string context, the largest possible expression is
evaluated, hence \kbd{print("a"[1])} is incorrect since \kbd{"a"} is not an
object whose first component can be extracted. On the other hand
\kbd{print("a", [1])} is correct (two distinct argument, each converted to a
string), and so is \kbd{print("a" 1)} (since \kbd{"a"1} is not a valid
expression, only \kbd{"a"} gets expanded, then \kbd{1}, and the result is
concatenated as explained above). In case of doubt, you can surround part of
your text by parenthesis to force immediate interpretation of a
subexpression: \kbd{print("a"([1]))} is another solution.

\subsec{Keywords}. Since there are cases where expansion is not desirable, we
now distinguish between ``Keywords'' and ``Strings''. String is what has been
described so far. Keywords are special relatives of Strings which are
automatically assumed to be quoted, whether you actually type in the quotes
or not. Thus expansion is never performed on them. They get concatenated,
though. The analyzer supplies automatically the quotes you have ``forgotten''
and treats Keywords just as normal strings otherwise. For instance, if you
type \kbd{"a"b+b} in Keyword context, you will get the string whose contents
are ab+b. In String context, on the other hand, you would get a2\kbd{*}b.

All GP functions have prototypes (described in Chapter~3 below) which
specify the types of arguments they expect: either generic PARI objects
(GEN), or strings, or keywords, or unevaluated expression sequences. In the
keyword case, only a very small set of words will actually be meaningful
(the \kbd{default} function is a prominent example).

\misctitle{Reference:} The arguments of the following functions are processed
in string context:

\settabs\+\indent&\cr
\+&\tet{Str}\cr
\+&\tet{addhelp} (second argument)\cr
\+&\tet{default} (second argument)\cr
\+&\tet{error}\cr
\+&\tet{extern}\cr
\+&\tet{plotstring} (second argument)\cr
\+&\tet{plotterm} (first argument)\cr
\+&\tet{read} and \tet{readvec}\cr
\+&\tet{system}\cr
\+&all the \tet{print}\var{xxx} functions\cr
\+&all the \tet{write}\var{xxx} functions\cr

\noindent The arguments of the following functions are processed as keywords:

\+&\tet{alias}\cr
\+&\tet{default} (first argument)\cr
\+&\tet{install} (all arguments but the last)\cr
\+&\tet{trap} (first argument)\cr
\+&\tet{whatnow}\cr

\subsec{Useful examples} The function \kbd{Str} converts its arguments into
strings and concatenate them. Coupled with \tet{eval}, it is very powerful.
The following example creates generic matrices\sidx{generic
matrix}\sidx{matrix}:
\bprog
? genmat(u,v,s="x") = matrix(u,v,i,j, eval( Str(s,i,j) ))
? genmat(2,3) + genmat(2,3,"m")
%1 =
[x11 + m11 x12 + m12 x13 + m13]
[x21 + m21 x22 + m22 x23 + m23]
@eprog\noindent

Two last examples: \kbd{hist(10,20)} returns all \idx{history} entries from
\kbd{\%10} to \kbd{\%20} neatly packed into a single
vector; \kbd{histlast(10)} returns the last $10$ history entries:
\bprog
  hist(a,b) = vector(b-a+1, i, eval(Str("%", a-1+i)))
  histlast(n) = vector(n, i, eval(Str("%", %#-i+1)))
@eprog

\section{Errors and error recovery}

\subsec{Errors}. Your input program is first compiled to a more efficient
bytecode; then the latter is evaluated, calling appropriate functions from
the PARI library. Accordingly, there are two kind of errors: syntax errors
occuring during the compilation phase, and runtime errors produced by
functions in the PARI library. Both kinds are fatal to your computation:
\kbd{gp} will report the error, perform some cleanup (restore variables
modified while evaluating the erroneous command, close open files, reclaim
unused memory, etc.), and output its usual prompt.

When reporting a \emph{syntax error}, \kbd{gp} gives meaningful
context by copying (part of) the expression it was trying to compile,
indicating where the error with a little caret, as in
\bprog
? factor()
  ***   too few arguments: factor()
  ***                             ^-
? 1+
  ***   syntax error, unexpected $end: 1+
  ***                                   ^-
@eprog\noindent
possibly enlarged to a full arrow given enough trailing context
\bprog
? if (isprime(1+, do_something())
  ***   syntax error, unexpected ',': if(isprime(1+,do_something()))
  ***                                              ^----------------
@eprog\noindent
These error messages may be mysterious, because \kbd{gp} cannot guess what
you were trying to do, and the error usually occurs once \kbd{gp} has been
sidetracked. The first error is straightforward: \kbd{factor} has one
mandatory argument, which is missing there.

The next two are simple typos involving an ill-formed addition
\kbd{1 + } missing its second operand. The error messages differ
because the parsing context is slightly different: in the first
case we reach the end of input (\kbd{\$end}) while still expecting a token,
and in the second one, we received an unexpected token (the comma).

Here is a more complicated one:
\bprog
? factor(x
  ***   syntax error, unexpected $end, expecting )-> or ',' or ')': factor(x
  ***                                                                      ^-
@eprog\noindent
The error is a missing parenthesis, but from \kbd{gp}'s point of view,
you might as well have intended to give further arguments to \kbd{factor}
(this is possible and useful, see the description of the function). In fact
\kbd{gp} expected either a closing parenthesis, or a second argument
separated from the first by a comma. And this is essentially what the error
message says: we reached the end  of the input (\kbd{\$end}) while expecting
a \kbd{')'} or a \kbd{','}.

Actually, a third possibility is mentionned in the error message \kbd{)->},
which could never be valid in the above context, but a subexpression like
\kbd{(x)->sin(x)}, defining an inline closure would be valid, and the parser is
not clever enough to rule that out, so we get the same message as in
\bprog
? (x
  ***   syntax error, unexpected $end, expecting )-> or ',' or ')': (x
  ***                                                                ^-
@eprog\noindent
where all three proposed continuations would be valid.

\emph{Runtime errors} from the evaluator are nicer because they answer a
correctly worded query, otherwise the bytecode compiler would have protested
first; here is a slightly pathological case:
\bprog
? if (siN(x) < eps, do_something())
  ***   at top-level: if(siN(x)<eps,do_someth
  ***                    ^--------------------
  ***   not a function: `siN'.
@eprog\noindent (no arrow!) The code is syntactically correct and compiled
correctly, even though the \kbd{siN} function, a typo for \kbd{sin}, was not
defined at this point. When trying to evaluate the bytecode, however, it
turned out that \kbd{siN} is still undefined so we cannot evaluate the
function call \kbd{siN(x)}.

\emph{Library runtime errors} are even nicer because they have more
mathematical content, which is easier to grasp than a parser's logic:
\bprog
? 1/0
  ***   at top-level: 1/0
  ***                  ^--
  *** _/_: division by zero
@eprog\noindent telling us that a runtime error occured while evaluating the
binary \kbd{/} operator (the \kbd{\_} surrounding the operator are
placeholders). More context is provided if the error occurs deep in the
call chain:
\bprog
? f(x) = 1/x;
? g(N) = for(i = -N, N, f(i));
? g(10)
  ***   at top-level: g(10)
  ***                 ^-----
  ***   in function g: for(i=-N,N,f(i))
  ***                             ^-----
  ***   in function f: 1/x
  ***                   ^--
  *** _/_: division by zero
@eprog

\subsec{Error recovery}.\sidx{error recovery}\label{se:errorrec}

It is quite annoying to wait for some program to finish and find out the hard
way that there was a mistake in it (like the division by 0 above), sending
you back to the prompt. First you may lose some valuable intermediate data.
Also, correcting the error may not be obvious; you might have to change your
program, adding a number of extra statements and tests to try and narrow down
the problem.

A slightly different situation, still related to error recovery, is when you
you actually foresee that some error may occur, are unable to prevent it, but
quite capable of recovering from it, given the chance. Examples include lazy
factorization (cf.~\tet{addprimes}), where you knowingly use a pseudo prime
$N$ as if it were prime; you may then encounter an ``impossible'' situation,
but this would usually exhibit a factor of $N$, enabling you to refine the
factorization and go on. Or you might run an expensive computation at low
precision to guess the size of the output, hence the right precision to use.
You can then encounter errors like ``precision loss in truncation'', e.g when
trying to convert \kbd{1E1000}, known to $28$ digits of accuracy, to an
integer; or ``division by 0'', e.g inverting \kbd{0E1000} when all accuracy
has been lost, and no significant digit remains. It would be enough to
restart part of the computation at a slightly higher precision.

We now describe \tev{error trapping}, a useful mechanism which alleviates
much of the pain in the first situation, and provides a satisfactory way out
of the second one. Everything is handled via the \tet{trap} function whose
different modes we now describe.

\subsec{Break loop}.\label{se:breakloop}

A \tev{break loop} is a special debugging mode that you enter whenever a
user interrupt (\kbd{Control-C}) or runtime error occurs, freezing the
\kbd{gp} state, and preventing cleanup until you get out of the loop. By
runtime error, we mean an error from the library or a user error (from
\tet{error}), but \emph{not} syntax errors. When a break loop starts, a
prompt is issued (\kbd{break>}). You can type in a \kbd{gp} command, which is
evaluated when you hit the \kbd{<Return>} key, and the result is printed as
during the main \kbd{gp} loop, except that no history of results is kept.
Then the break loop prompt reappears and you can type further commands as
long as you do not exit the loop. If you are using readline, the history of
commands is kept, and line editing is available as usual. If you type in a
command that results in an error, you are sent back to the break loop prompt
(errors do \var{not} terminate the loop).

To get out of a break loop, you can type 
\kbd{C-d} (\kbd{EOF}) or enter no input \emph{three times} consecutively
(hit \kbd{Return} thrice), any of which will let \kbd{gp}
perform its usual cleanup, and send you back to the \kbd{gp} prompt. If the
break loop was started by a user interrupt (and not by an error),
inputing an empty line, i.e hitting the \kbd{<Return>} key at the
\kbd{break>} prompt, resumes the temporarily interrupted computation.
A single empty line has no effect in case of a fatal error, to avoid
getting get out of the loop prematurely, thereby losing valuable debugging
data.

Break loops are useful as a debugging tool. You may inspect the values of
\kbd{gp} variables to understand why an error occurred, or change
\kbd{gp}'s state in the middle of a computation (increase debugging level,
start storing results in a logfile, set variables to different values\dots):
hit \kbd{C-c}, type in your modifications, then let the computation go on as
explained above. A break loop looks like this:
\bprog
? v = 0; 1/v
  ***   at top-level: v=0;1/v
  ***                      ^--
  *** _/_: division by zero
  ***   Break loop (type 'break' or Control-d to go back to GP)
break>
@eprog
\noindent So the standard error message is printed first. The
\kbd{break>} at the bottom is a prompt, and hitting \kbd{v} then
\kbd{<Return>}, we see:
\bprog
break> v
0
@eprog\noindent explaining the problem. We could have typed any \kbd{gp}
command, not only the name of a variable, of course. There is no special set
of commands becoming available during a break loop, as they would in most
debuggers.

Even lexically-scoped variables are accessible to the evaluator during the
break loop:
\bprog
? for(v = -2, 2, print(1/v))
-1/2
-1
  ***   at top-level: for(v=-2,2,print(1/v))
  ***                                   ^----
  *** _/_: division by zero
  ***   Break loop (type 'break' or Control-d to go back to GP)
break> v
0
@eprog\noindent
Even though loop indices are automatically lexically scoped and no longer
exist when the break loop is run, enough debugging information is retained in
the bytecode to reconstruct the evaluation context.

\misctitle{Note:} if you do not like this behaviour, you may disable it by
setting the default \tet{breakloop} to 0 in for \kbd{gprc}. A runtime error
will send you back to the prompt. Note that the break loop is automatically
disabled when running \kbd{gp} in non interactive mode, i.e.~when the
program's standard input is not attached to a terminal.

\misctitle{Technical Note:} When you enter a break loop due to a PARI stack
overflow, the PARI stack is reset so that you can run commands (otherwise the
stack would immediately overflow again). Still, as explained above, you do
not lose the value of any \kbd{gp} variable in the process.

\subsec{Protecting code}. Finally \kbd{trap} can define a temporary handler
used within the scope of a code fragment, protecting it from errors, by
providing replacement code should the trap be activated. The expression

  \kbd{trap}( , \var{recovery}, \var{statements})

\noindent evaluates and returns the value of \var{statements}, unless an
error occurs during the evaluation in which case the value of \var{recovery}
is returned. As in an if/else clause, with the difference that
\var{statements} has been partially evaluated, with possible side effects.
For instance one could define a fault tolerant inversion function as follows:
\bprog
? inv(x) = trap (, "oo", 1/x)
? for (i=-1,1, print(inv(i)))
-1
oo
1
@eprog\noindent Protected codes can be nested without adverse effect, the
last trap seen being the first to spring.

\subsec{Trapping specific exceptions}. We have not yet seen the use of the
first argument of trap, which has been omitted in all previous examples. It
simply indicates that only errors of a specific type should be intercepted,
see the documentation of \tet{trap} for the complete list. For instance, we
have

\kbd{gdiver}: division by 0

\kbd{invmoder}: impossible inverse modulo

\kbd{archer}: not available on this architecture or operating system

\kbd{typeer}: wrong type

\kbd{errpile}: the PARI stack overflows

\noindent Omitting the error name means we are trapping all errors. For
instance, the following can be used to check in a safe way whether
\tet{install} works correctly in your \kbd{gp}:
\bprog
broken_install() =
{
  trap(archer, return ("OS"),
    install(addii,GG)
  );
  trap(, "USE",
    if (addii(1,1) != 2, "BROKEN")
  )
}
@eprog
\noindent The function returns 0 if everything works (the omitted \var{else}
clause of the \kbd{if}), \kbd{OS} if the operating system does not support
\kbd{install}, \kbd{USE} if using an installed function triggers an error,
and \kbd{BROKEN} if the installed function did not behave as expected.

\section{Interfacing GP with other languages}
\noindent
The PARI library was meant to be interfaced with C programs. This specific
use is dealt with extensively in the \emph{User's guide to the PARI library}.
Of course, \kbd{gp} itself provides a convenient  interpreter to execute
rather intricate scripts (see \secref{se:programming}).

Scripts, when properly written, tend to be shorter and clearer than C
programs, and are certainly easier to write, maintain or debug. You don't
need to deal with memory management, garbage collection, pointers,
declarations, and so on. Because of their intrinsic simplicity, they are more
robust as well. They are unfortunately somewhat slower. Thus their use will
remain complementary: it is suggested that you test and debug your algorithms
using scripts, before actually coding them in C if speed is paramount.
The GP2C compiler often eases this part.

The \kbd{install} command (see~\secref{se:install}) efficiently imports
foreign functions for use under \kbd{gp}, which can of course be written
using other libraries than PARI. Thus you may code only critical parts
of your program in C, and still maintain most of the program as a GP script.

We are aware of four PARI-related Free Software packages to embed PARI in
other languages. We \emph{neither endorse nor support} any of them, but you
may want to give them a try if you are familiar with the languages they are
based on. The first is the \tet{Math::Pari} Perl module (see any CPAN
mirror), written by Ilya Zakharevich. The second is \tet{PariPython}%
\footnote{*}{see \kbd{http://www.fermigier.com/fermigier/PariPython/}},
%
by St\'efane Fermigier, which is no longer maintained. Starting from
Fermigier's work, William Stein has embedded PARI into his Python-based
SAGE\footnote{**}{see \kbd{http://modular.fas.harvard.edu/sage/}}
system. Finally, Michael Stoll has integrated PARI into \tet{CLISP}
\footnote{***}{see \kbd{http://clisp.cons.org}}, which is a Common Lisp
implementation by Bruno Haible, Marcus Daniels and others; this interface has
been updated for pari-2 by Sam Steingold.

These provide interfaces to \kbd{gp} functions for use in \kbd{perl},
\kbd{python} or \kbd{Lisp}\sidx{Perl}\sidx{Python}\sidx{Lisp} programs,
respectively.

\section{Defaults}\sidx{defaults}
\label{se:defaults}

\noindent There are many internal variables in \kbd{gp}, defining how the
system will behave in certain situations, unless a specific override has been
given. Most of them are a matter of basic customization (colors, prompt) and
will be set once and for all in your \idx{preferences file} (see
\secref{se:gprc}), but some of them are useful interactively (set timer on,
increase precision, etc.).

The function used to manipulate these values is called \kbd{default}, which
is described in \secref{se:default}. The basic syntax is

\kbd{default(\var{def}, \var{value})},

\noindent which sets the default \var{def} to \var{value}. In interactive
use, most of these can be abbreviated using \kbd{gp} metacommands
(mostly, starting with \b), which we shall describe in the next section.

Here we will only describe the available defaults and how they are used. Just
be aware that typing \kbd{default} by itself will list all of them, as well
as their current values (see \b{d}). Just after the default name, we give
between parentheses the initial value when \kbd{gp} starts, assuming you did
not tamper with factory settings using command-line switches or a~\tet{gprc}.

\misctitle{Note:} the suffixes \kbd{k}, \kbd{M} or \kbd{G} can be appended to
a \var{value} which is a numeric argument, with the effect of multiplying it
by $10^3$, $10^6$ and $10^9$ respectively. Case is not taken into account
there, so for instance \kbd{30k} and \kbd{30K} both stand for $30000$. This
is mostly useful to modify or set the defaults \kbd{primelimit} or
\kbd{stacksize} which typically involve a lot of trailing zeroes.

\misctitle{(somewhat technical) Note:} As we saw in \secref{se:strings},
the second argument to \kbd{default} is subject to string context
expansion, which means you can use run-time values. In other words, something
like
\bprog
  a = 3;
  default(logfile, "file" a ".log")
@eprog
logs the output in \kbd{file3.log}.

Some special defaults, corresponding to filenames and prompts, expand further
the resulting value at the time they are set. Two kinds of expansions may be
performed:

$\bullet$ \teb{time expansion}: the string is sent through the library
function \tet{strftime}. This means that \kbd{\%}\var{char} combinations have
a special meaning, usually related to the time and date. For instance,
\kbd{\%H} = hour (24-hour clock) and \kbd{\%M} = minute [00,59] (on a Unix
system, you can try \kbd{man strftime} at your shell prompt to get a complete
list). This is applied to \kbd{prompt}, \kbd{psfile}, and \kbd{logfile}. For
instance,

\kbd{default(prompt,"(\%H:\%M) ? ")}

\noindent
will prepend the time of day, in the form \kbd{(\var{hh}:\var{mm})}
to \kbd{gp}'s usual prompt.

$\bullet$ \teb{environment expansion}: When the string contains a sequence of
the form \kbd{\$\var{SOMEVAR}}, e.g.~\kbd{\$HOME}, the environment is
searched and if \var{SOMEVAR} is defined, the sequence is replaced by the
corresponding value. Also the \kbd{\til} symbol has the same meaning as in
many shells~--- \kbd{\til} by itself stands for your home directory, and
\kbd{\til{}user} is expanded to \kbd{user}'s home directory. This is applied
to all filenames\sidx{filename}. \label{se:envir}

We shall now describe all the available defaults, specifying each time
whether time and/or environment expansion is performed.

\subsecidx{breakloop} (default \kbd{1} if interactive, \kbd{0} otherwise):
if true, enables the ``break loop'' debugging mode, see \secref{se:breakloop}.

\subsecidx{colors} (default \kbd{""}): this default is only usable if
\kbd{gp}\label{se:colors}
is running within certain color-capable terminals. For instance \kbd{rxvt},
\kbd{color\_xterm} and modern versions of \kbd{xterm} under X Windows, or
standard Linux/DOS text consoles. It causes \kbd{gp} to use a small palette of
colors for its output. With xterms, the colormap used corresponds to the
resources \kbd{Xterm*color$n$} where $n$ ranges from $0$ to $15$ (see the
file \kbd{misc/color.dft} for an example). Accepted values for this
default are strings \kbd{"$a_1$,\dots,$a_k$"} where $k\le7$ and each
$a_i$ is either

\noindent $\bullet$ the keyword \kbd{no} (use the default color, usually
black on transparent background)

\noindent $\bullet$ an integer between 0 and 15 corresponding to the
aforementioned colormap

\noindent $\bullet$ a triple $[c_0,c_1,c_2]$ where $c_0$ stands for foreground
color, $c_1$ for background color, and $c_2$ for attributes (0 is default, 1
is bold, 4 is underline).

The output objects thus affected are respectively error messages,
history numbers, prompt, input line, output, help messages, timer (that's
seven of them). If $k < 7$, the remaining $a_i$ are assumed to be $no$. For
instance
%
\bprog
default(colors, "9, 5, no, no, 4")
@eprog
\noindent
typesets error messages in color $9$, history numbers in color $5$, output in
color $4$, and does not affect the rest.

A set of default colors for dark (reverse video or PC console) and light
backgrounds respectively is activated when \kbd{colors} is set to
\kbd{darkbg}, resp.~\kbd{lightbg} (or any proper prefix: \kbd{d} is
recognized as an abbreviation for \kbd{darkbg}). A bold variant of
\kbd{darkbg}, called \kbd{boldfg}, is provided if you find the former too
pale.

\emacs In the present version, this default is incompatible with PariEmacs.
Changing it will just fail silently (the alternative would be to display
escape sequences as is, since Emacs will refuse to interpret them).
You must customize color highlighting from the PariEmacs side, see its
documentation.

\misctitle{Technical note:}
If you use an old \kbd{readline} library (version number less than 2.0),
you should do as in the example above and leave $a_3$ and $a_4$ (prompt
and input line) strictly alone. Since old versions of \kbd{readline} did
not handle escape characters correctly (or more accurately, treated them
in the only sensible way since they did not care to check all your terminal
capabilities: it just ignored them), changing them would result in many
annoying display bugs.

The specific thing to look for is to check the \kbd{readline.h} include
file, wherever your readline include files are, for the string
\kbd{RL\_PROMPT\_START\_IGNORE}. If it is there, you are safe. Another
sensible way is to make some experiments, and get a more recent
\kbd{readline} if yours doesn't work the way you would like it to. See
the file \kbd{misc/gprc.dft} for some examples.

\subsecidx{compatible} (default \kbd{0}): The GP function names and syntax
have changed tremendously between versions 1.xx and 2.00. To help you cope
with this we provide some kind of backward compatibility, depending on the
value of this default:

\quad \kbd{compatible} = 0: no backward compatibility. In this mode, a very
handy function, to be described in \secref{se:whatnow}, is \kbd{whatnow},
which tells you what has become of your favourite functions, which \kbd{gp}
suddenly can't seem to remember.

\quad \kbd{compatible} = 1: warn when using obsolete functions, but
otherwise accept them. The output uses the new conventions though, and
there may be subtle incompatibilities between the behaviour of former and
current functions, even when they share the same name (the current function
is used in such cases, of course!). We thought of this one as a transitory
help for \kbd{gp} old-timers. Thus, to encourage switching to \kbd{compatible}=0,
it is not possible to disable the warning.

\quad \kbd{compatible} = 2: use only the old function naming scheme (as
used up to version 1.39.15), but \emph{taking case into account}. Thus
\kbd{I} (${}=\sqrt{-1}$) is not the same as \kbd{i} (user variable, unbound
by default), and you won't get an error message using \kbd{i} as a loop
index as used to be the case.

\quad \kbd{compatible} = 3: try to mimic exactly the former behaviour. This
is not always possible when functions have changed in a fundamental way.
But these differences are usually for the better (they were meant to,
anyway), and will probably not be discovered by the casual user.

One adverse side effect is that any user functions and aliases that have
been defined \emph{before} changing \kbd{compatible} will get erased if this
change modifies the function list, i.e.~if you move between groups
$\{0,1\}$ and $\{2,3\}$ (variables are unaffected). We of course strongly
encourage you to try and get used to the setting \kbd{compatible}=0.

Note that the default \tet{new_galois_format} is another compatibility setting,
which is completely independent of \kbd{compatible}.

\subsecidx{datadir} (default: the location of installed precomputed data):
the name of directory containing the optional data files. For now,
only the \kbd{galdata} and \kbd{elldata} packages.

\subsecidx{debug} (default \kbd{0}): debugging level. If it is non-zero,
some extra messages may be printed (some of it in French), according to
what is going on (see~\b{g}).

\subsecidx{debugfiles} (default \kbd{0}): file usage debugging level. If it
is non-zero, \kbd{gp} will print information on file descriptors in use, from
PARI's point of view (see~\b{gf}).

\subsecidx{debugmem} (default \kbd{0}): memory debugging level. If it is
non-zero, \kbd{gp} will regularly print information on memory usage. If it's
greater than 2, it will indicate any important garbage collecting and the
function it is taking place in (see~\b{gm}).

\noindent {\bf Important Note:} As it noticeably slows down the performance,
the first functionality (memory usage) is disabled if you're not running a
version compiled for debugging (see Appendix~A).

\subsecidx{echo} (default \kbd{0}): this is a toggle, which can be either 1
(on) or 0 (off). When \kbd{echo} mode is on, each command is reprinted before
being executed. This can be useful when reading a file with the \b{r} or
\kbd{read} commands. For example, it is turned on at the beginning of the
test files used to check whether \kbd{gp} has been built correctly (see
\b{e}).

\subsec{factor\_add\_primes} (default \kbd{0}):\kbdsidx{factor_add_primes} if
set, the integer factorization machinery calls \tet{addprimes} on primes
factor that were difficult to find (larger than $2^24$), so they are
automatically tried first in other factorizations. If a routine is performing
(or has performed) a factorization and is interrupted by an error or via
Control-C, this lets you recover the prime factors already found. The
downside is that a huge \kbd{addprimes} table unrelated to the current
computations will slow down arithmetic functions relying on integer
factorization; one should then empty the table using \tet{removeprimes}.

\subsec{factor\_proven} (default \kbd{0}):\kbdsidx{factor_proven} by
default, the factors output by the integer factorization machinery are
only pseudo-primes, not proven primes. If this is
set, a primality proof is done for each factor and all results depending on
integer factorization are fully proven. This flag does not affect partial
factorization when it is explicitely requested. Note that if affects all
other factorizations, and will in general lead to significant slowdowns.

\subsecidx{format} (default \kbd{"g.28"} and \kbd{"g.38"} on 32-bit and
64-bit machines, respectively): of the form x$.n$, where x (conversion style)
is a letter in $\{\kbd{e},\kbd{f},\kbd{g}\}$, and $n$ (precision) is an
integer; this affects the way real numbers are printed:
\label{se:format}

\item If the conversion style is \kbd{e}, real numbers are printed in
\idx{scientific format}, always with an explicit exponent,
e.g.~\kbd{3.3 E-5}.

\item In style \kbd{f}, real numbers are generally printed in \idx{fixed
floating point format} without exponent, e.g.~\kbd{0.000033}. A large
real number, whose integer part is not well defined (not enough significant
digits), is printed in style~\kbd{e}. For instance \kbd{10.\pow 100} known to
ten significant digits is always printed in style \kbd{e}.

\item In style \kbd{g}, non-zero real numbers are printed in \kbd{f} format,
except when their decimal exponent is $< -4$, in which case they are printed in
\kbd{e} format. Real zeroes (of arbitrary exponent) are printed in \kbd{e}
format.

The precision $n$ is the number of significant digits printed for real
numbers, except if $n<0$ where all the significant digits will be printed
(initial default 28, or 38 for 64-bit machines). For more powerful formatting
possibilities, see \tet{printf} and \tet{Strprintf}.

\subsecidx{graphcolormap} (default value: \kbd{["white", "black", "blue",
"violetred", "red", "green", "grey", "gainsboro"]}) A vector of colors, to be
used by hi-res graphing routines. Its length is arbitrary, but it must
contain at least 3 entries: the first 3 colors are used for background,
frame/ticks and axes respectively. All colors in the colormap may be freely
used in \tet{plotcolor} calls.

A color is either given as in the default by character strings or by an RGB
code. For valid character strings, see the standard \kbd{rgb.txt} file in X11
distributions, where we restrict to lowercase letters and remove all
whitespace from color names. An RGB code is a vector with 3 integer entries
between 0 and 255. For instance \kbd{[250, 235, 215]} and \kbd{"antique
white"} represent the same color. RGB codes are a little cryptic but often
easier to generate.

\subsecidx{graphcolors} (default: \kbd{[4,5]}) Entries in the
\tet{graphcolormap} that will be used to plot multi-curves. The successive
curves are drawn in colors

  \kbd{graphcolormap[graphcolors[1]]}, \kbd{graphcolormap[graphcolors[2]]},
  \dots

cycling when the \kbd{graphcolors} list is exhausted.

\subsecidx{help} (default: the location of the \kbd{gphelp} script): the
name of the external help program which will be used from within \kbd{gp} when
extended help is invoked, usually through a \kbd{??} or \kbd{???} request
(see \secref{se:exthelp}), or \kbd{M-H} under readline (see
\secref{se:readline}).

\subsecidx{histsize} (default \kbd{5000}): \kbd{gp} keeps a history of the last
\kbd{histsize} results computed so far, which you can recover using the
\kbd{\%} notation (see \secref{se:history}). When this number is exceeded,
the oldest values are erased. Tampering with this default is the only way to
get rid of the ones you do not need anymore.

\subsecidx{lines} (default \kbd{0}): if set to a positive value, \kbd{gp} prints at
most that many lines from each result, terminating the last line shown with
\kbd{[+++]} if further material has been suppressed. The various \kbd{print}
commands (see \secref{se:gp_program}) are unaffected, so you can always type
\kbd{print(\%)} or \b{a} to view the full result. If the actual
screen width cannot be determined, a ``line'' is assumed to be 80 characters
long.

\subsecidx{log} (default \kbd{0}): this can be either 0 (off) or 1, 2, 3
(on, see below for the various modes). When logging mode is turned on,
\kbd{gp}
opens a log file, whose exact name is determined by the \kbd{logfile}
default. Subsequently, all the commands and results will be written to that
file (see \b{l}). In case a file with this precise name already existed, it
will not be erased: your data will be \emph{appended} at the end.

The specific positive values of \kbd{log} have the following meaning

1: plain logfile

2: emit color codes to the logfile (if \kbd{colors} is set).

3: write LaTeX output to the logfile (can be further customized using
\tet{TeXstyle}).

\subsecidx{logfile} (default \kbd{"pari.log"}): name of the log file to be
used when the \kbd{log} toggle is on. Environment and time expansion are
performed.

\subsec{new\_galois\_format}\kbdsidx{new_galois_format} (default \kbd{0}): if
this is set, the \tet{polgalois} command will use a different, more
consistent, naming scheme for Galois groups. This default is provided to
ensure that scripts can control this behaviour and do not break unexpectedly.
Note that the default value of $0$ (unset) will change to $1$ (set) in the
next major version.

\subsecidx{output} (default \kbd{1}): there are four possible values: 0
(=~\var{raw}), 1 (=~\var{prettymatrix}), or 3
(=~\var{external prettyprint}). This
means that, independently of the default \kbd{format} for reals which we
explained above, you can print results in four ways: either in \tev{raw
format}, i.e.~a format which is equivalent to what you input, including
explicit multiplication signs, and everything typed on a line instead of
two dimensional boxes. This can have several advantages, for instance it
allows you to pick the result with a mouse or an editor, and to paste it
somewhere else.\label{se:output}

The second format is the \tev{prettymatrix format}. The only difference to
raw format is that matrices are printed as boxes instead of horizontally.
This is prettier, but takes more space and cannot be used for input. Column
vectors are still printed horizontally.

The third format is \tev{external prettyprint}, which pipes
all \kbd{gp} output in TeX format to an external prettyprinter, according to the
value of \tet{prettyprinter}. The default script (\tet{tex2mail}) converts
its input to readable two-dimensional text.

Independently of the setting of this default, an object can be printed
in any of the two formats at any time using the commands \b{a} and \b{m}
respectively (see below).

\subsecidx{parisize}\label{se:parisize} (default 4M, resp.~8M on a 32-bit,
resp.~64-bit machine): \kbd{gp}, and in fact any program using the PARI
library, needs a \tev{stack} in which to do its computations. \kbd{parisize}
is the stack size, in bytes. It is strongly recommended you increase this
default (using the \kbd{-s} command-line switch, or a \tet{gprc}) if you can
afford it. Don't increase it beyond the actual amount of RAM installed on
your computer or \kbd{gp} will spend most of its time paging.

In case of emergency, you can use the \tet{allocatemem} function to
increase \kbd{parisize}, once the session is started.

\subsecidx{path} (default \kbd{".:\til:\til/gp"} on UNIX systems,
\kbd{".;C:\bs;C:\bs GP"} on DOS, OS/2 and Windows, and \kbd{"."} otherwise):
This is a list of directories, separated by colons ':' (semicolons ';' in the
DOS world, since colons are preempted for drive names). When asked to read a
file whose name is not given by an absolute path (does not start with
\kbd{/}, \kbd{./} or \kbd{../}), \kbd{gp} will look for it in these
directories, in the order they were written in \kbd{path}. Here, as usual,
\kbd{.} means the current directory, and \kbd{..} its immediate parent.
Environment expansion is performed.

\subsecidx{prettyprinter} (default \kbd{"tex2mail -TeX -noindent
-ragged -by\_par"}) the name of an external prettyprinter to use when
\kbd{output} is~3 (\var{alternate prettyprinter}). Note that the default
\tet{tex2mail} looks much nicer than the built-in ``beautified
format'' ($\kbd{output} = 2$).

\subsecidx{primelimit} (default \kbd{500k}): \kbd{gp} precomputes a list of
all primes less than \kbd{primelimit} at initialization time. These are used
by many arithmetic functions, usually for trial division purposes. If you do
not plan to invoke any of them, you can just set this to 1. The maximal value
is a little less than $2^{32}$ (resp $2^{64}$) on a 32-bit (resp.~64-bit)
machine.

Since almost all arithmetic functions eventually require some table of prime
numbers, PARI currently guarantees that the first 6547 primes, up to and
including 65557, are precomputed, even if \kbd{primelimit} is $1$.

\subsecidx{prompt} (default \kbd{"? "}): a string that will be printed as
prompt. Note that most usual escape sequences are available there: \b{e} for
Esc, \b{n} for Newline, \dots, \kbd{\bs\bs} for \kbd{\bs}. Time expansion is
performed.

This string is sent through the library function \tet{strftime} (on a
Unix system, you can try \kbd{man strftime} at your shell prompt). This means
that \kbd{\%} constructs have a special meaning, usually related to the time
and date. For instance, \kbd{\%H} = hour (24-hour clock) and \kbd{\%M} =
minute [00,59] (use \kbd{\%\%} to get a real \kbd{\%}).

If you use \kbd{readline}, escape sequences in your prompt will result in
display bugs. If you have a relatively recent \kbd{readline} (see the comment
at the end of \secref{se:colors}), you can brace them with special sequences
(\kbd{\bs[} and \kbd{\bs]}), and you will be safe. If these just result in
extra spaces in your prompt, then you'll have to get a more recent
\kbd{readline}. See the file \kbd{misc/gprc.dft} for an example.

\emacs {\bf Caution}: PariEmacs needs to know about the prompt pattern to
separate your input from previous \kbd{gp} results, without ambiguity. It is
not a trivial problem to adapt automatically this regular expression to an
arbitrary prompt (which can be self-modifying!). See PariEmacs's
documentation.

\subsec{prompt\_cont}\kbdsidx{prompt_cont} (default \kbd{""}): a string that will be printed
to prompt for continuation lines (e.g. in between braces, or after a
line-terminating backslash). Everything that applies to \kbd{prompt}
applies to \kbd{prompt\_cont} as well.

\subsecidx{psfile} (default \kbd{"pari.ps"}): name of the default file where
\kbd{gp} is to dump its PostScript drawings (these are appended, so that no
previous data are lost). Environment and time expansion are performed.

\subsecidx{readline} (default \kbd{1}): switches readline line-editing
facilities on and off. This may be useful if you are running \kbd{gp} in a Sun
\tet{cmdtool}, which interacts badly with readline. Of course, until readline
is switched on again, advanced editing features like automatic completion
and editing history are not available.

% leave the long line for gphelp (expects ':' on the first line)
\subsecidx{realprecision} (default \kbd{28} and \kbd{38} on 32-bit and 64-bit machines respectively): the number of significant digits and, at the same
time, the number of printed digits of real numbers (see~\b{p}). Note that
PARI internal precision works on a word basis (32 or 64 bits), hence may not
coincide with the number of decimal digits you input. For instance to get 2
decimal digits you need one word of precision which, on a 32-bit machine,
actually gives you 9 digits ($9 < \log_{10}(2^{32}) < 10$):

\bprog
? default(realprecision, 2)
      realprecision = 9 significant digits (2 digits displayed)
@eprog

\subsecidx{recover} (default \kbd{1}): this is a toggle, which can be either 1
(on) or 0 (off). If you change this to $0$, any error becomes fatal and
causes the gp interpreter to exit immediately. Can be useful in batch job
scripts.

\subsecidx{secure} (default \kbd{0}): this is a toggle which can be either 1
(on) or 0 (off). If on, the \tet{system} and \tet{extern} command are
disabled. These two commands are potentially dangerous when you execute
foreign scripts since they let \kbd{gp} execute arbitrary UNIX commands.
\kbd{gp} will
ask for confirmation before letting you (or a script) unset this toggle.

\subsecidx{seriesprecision} (default \kbd{16}): number of significant terms
when converting a polynomial or rational function to a power series
(see~\b{ps}).

\subsecidx{simplify} (default \kbd{1}): this is a toggle which can be either
1 (on) or 0 (off). When the PARI library computes something, the type of the
result is not always the simplest possible. The only type conversions which
the PARI library does automatically are rational numbers to integers (when
they are of type \typ{FRAC} and equal to integers), and similarly rational
functions to polynomials (when they are of type \typ{RFRAC} and equal to
polynomials). This feature is useful in many cases, and saves time, but can
be annoying at times. Hence you can disable this and, whenever you feel like
it, use the function \kbd{simplify} (see Chapter 3) which allows you to
simplify objects to the simplest possible types recursively (see~\b{y}).
\sidx{automatic simplification}

\subsecidx{strictmatch} (default \kbd{1}): this is a toggle which can be
either 1 (on) or 0 (off). If on, unused characters after a sequence has been
processed will produce an error. Otherwise just a warning is printed. This
can be useful when you're not sure how many parentheses you have to close after
complicated nested loops.

\subsecidx{TeXstyle} (default \kbd{0}): the bits of this default allow
\kbd{gp} to use less rigid TeX formatting commands in the logfile. This
default is only taken into account when $\kbd{log} = 3$. The bits of
\kbd{TeXstyle} have the following meaning

2: insert \kbd{\bs right} / \kbd{\bs left} pairs where appropriate.

4: insert discretionary breaks in polynomials, to enhance the probability of
a good line break.

\subsecidx{timer} (default \kbd{0}): this is a toggle which can be either 1
(on) or 0 (off). If on, every instruction sequence (anything ended by a
newline in your input) is timed, to some accuracy depending on the hardware
and operating system. The time measured is the user \idx{CPU time},
\emph{not} including the time for printing the results (see \kbd{\#} and
\kbd{\#\#}).

\section{Simple metacommands}\label{se:meta}

\noindent
Simple metacommands are meant as shortcuts and should not be used in GP
scripts (see \secref{se:programming}). Beware that these, as all of \kbd{gp} input,
are \emph{case sensitive}. For example, \b{Q} is not identical to \b{q}. In
the following list, braces are used to denote optional arguments, with their
default values when applicable, e.g.~$\{n=0\}$ means that if $n$ is not
there, it is assumed to be~$0$. Whitespace (or spaces) between the
metacommand and its arguments and within arguments is optional. (This can
cause problems only with \b{w}, when you insist on having a filename whose
first character is a digit, and with \b{r} or \b{w}, if the filename itself
contains a space. In such cases, just use the underlying \tet{read} or
\tet{write} function; see~\secref{se:write}).

\subseckbd{?} $\{\var{command}\}$: \kbd{gp} on-line help interface.
If you type \kbd{?$n$} where $n$ is a number from 1 to 11, you will get the
list of functions in Section $3.n$ of the manual (the list of sections being
obtained by simply typing \kbd{?}). \label{se:exthelp}

These names are in general not informative enough. More details can be
obtained by typing \kbd{?\var{function}}, which gives a short explanation of
the function's calling convention and effects. Of course, to have complete
information, read Chapter 3 of this manual (the source code is at your
disposal as well, though a trifle less readable).

If the line before the copyright message indicates that extended help is
available (this means \kbd{perl} is present on your system and the PARI
distribution was correctly installed), you can add more \kbd{?} signs for
extended functionalities:

\kbd{??~\var{keyword}} yields the functions description as it stands in this
manual, usually in Chapter~2 or~3. If you're not satisfied with the default
chapter chosen, you can impose a given chapter by ending the keyword with
\kbd{@} followed by the chapter number, e.g.~\kbd{??~Hello@2} will look in
Chapter~2 for section heading \kbd{Hello} (which doesn't exist, by the way).

All operators (e.g.~\kbd{+}, \kbd{\&\&}, etc.) are accepted by this
extended help, as well as a few other keywords describing key \kbd{gp} concepts,
e.g.~\kbd{readline} (the line editor), \kbd{integer}, \kbd{nf} (``number
field'' as used in most algebraic number theory computations), \kbd{ell}
(elliptic curves), etc.

In case of conflicts between function and default names (e.g \tet{log},
\tet{simplify}), the function has higher priority. To get the default help,
use
\bprog
  ?? default(log)
  ?? default(simplify)
@eprog

\kbd{???~\var{pattern}} produces a list of sections in Chapter~3 of the
manual related to your query. As before, if \var{pattern} ends by \kbd{@}
followed by a chapter number, that chapter is searched instead; you also
have the option to append a simple \kbd{@} (without a chapter number) to
browse through the whole manual.

If your query contains dangerous characters (e.g \kbd{?} or blanks) it is
advisable to enclose it within double quotes, as for GP strings (e.g
\kbd{???~"elliptic curve"}).

Note that extended help is much more powerful than the short help, since
it knows about operators as well: you can type \kbd{??~*} or
\kbd{??~\&\&}, whereas a single \kbd{?} would just yield a not too helpful
\bprog
&&: unknown identifier.}
@eprog\noindent message. Also, you can ask for extended help on section
number~$n$ in Chapter~3, just by typing \kbd{??~$n$} (where \kbd{?$n$} would
yield merely a list of functions). Finally, a few key concepts in \kbd{gp} are
documented in this way: metacommands (e.g \kbd{??~"??"}), defaults (e.g
\kbd{??~psfile}) and type names (e.g \typ{INT} or \kbd{integer}), as well as
various miscellaneous keywords such as \kbd{edit} (short summary of line
editor commands), \kbd{operator}, \kbd{member}, \kbd{"user defined"},
\kbd{nf}, \kbd{ell}, \dots

Last but not least: \kbd{??} without argument will open a \kbd{dvi}
previewer (\kbd{xdvi} by default, \kbd{\$GPXDVI} if it is defined in your
environment) containing the full user's manual. \kbd{??tutorial} and
\kbd{??refcard} do the same with the \idx{tutorial} and \idx{reference card}
respectively.

\misctitle{Technical note:} these functionalities are provided by an
external \kbd{perl} script that you are free to use outside any \kbd{gp} session
(and modify to your liking, if you are perl-knowledgeable). It is called
\tet{gphelp}, lies in the \kbd{doc} subdirectory of your distribution
(just make sure you run \kbd{Configure} first, see Appendix~A) and is
really two programs in one. The one which is used from within \kbd{gp} is
\kbd{gphelp} which runs \TeX\ on a selected part of this manual, then opens
a previewer. \kbd{gphelp -detex} is a text mode equivalent, which looks
often nicer especially on a colour-capable terminal (see
\kbd{misc/gprc.dft} for examples). The default \kbd{help} selects which
help program will be used from within \kbd{gp}. You are welcome to improve this
help script, or write new ones (and we would like to know about it
so that we may include them in future distributions). By the way, outside
of \kbd{gp} you can give more than one keyword as argument to \kbd{gphelp}.

\subseckbd{/*...*/}: comment. Everything between the stars is ignored by
\kbd{gp}. These comments can span any number of lines.

\subseckbd{\bs\bs}: one-line comment. The rest of the line
is ignored by \kbd{gp}.

\subsec{\b{a}} $\{n\}$: prints the object number $n$ ($\%n$)
in raw format. If the number $n$ is omitted, print the latest computed object
($\%$). \label{se:history}

\subsec{\b{c}}:\sidx{available commands} prints the list of all available
hardcoded functions under \kbd{gp}, not including operators written as special
symbols (see \secref{se:operators}). More information can be obtained using
the \kbd{?} metacommand (see above). For user-defined functions / member
functions, see \b{u} and \b{um}.

\subsec{\b{d}}: prints the \idx{defaults} as described in the
previous section (shortcut for \kbd{default()}, see \secref{se:default}).

\subsec{\b{e}} $\{n\}$: switches the \tet{echo} mode on (1) or off (0). If
$n$ is explicitly given, set echo to $n$.

\subsec{\b{g}} $\{n\}$: sets the debugging level \tet{debug} to the
non-negative integer $n$.

\subsec{\b{gf}} $\{n\}$: sets the file usage debugging level \tet{debugfiles}
to the non-negative integer $n$.

\subsec{\b{gm}} $\{n\}$: sets the memory debugging level \tet{debugmem}
to the non-negative integer $n$.

\subsec{\b{h}} $\{m$\kbd{-}$n\}$: outputs some debugging info about the
hashtable. If the argument is a number $n$, outputs the contents of cell
$n$. Ranges can be given in the form $m$\kbd{-}$n$ (from cell $m$ to cell
$n$, \$ = last cell). If a function name is given instead of a number or
range, outputs info on the internal structure of the hash cell this
function occupies (a \kbd{struct entree} in C). If the range is reduced to
a dash ('\kbd{-}'), outputs statistics about hash cell usage.

\subsec{\b{l}} $\{$\var{logfile}$\}$: switches \tet{log} mode on and off.
If a \var{logfile} argument is given, change the default logfile name to
\var{logfile} and switch log mode on.

\subsec{\b{m}}: as \b{a}, but using prettymatrix format.

\subsec{\b{o}} $\{n\}$: sets \tet{output} mode to $n$ ($0$: raw, $1$:
prettymatrix, $3$: external prettyprint).

\subsec{\b{p}} $\{n\}$: sets \tet{realprecision} to $n$ decimal
digits. Prints its current value if $n$ is omitted.

\subsec{\b{ps}} $\{n\}$: sets \tet{seriesprecision} to $n$ significant terms.
Prints its current value if $n$ is omitted.

\subsec{\b{q}}: quits the \kbd{gp} session and returns to the system.
Shortcut for \tet{quit}\kbd{()} (see \secref{se:quit}).

\subsec{\b{r}} $\{$\var{filename}$\}$: \idx{read}s into \kbd{gp} all the
commands contained in the named file as if they had been typed from the
keyboard, one line after the other. Can be used in combination with the \b{w}
command (see below). Related but not equivalent to the function \kbd{read}
(see \secref{se:read}); in particular, if the file contains more than one
line of input, there will be one history entry for each of them, whereas
\kbd{read} would only record the last one. If \var{filename} is omitted,
re-read the previously used input file (fails if no file has ever been
successfully read in the current session). If a \kbd{gp} \tet{binary file}
(see \secref{se:writebin}) is read using this command, it is silently loaded,
without cluttering the history.

Assuming \kbd{gp} figures how to decompress files on your machine, this
command accepts compressed files in \tet{compress}ed (\kbd{.Z}) or
\tet{gzip}ped (\kbd{.gz} or \kbd{.z}) format. They will be uncompressed on
the fly as \kbd{gp} reads them, without changing the files themselves.

\subsec{\b{s}}: prints the state of the PARI \tev{stack} and \tev{heap}.
This is used primarily as a debugging device for PARI.

\subsec{\b{t}}: prints the \idx{internal longword format} of all the PARI
types. The detailed bit or byte format of the initial codeword(s) is
explained in Chapter~4, but its knowledge is not necessary for a \kbd{gp} user.

\subsec{\b{u}}: prints the definitions of all user-defined functions.

\subsec{\b{um}}: prints the definitions of all user-defined member functions.

\subsec{\b{v}}: prints the \idx{version number} and implementation architecture
(680x0, Sparc, Alpha, other) of the \kbd{gp} executable you are using.

\subsec{\b{w}} $\{n\}$ $\{$\var{filename}$\}$: writes the object number
$n$ ( $\%n$ ) into the named file, in raw format. If the number $n$ is
omitted, writes the latest computed object ( $\%$ ). If \var{filename} is
omitted, appends to \kbd{logfile} (the GP function \tet{write} is a trifle more
powerful, as you can have arbitrary filenames).

\subsec{\b{x}}: prints the complete tree with addresses and contents (in
hexadecimal) of the \idx{internal representation} of the latest computed
object in \kbd{gp}. As for \b{s}, this is used primarily as a debugging device for
PARI, and the format should be self-explanatory (a $*$ before an object --
typically a modulus -- means the corresponding component is out of stack).
However, used on a PARI integer, it can be used as a
decimal$\rightarrow$hexadecimal converter.

\subsec{\b{y}} $\{n\}$: switches \kbd{simplify} on (1) or off (0). If $n$
is explicitly given, set simplify to $n$.

\subseckbd{\#}: switches the \kbd{timer} on or off.

\subseckbd{\#\#}: prints the time taken by the latest computation.
Useful when you forgot to turn on the \kbd{timer}.


\section{The preferences file}\sidx{startup}\sidx{preferences file}
\label{se:gprc}

This file, called \tet{gprc} in the sequel, is used to modify or extend
\kbd{gp}
default behaviour, in all \kbd{gp} sessions: e.g customize \kbd{default} values or
load common user functions and aliases. \kbd{gp} opens the \kbd{gprc} file and
processes the commands in there, \emph{before} doing anything else,
e.g.~creating the PARI stack. If the file does not exist or cannot be read,
\kbd{gp} will proceed to the initialization phase at once, eventually emitting a
prompt. If any explicit command line switches are given, they override the
values read from the preferences file.

\subsec{Syntax}. The syntax in the \kbd{gprc} file (and valid in this file
only) is simple-minded, but should be sufficient for most purposes. The file
is read line by line; as usual, white space is ignored unless surrounded by
quotes and the standard multiline constructions using braces, \kbd{\bs}, or
\kbd{=} are available (multiline comments between \kbd{/*~\dots~*/} are also
recognized).

\subsubsec{Preprocessor:}
Two types of lines are first dealt with by a preprocessor:

$\bullet$ comments are removed. This applies to all text surrounded by
\kbd{/*~\dots~*/} as well as to everything following \kbd{\bs\bs} on a given
line.

$\bullet$ lines starting with \kbd{\#if} \var{boolean} are treated as
comments if \var{boolean} evaluates to \kbd{false}, and read normally
otherwise. The condition can be negated using either \kbd{\#if not} (or
\kbd{\#if !}). If the rest of the current line is empty, the test applies to
the next line (same behaviour as \kbd{=} under \kbd{gp}). Only three tests can be
performed:

\kbd{EMACS}: \kbd{true} if \kbd{gp} is running in an Emacs or TeXmacs shell (see
\secref{se:emacs}).

\kbd{READL}: \kbd{true} if \kbd{gp} is compiled with \kbd{readline} support (see
\secref{se:readline}).

\kbd{VERSION} \var{op} \var{number}: where \var{op} is in the set
$\{ \kbd{>}, \kbd{<}, \kbd{<=}, \kbd{>=} \}$, and \var{number} is a PARI
version number of the form \var{Major}.\var{Minor}.\var{patch}, where the
last two components can be omitted (i.e.~$1$ is understood as version $1.0.0$).
This is \kbd{true} if \kbd{gp}'s version number satisfies the required
inequality.

\subsubsec{Commands:}
After the preprocessing the remaining lines are executed as
sequence of expressions (as usual, separated by \kbd{;} if necessary). Only
two kinds of expressions are recognized:

$\bullet$ \var{default} \kbd{=} \var{value}, where \var{default} is one of
the available defaults (see \secref{se:defaults}), which will be set to
\var{value} on actual startup. Don't forget the quotes around strings
(e.g.~for \kbd{prompt} or \kbd{help}).

$\bullet$ \kbd{read "\var{some\_GP\_file}"} where \kbd{\var{some\_GP\_file}}
is a regular GP script this time, which will be read just before \kbd{gp} prompts
you for commands, but after initializing the defaults. In particular, file
input is delayed until the \kbd{gprc} has been fully loaded. This is the
right place to input files containing \kbd{alias} commands, or your favorite
macros.

\noindent For instance you could set your prompt in the following portable way:
\bprog
\\ self modifying prompt looking like @com\hbox{\rm(18:03) \key{gp}\kbd{ >}}
prompt   = "(%H:%M) \e[1mgp\e[m > "

\\ readline wants non-printing characters to be braced between ^A/^B pairs
#if READL prompt = "(%H:%M) ^A\e[1m^Bgp^A\e[m^B > "

\\ escape sequences not supported under emacs
#if EMACS prompt = "(%H:%M) gp > "
@eprog

\noindent Note that any of the last two lines could be broken in the
following way
\bprog
#if EMACS
  prompt = "(%H:%M) gp > "
@eprog
\noindent since the preprocessor directive applies to the next line if the
current one is empty.

A sample \kbd{gprc} file called \kbd{misc/gprc.dft} is provided in the
standard distribution. It is a good idea to have a look at it and customize
it to your needs. Since this file does not use multiline constructs, here is
one (note the terminating \kbd{;} to separate the expressions):
\bprog
#if VERSION > 2.2.3
{
  read "my_scripts";     \\ syntax errors in older versions
  new_galois_format = 1; \\ default introduced in 2.2.4
}
#if ! EMACS
{
  colors = "9, 5, no, no, 4, 1, 2";
  help   = "gphelp -detex -ch 4 -cb 0 -cu 2";
}
@eprog

\subsec{Where is it?}
When \kbd{gp} is started, it looks for a customization file, or \kbd{gprc} in the
following places (in this order, only the first one found will be loaded):

\noindent$\bullet$ On the Macintosh (only), \kbd{gp} looks in the directory
which contains the \kbd{gp} executable itself for a file called \kbd{gprc}.

\noindent$\bullet$ \kbd{gp} checks whether the environment variable
\tet{GPRC} is set. Under DOS, you can set it in \kbd{AUTOEXEC.BAT}. On Unix,
this can be done with something like: \smallskip

\settabs\+\indent&\kbd{GPRC=/my/dir/anyname; export GPRC}\quad&\cr

\+&\kbd{GPRC=/my/dir/anyname; export GPRC}\quad&in \kbd{sh} syntax
(for instance in your \kbd{.profile}),\cr

\+&\kbd{setenv GPRC /my/dir/anyname} &in \kbd{csh} syntax
(in your \kbd{.login} or \kbd{.cshrc} file).\cr

\noindent If so, the file named by \kbd{\$GPRC} is the \kbd{gprc}.

\noindent$\bullet$ If \kbd{GPRC} is not set, and if the environment variable
\kbd{HOME} is defined, \kbd{gp} then tries

\kbd{\$HOME/.gprc} on a Unix system

\kbd{\$HOME\bs\_$\,$gprc} on a DOS, OS/2, or Windows system.

\noindent$\bullet$ If \kbd{HOME} also leaves us clueless, we try

\strut\kbd{\til/.gprc} on a Unix system (where as usual \kbd{\til} stands for
your home directory), or

\kbd{\b{\_}$\,$gprc} on a DOS, OS/2, or Windows system.

\noindent$\bullet$ Finally, if no gprc was found among the user files
mentioned above we look for \kbd{/etc/gprc} (\kbd{\bs etc\bs gprc})
for a system-wide gprc file (you will need root privileges to set up such a
file yourself).

Note that on Unix systems, the \kbd{gprc}'s default name starts with a '.' and
thus is hidden to regular \kbd{ls} commands; you need to type \kbd{ls -a} to
list it.


\section{Using readline} \sidx{line editor}\sidx{completion}

This very useful library provides line editing and contextual completion
to \kbd{gp}. You are encouraged to read the \kbd{readline} user manual,
but we describe basic usage here.

\subsec{A (too) short introduction to readline}: \label{se:readline}
In the following, \kbd{C-} stands for ``the \kbd{Control} key combined with
another'' and the same for \kbd{M-} with the \kbd{Meta} key; generally
\kbd{C-} combinations act on characters, while the \kbd{M-} ones operate on
words. The \kbd{Meta} key might be called \kbd{Alt} on some keyboards, will
display a black diamond on most others, and can safely be replaced by
\kbd{Esc} in any case.

Typing any ordinary key inserts text where the cursor stands, the arrow keys
enabling you to move in the line. There are many more movement commands,
which will be familiar to the Emacs user, for instance \kbd{C-a}/\kbd{C-e}
will take you to the start/end of the line, \kbd{M-b}/\kbd{M-f} move the
cursor backward/forward by a word, etc. Just press the \kbd{<Return>} key at
any point to send your command to \kbd{gp}.

  All the commands you type at the \kbd{gp} prompt are stored in a history,
a multiline command being saved as a single concatenated line. The Up and Down
arrows (or \kbd{C-p}/\kbd{C-n}) will move you through the history,
\kbd{M-<}/\kbd{M->} sending you to the start/end of the history.
\kbd{C-r}/\kbd{C-s} will start an incremental backward/forward search. You
can kill text (\kbd{C-k} kills till the end of line, \kbd{M-d} to the end of
current word) which you can then yank back using the \kbd{C-y} key (\kbd{M-y}
will rotate the kill-ring). \kbd{C-\_} will undo your last changes
incrementally (\kbd{M-r} undoes all changes made to the current line).
\kbd{C-t} and \kbd{M-t} will transpose the character (word) preceding the
cursor and the one under the cursor.

  Keeping the \kbd{M-} key down while you enter an integer (a minus sign
meaning reverse behaviour) gives an argument to your next readline command
(for instance \kbd{M-- C-k} will kill text back to the start of line). If you
prefer \idx{Vi}--style editing, \kbd{M-C-j} will toggle you to Vi mode.

  Of course you can change all these default bindings. For that you need to
create a file named \kbd{.inputrc} in your home directory. For instance
(notice the embedding conditional in case you would want specific bindings
for \kbd{gp}):
%
\bprog
$if Pari-GP
  set show-all-if-ambiguous
  "\C-h": backward-delete-char
  "\e\C-h": backward-kill-word
  "\C-xd": dump-functions
  (: "\C-v()\C-b"       #@com can be annoying when copy-pasting !
  [: "\C-v[]\C-b"
$endif
@eprog
\noindent\kbd{C-x C-r} will re-read this init file, incorporating any
changes made to it during the current session.

\misctitle{Note:} By default, \kbd{(} and \kbd{[} are bound to the function
\kbd{pari-matched-insert} which, if ``electric parentheses'' are enabled
(default: off) will automatically insert the matching closure (respectively
\kbd{)} and \kbd{]}). This behaviour can be toggled on and off by giving
the numeric argument $-2$ to \kbd{(} (\kbd{M--2(}), which is useful if you
want, e.g to copy-paste some text into the calculator. If you do not want a
toggle, you can use \kbd{M--0} / \kbd{M--1} to specifically switch it on or
off).

\misctitle{Note:} In some versions of readline (2.1 for instance), the
\kbd{Alt} or \kbd{Meta} key can give funny results (output 8-bit accented
characters for instance). If you do not want to fall back to the \kbd{Esc}
combination, put the following two lines in your \kbd{.inputrc}:
%
\bprog
  set convert-meta on
  set output-meta off
@eprog

% don't remove this leading space (prevent gphelp from splitting the entry)
 \subsec{Command completion and online help}. Hitting
\kbd{<TAB>} will complete words for you. This mechanism is context-dependent:
\kbd{gp} will strive to only give you meaningful completions in a given
context (it will fail sometimes, but only under rare and restricted
conditions).

  For instance, shortly after a \kbd{\til}, we expect a user name, then a
path to some file. Directly after \kbd{default(} has been typed, we would
expect one of the \kbd{default} keywords. After \kbd{whatnow(} , we expect
the name of an old function, which may well have disappeared from this
version. After a '.', we expect a member keyword. And generally of course, we
expect any GP symbol which may be found in the hashing lists: functions (both
yours and GP's), and variables.

  If, at any time, only one completion is meaningful, \kbd{gp} will provide it
together with

$\bullet$ an ending comma if we are completing a default,

$\bullet$ a pair of parentheses if we are completing a function name. In
that case hitting \kbd{<TAB>} again will provide the argument list as given
by the online help\footnote{*}{recall that you can always undo the effect
of the preceding keys by hitting \kbd{C-\_}}.

Otherwise, hitting \kbd{<TAB>} once more will give you the list of possible
completions. Just experiment with this mechanism as often as possible,
you will probably find it very convenient. For instance, you can obtain
\kbd{default(seriesprecision,10)}, just by hitting \kbd{def<TAB>se<TAB>10},
which saves 18 keystrokes (out of 27).

  Hitting \kbd{M-h} will give you the usual short online help concerning the
word directly beneath the cursor, \kbd{M-H} will yield the extended help
corresponding to the \kbd{help} default program (usually opens a \idx{dvi}
previewer, or runs a primitive tex-to-ASCII program). None of these disturb
the line you were editing.

\section{GNU Emacs and PariEmacs}
\label{se:emacs}

If you install the PariEmacs package (see Appendix A), you may use \kbd{gp}
as a subprocess in \idx{Emacs}. You then need to include in your \kbd{.emacs}
file the following lines:
\bprog
  (autoload 'gp-mode "pari" nil t)
  (autoload 'gp-script-mode "pari" nil t)
  (autoload 'gp "pari" nil t)
  (autoload 'gpman "pari" nil t)

  (setq auto-mode-alist
    (cons '("\\.gp$" . gp-script-mode) auto-mode-alist))
@eprog
\noindent which autoloads functions from the PariEmacs package and ensures
that file with the \kbd{.gp} suffix are edited in gp-script mode.

Once this is done, under GNU Emacs if you type \kbd{M-x gp} (where as usual
\kbd{M} is the \kbd{Meta} key), a special shell will be started launching
\kbd{gp} with the default stack size and prime limit. You can then work as
usual under \kbd{gp}, but with all the facilities of an advanced text editor.
See the PariEmacs documentation for customizations, menus, etc.

\newpage
