% $Id$
% Copyright (c) 2000  The PARI Group
%
% This file is part of the PARI/GP documentation
%
% Permission is granted to copy, distribute and/or modify this document
% under the terms of the GNU General Public License
\chapter{Technical Reference Guide for Elliptic curves and arithmetic geometry}

This chapter is quite short, but is added as a placeholder, since
we expect the library to expand in that direction.

\section{Elliptic curves}
Elliptic curves are represented in the Weierstrass model
$$ (E): y^2z + a_1xyz + a_3 yz = x^3 + a_2 x^2z + a_4 xz^2 + a_6z^3, $$
by the $5$-tuple $[a_1,a_2,a_3,a_4,a_6]$. Points in the projective
plane are represented as follows: the point at infinity $(0:1:0)$ is coded
as \kbd{[0]}, a finite point $(x:y:1)$ outside the projective line at infinity
$z = 0$ is coded as $[x,y]$. Note that other points at infinity than $(0:1:0)$
cannot be represented; this is harmless, since they do not belong to any of
the elliptic curves $E$ above.

\emph{Points on the curve} are just projective points as described above,
they are not tied to a curve in any way: the same point may be used in
conjunction with different curves, provided it satisfies their equations (if
it does not, the result is usually undefined). In particular, the point at
infinity belongs to all elliptic curves.

As with \tet{factor} for polynomial factorization, the $5$-tuple
$[a_1,a_2,a_3,a_4,a_6]$ implicitly defines a base ring over which the curve
is defined. Point coordinates must be operation-compatible with this
base ring (\kbd{gadd}, \kbd{gmul}, \kbd{gdiv} involving them should not give
errors).

\subsec{Types of elliptic curves}

We call a $5$-tuble as above an \kbd{ell5}; most functions require an
\kbd{ell} structure, as returned by \tet{ellinit}, which contains additional
data (usually dynamically computed as needed), depending on the base field.

\fun{GEN}{ellinit}{GEN E, GEN D, long prec}, returns an \tet{ell} structure,
associated to the elliptic curve $E$ : either an \kbd{ell5}, a pair $[a_4,a_6]$
or a \typ{STR} in Cremona's notation, e.g. \kbd{"11a1"}. The optional $D$
(\kbd{NULL} to omit) describes the domain over which the curve is defined.

\subsec{Type checking}

\fun{void}{checkell}{GEN e} raise an error unless $e$ is a \var{ell}.

\fun{void}{checkell5}{GEN e} raise an error unless $e$ is an \var{ell},
a \var{smallell} or an \var{ell5}.

\fun{void}{checkellpt}{GEN z} raise an error unless $z$ is a point
(either finite or at infinity).

\fun{long}{ell_get_type}{GEN e} returns the domain type over which the curve
is defined, one of

  \tet{t_ELL_Q} the field of rational numbers;

  \tet{t_ELL_Qp} the field of $p$-adic numbers, for some prime $p$;

  \tet{t_ELL_Fp} a prime finite field, base field elements are represented as
  \kbd{Fp} (\typ{INT} reduced modulo $p$);

  \tet{t_ELL_Fq} a non-prime finite field (a prime finite field can also be
  represented by this subtype, but this is inefficient), base field elements
  are represented as \typ{FFELT};

  \tet{t_ELL_Rg} none of the above.

\fun{void}{checkell_Fq}{GEN e} checks whether $e$ is an \kbd{ell}, defined
over a finite field (either prime or non-prime), raises \tet{pari_err_TYPE}
otherwise.

\fun{void}{checkell_Q}{GEN e} checks whether $e$ is an \kbd{ell}, defined
over $\Q$, raises \tet{pari_err_TYPE} otherwise.

\fun{void}{checkell_Qp}{GEN e} checks whether $e$ is an \kbd{ell}, defined
over some $\Q_p$, raises \tet{pari_err_TYPE} otherwise.


\subsec{Extracting info from an \kbd{ell} structure}

These functions expect an \kbd{ell} argument. If the required data is not
part of the structure, it is computed then inserted, and the new value is
returned.

\subsubsec{All domains}

\fun{GEN}{ell_get_a1}{GEN e}

\fun{GEN}{ell_get_a2}{GEN e}

\fun{GEN}{ell_get_a3}{GEN e}

\fun{GEN}{ell_get_a4}{GEN e}

\fun{GEN}{ell_get_a6}{GEN e}

\fun{GEN}{ell_get_b2}{GEN e}

\fun{GEN}{ell_get_b4}{GEN e}

\fun{GEN}{ell_get_b6}{GEN e}

\fun{GEN}{ell_get_b8}{GEN e}

\fun{GEN}{ell_get_c4}{GEN e}

\fun{GEN}{ell_get_c6}{GEN e}

\fun{GEN}{ell_get_disc}{GEN e}

\fun{GEN}{ell_get_j}{GEN e}

\subsubsec{Curves over $\Q$}

\fun{GEN}{ellQ_get_N}{GEN e} returns the curve conductor

\fun{void}{ellQ_get_Nfa}{GEN e, GEN *N, GEN *faN} sets $N$ to the conductor
and \kbd{faN} to its factorization

\fun{long}{ellrootno_global}{GEN e} returns $[c, [c_{p_1}, \dots,c_{p_k}]]$,
where the \typ{INT} $c\in \{-1,1\}$ is the global root number, and the
$c_{p_i}$ are the local root numbers at all prime divisors of the conductor,
ordered as in \kbd{faN} above.

\fun{GEN}{elldatagenerators}{GEN E} returns generators for $E(\Q)$
extracted from Cremona's table.

\fun{GEN}{ell_apply_globalred}{GEN e} takes an \var{ell} over $\Q$
and returns a global minimal model for $e$.

\subsubsec{Curves over $\Q_p$}

\fun{GEN}{ellQp_get_p}{GEN E} returns $p$

\fun{long}{ellQp_get_prec}{GEN E} returns the default $p$-adic accuracy to
which we must compute approximate results associated to $E$.

\fun{GEN}{ellQp_get_zero}{GEN x} returns $O(p^n)$, where $n$ is the default
$p$-adic accuracy as above.

The following functions are only defined when $E$ has multiplicative
reduction (Tate curves):

\fun{GEN}{ellQp_Tate_uniformization}{GEN E, long prec} returns a
\typ{VEC} containing $u^2, u, q, [a,b]$, at $p$-adic precision \kbd{prec}.

\fun{GEN}{ellQp_u}{GEN E, long prec} returns $u$.

\fun{GEN}{ellQp_u2}{GEN E, long prec} returns $u^2$.

\fun{GEN}{ellQp_q}{GEN E, long prec} returns the Tate periode $q$.

\fun{GEN}{ellQp_ab}{GEN E, long prec} returns $[a,b]$.

\fun{GEN}{ellQp_root}{GEN E, long prec} returns $e_1$.

\subsubsec{Curves over a finite field $\F_q$}

\fun{GEN}{ellff_get_p}{GEN E} returns the characteristic

\fun{GEN}{ellff_get_field}{GEN E} returns $p$ if $\F_q$ is a prime field, and
a \typ{FFELT} belonging to $\F_q$ otherwise.

\fun{GEN}{ellff_get_card}{GEN E} returns $\#E(\F_q)$

\fun{GEN}{ellff_get_gens}{GEN E} returns a minimal set of generators for
$E(\F_q)$.

\fun{GEN}{ellff_get_group}{GEN E} returns \kbd{ellgroup}$(E)$.

\fun{GEN}{ellff_get_o}{GEN E} returns $[d, \kbd{factor{d}}]$, where $d$ is
the exponent of $E(\F_q)$.

\fun{GEN}{ellff_get_a4a6}{GEN E} returns a canonical ``short model'' for $E$
and the corresponding change of variable. For $p\neq,2,3$,
this is $[A_4,A_6,[u,r,s,t]]$, corresponding to $y^2 = x^3 + A_4x + A_6$,
where $A_4 = -27c_4$, $A_6 = -54c_6$, $[u,r,s,t] = [6, 3b_2,3a_1,108a_3]$.

If $p = 3$ and the curve is ordinary ($b_2\neq 0$), this is
$[b_2, A_6, [1,v,-a_1,-a_3]]$, corresponding to $y^2 = x^3 + b_2 x + A_6$,
where $v = b_4/b_2$, $A_6 = b_6 - v(b_4+v^2)$.

If $p = 3$ and the curve is supersingular ($b_2 = 0$), this is
$[-b_4, b_6, [1,0,-a_1,-a_3]]$, corresponding to $y^2 = x^3 - b_4 x + b_6$.

If $p = 2$ and $a1 \neq 0$, return
$[d_2,d_6,[a_1^{-1}, da_1^{-2}, 0, (a_4+d^2)a_1^{-1}]$, where
$d = a_3/a_1$, $a_1^2 d_2 = (a_2 + d)$ and
$$ a_1^6 d_6 = d^3 + a_2 d^2 + a_4 d + a_6 + (a_4 + d^2)^2a_1^{-2}.$$

If $p = 2$ and $a3 \neq 0$, return
$[[a_3, a_2^2 + a_4, 1/a_3], a_2a_4 + a_6, [1,a_2,0,0]]$

\subsubsec{Curves over $\C$} (This includes curves over $\Q$!)

\fun{long}{ellR_get_prec}{GEN E} returns the default accuracy to
which we must compute approximate results associated to $E$.

\fun{GEN}{ellR_ab}{GEN E, long prec} returns $[a,b]$

\fun{GEN}{ellR_omega}{GEN x, long prec} returns periods
$[\omega_1,\omega_2]$.

\fun{GEN}{ellR_eta}{GEN E, long prec} returns quasi-periods
$[\eta_1,\eta_2]$.

\fun{GEN}{ellR_roots}{GEN E, long prec} returns $[e_1,e_2,e_3]$. If $E$ is
defined over $\R$, then $e_1$ is real. If furthermore $\disc E > 0$, then
$e_1 > e_2 > e_3$.

\fun{long}{ellR_get_sign}{GEN E} if $E$ is defined over $\R$ returns the
signe of its discriminant, otherwise return $0$.

\subsec{Points}

\fun{int}{ell_is_inf}{GEN z} tests whether the point $z$ is the point at
infinity.

\fun{GEN}{ellinf}{} returns the point at infinity \kbd{[0]}.

\subsec{Change of variables}
\fun{GEN}{ellchangeinvert}{GEN w} given a change of variables $w =
[u,r,s,t]$, returns the inverse change of variables $w'$, such that if $E' =
\kbd{ellchangecurve(E, w)}$, then $E = \kbd{ellchangecurve}(E, w')$.

\subsec{Functions to handle elliptic curves over finite fields}

\subsubsec{Tolerant routines}

\fun{GEN}{ellap}{GEN E, GEN p} given a prime number $p$ and an elliptic curve
defined over $\Q$ or $\Q_p$ (assumed integral and minimal at $p$), computes
the  trace of  Frobenius  $a_p = p+1 - \#E(\F_p)$. If $E$ is defined over
a non-prime finite field $\F_q$, ignore $p$ and return $q+1 - \#E(\F_q)$.
When $p$ is implied ($E$ defined over $\Q_p$ or a finite field), $p$ can be
omitted (set to \kbd{NULL}).

\fun{GEN}{ellsea}{GEN E, GEN p, long s} available if the \kbd{seadata}
package is installed. This function returns $\#E(\F_p)$, using the
Schoof-Elkies-Atkin algorithm; it is called by \kbd{ellap}: same conditions
as above for $E$, except that \tet{t_ELL_Fq} are not allowed. The extra flag
\kbd{s}, if set to a non-zero value, causes the computation to return
\kbd{gen\_0} (an impossible cardinality) if one of the small primes $\ell>s$
divides the curve order. For cryptographic applications, where one is usually
interested in curves of prime order, setting $s=1$ efficiently weeds out most
uninteresting curves; if curves of order a power of $2$ times a prime are
acceptable, set $s=2$. There is no guarantee that the resulting cardinality
is prime, only that it has no small prime divisor larger than $s$.

\subsubsec{Curves defined a non-prime finite field}
In this subsection, we assume that \tet{ell_get_type}$(E)$ is \tet{t_ELL_Fq}.
(As noted above, a curve defined over $\Z/p\Z$ can be represented as a
\tet{t_ELL_Fq}.)

\fun{GEN}{FF_ellmul}{GEN E, GEN P, GEN n} returns $[n]P$ where $n$ is an
integer and $P$ is a point on the curve $E$.

\fun{GEN}{FF_ellrandom}{GEN E} returns a random point in $E(\F_q)$

\fun{GEN}{FF_ellorder}{GEN E, GEN P, GEN o} returns the order of the point
$P$, where $o$ is a multiple of the order of $P$, or its factorization.

\fun{GEN}{FF_ellcard}{GEN E} returns $\#E(\F_q)$.

\fun{GEN}{FF_ellgens}{GEN E} returns the generators of the group $E(\F_q)$.

\fun{GEN}{FF_elllog}{GEN E, GEN P, GEN G, GEN o} Let \kbd{G} be a point of
order \kbd{o}, return $e$ such that $[e]P=G$. If $e$ does not exists, the
result is undefined.

\fun{GEN}{FF_ellgroup}{GEN E} returns the Abelian group $E(\F_q)$ in the form
$[h,\kbd{cyc},\kbd{gen}]$.

\fun{GEN}{FF_ellweilpairing}{GEN E, GEN P, GEN Q, GEN m} returns the
Weil pairing of the points of $m$-torsion $P$ and $Q$.

\fun{GEN}{FF_elltatepairing}{GEN E, GEN P, GEN Q, GEN m} returns the Tate
pairing of $P$ and $Q$, where $[m]P = 0$.

\section{Arithmetic on elliptic curve over a finite field in simple form}

The functions in this section no longer operate on elliptic curve structures,
as seen up to now. They are used to implement those higher-level functions
without using cached information and thus require suitable explicitly
enumerated data.

\subsec{Helper functions}

\fun{GEN}{elltrace_extension}{GEN t, long n, GEN q} Let $E$ some elliptic curve
over $\F_q$ such that the trace of the Frobenius is $t$, returns the trace of
the Frobenius over $\F_q^n$.

\subsec{Elliptic curves over $\F_p$, $p>3$}

Let $p$ a prime number and $E$ the elliptic curve given by the equation
$E:y^2=x^3+a_4\*x+a_6$, with $a_4$ and $a_6$ in $\F_p$. A \kbd{FpE} is a
point of $E(\F_p)$.  Since an affine point and $a_4$ determine an unique
$a6$, most functions do not take $a_6$ as an argument. A \kbd{FpE} is either
the point at infinity (\kbd{ellinf()}) or a $FpV$ whith two components. The
parameters $a_4$ and $a_6$ are given as \typ{INT}s when required.

\fun{GEN}{Fp_ellj}{GEN a4, GEN a6, GEN p}
returns the $j$-invariant of the curve $E$.

\fun{GEN}{Fp_ellcard}{GEN a4, GEN a6, GEN p} returns the cardinal of the group
$E(\F_p)$.

\fun{GEN}{Fp_ellcard_SEA}{GEN a4, GEN a6, GEN p, long s} same as
\tet{ellsea} when only $[a_4,a_6]$ are given.

\fun{GEN}{Fq_ellcard_SEA}{GEN a4, GEN a6, GEN q, GEN T, GEN p, long s} same
when only $[a_4,a_6]$ are given, over $\F_p[t]/(T)$, assume $p\neq 2,3$.


\fun{GEN}{Fp_ffellcard}{GEN a4, GEN a6, GEN q, long n, GEN p} returns the
cardinal of the group $E(\F_q)$ where $q=p^n$.

\fun{GEN}{Fp_ellgroup}{GEN a4, GEN a6, GEN N, GEN p, GEN *pt_m} returns the
group structure $D$ of the group $E(\F_p)$, which is assumed to be of order $N$
and set $*pt_m=m$.

\fun{GEN}{Fp_ellgens}{GEN a4, GEN a6, GEN ch, GEN D, GEN m, GEN p} returns
generators of the group $E(\F_p)$ with the base change \kbd{ch} (see
\kbd{FpE\_changepoint}), where $D$ and $m$ are as returned by
\kbd{Fp\_ellgroup}.

\fun{GEN}{Fp_elldivpol}{GEN a4, GEN a6, long n, GEN p} returns the $n$-division
polynomial of the elliptic curve $E$.

\subsec{\kbd{FpE}}

\fun{GEN}{FpE_add}{GEN P, GEN Q, GEN a4, GEN p} returns the sum $P+Q$
in the group $E(\F_p)$, where $E$ is defined by $E:y^2=x^3+a_4\*x+a_6$,
for any value of $a_6$ compatible with the points given.

\fun{GEN}{FpE_sub}{GEN P, GEN Q, GEN a4, GEN p} returns $P-Q$.

\fun{GEN}{FpE_dbl}{GEN P, GEN a4, GEN p} returns $2.P$.

\fun{GEN}{FpE_neg}{GEN P, GEN p} returns $-P$.

\fun{GEN}{FpE_mul}{GEN P, GEN n, GEN a4, GEN p} return $n.P$.

\fun{GEN}{FpE_changepoint}{GEN P, GEN m, GEN a4, GEN p} returns the image
$Q$ of the point $P$ on the curve $E:y^2=x^3+a_4\*x+a_6$ by the coordinate
change $m$ (which is a \kbd{FpV}).

\fun{GEN}{FpE_changepointinv}{GEN P, GEN m, GEN a4, GEN p} returns the image
$Q$ on the curve $E:y^2=x^3+a_4\*x+a_6$ of the point $P$ by the inverse of the
coordinate change $m$ (which is a \kbd{FpV}).

\fun{GEN}{random_FpE}{GEN a4, GEN a6, GEN p} returns a random point on
$E(\F_p)$, where $E$ is defined by $E:y^2=x^3+a_4\*x+a_6$.

\fun{GEN}{FpE_order}{GEN P, GEN o, GEN a4, GEN p} returns the order of $P$ in
the group $E(\F_p)$, where $o$ is a multiple of the order of $P$, or its
factorization.

\fun{GEN}{FpE_log}{GEN P, GEN G, GEN o, GEN a4, GEN p} Let \kbd{G} be a
point of order \kbd{o}, return $e$ such that $e.P=G$. If $e$ does not exists,
the result is currently undefined.

\fun{GEN}{FpE_tatepairing}{GEN P, GEN Q, GEN m, GEN a4, GEN p} returns the
Tate pairing of the point of $m$-torsion $P$ and the point $Q$.

\fun{GEN}{FpE_weilpairing}{GEN P, GEN Q, GEN m, GEN a4, GEN p} returns the
Weil pairing of the points of $m$-torsion $P$ and $Q$.

\fun{GEN}{FpE_to_mod}{GEN P, GEN p} returns $P$ as a vector of \typ{INTMOD}s.

\fun{GEN}{RgE_to_FpE}{GEN P, GEN p} returns the \kbd{FpE} obtained by applying
\kbd{Rg\_to\_Fp} coefficientwise.

\subsec{Elliptic curves over $\F_{2^n}$}
Let $T$ be an irreducible \kbd{F2x} and $E$ the
elliptic curve given by either the equation
$E:y^2+x*y=x^3+a_2\*x^2+a_6$ with $a_2$ and $a_6$ being \kbd{F2x} in $\F_2[X]/(T)$
(ordinary case)
or
$E:y^2+a_3*y=x^3+a_4\*x+a_6$ with $a_3$, $a_4$ and $a_6$ being \kbd{F2x} in $\F_2[X]/(T)$
(supersingular case).

A \kbd{F2xqE} is a point of $E(\F_2[X]/(T))$.  In the supersingular case, the
parameter \kbd{a2} is actually the \typ{VEC} $[a_3,a_4,a_3^{-1}]$.

\fun{GEN}{F2xq_ellcard}{GEN a2, GEN a6, GEN T}
Return the order of the group $E(\F_2[X]/(T))$.

\fun{GEN}{F2xq_ellgroup}{GEN a2, GEN a6, GEN N, GEN T, GEN *pt_m}
Return the group structure $D$ of the group $E(\F_2[X]/(T))$,
which is assumed to be of order $N$ and set $*pt_m=m$.

\fun{GEN}{F2xq_ellgens}{GEN a2, GEN a6, GEN ch, GEN D, GEN m, GEN T}
Returns generators of the group $E(\F_2[X]/(T))$ with the base change \kbd{ch}
(see \kbd{F2xqE\_changepoint}), where $D$ and $m$ are as returned by
\kbd{F2xq\_ellgroup}.

\subsec{\kbd{F2xqE}}

\fun{GEN}{F2xqE_changepoint}{GEN P, GEN m, GEN a2, GEN T} returns the image
$Q$ of the point $P$ on the curve $E:y^2+x*y=x^3+a_2\*x^2+a_6$ by the coordinate
change $m$ (which is a \kbd{F2xqV}).

\fun{GEN}{F2xqE_changepointinv}{GEN P, GEN m, GEN a2, GEN T} returns the image
$Q$ on the curve $E:y^2=x^3+a_4\*x+a_6$ of the point $P$ by the inverse of the
coordinate change $m$ (which is a \kbd{F2xqV}).

\fun{GEN}{F2xqE_add}{GEN P, GEN Q, GEN a2, GEN T}

\fun{GEN}{F2xqE_sub}{GEN P, GEN Q, GEN a2, GEN T}

\fun{GEN}{F2xqE_dbl}{GEN P, GEN a2, GEN T}

\fun{GEN}{F2xqE_neg}{GEN P, GEN a2, GEN T}

\fun{GEN}{F2xqE_mul}{GEN P, GEN n, GEN a2, GEN T}

\fun{GEN}{random_F2xqE}{GEN a2, GEN a6, GEN T}

\fun{GEN}{F2xqE_order}{GEN P, GEN o, GEN a2, GEN T} returns the order of $P$ in
the group $E(\F_2[X]/(T))$, where $o$ is a multiple of the order of $P$, or its
factorization.

\fun{GEN}{F2xqE_log}{GEN P, GEN G, GEN o, GEN a2, GEN T} Let \kbd{G} be a
point of order \kbd{o}, return $e$ such that $e.P=G$. If $e$ does not exists,
the result is currently undefined.

\fun{GEN}{F2xqE_tatepairing}{GEN P, GEN Q, GEN m, GEN a2, GEN T} returns the
Tate pairing of the point of $m$-torsion $P$ and the point $Q$.

\fun{GEN}{F2xqE_weilpairing}{GEN Q, GEN Q, GEN m, GEN a2, GEN T} returns the
Weil pairing of the points of $m$-torsion $P$ and $Q$.

\fun{GEN}{RgE_to_F2xqE}{GEN P, GEN T} returns the \kbd{F2xqE} obtained by applying
\kbd{Rg\_to\_F2xq} coefficientwise.

\subsec{Elliptic curves over $\F_q$, small characteristic $p>3$ }
Let $p$ be a prime \kbd{ulong}, $T$ an irreducible \kbd{Flx} mod $p$, and $E$ the
elliptic curve given by the equation $E:y^2=x^3+a_4\*x+a_6$ with $a_4$ and $a_6$
being \kbd{Flx} in $\F_p[X]/(T)$.  A \kbd{FlxqE} is a point of $E(\F_p[X]/(T))$.

In the special case of characteristic $3$, an ordinary elliptic curve can be
given by the equation $E:y^2=x^3+a_2\*x^2+a_6$ with $a_2$ and $a_6$ being
\kbd{Flx} in $\F_3[X]/(T)$. In that case, the parameter parameter \kbd{a4} is
actually the \typ{VEC} $[a_2]$.

\fun{GEN}{Flxq_ellj}{GEN a4, GEN a6, GEN T, ulong p}
returns the $j$-invariant of the curve $E$.

\fun{GEN}{Flxq_ellcard}{GEN a4, GEN a6, GEN T, ulong p}
returns the order of the group $E(\F_p[X]/(T))$.

\fun{GEN}{Flxq_ellgroup}{GEN a4, GEN a6, GEN N, GEN T, ulong p, GEN *pt_m}
returns the group structure $D$ of the group $E(\F_p[X]/(T))$,
which is assumed to be of order $N$ and set $*pt_m=m$.

\fun{GEN}{Flxq_ellgens}{GEN a4, GEN a6, GEN ch, GEN D, GEN m, GEN T, ulong p}
returns generators of the group $E(\F_p[X]/(T))$ with the base change \kbd{ch}
(see \kbd{FlxqE\_changepoint}), where $D$ and $m$ are as returned by
\kbd{Flxq\_ellgroup}.

\subsec{\kbd{FlxqE}}

\fun{GEN}{FlxqE_changepoint}{GEN P, GEN m, GEN a4, GEN T, ulong p} returns the image
$Q$ of the point $P$ on the curve $E:y^2=x^3+a_4\*x+a_6$ by the coordinate
change $m$ (which is a \kbd{FlxqV}).

\fun{GEN}{FlxqE_changepointinv}{GEN P, GEN m, GEN a4, GEN T, ulong p} returns the image
$Q$ on the curve $E:y^2=x^3+a_4\*x+a_6$ of the point $P$ by the inverse of the
coordinate change $m$ (which is a \kbd{FlxqV}).

\fun{GEN}{FlxqE_add}{GEN P, GEN Q, GEN a4, GEN T, ulong p}

\fun{GEN}{FlxqE_sub}{GEN P, GEN Q, GEN a4, GEN T, ulong p}

\fun{GEN}{FlxqE_dbl}{GEN P, GEN a4, GEN T, ulong p}

\fun{GEN}{FlxqE_neg}{GEN P, GEN T, ulong p}

\fun{GEN}{FlxqE_mul}{GEN P, GEN n, GEN a4, GEN T, ulong p}

\fun{GEN}{random_FlxqE}{GEN a4, GEN a6, GEN T, ulong p}

\fun{GEN}{FlxqE_order}{GEN P, GEN o, GEN a4, GEN T, ulong p} returns the order of $P$ in
the group $E(\F_p[X]/(T))$, where $o$ is a multiple of the order of $P$, or its
factorization.

\fun{GEN}{FlxqE_log}{GEN P, GEN G, GEN o, GEN a4, GEN T, ulong p} Let \kbd{G} be a
point of order \kbd{o}, return $e$ such that $e.P=G$. If $e$ does not exists,
the result is currently undefined.

\fun{GEN}{FlxqE_tatepairing}{GEN P, GEN Q, GEN m, GEN a4, GEN T, ulong p} returns the
Tate pairing of the point of $m$-torsion $P$ and the point $Q$.

\fun{GEN}{FlxqE_weilpairing}{GEN P, GEN Q, GEN m, GEN a4, GEN T, ulong p} returns the
Weil pairing of the points of $m$-torsion $P$ and $Q$.

\fun{GEN}{RgE_to_FlxqE}{GEN P, GEN T, ulong p} returns the \kbd{FlxqE} obtained by applying
\kbd{Rg\_to\_Flxq} coefficientwise.


\subsec{Elliptic curves over $\F_q$, large characteristic }

Let $p$ be a prime number, $T$ an irreducible polynomial mod $p$, and $E$ the
elliptic curve given by the equation $E:y^2=x^3+a_4\*x+a_6$ with $a_4$ and $a_6$
in $\F_p[X]/(T)$.  A \kbd{FpXQE} is a point of $E(\F_p[X]/(T))$.

\fun{GEN}{FpXQ_ellj}{GEN a4, GEN a6, GEN T, GEN p}
returns the $j$-invariant of the curve $E$.

\fun{GEN}{FpXQ_ellcard}{GEN a4, GEN a6, GEN T, GEN p}
Return the order of the group $E(\F_p[X]/(T))$.

\fun{GEN}{FpXQ_ellgroup}{GEN a4, GEN a6, GEN N, GEN T, GEN p, GEN *pt_m}
Return the group structure $D$ of the group $E(\F_p[X]/(T))$,
which is assumed to be of order $N$ and set $*pt_m=m$.

\fun{GEN}{FpXQ_ellgens}{GEN a4, GEN a6, GEN ch, GEN D, GEN m, GEN T, GEN p}
Returns generators of the group $E(\F_p[X]/(T))$ with the base change \kbd{ch}
(see \kbd{FpXQE\_changepoint}), where $D$ and $m$ are as returned by
\kbd{FpXQ\_ellgroup}.

\fun{GEN}{FpXQ_elldivpol}{GEN a4, GEN a6, long n, GEN T, GEN p} returns the $n$-division
polynomial of the elliptic curve $E$.

\fun{GEN}{Fq_elldivpolmod}{GEN a4, GEN a6, long n, GEN h, GEN T, GEN p} returns the $n$-division
polynomial of the elliptic curve $E$ modulo the polynomial $h$.

\subsec{\kbd{FpXQE}}

\fun{GEN}{FpXQE_changepoint}{GEN P, GEN m, GEN a4, GEN T, GEN p} returns the image
$Q$ of the point $P$ on the curve $E:y^2=x^3+a_4\*x+a_6$ by the coordinate
change $m$ (which is a \kbd{FpXQV}).

\fun{GEN}{FpXQE_changepointinv}{GEN P, GEN m, GEN a4, GEN T, GEN p} returns the image
$Q$ on the curve $E:y^2=x^3+a_4\*x+a_6$ of the point $P$ by the inverse of the
coordinate change $m$ (which is a \kbd{FpXQV}).

\fun{GEN}{FpXQE_add}{GEN P, GEN Q, GEN a4, GEN T, GEN p}

\fun{GEN}{FpXQE_sub}{GEN P, GEN Q, GEN a4, GEN T, GEN p}

\fun{GEN}{FpXQE_dbl}{GEN P, GEN a4, GEN T, GEN p}

\fun{GEN}{FpXQE_neg}{GEN P, GEN T, GEN p}

\fun{GEN}{FpXQE_mul}{GEN P, GEN n, GEN a4, GEN T, GEN p}

\fun{GEN}{random_FpXQE}{GEN a4, GEN a6, GEN T, GEN p}

\fun{GEN}{FpXQE_log}{GEN P, GEN G, GEN o, GEN a4, GEN T, GEN p} Let \kbd{G} be a
point of order \kbd{o}, return $e$ such that $e.P=G$. If $e$ does not exists,
the result is currently undefined.

\fun{GEN}{FpXQE_order}{GEN P, GEN o, GEN a4, GEN T, GEN p} returns the order of $P$ in
the group $E(\F_p[X]/(T))$, where $o$ is a multiple of the order of $P$, or its
factorization.

\fun{GEN}{FpXQE_tatepairing}{GEN P, GEN Q, GEN m, GEN a4, GEN T, GEN p} returns the
Tate pairing of the point of $m$-torsion $P$ and the point $Q$.

\fun{GEN}{FpXQE_weilpairing}{GEN P, GEN Q, GEN m, GEN a4, GEN T, GEN p} returns the
Weil pairing of the points of $m$-torsion $P$ and $Q$.

\fun{GEN}{RgE_to_FpXQE}{GEN P, GEN T, GEN p} returns the \kbd{FpXQE} obtained by applying
\kbd{Rg\_to\_FpXQ} coefficientwise.

\section{Other curves}

The following functions deal with hyperelliptic curves in weighted projective
space $\P_{(1,d,1)}$, with coordinates $(x,y,z)$ and a model of the form
$ y^2 = T(x,z)$, where $T$ is homogeneous of degree $2d$, and squarefree.
Thus the curve is nonsingular of genus $d-1$.

\fun{long}{hyperell_locally_soluble}{GEN T, GEN p} assumes that $T\in\Z[X]$ is
integral. Returns $1$ if the curve is locally soluble over $\Q_p$, $0$
otherwise.

\fun{long}{nf_hyperell_locally_soluble}{GEN nf, GEN T, GEN pr} let $K$
be a number field, associated to \kbd{nf}, \kbd{pr} a \var{prid} associated
to some maximal ideal $\goth{p}$; assumes that $T\in\Z_K[X]$ is integral.
Returns $1$ if the curve is locally soluble over $K_{\goth{p}}$.

\newpage
