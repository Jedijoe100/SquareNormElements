% $Id$
% Copyright (c) 2000  The PARI Group
%
% This file is part of the PARI/GP documentation
%
% Permission is granted to copy, distribute and/or modify this document
% under the terms of the GNU General Public License
\chapter{Technical Reference Guide for Elliptic curves and arithmetic geometry}

This chapter is quite short, but is added as a placeholder, since
we expect the library to expand in that direction.

\section{Elliptic curves}
Elliptic curves are represented in the Weierstrass model
$$ (E): y^2z + a_1xyz + a_3 yz = x^3 + a_2 x^2z + a_4 xz^2 + a_6z^3, $$
by the $5$-tuple $[a_1,a_2,a_3,a_4,a_6]$. Points in the projective
plane are represented as follows: the point at infinity $(0:1:0)$ is coded
as \kbd{[0]}, a finite point $(x:y:1)$ outside the projective line at infinity
$z = 0$ is coded as $[x,y]$. Note that other points at infinity than $(0:1:0)$
cannot be represented; this is harmless, since they do not belong to any of
the elliptic curves $E$ above.

\emph{Points on the curve} are just projective points as described above,
they are not tied to a curve in any way: the same point may be used in
conjunction with different curves, provided it satisfies their equations (if
it does not, the result is usually undefined). In particular, the point at
infinity belongs to all elliptic curves.

As with \tet{factor} for polynomial factorization, the $5$-tuple
$[a_1,a_2,a_3,a_4,a_6]$ implicitly defines a base ring over which the curve
is defined. Point coordinates must be operation-compatible with this
base ring (\kbd{gadd}, \kbd{gmul}, \kbd{gdiv} involving them should not give
errors).

\subsec{Types of elliptic curves}

There are three types of elliptic curves structures: by increasing order of
complexity, \kbd{ell5} (a $5$-tuple as above), \kbd{smallell} (containing
algebraic data defined over any domain), and \kbd{ell} (contains additional
analytic data for curves defined over $\R$ or $\Q_p$). The last two types are
produced by :

\fun{GEN}{smallellinit}{GEN x}

\fun{GEN}{ellinit}{GEN x, long prec}, where $x$ is an \var{ell5}

The last function \kbd{ellinit} generates a $p$-adic
curve if and only if one of $a_1$, $a_2$, $a_3$, $a_4$, $a_6$ has type
\typ{PADIC}, at the accuracy which is the minimum of their $p$-adic accuracy;
otherwise a curve over $\R$. You may also call directly the underlying
functions, which are not memory-clean:

\fun{GEN}{ellinit_padic}{GEN x, GEN p, long e} initializes an \var{ell} over
$\Q_p$, computing mod $p^e$. In this case the entries of $x$
may have arbitrary type, provided that they can be converted to \typ{PADIC}s
of accuracy $e$ (via \kbd{cvtop}). Shallow function.

\fun{GEN}{ellinit_real}{GEN x, long prec} initializes an \var{ell} over
$\R$. Shallow function.

\subsec{Extracting info from an \kbd{ell} structure}

These functions expect either a \kbd{smallell} or an \kbd{ell} argument.
Both $p$-adic adic and real curves are supported in the latter case.

\fun{GEN}{ell_get_a1}{GEN e}

\fun{GEN}{ell_get_a2}{GEN e}

\fun{GEN}{ell_get_a3}{GEN e}

\fun{GEN}{ell_get_a4}{GEN e}

\fun{GEN}{ell_get_a6}{GEN e}

\fun{GEN}{ell_get_b2}{GEN e}

\fun{GEN}{ell_get_b4}{GEN e}

\fun{GEN}{ell_get_b6}{GEN e}

\fun{GEN}{ell_get_b8}{GEN e}

\fun{GEN}{ell_get_c4}{GEN e}

\fun{GEN}{ell_get_c6}{GEN e}

\fun{GEN}{ell_get_disc}{GEN e}

\fun{GEN}{ell_get_j}{GEN e}

\fun{GEN}{ell_get_roots}{GEN e}

\subsec{Type checking}

\fun{void}{checkell}{GEN e} raise an error unless $e$ is a \var{ell}.

\fun{void}{checksmallell}{GEN e} raise an error unless $e$ is an \var{ell}
or a \var{smallell}.

\fun{void}{checkell5}{GEN e} raise an error unless $e$ is an \var{ell},
a \var{smallell} or an \var{ell5}.

\fun{int}{ell_is_padic}{GEN e} return $1$ if $e$ is an \var{ell} defined
over $\Q_p$.

\fun{int}{ell_is_real}{GEN e} return $1$ if $e$ is an \var{ell} defined over
$\R$.

\fun{void}{checkell_real}{GEN e} combines \tet{checkell} and \tet{ell_is_real}.

\fun{void}{checkell_padic}{GEN e} combines \tet{checkell} and
\tet{ell_is_padic}.

\fun{void}{checkellpt}{GEN z} raise an error unless $z$ is a point
(either finite or at infinity).

\subsec{Points}

\fun{int}{ell_is_inf}{GEN z} tests whether the point $z$ is the point at
infinity.

\fun{GEN}{ellinf}{} returns the point at infinity \kbd{[0]}.

\subsec{Point counting}

\fun{GEN}{ellap}{GEN E, GEN p} computes  the  trace  of  Frobenius  $a_p
= p+1 - \#E(\F_p)$  for the elliptic curve $E/\F_p$  and the prime number p. The
coefficients of the curve may belong to an arbitrary domain that \tet{Rg_to_Fp}
can handle. The equation must be minimal at $p$.

\fun{GEN}{ellsea}{GEN E, GEN p, long s} available if the \kbd{seadata}
package is installed. This function returns directly $\#E(\F_p)$, by
computing it modulo $\ell$ for many small $\ell$; it is called by
\kbd{ellap}: same conditions as above for $E$. The extra flag
\kbd{s}, if set to a non-zero value, causes the computation to
return \kbd{gen\_0} (an impossible cardinality) if one of the small primes
$\ell>s$ divides the curve order. For cryptographic applications, where one is
usually interested in curves of prime order, setting $s=1$ efficiently weeds
out most uninteresting curves (if curves of order twice a prime are acceptable
set $s=2$); there is no guarantee that the resulting cardinality is prime, only
that it has no small prime divisor larger than $s$.

\section{Other curves}

The following functions deal with hyperelliptic curves in weighted projective
space $\P_{(1,d,1)}$, with coordinates $(x,y,z)$ and a model of the form
$ y^2 = T(x,z)$, where $T$ is homogeneous of degree $2d$, and squarefree.
Thus the curve is nonsingular of genus $d-1$.

\fun{long}{hyperell_locally_soluble}{GEN T, GEN p} assumes that $T\in\Z[X]$ is
integral. Returns $1$ if the curve is locally soluble over $\Q_p$, $0$
otherwise.

\fun{long}{nf_hyperell_locally_soluble}{GEN nf, GEN T, GEN pr} let $K$
be a number field, associated to \kbd{nf}, \kbd{pr} a \var{prid} associated
to some maximal ideal $\goth{p}$; assumes that $T\in\Z_K[X]$ is integral.
Returns $1$ if the curve is locally soluble over $K_{\goth{p}}$.

\newpage
