% $Id$
\chapter{Overview of the PARI system}

\section{Introduction}

\noindent
The PARI system is a package which is capable of doing formal computations
on recursive types at high speed; it is primarily aimed at number
theorists, but can be used by anybody whose primary need is speed.

Although quite an amount of symbolic manipulation is possible in PARI, this
system does very badly compared to much more sophisticated systems like
Axiom, Macsyma, Maple, Mathematica or Reduce on such manipulations
(e.g.~multivariate polynomials, formal integration, etc\dots). On the other
hand, the three main advantages of the system are its speed (which can be
between 5 and 100 times better on many computations), the possibility of
using directly data types which are familiar to mathematicians, and its
extensive algebraic number theory module which has no equivalent in the
above-mentioned systems.

It is possible to use PARI in two different ways:

\quad 1) as a library, which can be called from an upper-level language
application (for instance written in C, C$++$, Pascal or Fortran);

\quad 2) as a sophisticated programmable calculator, named {\bf GP}, which
contains most of the control instructions of a standard language like C.

The use of GP is explained in chapters 2 and 3, and the programming in library
mode is explained in chapters 3, 4 and 5. In the present Chapter 1, we give
an overview of the system.

\subsectitle{Important note:} A tutorial for GP is provided in the standard
distribution (\kbd{tutorial.dvi}) and you should read this first (at
least the beginning of it, you can skip the specialized topics you're not
interested in). You can then start over and read the more boring stuff which
lies ahead. But you should do that eventually, at the very least the various
Chapter headings. You can have a quick idea of what is available by looking
at the GP reference card (\kbd{refcard.dvi} or \kbd{refcard.ps}). In case
of need, you can then refer to the complete function description in Chapter 3.

\subsectitle{How to get the latest version?}

\noindent
This package can be obtained by anonymous ftp from quite a number of sites
(ask \kbd{archie} or your favourite Web search engine for the site nearest to
you). But, if you want the very latest version (including development
versions), you should use the anonymous ftp address

\kbd{ftp://megrez.math.u-bordeaux.fr/pub/pari}

\noindent
where you will find all the different ports and possibly some
binaries. A lot of version information, mailing list archives, and various
tips can be found on PARI's (fledgling) home page:

\kbd{\wwwsite}

\subsectitle{Implementation notes:} (You can skip this section and switch to
\secref{se:start} if you're not interested in hardware technicalities. You
won't miss anything that would be mentioned here.)

The PARI package contains essentially three versions. The first one is a
specific implementation for 680x0 based computers which contains a kernel
(for the elementary arithmetic operations on multiprecise integers and real
numbers, and binary/decimal conversion routines) entirely written in
MC68020 assembly language (around 6000 lines), the rest being at present
entirely written in ANSI C with a C++-compatible syntax. The system runs on
SUN-3/xx, Sony News, NeXT cubes and on 680x0 based Macs with x$\ge$2. It
should be very easy to port on any other 680x0 based machine like for
instance the Apollo Domain workstations.

Note that the assembly language source code uses the SUN syntax, which for
some strange reason differs from the Motorola standard used by most other
680x0 machines in the world. In the Mac distribution, we have included a
program which automatically converts from the SUN syntax into the standard
one, at least for the needed PARI assembly file. On the Unix distribution,
we have included other versions of the assembly file, using different
syntaxes. {\bf This version is not really maintained anymore since we lack
the hardware to update/test it.}

The second version is a version where most of the kernel routines are written
in C, but the time-critical parts are written in a few hundred lines
of assembler at most. At present there exist three versions for the Sparc
architecture: one for Sparc version 7 (e.g.~Sparcstation 1, 1+, IPC, IPX or 2),
one for Sparc version 8 with supersparc processors (e.g.~Sparcstation 10
and 20) and one for Sparc version 8 with microsparc I or II processors
(e.g.~Sparcclassic or Sparcstation 4 and 5). No specific version is written
for the Ultrasparc since it can use the microsparc II version. In addition,
versions exist for the HP-PA architecture, for the PowerPC architecture
(only for the 601), for the Intel family starting at the 386 (under Linux,
OS/2, MSDOS, or Windows), and finally for the DEC Alpha 64-bit processors.

Finally, a third version is written entirely in C, and should be portable
without much trouble to any 32 or 64-bit computer having no real memory
constraints. It is about 2 times slower than versions with a small assembly
kernel. This version has been tested for example on MIPS based DECstations
3100 and 5000 and SGI computers.

In addition to Unix workstations and Macs, PARI has been ported to a
considerable number of smaller and larger machines, for example the VAX,
68000-based machines like the Atari, Mac Classic or Amiga 500, 68020 machines
such as the Amiga 2500 or 3000, and even to MS-DOS 386 or better machines,
using the \tet{EMX} port of the GNU C compiler and DOS-extender.

\section{The PARI types}
\label{se:start}\sidx{types}

\noindent
The crucial word in PARI is \idx{recursiveness}: most of the types it knows
about are recursive. For example, the basic type {\bf Complex} exists (actually
called \typ{COMPLEX}). However, the components (i.e.~the real and imaginary
part) of such a ``complex number'' can be of any type. The only sensible ones
are integers (we are then in $\Z[i]$), rational numbers ($\Q[i]$), real
numbers ($\R[i]=\C$), or even elements of $\Z/n\Z$ ($(\Z/n\Z)[i]$ when this
makes sense), or $p$-adic numbers when $p\equiv 3 \mod 4$ ($\Q_{p}[i]$).

This feature must of course not be used too rashly: for example you are in
principle allowed to create objects which are ``complex numbers of complex
numbers'', but don't expect PARI to make sensible use of such objects: you
will mainly get nonsense.

On the other hand, one thing which {\it is\/} allowed is to have components
of different, but compatible, types. For example, taking again complex
numbers, the real part could be of type integer, and the imaginary part of
type rational.

By compatible, we mean types which can be freely mixed in operations like $+$
or $\times$. For example if the real part is of type real, the imaginary part
cannot be of type integermod (integers modulo a given number $n$).

Let us now describe the types. As explained above, they are built recursively
from basic types which are as follows. We use the letter $T$ to designate any
type; the symbolic names correspond to the internal representations of the
types.\medskip
\settabs\+xxx&typexxxxxxxxxxxx&xxxxxxxxxxxxxxxx&xxxxxxxxxxxxxxxxxxxxxxxxxxxxxxxxxxxxxxxxxxxxxxxx\cr
%
\+&type \tet{t_INT}& $\Z$& Integers (with
arbitrary precision)\sidx{integer}\cr
%
\+&type \tet{t_REAL}& $\R$& Real numbers
(with arbitrary precision)\sidx{real number}\cr
%
\+&type \tet{t_INTMOD}& $\Z/n\Z$&
Integermods (integers modulo $n$)\sidx{integermod}\cr
%
\+&type \tet{t_FRAC}& $\Q$& Rational numbers
(in irreducible form)\sidx{rational number}\cr
%
\+&type \tet{t_FRACN}& $\Q$& Rational numbers
(not necessarily in irreducible form)\cr
%
\+&type \tet{t_COMPLEX}& $T[i]$& Complex
numbers\sidx{complex number}\cr
%
\+&type \tet{t_PADIC}& $\Q_p$&
$p$-adic\sidx{p-adic number} numbers\cr
%
\+&type \tet{t_QUAD}& $\Q[w]$& Quadratic Numbers
(where $[\Z[w]:\Z]=2$)\sidx{quadratic number}\cr
%
\+&type \tet{t_POLMOD}& $T[X]/P(X)T[X]$&
Polmods (polynomials modulo $P$)\sidx{polmod}\cr
%
\+&type \tet{t_POL}& $T[X]$& Polynomials
\sidx{polynomial}\cr
%
\+&type \tet{t_SER}& $T((X))$& Power series
(finite Laurent series)\sidx{power series}\cr
%
\+&type \tet{t_RFRAC}& $T(X)$& Rational
functions (in irreducible form)\sidx{rational function}\cr
%
\+&type \tet{t_RFRACN}& $T(X)$& Rational functions
(not necessarily in irreducible form)\cr
%
\+&type \tet{t_VEC}& $T^n$& Row (i.e.~horizontal)
vectors\sidx{row vector}\cr
%
\+&type \tet{t_COL}& $T^n$& Column (i.e.~vertical)
vectors\sidx{column vector}\cr
%
\+&type \tet{t_MAT}& ${\cal M}_{m,n}(T)$&
Matrices\sidx{matrix}\cr
%
\+&type \tet{t_LIST}& $T^n$&
Lists\sidx{list}\cr
%
\+&type \tet{t_STR}&     &
Character strings\sidx{string}\cr
\noindent
and where the types $T$ in recursive types can be different in each component.

In addition, there exist types \tet{t_QFR} and \tet{t_QFI} for binary
quadratic forms of respectively positive and negative
discriminants,\sidx{binary quadratic form} which can be used in specific
operations, but which may disappear in future versions.

Every PARI object (called \tet{GEN} in the sequel) belongs to one of these
basic types. Let us have a closer look.

\subsec{Integers and reals}:\sidx{integer}\sidx{real number}
they are of arbitrary and varying length (each number carrying in its
internal representation its own length or precision) with the following mild
restrictions (given for 32-bit machines, the restrictions for 64-bit machines
being so weak as to be considered inexistent): integers must be in absolute
value less than $2^{268435454}$ (i.e.~roughly 80807123 digits).  The
precision of real numbers is also at most 80807123 significant decimal
digits, and the binary exponent must be in absolute value less than
$2^{23}=8388608$.

Note that PARI has been optimized so that it works as fast as possible on
numbers with at most a few thousand decimal digits. In particular, not too
much effort has been put into fancy multiplication techniques (only the
Karatsuba multiplication is implemented). Hence, although it is possible to
use PARI to do computations with $10^7$ decimal digits, much better programs
can be written for such huge numbers.

Integers and real numbers are completely non-recursive types and are
sometimes called the \tet{leaves}.

\subsec{Integermods, rational numbers (irreducible or not),
$p$-adic numbers, polmods, and
rational functions}:\sidx{integermod}\sidx{rational number}\sidx{p-adic number}
\sidx{polmod} these are recursive, but in a restricted way.

For integermods or polmods, there are two components: the modulus, which
must be of type integer (resp.\ polynomial), and the representative number
(resp.\ polynomial).

For rational numbers or rational functions, there are also only two
components: the numerator and the denominator, which must both be of type
integer (resp.\ polynomial).

\def\limproj{{\displaystyle\lim_{\textstyle\longleftarrow}}}

Finally, $p$-adic numbers have three components: the prime $p$, the
``modulus'' $p^k$, and an approximation to the $p$-adic number. Here $\Z_p$
is considered as $\limproj \Z/p^k\Z$, and $\Q_p$ as its field of
fractions.  Like real numbers, the codewords contain an exponent (giving
essentially the $p$-adic valuation of the number) and also the information on
the precision of the number (which is in fact redundant with $p^k$, but is
included for the sake of efficiency).

\subsec{Complex numbers and quadratic numbers}:
\sidx{complex number}\sidx{quadratic number}
quadratic numbers are numbers of the form $a+bw$, where $w$ is such that
$[\Z[w]:\Z]=2$, and more precisely $w=\sqrt d/2$ when $d\equiv 0 \mod 4$,
and $w=(1+\sqrt d)/2$ when $d\equiv 1 \mod 4$, where $d$ is the discriminant
of a quadratic order. Complex numbers correspond to the very important
special case $w=\sqrt{-1}$.\label{se:compquad}

Complex and quadratic numbers are partially recursive: the two components
$a$ and $b$ can be of type integer, real, rational, integermod or $p$-adic,
and can be mixed, subject to the limitations mentioned above. For example,
$a+bi$ with $a$ and $b$ $p$-adic is in $\Q_p[i]$, but this is equal to
$\Q_p$ when $p\equiv 1 \mod 4$, hence we must exclude these $p$ when one
explicitly uses a complex $p$-adic type.

\subsec{Polynomials, power series, vectors, matrices and lists}:
\sidx{polynomial}\sidx{power series}\sidx{vector}\sidx{matrix}
they are completely recursive: their components can be of any type, and types
can be mixed (however beware when doing operations). Note in particular that
a polynomial in two variables is simply a polynomial with polynomial
coefficients.

Note that in the present version \vers{} of PARI, there is a bug in the
handling of power series of power series (i.e.~power series in several
variables).  However power series of polynomials (which are power series in
several variables of a special type) are OK. The reason for this bug is
known, but it is difficult to correct because the mathematical problem itself
contains some amount of imprecision.

\subsec{Strings}: These contain objects just as they would be printed by the
GP calculator.

\subsec{Notes}:

\subsubsec{Exact and imprecise objects}: \sidx{imprecise object}we have
already said that integers and reals are called the \idx{leaves} because they
are ultimately at the end of every branch of a tree representing a PARI
object. Another important notion is that of an {\bf \idx{exact object}}: by
definition, numbers of basic type real, $p$-adic or power series are
imprecise, and we will say that a PARI object having one of these imprecise
types anywhere in its tree is not exact. All other PARI objects will be
called exact. This is a very important notion since no numerical analysis is
involved when dealing with exact objects.

\subsubsec{Scalar types}:\sidx{scalar type} the first nine basic types, from
\typ{INT} to \typ{POLMOD}, will be called scalar types because they
essentially occur as coefficients of other more complicated objects. Note
that type \typ{POLMOD} is used to define algebraic extensions of a base ring,
and as such is a scalar type.

\subsubsec{What is zero?} This is a crucial question in all computer
systems. The answer we give in PARI is the following. For exact types, all
zeros are equivalent and are exact, and thus are usually represented as an
integer \idx{zero}. The problem becomes non-trivial for imprecise types. For
$p$-adics the answer is as follows: every $p$-adic number (including 0) has
an exponent $e$ and a ``mantissa'' (a purist would say a ``significand'') $u$
which is a $p$-adic unit, except when the number is zero (in which case $u$
is zero), the significand having a certain ``precision'' $k$ (i.e.~being
defined modulo $p^k$). Then this $p$-adic zero is understood to be equal to
$O(p^e)$, i.e.~there are infinitely many distinct $p$-adic zeros. The number
$k$ is thus irrelevant.

For power series the situation is similar, with $p$ replaced by $X$, i.e.~a
power series zero will be $O(X^e)$, the number $k$ (here the length of the
power series) being also irrelevant.\label{se:whatzero}

For real numbers, the precision $k$ is also irrelevant, and a real zero will
in fact be $O(2^e)$ where $e$ is now usually a negative binary exponent. This
of course will be printed as usual for a real number ($0.0000\cdots$ in
\kbd{f} format or $0.Exx$ in \kbd{e} format) and not with a $O()$ symbol as
with $p$-adics or power series. With respect to the natural ordering on the
reals we make the following convention: whatever its exponent a real
zero is smaller than any positive number, and any two real zeroes are equal.

\section{Operations and functions}

\subsec{The PARI philosophy}.
The basic philosophy which governs PARI is that operations and functions
should, firstly, give as exact a result as possible, and secondly, be
permitted if they make any kind of sense.

More specifically, if you do an operation (not a transcendental one) between
exact objects, you will get an exact object. For example, dividing 1 by 3
does not give $0.33333\cdots$ as you might expect, but simply the rational
number $(1/3)$. If you really want the result in type real, evaluate $1./3$
or add $0.$ to $(1/3)$.

The result of operations between imprecise objects will be as precise as
possible. Consider for example one of the most difficult cases, that is the
addition of two real numbers $x$ and $y$. The \idx{accuracy} of the result is
{\it a priori\/} unpredictable; it depends on the precisions of $x$ and $y$,
on their sizes (i.e.~their exponents), and also on the size of $x+y$. PARI
works out automatically the right precision for the result, even when it is
working in calculator mode GP where there is a \idx{default precision}.

In particular, this means that if an operation involves objects of
different accuracies, some digits will be disregarded by PARI. It is a
common source of errors to forget, for instance, that a real number is
given as $r + 2^e \varepsilon$ where $r$ is a rational approximation, $e$ a
binary exponent and $\varepsilon$ is a nondescript real number less than 1 in
absolute value\footnote{*}{this is actually not quite true: internally, the
format is $2^b (a + \varepsilon)$, where $a$ and $b$ are integers}. Hence,
any number less than $2^e$ may be treated as an exact zero:

\bprog
? 0.E-28 + 1.E-100
%1 = 0.E-28
? 0.E100 + 1
%2 = 0.E100
@eprog
\noindent As an exercise, if \kbd{a = 2\pow (-100)}, why do \kbd{a + 0.} and
\kbd{a * 1.} differ ?

The second part of the PARI philosophy is that PARI operations are in general
quite permissive. For instance taking the exponential of a vector should not
make sense. However, it frequently happens that a computation comes out with a
result which is a vector with many components, and one wants to get the
exponential of each one. This could easily be done either under GP or in
library mode, but in fact PARI assumes that this is exactly what you want to
do when you take the exponential of a vector, so no work is necessary. Most
transcendental functions work in the same way (see Chapter 3 for details).

An ambiguity would arise with square matrices. PARI always considers that you
want to do componentwise function evaluation, hence to get for example the
exponential of a square matrix you would need to use a function with a
different name, \kbd{matexp} for instance. In the present version \vers, this
is not yet implemented. See however the program in Appendix C, which is a
first attempt for this particular function.

The available operations and functions in PARI are described in detail in
Chapter 3. Here is a brief summary:

\subsec{Standard operations}.

\noindent
Of course, the four standard operators \kbd{+}, \kbd{-}, \kbd{*}, \kbd{/}
exist. It should once more be emphasized that division is, as far as possible,
an exact operation: $4$ divided by $3$ gives \kbd{(4/3)}. In addition to
this, operations on integers or polynomials, like \b{} (Euclidean
division), \kbd{\%} (Euclidean remainder) exist (and for integers, {\b{/}}
computes the quotient such that the remainder has smallest possible absolute
value). There is also the exponentiation operator \kbd{\pow }, when the
exponent is of type integer. Otherwise, it is considered as a transcendental
function. Finally, the logical operators \kbd{!} (\kbd{not} prefix operator),
\kbd{\&\&} (\kbd{and} operator), \kbd{||} (\kbd{or} operator) exist, giving
as results \kbd{1} (true) or \kbd{0} (false). Note that \kbd{\&} and \kbd{|}
are also accepted as synonyms respectively for \kbd{\&\&} and \kbd{||}.
However, there is no bitwise \kbd{and} or \kbd{or}.

\subsec{Conversions and similar functions}.

\noindent
Many conversion functions are available to convert between different types.
For example floor, ceiling, rounding, truncation, etc\dots.  Other simple
functions are included like real and imaginary part, conjugation, norm,
absolute value, changing precision or creating an integermod or a polmod.

\subsec{Transcendental functions}.

\noindent
They usually operate on any object in $\C$, and some also on $p$-adics.
The list is everexpanding and of course contains all the elementary
functions, plus already a number of others. Recall that by extension, PARI
usually allows a transcendental function to operate componentwise on vectors
or matrices.

\subsec{Arithmetic functions}.

\noindent
Apart from a few like the factorial function or the Fibonacci numbers, these
are functions which explicitly use the prime factor decomposition of
integers. The standard functions are included.  In the present version \vers,
a primitive, but useful version of Lenstra's Elliptic Curve Method (ECM) has
been implemented.

There is now a very large package which enables the number theorist to work
with ease in algebraic number fields. All the usual operations on elements,
ideals, prime ideals, etc\dots are available.

More sophisticated functions are also implemented, like solving Thue
equations, finding integral bases and discriminants of number fields,
computing class groups and fundamental units, computing in relative number
field extensions (including explicit class field theory), and also many
functions dealing with elliptic curves over $\Q$ or over local fields.

\subsec{Other functions}.

\noindent
Quite a number of other functions dealing with polynomials (e.g.~finding
complex or $p$-adic roots, factoring, etc), power series (e.g.~substitution,
reversion), linear algebra (e.g.~determinant, characteristic polynomial,
linear systems), and different kinds of recursions are also included. In
addition, standard numerical analysis routines like Romberg integration (open
or closed, on a finite or infinite interval), real root finding (when the
root is bracketed), polynomial interpolation, infinite series evaluation, and
plotting are included.  See the last sections of Chapter~3 for details.
\vfill\eject
