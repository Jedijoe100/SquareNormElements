% $Id$
% Copyright (c) 2000  The PARI Group
%
% This file is part of the PARI/GP documentation
%
% Permission is granted to copy, distribute and/or modify this document
% under the terms of the GNU Free Documentation License
\chapter{Programming PARI in Library Mode}

\section{Introduction: initializations, universal objects}
\label{se:intro4}

\noindent
To be able to use PARI in \idx{library mode}, you must write a C program and
link it to the PARI library. See the installation guide (in Appendix~A)
on how to create and install the library and include files. A sample Makefile
is presented in Appendix~B.

Probably the best way to understand how programming is done is to work
through a complete example. We will write such a program in
\secref{se:prog}. Before doing this, a few explanations are in order.

First, one must explain to the outside world what kind of objects and
routines we are going to use. This is done simply with the statement

\bprog
#include <pari.h>
@eprog
\noindent
This file \tet{pari.h} imports all the necessary constants,
variables and functions, defines some important macros, and also defines the
fundamental type for all PARI objects: the type \teb{GEN}, which is
simply a pointer to \kbd{long}.

\misctitle{Technical note}: we would have liked to define a type \kbd{GEN}
to be a pointer to itself. This unfortunately is not possible in C, except by
using structures, but then the names become unwieldy. The result of this is
that when we use a component of a PARI object, it will be a \kbd{long},
hence will need to be typecast to a \kbd{GEN} again if we want to avoid
warnings from the compiler. This will sometimes be quite tedious, but of
course is trivially done. See the discussion on typecasts in the next
section.

After declaring the use of the file \kbd{pari.h}, the first executable
statement of a main program should be to initialize the PARI system, and in
particular the PARI stack which will be both a scratchboard and a repository
for computed objects. This is done with a call to the function

\fun{void}{pari_init}{long size, long maxprime)}

\noindent The first argument is the number of bytes given to PARI to work
with (it should not reasonably be taken below 500000), and the second is the
upper limit on a precomputed prime number table. If you don't want prime
numbers, just put $\tet{maxprime} = 2$. Be careful because lots a PARI
functions need this table (certainly all the ones of interest to number
theorists). If you wind up with the error message ``not enough precomputed
primes'', try to increase this value.

\noindent We have now at our disposal:

$\bullet$ a large PARI \tev{stack} containing nothing. It's a big
connected chunk of memory whose size you chose when invoking
\tet{pari_init}. All your computations are going to take place here.
When doing large computations, unwanted intermediate results clutter up
memory very fast so some kind of garbage collecting is needed. Most large
systems do garbage collecting when the memory is getting scarce, and this slows
down the performance. In PARI we have taken a different approach: you must do
your own cleaning up when the intermediate results are not needed anymore.
Special purpose routines have been written to do this; we will see later how
(and when, if at all) you should use them.

$\bullet$ the following {\it universal objects\/} (by definition, objects
which do not belong on the stack): the integers 0, 1 and 2 (respectively
called \teb{gzero}, \teb{gun}, and \teb{gdeux}), the
fraction $\dfrac{1}{2}$ (\teb{ghalf}), the complex number $i$
(\teb{gi}). All of these are of type \kbd{GEN}.

In addition, space is reserved for the polynomials $x_v$
\sidx{variable}
(\teb{polx}\kbd{[$v$]}), and the polynomials $1_v$ (\teb{polun}\kbd{[$v$]}).
Here, $x_v$ is the name of variable number $v$, where $0\le v\le
\tet{MAXVARN}$ (the exact value of which depends on your machine, at least
16383 in any case). Both \kbd{polun} and \kbd{polx} are arrays of
\kbd{GEN}s, the index being the polynomial variable number.

However, except for the ones corresponding to variables $0$ and \kbd{MAXVARN},
these polynomials are {\it not\/} created upon initialization. It
is the programmer's responsibility to fill them before use. We'll see how
this is done in \secref{se:vars} ({\it never\/} through direct assignment).

$\bullet$ a {\it \idx{heap}\/} which is just a linked list of permanent
universal objects. For now, it contains exactly the ones listed above. You
will probably very rarely use the heap yourself; and if so, only as a
collection of individual copies of objects taken from the stack
(called \idx{clone}s in the sequel). Thus you need not bother with its
internal structure, which may change as PARI evolves. Some complex PARI
functions may create clones for special garbage collecting purposes, usually
destroying them when returning.

$\bullet$ a table of primes (in fact of {\it differences\/} between
consecutive primes), called \teb{diffptr}, of type \kbd{byteptr}
(pointer to \kbd{unsigned char}). Its use is described in appendix~C.

$\bullet$ access to all the built-in functions of the PARI library.
These are declared to the outside world when you include \kbd{pari.h}, but
need the above things to function properly. So if you forget the call to
\tet{pari_init}, you will immediately get a fatal error when running your
program (usually a segmentation fault).

\section{Important technical notes}

\subsec{Typecasts}.\label{se:typecast}\sidx{typecast}

\noindent
We have seen that, due to the non-recursiveness of the PARI types from the
compiler's point of view, many typecasts will be necessary when programming
in PARI. To take an example, a vector $V$ of dimension 2 (two components)
will be represented by a chunk of memory pointed to by the \kbd{GEN}~\kbd{V}.
\kbd{V[0]} contains coded information, in particular about the type of the
object, its length, etc. \kbd{V[1]} and \kbd{V[2]} contain pointers to
the two components of \kbd{V}. Those coefficients \kbd{V[i]} themselves are in
chunks of memory whose complexity depends on their own types, and so on. This
is where typecasting will be necessary: a priori, \kbd{V[i]} (for
$\kbd{i}=1,2$) is a \kbd{long}, but we will want to use it as a \kbd{GEN}.
The following two constructions will be exceedingly frequent (\kbd{x} and
\kbd{V} are \kbd{GEN}s):
%
\bprog
  V[i] = (long) x;
  x = (GEN) V[i];
@eprog
\noindent Note that a typecast is not a valid lvalue (cannot be put on the
left side of an assignment), so \kbd{(GEN)V[i] = x} would be incorrect, though
some compilers may accept it.

Due to this annoyance, the PARI functions and variables that occur most
frequently have analogues which are macros including the typecast. The complete
list can be found in the file \kbd{paricast.h} (which is included by
\kbd{pari.h} and can be found at the same place). For instance you can
abbreviate:
%
\bprog
(long) gzero     ----->  zero
(long) gun       ----->  un
(long) polx[v]   ----->  lpolx[v]
(long) gadd(x,y) ----->  ladd(x,y)
@eprog\noindent%
\sidx{zero}%
\sidx{un}%
\sidx{lpolx}
In general, replacing a leading \kbd{g} by an \kbd{l} in the name of a PARI
function will typecast the result to \kbd{long}. Note that \kbd{ldiv} is an
ANSI C function which is is hidden in PARI by a macro of the same name
representing \kbd{(long)gdiv}.

The macro $\teb{coeff}(x,m,n)$ exists with exactly the meaning of \kbd{x[m,n]}
under GP when \kbd{x} is a matrix. This is a purely syntactical trick
to reduce the number of typecasts and thus does not create a copy of the
coefficient (contrary to all the library {\it functions\/}). It can be put on
the left side of an assignment statement, and its value, of type \kbd{long}
integer, is a pointer to the desired coefficient object. The macro
\kbd{gcoeff} is a synonym for \kbd{(GEN) coeff}, hence cannot be put on
the left side of an assignment.

To retrieve the values of elements of lists of \dots\ of
lists of vectors, without getting infuriated by gigantic lists of typecasts,
we have the \teb{mael} macros (for {\bf m}ultidimensional {\bf a}rray {\bf
el}ement). The syntax is $\key{mael}n(x,a_1,\dots,a_n)$, where $x$ is a
\kbd{GEN}, the $a_i$ are indexes, and $n$ is an integer between $2$ and $5$
(with a standalone \kbd{mael} as a synonym for \kbd{mael2}). This stands
for $x[a_1][a_2]\dots[a_n]$ (with all the necessary typecasts), and returns a
\kbd{long} (i.e.~they are valid lvalues). The $\kbd{gmael}n$ macros are
synonyms for $\kbd{(GEN) mael}n$. Note that due to the implementation of matrix
types in PARI (i.e.~as horizontal lists of vertical vectors), \kbd{coeff(x,y)}
is actually completely equivalent to \kbd{mael(y,x)}. It is suggested that
you use \kbd{coeff} in matrix context, and \kbd{mael} otherwise.

\subsec{Variations on basic functions}.\label{se:low_level} In the library
syntax descriptions in Chapter~3, we have only given the basic names of the
functions. For example \kbd{gadd}$(x,y)$ assumes that $x$ and $y$ are PARI
objects (of type \kbd{GEN}), and {\it creates\/} the result $x+y$ on the PARI
stack. For most of the basic operators and functions, many other variants
are available. We give some examples for \kbd{gadd}, but the same is true for
all the basic operators, as well as for some simple common functions (a
more complete list is given in Chapter~5):

\fun{GEN}{gaddgs}{GEN x, long y}

\fun{GEN}{gaddsg}{long x, GEN y}

\noindent In the following three, \kbd{z} is a preexisting \kbd{GEN} and the
result of the corresponding operation is put into~\kbd{z}. The size of the PARI
stack does not change:

\fun{void}{gaddz}{GEN x, GEN y, GEN z}

\fun{void}{gaddgsz}{GEN x, long y, GEN z}

\fun{void}{gaddsgz}{GEN x, GEN y, GEN z}

\noindent There are also low level functions which are special cases of the
above:

\fun{GEN}{addii}{GEN x, GEN y}: here $x$ and $y$ are \kbd{GEN}s of type
\typ{INT} (this is not checked).

\fun{GEN}{addrr}{GEN x, GEN y}: here $x$ and $y$ are \kbd{GEN} reals
(type \typ{REAL}).

\noindent
There also exist functions \teb{addir}, \teb{addri}, \teb{mpadd} (whose
two arguments can be of type integer or real), \teb{addis} (to add a \typ{INT}
and a \kbd{long}) and so on.

All these functions can of course be called by the user but we feel that
the few microseconds lost in calling more general functions (in this case
\kbd{gadd}) are compensated by the fact that one needs to remember a much
smaller number of functions, and also because there is a hidden danger here:
the types of the objects that you use, if they are themselves results of a
previous computation, are not completely predetermined. For instance the
multiplication of a type real \typ{REAL} by a type integer \typ{INT}
{\it usually\/} gives a result of type real, except when the integer is~0, in
which case according to the PARI philosophy the result is the exact integer~0.
Hence if afterwards you call a function which specifically needs a real
type argument, you are going to be in trouble.

If you really want to use these functions, their names are self-explanatory
once you know that {\bf i} stands for a PARI integer, {\bf r} for a PARI
real, {\bf mp} for i or r, {\bf s} for an ordinary signed long, whereas {\bf
z} (as a suffix) means that the result is not created on the PARI
stack but assigned to a preexisting GEN object passed as an extra argument.

For completeness, Chapter 5 gives a description of all these
low-level functions.

Please note that in the present version \vers{} the names of the functions
are not always consistent. This will be changed. Hence anyone programming in
PARI must be aware that the names of almost all functions that he uses might
be subject to change. If the need arises (i.e.~if there really are people out
there who delve into the innards of PARI), updated versions with no name
changes will be released.

\subsec{Portability: 32-bit / 64-bit architectures}.

\noindent
PARI supports both 32-bit and 64-bit based machines, but not simultaneously!
The library will have been compiled assuming a given architecture (a~priori
following a guess by the \kbd{Configure} program, see Appendix~A), and some
of the header files you include (through \kbd{pari.h}) will have been modified
to match the library.

Portable macros are defined to bypass most machine dependencies. If you
want your programs to run identically on 32-bit and 64-bit machines, you will
have to use these, and not the corresponding numeric values, whenever the
precise size of your \kbd{long} integers might matter. Here are the most
important ones:

\settabs\+
-----------------------------&---------------&------------&\cr
\+
                    & 64-bit  & 32-bit
\cr\+
\tet{BITS_IN_LONG}  & 64      & 32
\cr\+
\tet{LONG_IS_64BIT} & defined & undefined
\cr\+
\tet{DEFAULTPREC}   & 3       & 4 & ($\approx$ 19 decimal digits, %
 see formula below)
\cr\+
\tet{MEDDEFAULTPREC}& 4       & 6 & ($\approx$ 38 decimal digits)
\cr\+
\tet{BIGDEFAULTPREC}& 5       & 8 & ($\approx$ 57 decimal digits)
\cr
\noindent
For instance, suppose you call a transcendental function, such as

\kbd{GEN \key{gexp}(GEN x, long prec)}.

\noindent The last argument \kbd{prec} is only used if \kbd{x} is an exact
object (otherwise the relative precision is determined by the precision
of~\kbd{x}). But since \kbd{prec} sets the size of the inexact result counted
in (\kbd{long}) {\it words\/} (including codewords), the same value of
\kbd{prec} will yield different results on 32-bit and 64-bit machines. Real
numbers have two codewords (see~\secref{se:impl}), so the formula for
computing the bit accuracy is
$$ \tet{bit_accuracy}(\kbd{prec}) = (\kbd{prec} - 2) * \tet{BITS_IN_LONG}$$
(this is actually the definition of a macro). The corresponding accuracy
expressed in decimal digits would be
%
$$ \kbd{bit\_accuracy(prec)} * \log(2) / \log(10).$$
%
For example if the value of \kbd{prec} is 5, the corresponding accuracy for
32-bit machines is $(5-2)*\log(2^{32})/\log(10)\approx 28$ decimal digits,
while for 64-bit machines it is $(5-2)*\log(2^{64})/\log(10)\approx 57$
decimal digits.

Thus, you must take care to change the \kbd{prec} parameter you are supplying
according to the bit size, either using the default precisions given by the
various \kbd{DEFAULTPREC}s, or by using conditional constructs of the form:
%
\bprog
#ifndef LONG_IS_64BIT
  prec = 4;
#else
  prec = 6;
#endif
@eprog
\noindent which is in this case equivalent to the statement
\kbd{prec = MEDDEFAULTPREC;}.

Note that for parity reasons, half the accuracies available on 32-bit
architectures (the odd ones) have no precise equivalents on 64-bit machines.

\section{Creation of PARI objects, assignments, conversions}

\subsec{Creation of PARI objects}.\sidx{creation}
The basic function which creates a PARI object is the function
\teb{cgetg} whose prototype is:

\kbd{GEN \key{cgetg}(long length, long type)}.

\noindent
Here \kbd{length} specifies the number of longwords to be allocated to the
object, and type is the type number of the object, preferably in symbolic
form (see \secref{se:impl} for the list of these). The precise effect of
this function is as follows: it first creates on the PARI {\it stack\/} a
chunk of memory of size \kbd{length} longwords, and saves the address of the
chunk which it will in the end return.  If the stack has been used up, a
message to the effect that ``the PARI stack overflows'' will be printed,
and an error raised. Otherwise, it sets the type and length of the PARI object.
In effect, it fills correctly and completely its first codeword (\kbd{z[0]} or
\kbd{*z}).  Many PARI objects also have a second codeword (types \typ{INT},
\typ{REAL},  \typ{PADIC}, \typ{POL}, and \typ{SER}). In case you want to
produce one of those from scratch (this should be exceedingly rare), {\it it
is your responsibility to fill this second codeword}, either explicitly (using
the macros described in \secref{se:impl}), or implicitly using an assignment
statement (using \kbd{gaffect}).

Note that the argument \kbd{length} is predetermined for a number of types:
3 for types \typ{INTMOD}, \typ{FRAC}, \typ{FRACN}, \typ{COMPLEX},
\typ{POLMOD}, \typ{RFRAC} and \typ{RFRACN}, 4 for type \typ{QUAD}
and \typ{QFI}, and 5 for type \typ{PADIC} and \typ{QFR}. However for the sake
of efficiency, no checking is done in the function \kbd{cgetg}, so
disasters will occur if you give an incorrect length.

\misctitle{Notes}:
1)  The main use of this function is to prepare for later assigments
(see \secref{se:assign}). Most of the time you will use \kbd{GEN} objects
as they are created and returned by PARI functions. In this case you don't need
to use \kbd{cgetg} to create space to hold them.

\noindent 2) For the creation of leaves, i.e.~integers or reals, which is
very common,

\fun{GEN}{cgeti}{long length}

\fun{GEN}{cgetr}{long length}

\noindent should be used instead of \kbd{cgetg(length, t\_INT)} and
\kbd{cgetg(length, t\_REAL)} respectively.

\noindent 3) The macros \teb{lgetg}, \teb{lgeti}, \teb{lgetr} are
predefined as \kbd{(long)cgetg}, \kbd{(long)cgeti}, \kbd{(long)cgetr},
respectively.

\misctitle{Examples}: 1) \kbd{z = cgeti(DEFAULTPREC)} and
\kbd{cgetg(DEFAULTPREC, t\_INT)} create an integer object whose ``precision''
is \kbd{bit\_accuracy(DEFAULTPREC)} = 64. This means \kbd{z} can hold rational
integers of absolute value less than $2^{64}$. Note that in both cases, the
second codeword will {\it not\/} be filled. Of course we could use numerical
values, e.g.~\kbd{cgeti(4)}, but this would have different meanings on
different machines as \kbd{bit\_accuracy(4)} equals 64 on 32-bit machines,
but 128 on 64-bit machines.

\noindent 2) The following creates a type ``complex'' object whose real and
imaginary parts can hold real numbers of precision
$\kbd{bit\_accuracy(MEDDEFAULTPREC)} = 96\hbox{ bits:}$
%
\bprog
  z = cgetg(3, t_COMPLEX);
  z[1] = lgetr(MEDDEFAULTPREC);
  z[2] = lgetr(MEDDEFAULTPREC);
@eprog

\noindent 3) To create a matrix object for $4\times 3$ matrices:
%
\bprog
  z = cgetg(4, t_MAT);
  for(i=1; i<4; i++) z[i] = lgetg(5, t_COL);
@eprog
%
\noindent If one wishes to create space for the matrix elements themselves,
one has to follow this with a double loop to fill each column vector.

These last two examples illustrate the fact that since PARI types are
recursive, all the branches of the tree must be created. The function
\teb{cgetg} creates only the ``root'', and other calls to \kbd{cgetg} must be
made to produce the whole tree.  For matrices, a common mistake is to think
that \kbd{z = cgetg(4, t\_MAT)} (for example) will create the root of the
matrix: one needs also to create the column vectors of the matrix (obviously,
since we specified only one dimension in the first \kbd{cgetg}!). This is
because a matrix is really just a row vector of column vectors (hence a
priori not a basic type), but it has been given a special type number so that
operations with matrices become possible.

\subsec{Assignments}.
Firstly, if \kbd{x} and \kbd{y} are both declared as \kbd{GEN} (i.e.~pointers
to something), the ordinary C assignment \kbd{y = x} makes perfect sense: we
are just moving a pointer around. However, physically modifying either
\kbd{x} or \kbd{y} (for instance, \kbd{x[1] = 0}) will also change the other
one, which is usually not desirable. \label{se:assign}

\misctitle{Very important note}: Using the functions described in this
paragraph is very inefficient and often awkward: one of the \tet{gerepile}
functions (see~\secref{se:garbage}) should be preferred. See the paragraph
end for some exceptions to this rule.

\noindent
The general PARI \idx{assignment} function is the function \teb{gaffect} with
the following syntax:

\fun{void}{gaffect}{GEN x, GEN y}

\noindent
Its effect is to assign the PARI object \kbd{x} into the {\it preexisting\/}
object \kbd{y}. This copies the whole structure of \kbd{x} into \kbd{y} so
many conditions must be met for the assignment to be possible. For instance
it is allowed to assign an integer into a real number, but the converse is
forbidden.  For that, you must use the truncation or rounding function of
your choice (see section 3.2). It can also happen that \kbd{y} is not large
enough or does not have the proper tree structure to receive the object
\kbd{x}. As an extreme example, assume \kbd{y} is the zero integer with length
equal to 2. Then all assignments of a non-zero integer into \kbd{y} will
result in an error message since \kbd{y} is not large enough to accommodate
a non-zero integer.  In general common sense will tell you what is possible,
keeping in mind the PARI philosophy which says that if it makes sense it is
legal. For instance, the assignment of an imprecise object into a precise one
does {\it not\/} make sense. However, a change in precision of imprecise
objects is allowed.

All functions ending in ``\kbd{z}'' such as \teb{gaddz}
(see~\secref{se:low_level}) implicitly use this function. In fact what they
exactly do is record {\teb{avma}} (see~\secref{se:garbage}), perform the
required operation, \teb{gaffect} the result to the last operand, then
restore the initial \kbd{avma}.

You can assign ordinary C long integers into a PARI object (not necessarily
of type \typ{INT}). Use the function \teb{gaffsg} with the following
syntax:

\fun{void}{gaffsg}{long s, GEN y}

\misctitle{Note}: due to the requirements mentionned above, it's usually
a bad idea to use \tet{gaffect} statements. Two exceptions:

$\bullet$ for simple objects (e.g.~leaves) whose size is controlled, they can
be easier to use than gerepile, and about as efficient.

$\bullet$ to coerce an inexact object to a given precision. For instance
\bprog
gaffect(x, (tmp=cgetr(3))); x = tmp;
@eprog
\noindent at the beginning of a routine where precision can be kept to a
minimum (otherwise the precision of \kbd{x} will be used in all subsequent
computations, which will be a disaster if \kbd{x} is known to thousands of
digits).

\subsec{Copy}. It is also very useful to \idx{copy} a PARI object, not
just by moving around a pointer as in the \kbd{y = x} example, but by
creating a copy of the whole tree structure, without pre-allocating a
possibly complicated \kbd{y} to use with \kbd{gaffect}. The function which
does this is called \teb{gcopy}, with the predefined macro
\teb{lcopy} as a synonym for \kbd{(long)gcopy}. Its syntax is:

\fun{GEN}{gcopy}{GEN x}

\noindent and the effect is to create a new copy of x on the PARI stack.
Beware that universal objects which occur in specific components of certain
types (mainly moduli for types \typ{INTMOD} and \typ{PADIC}) are not
copied, as they are assumed to be permanent. In this case, \kbd{gcopy} only
copies the pointer. Use \fun{GEN}{forcecopy}{GEN x} if you want a complete
copy.

Please be sure at this point that you really understand the difference between
\kbd{y = x}, \kbd{y = gcopy(x)}, and \kbd{gaffect(x,y)}: this will save you
from many ``obvious'' mistakes later on.

\subsec{Clones}.\sidx{clone}
Sometimes, it may be more efficient to create a {\it permanent\/} copy of a
PARI object. This will not be created on the stack but on the heap. The
function which does this is called \teb{gclone}, with the predefined macro
\teb{lclone} as a synonym for \kbd{(long)gclone}. Its syntax is:

\fun{GEN}{gclone}{GEN x}

A clone can be removed from the heap (thus destroyed) using

\fun{void}{gunclone}{GEN x}

\noindent No PARI object should keep references to a clone which has been
destroyed. If you want to copy a clone back to the stack then delete it, use
\tet{forcecopy} and not \tet{gcopy}, otherwise some components might not be
copied (moduli of \typ{INTMOD}s and \typ{POLMOD}s for instance).

\subsec{Conversions}.\sidx{conversions}
The following functions convert C objects to PARI objects (creating them on
the stack as usual):

\fun{GEN}{stoi}{long s}: C \kbd{long} integer  (``small'') to PARI integer
(\typ{INT})

\fun{GEN}{dbltor}{double s}: C \kbd{double} to PARI real (\typ{REAL}). The
accuracy of the result is 19 decimal digits, i.e.~a type \typ{REAL} of
length \kbd{DEFAULTPREC}, although on 32-bit machines only 16 of them will
be significant.

\noindent We also have the converse functions:

\fun{long}{itos}{GEN x}: \kbd{x} must be of type \typ{INT},

\fun{double}{rtodbl}{GEN x}: \kbd{x} must be of type \typ{REAL},

\noindent as well as the more general ones:

\fun{long}{gtolong}{GEN x},

\fun{double}{gtodouble}{GEN x}.

\section{Garbage collection}\label{se:garbage}\sidx{garbage collecting}

\subsec{Why and how}.

\noindent
As we have seen, the \kbd{pari\_init} routine allocates a big range of
addresses (the {\it stack\/}) that are going to be used throughout. Recall that
all PARI objects are pointers. But for universal objects, they will all point
at some part of the stack.

The stack starts at the address \kbd{bot} and ends just before \kbd{top}. This
means that the quantity
%
$$ (\kbd{top} - \kbd{bot})\,/\,\kbd{sizeof(long)} $$
%
is equal to the \kbd{size} argument of \kbd{pari\_init}. The PARI
stack also has a ``current stack pointer'' called \teb{avma}, which
stands for {\bf av}ailable {\bf m}emory {\bf a}ddress. These three variables
are global (declared for you by \kbd{pari.h}). For historical reasons they
are of type \kbd{ulong} and not of type \kbd{GEN} as would seem more natural.

The stack is oriented upside-down: the more recent an object, the closer to
\kbd{bot}. Accordingly, initially \kbd{avma} = \kbd{top}, and \kbd{avma} gets
{\it decremented\/} as new objects are created. As its name indicates,
\kbd{avma} always points just {\it after\/} the first free address on the
stack, and \kbd{(GEN)avma} is always (a pointer to) the latest created object.
When \kbd{avma} reaches \kbd{bot}, the stack overflows, aborting all
computations, and an error message is issued. To avoid this {\it you\/} will
need to clean up the stack from time to time, when some bunch of intermediate
objects will not be needed anymore. This is called ``{\it garbage
collecting}.''

We are now going to describe briefly how this is done. We will see many
concrete examples in the next subsection.

\noindent$\bullet$
First, PARI routines will do their own garbage collecting, which
means that whenever a documented function from the library returns, only its
result(s) will have been added to the stack (non-documented ones may not do
this, for greater speed). In particular, a PARI function that does not return
a \kbd{GEN} does not clutter the stack. Thus, if your computation is small
enough (i.e.~you call few PARI routines, or most of them return \kbd{long}
integers), then you don't need to do any garbage collecting. This will probably
be the case in many of your subroutines. Of course the objects that were on
the stack {\it before\/} the function call are left alone. Except for the ones
listed below, PARI functions only collect their own garbage.

\noindent$\bullet$
It may happen that all objects that were created after a certain point can
be deleted~--- for instance, if the final result you need is not a
\kbd{GEN}, or if some search proved futile. Then, it is enough to record
the value of \kbd{avma} just {\it before\/} the first garbage is created,
and restore it upon exit:

\bprog
ulong ltop = avma; /*@Ccom record initial avma */

garbage ...
avma = ltop; /*@Ccom restore it */
@eprog
\noindent All objects created in the \kbd{garbage} zone will eventually
be overwritten: they should not be accessed anymore once \kbd{avma} has been
restored.

\noindent$\bullet$
If you want to destroy (i.e.~give back the memory occupied by) the
latest PARI object on the stack (e.g.~the latest one obtained from a function
call), and the above method is not available (because the initial value of
\kbd{avma} has been lost), just use the function\sidx{destruction}%
\vadjust{\penalty500}%discourage page break

\fun{void}{cgiv}{GEN z}

\noindent where \kbd{z} is the object you want to give back.

\noindent$\bullet$
Unfortunately life is not so simple, and sometimes you will want
to give back accumulated garbage {\it during\/} a computation without losing
recent data. For this you need the \kbd{gerepile} function (or one of its
variants described hereafter):

\fun{GEN}{gerepile}{ulong ltop, ulong lbot, GEN q}

\noindent
This function cleans up the stack between \kbd{ltop} and \kbd{lbot}, where
$\kbd{lbot} < \kbd{ltop}$, and returns the updated object \kbd{q}. This means:

1) we translate (copy) all the objects in the interval
$[\kbd{avma}, \kbd{lbot}[$, so that its right extremity abuts the address
\kbd{ltop}. Graphically

\vbox{\bprog
             bot           avma   lbot          ltop     top
End of stack  |-------------[++++++[-/-/-/-/-/-/-|++++++++|  Start
                free memory            garbage
@eprog
\noindent becomes:
\bprog
             bot                         avma   ltop     top
End of stack  |---------------------------[++++++[++++++++|  Start
                       free memory
@eprog
}
\noindent where \kbd{++} denote significant objects, \kbd{--} the unused part
of the stack, and \kbd{-/-} the garbage we remove.

2) The function then inspects all the PARI objects between \kbd{avma} and
\kbd{lbot} (i.e.~the ones that we want to keep and that have been translated)
and looks at every component of such an object which is not a codeword. Each
such component is a pointer to an object whose address is either

--- between \kbd{avma} and \kbd{lbot}, in which case it will be suitably
updated,

--- larger than or equal to \kbd{ltop}, in which case it will not change, or

--- between \kbd{lbot} and \kbd{ltop} in which case \kbd{gerepile} will
scream an error message at you (``significant pointers lost in gerepile'').

3) \key{avma} is updated (we add $\kbd{ltop} - \kbd{lbot}$ to the old value).

4) We return the (possibly updated) object \kbd{q}: if \kbd{q} initially
pointed between \kbd{avma} and \kbd{lbot}, we return the translated
address, as in~2). If not, the original address is still valid (and we return
it!).

As stated above, no component of the remaining objects (in particular
\kbd{q}) should belong to the erased segment [\kbd{lbot}, \kbd{ltop}[, and
this is checked within \kbd{gerepile}. But beware as well that the addresses
of all the objects in the translated zone will have changed after a call to
\kbd{gerepile}: every pointer you may have kept around elsewhere, outside the
stack objects, which previously pointed into the zone below \kbd{ltop}
must be discarded. If you need to recover more than one object, use one of
the \kbd{gerepilemany} functions below.

As a consequence of the preceding explanation, we must now state the most
important law about programming in PARI:

{\bf If a given PARI object is to be relocated by \hbox{gerepile} then,
apart from universal objects, the chunks of memory used by its components
should be in consecutive memory locations}. All \kbd{GEN}s created by
documented PARI function are guaranteed to satisfy this.

This is because the \kbd{gerepile} function knows only about {\it two
connected zones\/}: the garbage that will be erased (between \kbd{lbot} and
\kbd{ltop}) and the  significant pointers that will be copied and updated.
If there is garbage interspersed with your objects, disasters will occur when
we try to update them and consider the corresponding ``pointers''. So be
{\it very\/} wary when you allow objects to become disconnected. Have a look
at the examples, it's not as complicated as it seems.

\noindent In practice this is achieved by the following programming idiom:
\bprog
  ltop=avma; garbage(); lbot=avma; q=anything();
  return gerepile(ltop, lbot, q); /*@Ccom returns the updated q */
@eprog

\noindent Beware that
\bprog
  ltop=avma; garbage();
  return gerepile(ltop, avma, anything())
@eprog

\noindent might work, but should be frowned upon. We can't predict whether
\kbd{avma} is going to be evaluated after or before the call to
\kbd{anything()}: it depends on the compiler. If we are out of luck, it will
be {\it after\/} the call, so the result will belong to the garbage zone and
the \kbd{gerepile} statement becomes equivalent to \kbd{avma = ltop}. Thus we
would return a pointer to random garbage.

\noindent$\bullet$ A simple variant is

\fun{GEN}{gerepileupto}{ulong ltop, GEN q}

\noindent which cleans the stack between \kbd{ltop} and the {\it connected\/}
object \kbd{q} and returns \kbd{q} updated. For this to work, \kbd{q} must
have been created {\it before\/} all its components, otherwise they would
belong to the garbage zone! Documented PARI functions guarantee this. If you
stumble upon one that does not, consider it a bug worth reporting.

\noindent$\bullet$
To cope with complicated cases where many objects have to be
preserved, you can use

\fun{void}{gerepilemany}{ulong ltop, GEN *gptr[], long n)}

\noindent which cleans up the most recent part of the stack (between
\kbd{ltop} and \kbd{avma}). All the \kbd{GEN}s pointed at by the elements of
the array \kbd{gptr} (of length \kbd{n}) are updated. A copy is done just
before the cleaning to preserve them, so they don't need to be connected
before the call. This is the most robust of the gerepile functions (the less
prone to user error), but also the slowest.

\noindent$\bullet$ More efficient, but trickier to use is

\fun{void}{gerepilemanysp}{ulong ltop, ulong lbot, GEN *gptr[], long n}

\noindent which cleans the stack between \kbd{lbot} and \kbd{ltop} and
updates the \kbd{GEN}s pointed at by the elements of \kbd{gptr} without doing
any copying. This is subject to the same restrictions as \kbd{gerepile}, the
only difference being that more than one address gets updated.

\subsec{Examples}.

\noindent
Let \kbd{x} and \kbd{y} be two preexisting PARI objects and suppose that we
want to compute $\kbd{x}^2+ \kbd{y}^2$. This can trivially be done using the
following program (we skip the necessary declarations; everything in sight is
a \kbd{GEN}):
\bprog
p1 = gsqr(x);
p2 = gsqr(y); z = gadd(p1,p2);
@eprog\noindent
The \kbd{GEN} \kbd{z} indeed points at the desired quantity. However,
consider the stack: it contains as unnecessary garbage \kbd{p1} and \kbd{p2}.
More precisely it contains (in this order) \kbd{z}, \kbd{p2}, \kbd{p1}
(recall that, since the stack grows downward from the top, the most recent
object comes first). We need a way to get rid of this garbage (in this case
it causes no harm except that it occupies memory space, but in other cases
it could disconnect other PARI objects and this is dangerous).

It would not have been possible to get rid of \kbd{p1}, \kbd{p2} before
\kbd{z} is computed, since they are used in the final operation. We cannot
record \kbd{avma} before \kbd{p1} is computed and restore it later, since
this would destroy \kbd{z} as well. It is not possible either to use the
function \kbd{cgiv} since \kbd{p1} and \kbd{p2} are not at the bottom of the
stack and we don't want to give back~\kbd{z}.

But using \kbd{gerepile}, we can give back the memory locations corresponding
to \kbd{p1}, \kbd{p2}, and move the object \kbd{z} upwards so that no
space is lost. Specifically:
\bprog
  ltop = avma; /*@Ccom remember the current address of the top of the stack */
  p1 = gsqr(x); p2 = gsqr(y);
  lbot = avma; /*@Ccom keep the address of the bottom of the garbage pile */
  z = gadd(p1, p2); /*@Ccom z is now the last object on the stack */
  z = gerepile(ltop, lbot, z); /*@Ccom garbage collecting */
@eprog
\noindent Of course, the last two instructions could also have been
written more simply:

\kbd{z = gerepile(ltop, lbot, gadd(p1,p2));}

\noindent In fact \kbd{gerepileupto} is even simpler to use, because the
result of \kbd{gadd} will be the last object on the stack and \kbd{gadd} is
guaranteed to return an object suitable for \kbd{gerepileupto}:
\bprog
  ltop = avma;
  z = gerepileupto(ltop, gadd(gsqr(x), gsqr(y)));
@eprog
\noindent
As you can see, in simple conditions the use of \kbd{gerepile} is not really
difficult. However make sure you understand exactly what has happened before
you go on (use the figure from the preceding section).

\misctitle{Important remark}: as we will see presently it is often
necessary to do several \kbd{gerepile}s during a computation. However, the
fewer the better.  The only condition for \kbd{gerepile} to work is that the
garbage be connected. If the computation can be arranged so that there is a
minimal number of connected pieces of garbage, then it should be done that
way.

For example suppose we want to write a function of two \kbd{GEN} variables
\kbd{x} and \kbd{y} which creates the vector $\kbd{[x}^2+\kbd{y},
\kbd{y}^2+\kbd{x]}$. Without garbage collecting, one would write:
%
\bprog
  p1 = gsqr(x); p2 = gadd(p1, y);
  p3 = gsqr(y); p4 = gadd(p3, x);
  z = cgetg(3,t_VEC);
  z[1] = (long)p2;
  z[2] = (long)p4;
@eprog
\noindent
This leaves a dirty stack containing (in this order) \kbd{z}, \kbd{p4},
\kbd{p3}, \kbd{p2}, \kbd{p1}. The garbage here consists of \kbd{p1} and
\kbd{p3}, which are separated by \kbd{p2}. But if we compute \kbd{p3}
{\it before\/} \kbd{p2} then the garbage becomes connected, and we get the
following program with garbage collecting:
%
\bprog
  ltop = avma; p1 = gsqr(x); p3 = gsqr(y); lbot = avma;
  z = cgetg(3,t_VEC);
  z[1] = ladd(p1,y);
  z[2] = ladd(p3,x);
  z = gerepile(ltop,lbot,z);
@eprog

\noindent Finishing by \kbd{z = gerepileupto(ltop, z)} would be ok as well.
But when you have the choice, it's usually clearer to brace the garbage
between \kbd{ltop}~/ \kbd{lbot} pairs.

\noindent Beware that
\bprog
  ltop = avma; p1 = gadd(gsqr(x), y); p3 = gadd(gsqr(y), x);
  z = cgetg(3,t_VEC);
  z[1] = (long)p1;
  z[2] = (long)p3
  z = gerepileupto(ltop,z); /*@Ccom WRONG !!! */
@eprog
\noindent would be a disaster since \kbd{p1} and \kbd{p3} would be created
before \kbd{z}, so the call to \kbd{gerepileupto} would overwrite them,
leaving \kbd{z[1]} and \kbd{z[2]} pointing at random data!

We next want to write a program to compute the product of two complex numbers
$x$ and $y$, using a method which takes only 3 multiplications instead of 4.
Let $z = x*y$, and set $x = x_r + i*x_i$ and similarly for $y$ and $z$. The
well-known trick is to compute $p_1 = x_r*y_r$, $p_2=x_i*y_i$,
$p_3=(x_r+x_i)*(y_r+y_i)$, and then we have $z_r=p_1-p_2$,
$z_i=p_3-(p_1+p_2)$. The program is essentially as follows:
%
\bprog
ltop = avma;
p1 = gmul(x[1],y[1]);
p2 = gmul(x[2],y[2]);
p3 = gmul(gadd(x[1],x[2]), gadd(y[1],y[2]));
p4 = gadd(p1,p2); lbot = avma;
z = cgetg(3,t_COMPLEX);
z[1] = lsub(p1,p2);
z[2] = lsub(p3,p4);
z = gerepile(ltop,lbot,z);
@eprog
\noindent
``Essentially'', because for instance \kbd{x[1]} is a \kbd{long} and not a
\kbd{GEN}, so we need to insert many annoying typecasts:
\kbd{p1 = gmul((GEN)x[1], (GEN)y[1])} and so on.

Let us now look at a less trivial example where more than one \kbd{gerepile}
is needed in practice (at the expense of efficiency, one can always use only
one using \kbd{gcopy}; see below). Suppose that we want to write a function
which multiplies a line vector by a matrix (such a function is of course
already part of \kbd{gmul}, but let's ignore this for a moment). Then the
most natural way is to do a \kbd{cgetg} of the result immediately, and then a
\kbd{gerepile} for each coefficient of the result vector to get rid of the
garbage which has accumulated while this particular coefficient was computed.
We leave the details to the reader, who can look at the answer in the file
\kbd{basemath/gen1.c}, in the function \kbd{gmul}, case \typ{VEC} times case
\typ{MAT}. It would theoretically be possible to have a single connected
piece of garbage, but it would be a much less natural and unnecessarily
complicated program.

Let us now take an example which is probably the least trivial way of using
\kbd{gerepile}, but is unfortunately sometimes necessary. Although it is not
an infrequent occurrence, we will not give a specific example but a general
one: suppose that we want to do a computation (usually inside a larger
function) producing more than one PARI object as a result, say two for
instance. Then even if we set up the work properly, before cleaning up we
will have a stack which has the desired results \kbd{z1}, \kbd{z2} (say),
and then connected garbage from lbot to ltop. If we write

\kbd{z1 = gerepile(ltop,lbot,z1);}

\noindent
then the stack will be cleaned, the pointers fixed up, but we will have lost
the address of \kbd{z2}. This is where we need one of the \idx{gerepilemany}
functions: we declare
\bprog
  GEN *gptr[2]; /*@Ccom Array of pointers to GENs */
  gptr[0] = &z1; gptr[1] = &z2;
@eprog
\noindent and now the call \kbd{gerepilemany(ltop, gptr, 2)} copies \kbd{z1}
and \kbd{z2} to new locations, cleans the stack from \kbd{ltop} to the old
\kbd{avma}, and updates the pointers \kbd{z1} and \kbd{z2}. Here we don't
assume anything about the stack: the garbage can be disconnected and
\kbd{z1}, \kbd{z2} need not be at the bottom of the stack. If all of these
assumptions are in fact satisfied, then we can call \kbd{gerepilemanysp}
instead, which will usually be faster since we don't need the initial copy
(on the other hand, it is less cache friendly).

Another important usage is ``random'' garbage collection during loops
whose size requirements we cannot (or don't bother to) control in advance:
\bprog
  ulong ltop = avma, limit = (avma+bot)/2;
  GEN x, y;

  while (...)
  {
    garbage(); x = anything();
    garbage(); y = anything()
    garbage();
    if (avma < limit) /*@Ccom memory is running low (half spent since entry) */
    {
       GEN *gptr[2];
       gptr[0] = &x; gptr[1] = &y;
       gerepilemany(ltop, gptr, 2);
    } 
  } 
@eprog
\noindent Here we assume that only \kbd{x} and \kbd{y} are needed from one
iteration to the next. As it would be too costly to call gerepile once for
each iteration, we only do it when it seems to have become necessary. Of
course, when the need arises, you can use bigger \kbd{gptr} arrays: in the
PARI library source code, we once needed to preserve up to 10 objects at a
time (in a variant of the LLL algorithm)!

\misctitle{Technical note:} the statement \kbd{limit = (avma+bot)/2} is
dangerous since the addition can overflow, which would result in
\kbd{limit} being negative. This will prevent garbage collection in the
loop. To avoid this problem, we provide a robust macro
\tet{stack_lim}\kbd{(avma,$n$)}, which denotes an address where $2^{n-1} /
(2^{n-1}+1)$ of the total stack space is exhausted ($1/2$ for $n=1$, $2/3$
for $n=2$). Hence, the above snippet should be written as

\bprog
  ulong ltop = avma, limit = stack_lim(avma,1);
  @dots
@eprog

\subsec{Some hints and tricks}. In this section, we give some indications
on how to avoid most problems connected with garbage collecting:

First, although it looks complicated, \kbd{gerepile} has turned out to be a
very flexible and fast garbage collector, which compares very favorably
with much more sophisticated methods used in other systems. A few tests that
we have done indicate that the price paid for using \kbd{gerepile}, when
properly used, is usually around 1 or 2 percents of the total time, which is
quite acceptable.

Secondly, in many cases, in particular when the tree structure and the size of
the PARI objects which will appear in a computation are under control, one
can avoid \kbd{gerepile} altogether by creating sufficiently large objects at
the beginning (using \teb{cgetg}), and then using assignment statements and
operations ending with z (such as \teb{gaddz}). Coming back to our first
example, note that if we know that x and y are of type real and of length
less than or equal to 5, we can program without using \kbd{gerepile} at all:

\bprog
  z = cgetr(5); ltop = avma;
  p1 = gsqr(x); p2 = gsqr(y); gaddz(p1,p2,z);
  avma = ltop;
@eprog
\noindent This practice will usually be {\it slower\/} than a craftily used
\kbd{gerepile} though, and is certainly more cumbersome to use. As a rule,
assignment statements should generally be avoided.
\smallskip

Thirdly, the philosophy of \kbd{gerepile} is the following: keep the value of
the stack pointer \kbd{avma} at the beginning, and just {\it before\/} the
last operation.  Afterwards, it would be too late since the lower end address
of the garbage zone would have been lost. Of course you can always use
\kbd{gerepileupto}, but you will have to assume that the object was created
{\it before\/} its components.

Finally, if everything seems hopeless, at the expense of speed you can do the
following: after saving the value of \kbd{avma} in \kbd{ltop}, perform your
computation as you wish, in any order, leaving a messy stack. Let \kbd{z} be
your result. Then write the following:

\kbd{z = gerepileupto(ltop, gcopy(z));}

\noindent
The trick is to force a copy of \kbd{z} to be created at the bottom of the
stack, hence all the rest including the initial \kbd{z} becomes connected
garbage. If you need to keep more than one result, use \kbd{gerepilemany}
(of which the above is just a special case).

\smallskip If you followed us this far, congratulations, and rejoice: the
rest is much easier.

\section{Implementation of the PARI types}
\label{se:impl}

\noindent
Although it is a little tedious, we now go through each type and explain its
implementation. Let \kbd{z} be a \kbd{GEN}, pointing at a PARI object. In
the following paragraphs, we will constantly mix two points of view: on the
one hand, \kbd{z} will be treated as the C pointer it is (in the context of
program fragments like \kbd{z[1]}), on the other, as PARI's handle on (the
internal representation of) some mathematical entity, so we will shamelessly
write $\kbd{z} \ne 0$ to indicate that the {\it value\/} thus represented
is nonzero (in which case the {\it pointer\/}~\kbd{z} certainly will be
non-\kbd{NULL}). We offer no apologies for this style. In fact, you had
better feel comfortable juggling both views simultaneously in your mind if
you want to write correct PARI programs.

Common to all the types is the
first codeword \kbd{z[0]}, which we don't have to worry about since this is
taken care of by \kbd{cgetg}. Its precise structure will depend on the
machine you are using, but it always contain the following data: the
{\it internal \idx{type number}\/} associated to the symbolic type name, the
{\it\idx{length}\/} of the root in longwords, and a technical bit which
indicates whether the object is a clone (see below) or not. This last one is
used by GP for internal garbage collecting, you won't have to worry about it.

\noindent These data can be handled through the following {\it macros\/}:

\fun{long}{typ}{GEN z} returns the type number of \kbd{z}.

\fun{void}{settyp}{GEN z, long n} sets the type number of \kbd{z} to
\kbd{n} (you should not have to use this function if you use \kbd{cgetg}).

\fun{long}{lg}{GEN z} returns the length (in longwords) of the root of \kbd{z}.

\fun{long}{setlg}{GEN z, long l} sets the length of \kbd{z} to \kbd{l} (you
should not have to use this function if you use \kbd{cgetg}; however, see
an advanced example in \secref{se:prog}).

\noindent
(If you know enough about PARI to need to access the ``clone'' bit, then you'll
be able to find usage hints in the code (esp. \kbd{killbloc()} and
\kbd{matrix\_block()}). It {\it is\/} technical after all.)

These macros are written in such a way that you don't need to worry about
type casts when using them: i.e.~if \kbd{z} is a \kbd{GEN}, \kbd{typ(z[2])}
will be accepted by your compiler, without your having to explicitly type
\kbd{typ((GEN)z[2])}. Note that for the sake of efficiency, none of the
codeword-handling macros check the types of their arguments even when there
are stringent restrictions on their use.

The possible second codeword is used differently by the different types, and
we will describe it as we now consider each of them in turn:


\subsec{Type \typ{INT} (integer):}\sidx{integer}\kbdsidx{t_INT} this type has
a second codeword \kbd{z[1]} which contains the following information:

the sign of \kbd{z}: coded as $1$, $0$ or $-1$ if $\kbd{z} > 0$, $\kbd{z} = 0$,
$\kbd{z} < 0$ respectively.

the {\it effective length\/} of \kbd{z}, i.e.~the total number of significant
longwords. This means the following: apart from the integer 0, every integer
is ``normalized'', meaning that the first mantissa longword (i.e.~\kbd{z[2]})
is non-zero. However, the integer may have been created with a longer length.
Hence the ``length'' which is in \kbd{z[0]} can be larger than the
``effective length'' which is in \kbd{z[1]}. Accessing \kbd{z[i]} for \kbd{i}
larger than or equal to the effective length will yield random results.

\noindent This information is handled using the following macros:

\fun{long}{signe}{GEN z} returns the sign of \kbd{z}.

\fun{void}{setsigne}{GEN z, long s} sets the sign of \kbd{z} to \kbd{s}.

\fun{long}{lgefint}{GEN z} returns the \idx{effective length} of \kbd{z}.

\fun{void}{setlgefint}{GEN z, long l} sets the effective length
of \kbd{z} to \kbd{l}.

The integer 0 can be recognized either by its sign being~0, or by its
effective length being equal to~2. When $\kbd{z} \ne 0$, the word
\kbd{z[2]} exists and is non-zero, and the absolute value of \kbd{z}
is (\kbd{z[2]},\kbd{z[3]},\dots,\kbd{z[lgefint(z)-1]}) in base
\kbd{2\pow BITS\_IN\_LONG}, where as usual in this notation \kbd{z[2]} is
the highest order longword.

\noindent The following further macros are available:

\fun{long}{mpodd}{GEN x} which is 1 if \kbd{x} is odd, and 0 otherwise.

\fun{long}{mod2}{GEN x}, \fun{}{mod4}{x}, and so on up to \fun{}{mod64}{x},
which give the residue class of \kbd{x} modulo the corresponding power of
2, for {\it positive\/}~\kbd{x} (you will obtain weird results if you use
these on the integer 0 or on negative numbers).

These macros directly access the binary data and are thus much faster than
the generic modulo functions. Besides, they return long integers instead of
\kbd{GEN}s, so they don't clutter up the stack.

\subsec{Type \typ{REAL} (real number):}\kbdsidx{t_REAL}\sidx{real number}
this type has a second codeword z[1] which also encodes its sign (obtained
or set using the same functions as for the integers), and a biased binary
exponent (i.e.~the actual exponent value plus some constant bias, actually
a power of~2, whose value is given by \kbd{HIGHEXPOBIT}). This exponent can
be handled using the following macros:

\fun{long}{expo}{GEN z} returns the true (unbiased) exponent of \kbd{z}.
This is defined even when \kbd{z} is equal to zero, see
\secref{se:whatzero}.

\fun{void}{setexpo}{GEN z, long e} sets the exponent of \kbd{z} to \kbd{e},
of course after adding the bias.

\noindent Note the functions:

\fun{long}{gexpo}{GEN z} which tries to return an exponent for \kbd{z},
even if \kbd{z} is not a real number.

\fun{long}{gsigne}{GEN z} which returns a sign for \kbd{z}, even when
\kbd{z} is neither real nor integer (a rational number for instance).

The real zero is characterized by having its sign equal to 0. However,
usually the first mantissa word \kbd{z[2]} is defined and equal to~0. This
fact must {\it never\/} be used to recognize a real~0. If \kbd{z} is not
equal to~0, the first mantissa word \kbd{z[2]} is normalized, i.e.~its most
significant bit is~1. The mantissa is (\kbd{z[2]},\kbd{z[3]},\dots,%
\kbd{z[lg(z]-1]}) in base \kbd{2\pow BITS\_IN\_LONG}. Here, \kbd{z[2]} is
the most significant longword, and the mantissa takes values between
1 (included) and 2 (excluded). Thus, assume that \kbd{sizeof(long)} is 32 and
the length is 4, the real number $3.5$ is represented as \kbd{z[0]} (encoding
$\kbd{type} = \typ{REAL}$, $\kbd{lg} = 4$), \kbd{z[1]} (encoding $\kbd{sign} =
1$, $\kbd{expo} = 1$), $\kbd{z[2]} = \kbd{0xe0000000}$,
$\kbd{z[3]} = \kbd{0x0}$.

\subsec{Type \typ{INTMOD} (integermod):}\kbdsidx{t_INTMOD}\sidx{integermod}
\kbd{z[1]} points to the modulus, and \kbd{z[2]} at the number representing
the class \kbd{z}. Both are separate \kbd{GEN} objects, and both must be of
type integer, satisfying the inequality $0 \le \kbd{z[2]} < \kbd{z[1]}$.

It is good practice to keep the modulus object on the heap, so that new
integermods resulting from operations can point at this common object,
instead of carrying along their own copies of it on the stack. The library
functions implement this practice almost by default.

\subsec{Type \typ{FRAC} and \typ{FRACN} (rational number):}%
\kbdsidx{t_FRAC}\kbdsidx{t_FRACN}\sidx{rational number}
\kbd{z[1]} points to the numerator, and \kbd{z[2]} to the denominator. Both
must be of type integer. In principle $\kbd{z[2]} > 0$, but this rule does not
have to be strictly obeyed. Note that a type \typ{FRACN} rational number can be
converted to irreducible form using the function \fun{GEN}{gred}{GEN x}.

\subsec{Type \typ{COMPLEX} (complex number):}%
\kbdsidx{t_COMPLEX}\sidx{complex number}
\kbd{z[1]} points to the real part, and \kbd{z[2]} to the imaginary part. A
priori \kbd{z[1]} and \kbd{z[2]} can be of any type, but only certain types
are useful and make sense.

\subsec{Type \typ{PADIC} ($p$-adic numbers):}%
\sidx{p-adic number}\kbdsidx{t_PADIC} this type has a second codeword
\kbd{[1]} which contains the following information: the $p$-adic precision
(the exponent of $p$ modulo which the $p$-adic unit corresponding to
\kbd{z} is defined if \kbd{z} is not~0), i.e.~one less than the number of
significant $p$-adic digits, and the biased exponent of \kbd{z} (the bias
being equal to \kbd{HIGHVALPBIT} here). This information can be handled
using the following functions:

\fun{long}{precp}{GEN z} returns the $p$-adic precision of \kbd{z}.

\fun{void}{setprecp}{GEN z, long l} sets the $p$-adic precision of \kbd{z}
to \kbd{l}.

\fun{long}{valp}{GEN z} returns the $p$-adic valuation of \kbd{z} (i.e. the
unbiased exponent). This is defined even if \kbd{z} is equal to~0, see
\secref{se:whatzero}.

\fun{void}{setvalp}{GEN z, long e} sets the $p$-adic valuation of \kbd{z}
to \kbd{e}.

In addition to this codeword, \kbd{z[2]} points to the prime $p$,
\kbd{z[3]} points to $p^{\text{precp(z)}}$, and \kbd{z[4]} points to an
integer representing the $p$-adic unit associated to \kbd{z} modulo
\kbd{z[3]} (and points to zero if \kbd{z} is zero). To summarize, if $z\neq
0$, we have the equality:
$$ \kbd{z} = p^{\text{valp(z)}} * (\kbd{z[4]} + O(\kbd{z[3]})) =
     p^{\text{valp(z)}} * (\kbd{z[4]} + O(p^{\text{precp(z)}})) $$

\subsec{Type \typ{QUAD} (quadratic number):}\sidx{quadratic
number}\kbdsidx{t_QUAD} \kbd{z[1]} points to the polynomial defining the
quadratic field, \kbd{z[2]} to the ``real part'' and \kbd{z[3]} to the
``imaginary part'', which are to be taken as the coefficients of \kbd{z}
with respect to the ``canonical'' basis $(1,w)$, see~\secref{se:compquad}.
Complex numbers are a particular case of quadratics but deserve a separate
type.

\subsec{Type \typ{POLMOD} (polmod):}\kbdsidx{t_POLMOD}\sidx{polmod} exactly as
for integermods, \kbd{z[1]} points to the modulus, and \kbd{z[2]} to a
polynomial representing the class of~\kbd{z}. Both must be of type
\typ{POL} in the same variable. However, \kbd{z[2]} is allowed to be a
simplification of such a polynomial, e.g a scalar. This is quite tricky
considering the hierarchical structure of the variables; in particular, a
polynomial in variable of \var{lesser} priority than the modulus variable
is valid, since it can be considered as the constant term of a polynomial
of degree 0 in the correct variable. On the other hand a variable of
\var{greater} priority would not be acceptable.

\subsec{Type \typ{POL} (polynomial):}\kbdsidx{t_POL}\sidx{polynomial} this
type has a second codeword which is analogous to the one for integers. It
contains a ``{\it sign\/}'': 0 if the polynomial is equal to~0, and 1 if
not (see however the important remark below), a {\it variable number\/}
(e.g.~0 for $x$, 1 for $y$, etc\dots), and an {\it effective length}.

\noindent These data can be handled with the following macros:

\teb{signe} and \teb{setsigne} as for reals and integers.

\fun{long}{lgef}{GEN z} returns the \idx{effective length} of \kbd{z}.

\fun{void}{setlgef}{GEN z, long l} sets the effective length of \kbd{z} to
\kbd{l}.

\fun{long}{varn}{GEN z} returns the variable number of the object \kbd{z}.

\fun{void}{setvarn}{GEN z, long v} sets the variable number of \kbd{z} to
\kbd{v}.

Note also the function \fun{long}{gvar}{GEN z} which tries to return a
\idx{variable number} for \kbd{z}, even if \kbd{z} is not a polynomial or
power series. The variable number of a scalar type is set by definition
equal to \tet{BIGINT}.

The components \kbd{z[2]}, \kbd{z[3]},\dots \kbd{z[lgef(z)-1]} point to the
coefficients of the polynomial {\it in ascending order}, with \kbd{z[2]}
being the constant term and so on. Note that the {\it \idx{degree}\/} of the
polynomial is equal to its effective length minus three. The function

\fun{long}{degree}{GEN x} returns the degree of \kbd{x} with respect to its
main variable even when \kbd{x} is not a polynomial (a rational function
for instance). By convention, the degree of $0$ is~$-1$.

\misctitle{Important remark}. A zero polynomial can be characterized by the
fact that its sign is~0. However, its effective length may be equal to 2, or
greater than 2. If it is greater than 2, this means that all the coefficients
of the polynomial are equal to zero (as they should for a zero polynomial),
but not all of these zeros are exact zeros, and more precisely the leading
term \kbd{z[lgef(z)-1]} is not an exact zero.

\subsec{Type \typ{SER} (power series):}\kbdsidx{t_SER}\sidx{power series} This
type also has a second codeword, which encodes a ``{\it sign\/}'', i.e.~0
if the power series is 0, and 1 if not, a {\it variable number\/} as for
polynomials, and a {\it biased exponent\/} with a bias of
\kbd{HIGHVALPBIT}. This information can be handled with the following
functions: \teb{signe}, \teb{setsigne}, \teb{varn}, \teb{setvarn} as for
polynomials, and \teb{valp}, \teb{setvalp} for the exponent as for $p$-adic
numbers. Beware: do {\it not\/} use \teb{expo} and \teb{setexpo} on power
series.

If the power series is non-zero, \kbd{z[2]}, \kbd{z[3]},\dots
\kbd{z[lg(z)-1]} point to the coefficients of \kbd{z} in ascending order,
\kbd{z[2]} being the first non-zero coefficient.  Note that the exponent of a
power series can be negative, i.e.~we are then dealing with a Laurent series
(with a finite number of negative terms).

\subsec{Type \typ{RFRAC} and \typ{RFRACN} (rational function):}%
\kbdsidx{t_RFRAC}\kbdsidx{t_RFRACN}\sidx{rational function}
\kbd{z[1]} points to the numerator, and \kbd{z[2]} on the denominator. The
denominator must be of type polynomial. Note that a type \typ{RFRACN}
rational function can be converted to irreducible form using the function
\teb{gred}.

\subsec{Type \typ{QFR} (indefinite binary quadratic form):}%
\kbdsidx{t_QFR}\sidx{indefinite binary quadratic form}
\kbd{z[1]}, \kbd{z[2]}, \kbd{z[3]} point to the three coefficients of the
form and should be of type integer. \kbd{z[4]} is Shanks's distance
function, and should be of type real.

\subsec{Type \typ{QFI} (definite binary quadratic form):}%
\kbdsidx{t_QFI}\sidx{definite binary quadratic form}
\kbd{z[1]}, \kbd{z[2]}, \kbd{z[3]} point to the three coefficients of the
form. All three should be of type integer.

\subsec{Type \typ{VEC} and \typ{COL} (vector):}%
\kbdsidx{t_VEC}\kbdsidx{t_COL}\sidx{row vector}\sidx{column vector}
\kbd{z[1]}, \kbd{z[2]},\dots \kbd{z[lg(z)-1]} point to the components of
the vector.

\subsec{Type \typ{MAT} (matrix):}\kbdsidx{t_MAT}\sidx{matrix}
\kbd{z[1]}, \kbd{z[2]},\dots \kbd{z[lg(z)-1]} point to the column vectors
of \kbd{z}, i.e.~they must be of type \typ{COL} and of the same length.

\noindent The last two were introduced for specific GP use, and you'll be
much better off using the standard malloc'ed C constructs when programming
in library mode. We quote them just for completeness (advising you not to
use them):

\subsec{Type \typ{LIST} (list):}\kbdsidx{t_LIST}\sidx{list}
This one has a second codeword which contains an effective length (handled
through \teb{lgef}~/ \teb{setlgef}). \kbd{z[2]},\dots, \kbd{z[lgef(z)-1]}
contain the components of the list.

\subsec{Type \typ{STR} (character string):}%
\kbdsidx{t_STR}\sidx{character string}
\fun{char *}{GSTR}{z} (= \kbd{(z+1)}) points to the first character of the
(\kbd{NULL}-terminated) string.

\section{PARI variables}\label{se:vars}
\subsec{Multivariate objects}

\noindent
We now consider variables and formal computations. As we have seen in
\secref{se:impl}, the codewords for types \typ{POL} and \typ{SER}
encode a ``variable number''. This is an integer, ranging from $0$ to
\kbd{MAXVARN}. The lower it is, the higher the variable priority. PARI does
not know anything about intelligent ``sparse'' representation of polynomials.
So a multivariate polynomial in PARI is just a polynomial (in one variable),
whose coefficients are themselves (arbitrary) polynomials. All computations
are then just done formally on the coefficients as if the polynomial was
univariate.

In fact, the way an object will be considered in formal computations depends
entirely on its ``principal variable number'' which is given by the function

\fun{long}{gvar}{GEN z}

\noindent which returns a \idx{variable number} for \kbd{z}, even if \kbd{z}
is not a polynomial or power series. The variable number of a scalar type is
set by definition equal to \tet{BIGINT} which is bigger than any legal
variable number. The variable number of a recursive type which is not a
polynomial or power series is the minimal variable number of its components.
But for polynomials and power series only the ``outermost'' number counts:
the representation is not symmetrical at all.

Under GP, one need not worry too much since the interpreter will define
the variables as it sees them and do the right thing with the polynomials
produced (however, have a look at the remark in \secref{se:rempolmod}). But
in library mode, they are tricky objects if you intend to build polynomials
yourself (and not just let PARI functions produce them, which is usually less
efficient). For instance, it does not make sense to have a variable number
occur in the components of a polynomial whose main variable has a higher
number (lower priority), even though there's nothing PARI can do to prevent
you from doing it.

\subsec{Creating variables}
A basic difficulty is to ``create'' a variable. As we have seen in
\secref{se:intro4}, a plethora of objects is associated to variable
number~$v$. Here is the complete list: \teb{polun}$[v]$ and
\teb{polx}$[v]$, which you can use in library mode and which represent,
respectively, the monic monomials of degrees 0 and 1 in~$v$;
\teb{varentries}$[v]$, and \teb{polvar}$[v]$. The latter two are only
meaningful to GP, but they have to be set nevertheless. All of them must be
properly defined before you can use a given integer as a variable number.

Initially, this is done for $0$ (the variable \kbd{x} under GP), and
\tet{MAXVARN}, which is there to address the need for a ``temporary'' new
variable, which would not be used in regular objects (created by the
library). We call the latter type a ``temporary variable''. The regular
variables meant to be used in regular objects, are called ``user
variables\sidx{variable (user)}''.

\subsubsec{User variables}\sidx{variable (user)}:
When the program starts, \kbd{x} is the only user variable (number~$0$). To
define new ones, use

\fun{long}{fetch_user_var}{char *$s$}

\noindent which inspects the user variable named $s$ (creating it if
needed), and returns its variable number. 
\bprog
long v = fetch_user_var("y");
GEN gy = polx[v];
@eprog
This function raises an error if $s$ is already known as a function name to
the interpreter.

\misctitle{Caveat:} it is possible to use \teb{flissexpr}
(see~\secref{se:flisexpr}) to execute a GP command and create GP variables
on the fly as needed:

\bprog
GEN gy = flissexpr("y"); /*@Ccom supposedly returns polx[$v$], for some $v$ */
long v = gvar(gy);
@eprog

\noindent This is dangerous, especially when programming functions that
will be used under GP. The code above reads the value of \kbd{y}, as it is
currently known by the GP interpreter (possibly creating it in the
process). All is well and good if \kbd{y} hasn't been tampered with in
previous GP commands. But if \kbd{y} has been modified (e.g \kbd {y = 1}),
then the value of \kbd{gy} is not what you expected it to be and corresponds
instead to the current value of the GP variable (e.g \kbd{gun}).

\subsubsec{Temporary variables}\sidx{variable (temporary)}: 
\kbd{MAXVARN} is available, but is better left to pari internal functions
(some of which don't check that \kbd{MAXVARN} is free for them to use,
which can be considered a bug). You can create more temporary variables
using

\fun{long}{fetch_var}{}\label{se:fetch_var}

\noindent
This returns a variable number which is guaranteed to be unused by the
library at the time you get it and as long as you do not delete it (we'll see
how to do that shortly). This has {\it lower\/} number (i.e.~{\it higher\/}
priority) than any temporary variable produced so far (\kbd{MAXVARN} is
assumed to be the first such). This call updates all the aforementioned
internal arrays. In particular, after the statement \kbd{v = fetch\_var()},
you can use \kbd{polun[v]} and \kbd{polx[v]}. The variables created in this
way have no identifier assigned to them though, and they will be printed as
\kbd{\#<\text{number}>}, except for \kbd{MAXVARN} which will be printed
as~\kbd{\#}. You can assign a name to a temporary variable, after creating
it, by calling the function

\fun{void}{name_var}{long n, char *s}

\noindent after which the output machinery will use the name \kbd{s} to
represent the variable number~\kbd{n}. The GP parser will {\it not\/}
recognize it by that name, however, and calling this on a variable known
to~GP will raise an error. Temporary variables are meant to be used as free
variables, and you should never assign values or functions to them as you
would do with variables under~GP. For that, you need a user variable.

All objects created by \kbd{fetch\_var} are on the heap and not on the stack,
thus they are not subject to standard garbage collecting (they won't be
destroyed by a \kbd{gerepile} or \kbd{avma = ltop} statement). When you don't
need a variable number anymore, you can delete it using

\fun{long}{delete_var}{}

\noindent which deletes the {\it latest\/} temporary variable created and
returns the variable number of the previous one (or simply returns 0 if you
try, in vain, to delete \kbd{MAXVARN}). Of course you should make sure that
the deleted variable does not appear anywhere in the objects you use later
on. Here is an example:

\bprog
{
  long first = fetch_var();
  long n1 = fetch_var();
  long n2 = fetch_var(); /*@Ccom prepare three variables for internal use */
  ...
  /*@Ccom delete all variables before leaving */
  do { num = delete_var(); } while (num && num <= first);
}@eprog

\noindent
The (dangerous) statement

\bprog
while (delete_var()) /*@Ccom empty */;
@eprog

\noindent removes all temporary variables that were in use, except
\kbd{MAXVARN} which cannot be deleted.

\section{Input and output}

\noindent
Two important aspects have not yet been explained which are specific to
library mode: input and output of PARI objects.

\subsec{Input}.

\noindent
For \idx{input}, PARI provides you with two powerful high level functions
which enables you to input your objects as if you were under GP. In fact,
the second one {\it is\/} essentially the GP syntactical parser, hence you
can use it not only for input but for (most) computations that you can do
under GP. These functions are called \teb{flisexpr} and \teb{flisseq}. The
first one has the following syntax:\label{se:flisexpr}

\fun{GEN}{flisexpr}{char *s}

\noindent
Its effect is to analyze the input string s and to compute the result as in
GP.  However it is limited to one expression. If you want to read and
evaluate a sequence of expressions, use

\fun{GEN}{flisseq}{char *s}

\noindent\sidx{filter}
In fact these two functions start by {\it filtering\/} out all spaces and
comments in the input string (that's what the initial \kbd{f} stands for).
They then call the underlying basic functions, the GP parser proper:
\fun{GEN}{lisexpr}{char *s} and \fun{GEN}{lisseq}{char *s}, which are
slightly faster but which you probably don't need.

To read a \kbd{GEN} from a file, you can use the simpler interface

\fun{GEN}{lisGEN}{FILE *file}

which reads a character string of arbitrary length from the stream \kbd{file}
(up to the first newline character), applies \kbd{flisexpr} to it, and
returns the resulting \kbd{GEN}. This way, you won't have to worry about
allocating buffers to hold the string. To interactively input an expression,
use \kbd{lisGEN(stdin)}. This function returns \kbd{NULL} if \kbd{EOF} is
encountered before a complete expression could be read.

Once in a while, it may be necessary to evaluate a GP expression sequence
involving a call to a function you have defined in~C. This is easy using
\teb{install} which allows you to manipulate quite an arbitrary function (GP
knows about pointers!). The syntax is

\fun{void}{install}{void *f, char *name, char *code}

\noindent where \kbd{f} is the (address of) the function (cast to the C type
\kbd{void*}), \kbd{name} is the name by which you want to access your
function from within your GP expressions, and \kbd{code} is a character
string describing the function call prototype (see~\secref{se:gp.interface}
for the precise description of prototype strings). In case the function
returns a \kbd{GEN}, it should satisfy \kbd{gerepileupto} assumptions (see
\secref{se:garbage}).

\subsec{Output}.

\noindent
For \idx{output}, there exist essentially three different functions (with
variants), corresponding to the three main GP output formats (as described in
\secref{se:output}), plus three extra ones, respectively devoted to
\TeX\ output, string output, and (advanced) debugging.

\noindent $\bullet$ ``raw'' format, obtained by using the function
\teb{brute} with the following syntax:

\fun{void}{brute}{GEN obj, char x, long n}

\noindent
This prints the PARI object \kbd{obj} in \idx{format} \kbd{x0.n}, using the
notations from \secref{se:format}. Recall that here \kbd{x} is either
\kbd{'e'}, \kbd{'f'} or \kbd{'g'} corresponding to the three numerical output
formats, and \kbd{n} is the number of printed significant digits, and should
be set to $-1$ if all of them are wanted (these arguments only affect the
printing of real numbers). Usually you won't need that much flexibility, so
most of the time you will get by with the function

\fun{void}{outbrute}{GEN obj}, which is equivalent to \kbd{brute(x,'g',-1)},

\noindent or even better, with

\fun{void}{output}{GEN obj} which is equivalent to \kbd{outbrute(obj)}
followed by a newline and a buffer flush. This is especially nice during
debugging. For instance using \kbd{dbx} or \kbd{gdb}, if \kbd{obj} is a
\kbd{GEN}, typing \kbd{print output(obj)} will enable you to see the
content of \kbd{obj} (provided the optimizer has not put it into a
register, but it's rarely a good idea to debug optimized code).

\noindent $\bullet$ ``prettymatrix'' format: this format is identical to the
preceding one except for matrices. The relevant functions are:

\fun{void}{matbrute}{GEN obj, char x, long n}

\fun{void}{outmat}{GEN obj}, which is followed by a newline and a buffer flush.

\noindent $\bullet$ ``prettyprint'' format: the basic function has an
additional parameter \kbd{m}, corresponding to the (minimum) field width
used for printing integers:

\fun{void}{sor}{GEN obj, char x, long n, long m}

\noindent The simplified version is

\fun{void}{outbeaut}{GEN obj} which is equivalent to
\kbd{sor(obj,'g',-1,0)} followed by a newline and a buffer flush.

\noindent $\bullet$ The first extra format corresponds to the \teb{texprint}
function of GP, and gives a \TeX\ output of the result. It is obtained by
using:

\fun{void}{exe}{GEN obj, char x, long n}

\noindent $\bullet$ The second one is the function \teb{GENtostr} which
converts a PARI \kbd{GEN} to an ASCII string. The syntax is

\fun{char*}{GENtostr}{GEN obj}, wich returns a \kbd{malloc}'ed character
string (which you should \kbd{free} after use).

\noindent $\bullet$ The third and final one outputs the \idx{hexadecimal tree}
corresponding to the GP command \kbd{\b x} using the function

\fun{void}{voir}{GEN obj, long nb}, which will only output the first
\kbd{nb} words corresponding to leaves (very handy when you have a look at
big recursive structures). If you set this parameter to $-1$ all
significant words will be printed. Usually this last type of output would
only be used for debugging purposes.

\misctitle{Remark}. Apart from \teb{GENtostr}, all PARI output is done on
the stream \teb{outfile}, which by default is initialized to \teb{stdout}. If
you want that your output be directed to another file, you should use the
function \fun{void}{switchout}{char *name} where \kbd{name} is a
character string giving the name of the file you are going to use. The
output will be {\it appended\/} at the end of the file. In order to close
the file, simply call \kbd{switchout(NULL)}.

Similarly, errors are sent to the stream \teb{errfile} (\teb{stderr}
by default), and input is done on the stream \teb{infile}, which you can change
using the function \teb{switchin} which is analogous to \teb{switchout}.

\misctitle{(Advanced) Remark}. All output is done according to the values
of the \teb{pariOut}~/ \teb{pariErr} global variables which are pointers to
structs of pointer to functions. If you really intend to use these, this
probably means you are rewriting GP. In that case, have a look at the code in
\kbd{language/es.c} (\kbd{init80()} or \kbd{GENtostr()} for instance).

\subsec{Errors}.\sidx{error}\sidx{talker}

\noindent
If you want your functions to issue error messages, you can use the general
error handling routine \teb{err}. The basic syntax is
%
\bprog
  err(talker, "error message");
@eprog

\noindent
This will print the corresponding error message and exit the program (in
library mode; go back to the GP prompt otherwise).\label{se:err} You can
also use it in the more versatile guise
\bprog
  err(talker, format, ...);
@eprog\noindent
where \kbd{format} describes the format to use to write the remaining
operands, as in the \teb{printf} function (however, see the next section).
The simple syntax above is just a special case with a constant format and no
remaining arguments.

\noindent
The general syntax is

\fun{void}{err}{numerr,...}

\noindent where \kbd{numerr} is a codeword which indicates what to do with
the remaining arguments and what message to print. The list of valid keywords
is in \kbd{language/errmessages.c} together with the basic corresponding
message. For instance, \kbd{err(typeer,"matexp")} will print the message:

\kbd{ ***   incorrect type in matexp.}

\noindent
Among the codewords are {\it warning\/} keywords (all those which start with
the prefix \kbd{warn}). In that case, \teb{err} does {\it not\/} abort the
computation, just print the requested message and go on. The basic example is

\kbd{err(warner, "Strategy 1 failed. Trying strategy 2")}

\noindent which is the exact equivalent of \kbd{err(talker,...)} except that
you certainly don't want to stop the program at this point, just inform the
user that something important has occured (in particular, this output would be
suitably highlighted under GP, whereas a simple \kbd{printf} would not).

\subsec{Debugging output}.\sidx{debugging}\sidx{format}\label{se:dbg_output}

\noindent
The global variables \teb{DEBUGLEVEL} and \teb{DEBUGMEM} (corresponding
to the default \teb{debug} and \teb{debugmem}, see \secref{se:defaults})
are used throughout the PARI code to govern the amount of diagnostic and
debugging output, depending on their values. You can use them to debug your
own functions, especially after having made them accessible under GP through
the command \teb{install} (see \secref{se:install}).

For debugging output, you can use \kbd{printf} and the standard output
functions (\teb{brute} or \teb{output} mainly), but also some special purpose
functions which embody both concepts, the main one being

\fun{void}{fprintferr}{char *pariformat, ...}

\noindent
Now let's define what a PARI format is. It is a character string, similar
to the one \kbd{printf} uses, where \kbd{\%} characters have a special
meaning. It describes the format to use when printing the remaining operands.
But, in addition to the standard format types, you can use \kbd{\%Z} to
denote a \kbd{GEN} object (we would have liked to pick \kbd{\%G} but it was
already in use!). For instance you could write:
\bprog
err(talker, "x[%d] = %Z is not invertible!", i, x[i])
@eprog
\noindent since the \teb{err} function accepts PARI formats. Here \kbd{i}
is an \kbd{int}, \kbd{x} a \kbd{GEN} which is not a leaf and this would
insert in raw format the value of the \kbd{GEN} \kbd{x[i]}.

\subsec{Timers and timing output}.

\noindent
To profile your functions, you can use the PARI timer. The functions
\fun{long}{timer}{} and \fun{long}{timer2}{} return the elapsed time since
the last call of the same function (in milliseconds). Two different
functions (identical except for their independent time-of-last-call
memories!) are provided so you can have both global timing and fine tuned
profiling.

You can also use \fun{void}{msgtimer}{char *format,...}, which prints
prints \kbd{Time}, then the remaining arguments as specified by
\kbd{format} (which is a PARI format), then the output of \kbd{timer2}.

This mechanism is simple to use but not foolproof. If some other function
uses these timers, and many PARI functions do when DEBUGLEVEL is high enough,
the timings will be meaningless. The functions \fun{long}{gentimer}{long
id} or \fun{long}{genmsgtimer}{long id, char *format,...} are equivalent to
\kbd{timer} and \kbd{msgtimer} respectively, except they will use a unique
timer denoted by \kbd{id}. To get a valid identifier, use
\kbd{id = }\tet{get_timer}\kbd{(0)}.

This timer has to be deleted when it's not needed anymore
(i.e.~when the main function returns), with a call to \kbd{get\_timer(id)}.
After such a call, the reserved identifier becomes available again.

\section{A complete program}
\label{se:prog}

\noindent
Now that the preliminaries are out of the way, the best way to learn how to
use the library mode is to work through a detailed non-trivial example of a
main program. We will write a program which computes the exponential of a
square matrix~$x$.  The complete listing is given in Appendix~B, but each
part of the program will be produced and explained here. We will use an
algorithm which is not optimal but is not far from the one used for the PARI
function \teb{gexp} (in fact embodied in the function \kbd{mpexp1}). This
consists in calculating the sum of the series:
$$e^{x/(2^n)}=\sum_{k=0}^\infty \dfrac{(x/(2^n))^k}{k!}$$
for a suitable positive integer $n$, and then computing $e^x$ by repeated
squarings.  First, we will need to compute the $L^2$-norm of the matrix~$x$,
i.e.~the quantity:
$$z=\|x\|_2=\sqrt{\sum x_{i,j}^2}.$$
We will then choose the integer $n$ such that the $L^2$-norm of $x/(2^n)$ is
less than or equal to~1, i.e.
$$ n = \left\lceil{\ln(z)}\big/{\ln(2)}\right\rceil $$
if $z\ge1$, and~$n=0$ otherwise. Then the series will converge at least as
fast as the usual one for $e^1$, and the cutoff error will be easy to
estimate. In fact a larger value of $n$ would be preferable, but this is
slightly machine dependent and more complicated, and will be left to the
reader.

Let us start writing our program. So as to be able to use it in other
contexts, we will structure it in the following way: a main program which
will do the input and output, and a function which we shall call \kbd{matexp}
which does the real work. The main program is easy to write. It can be
something like this:

\bprog
#include <pari.h>

GEN matexp(GEN x, long prec);

int
main()
{
  long d, prec = 3;
  GEN x;

  /*@Ccom take a stack of $10^6$ bytes, no prime table */
  pari_init(1000000, 2);
  printf("precision of the computation in decimal digits:\n");
  d = itos(lisGEN(stdin));
  if (d > 0) prec = (long) (d*pariK1+3);

  printf("input your matrix in GP format:\n");
  x = matexp(lisGEN(stdin), prec);

  sor(x, 'g', d, 0);
  exit(0);
}@eprog
\noindent
The variable \kbd{prec} represents the length in longwords of the real
numbers used. \teb{pariK1} is a constant (defined in \kbd{paricom.h}) equal
to $\ln(10) / (\ln(2) * \kbd{BITS\_IN\_LONG})$, which allows us to convert
from a number of decimal digits to a number of longwords, independently of
the actual bit size of your long integers. The function \teb{lisGEN} reads an
expression (here from standard input) and converts it to a \kbd{GEN}, like
the GP parser itself would. This means it takes care of whitespace etc.\ in
the input, and can do computations (e.g.~\kbd{matid(2)} or \kbd{[1,0; 0,1]}
are equally valid inputs).

Finally, \kbd{sor} is the general output routine. We have chosen to give
\kbd{d} significant digits since this is what was asked for. Note that there
is a trick hidden here: if a negative \kbd{d} was input, then the computation
will be done in precision 3 (i.e.~about 9.7 decimal digits for 32-bit
machines and 19.4 for 64-bit machines) and in the function \kbd{sor}, giving
a negative third argument outputs all the significant digits, which is entirely
appropriate. Now let us attack the main course, the function \kbd{matexp}:
%
\bprog
GEN
matexp(GEN x, long prec)
{
  long lx=lg(x),i,k,n,lbot, ltop = avma;
  GEN y,r,s,p1,p2;

  /*@Ccom check that x is a square matrix */
  if (typ(x) != t_MAT) err(talker,"this expression is not a matrix");
  if (lx == 1) return cgetg(1, t_MAT);
  if (lx != lg(x[1])) err(talker,"not a square matrix");

  /*@Ccom compute the $L_2$ norm of x */
  s = gzero;
  for (i=1; i<lx; i++)
    s = gadd(s, gnorml2((GEN)x[i]));
  if (typ(s) == t_REAL) setlg(s,3);
  s = gsqrt(s,3); /*@Ccom we do not need much precision on s */

  /*@Ccom if $s<1$, we are happy */
  k = expo(s);
  if (k < 0) { n = 0; p1 = x; }
  else { n = k+1; p1 = gmul2n(x,-n); setexpo(s,-1); }@eprog
\noindent
Before continuing, several remarks are in order.

First, before starting this computation which will produce garbage on the
stack, we have carefully saved the value of the stack pointer \kbd{avma} in
\kbd{ltop}. Note that we are going to assume throughout that the garbage does
not overflow the currently available stack. If it ever did, we would
have several options~--- allocate a larger stack in the main program (for
instance change 1000000 into 2000000), do some \kbd{gerepile}ing along the
way, or (if you know what you are doing) use \tet{allocatemoremem}.

Secondly, the \teb{err} function is the general error handler for the
PARI library. This will abort the program after printing the required
message.

Thirdly, notice how we handle the special case $\kbd{lx} = 1$ (empty matrix)
{\it before\/} accessing \kbd{lx(x[1])}. Doing it the other way round could
produce a fatal error (a segmentation fault or a bus error, most probably).
Indeed, if \kbd{x} is of length 1, then \kbd{x[1]} is not a component of
\kbd{x}. It is just the contents of the memory cell which happens to follow
the one pointed to by \kbd{x}, and thus has no reason to be a valid \kbd{GEN}.
Now recall that none of the codeword handling macros do any kind of type
checking (see \secref{se:impl}), thus \teb{lg} would consider \kbd{x[1]}
as a valid address, and try to access \kbd{*((GEN)x[1])} (the first codeword)
which is unlikely to be a legal memory address.

In the fourth place, to compute the square of the $L^2$-norm of \kbd{x} we
just add the squares of the $L^2$-norms of the column vectors which we obtain
using the library function \teb{gnorml2}. Had this function not existed, the
norm computation would of course have been just as easy to write, but we would
have needed a double loop.

We then take the square root of \kbd{s}, in precision~3 (the smallest
possible). The \kbd{prec} argument of transcendental functions (here~$3$) is
only taken into account when the arguments are {\it exact\/} objects, and thus
no a priori precision can be determined from the objects themselves. To cater
for this possibility, if \kbd{s} is of type \typ{REAL}, we use the function
\teb{setlg} which effectively sets the precision of \kbd{s} to the required
value. Note that here, since we are using a numeric value for a \kbd{cget}
function, the program will run slightly differently on 32-bit and 64-bit
machines: we want to use the smallest possible bit accuracy, and this is equal
to \kbd{BITS\_IN\_LONG}.

Note that the matrix \kbd{x} is allowed to have complex entries, but the
function \kbd{gnorml2} guarantees that \kbd{s} is a non-negative real number
(not necessarily of type \typ{REAL} of course). If we had not known this fact,
we would simply have added the instruction \kbd{s = greal(s);} just after the
\kbd{for} loop.

Note also that the function \kbd{gnorml2} works as desired on matrices, so we
really did not need this loop at all (\kbd{s = gnorml2(x)} would have been
enough), but we wanted to give examples of function usage. Similarly, it is of
course not necessary to take the square root for testing whether the norm
exceeds~$1$.

In the fifth place, note that we initialized the sum \kbd{s} to \teb{gzero},
which is an exact zero. This is logical, but has some disadvantages: if all
the entries of the matrix are integers (or rational numbers), the computation
will take rather long, about twice as long as with real numbers of the same
length. It would be better to initialize \kbd{s} to a real zero, using for
instance the instructions:

\kbd{s = cgetr(prec+1); gaffsg(0,s);}

\noindent
This raises the question: which real zero does this produce (have a look
at~\secref{se:whatzero})? In fact, the following choice has been made: it will
give you the zero with exponent equal to $-\kbd{BITS\_IN\_LONG}$ times the
number of longwords in the mantissa, i.e.~$-\kbd{bit\_accuracy(lg(s))}$.
Instead of the above idiom, you can also use the function
\fun{GEN}{realzero}{long prec}, which simply returns a real zero to
accuracy $-\tet{bit_accuracy}(\var{prec})$.

The sixth remark here is about how to determine the approximate size
of a real number. The fastest way to do this is to look at its binary
exponent. Hence we need to have \kbd{s} actually represented as a real number,
and not as an integer or a rational number. The result of transcendental
functions is guaranteed to be of type \typ{REAL}, or complex with \typ{REAL}
components, thus this is indeed the case after the call to \kbd{gsqrt} since
its argument is a nonnegative (real) number.

Finally, note the use of the function \teb{gmul2n}. It has the following
syntax:

\fun{GEN}{gmul2n}{GEN x, long n}

\noindent and the effect is simply to multiply \kbd{x} by $2^{\text n}$,
where \kbd{n} can be positive or negative. This is much faster than
\kbd{gmul} or \kbd{gmulgs}.

There is another function \teb{gshift} with exactly the same syntax.
When \kbd{n} is non-negative, the effects of these two functions are the same.
However, when \kbd{n} is negative, \kbd{gshift} acts like a right shift of
\kbd{-n}, hence does not noramlly perform an exact division on integers. The
function  \kbd{gshift} is the PARI analogue of the C or GP operators \kbd{<<}
and~\kbd{>>}.

We now come to the heart of the function. We have a \kbd{GEN} \kbd{p1} which
points to a certain matrix of which we want to take the exponential. We will
want to transform this matrix into a matrix with real (or complex of real)
entries before starting the computation. To do this, we simply multiply by
the real number 1 in precision $\kbd{prec}+1$ (to be on the side of safety).
To sum the series, we will use three variables: a variable \kbd{p2} which at
stage $k$ will contain $\kbd{p1}^k/k!$, a variable \kbd{y} which will contain
$\sum_{i=0}^k \kbd{p1}^i/i!$, and a variable \kbd{r} which will contain the
size estimate $\kbd{s}^k/k!$. Note that we do not use Horner's rule. This is
simply because we are lazy and do not want to compute in advance the number
of terms that we need. We leave this modification (and many other
improvements!) to the reader. The program continues as follows:

\bprog
  /*@Ccom initializations before the loop */
  r = cgetr(prec+1); gaffsg(1,r); p1 = gmul(r,p1);
  y = gscalmat(r,lx-1); /*@Ccom creates scalar matrix with r on diagonal */
  p2 = p1; r = s; k = 1;
  y = gadd(y,p2);

  /*@Ccom now the main loop */
  while (expo(r) >= -BITS_IN_LONG*(prec-1))
  {
    k++; p2 = gdivgs(gmul(p2,p1),k);
    r = gdivgs(gmul(s,r),k); y = gadd(y,p2);
  }

  /*@Ccom now square back n times if necessary */
  if (!n) { lbot = avma; y = gcopy(y); }
  else
  {
    for (i=0; i<n; i++) { lbot = avma; y = gsqr(y); }
  }
  return gerepile(ltop,lbot,y);
}@eprog
\noindent
A few remarks once again. First note the use of the function
\teb{gscalmat} with the following syntax:

\fun{GEN}{gscalmat}{GEN x, long m}

\noindent
The effect of this function is to create the $\kbd{m}\times\kbd{m}$ scalar
matrix whose diagonal entries are~\kbd{x}. Hence the length of the matrix
including the codeword will in fact be \kbd{m+1}.  There is a corresponding
function \teb{gscalsmat} which takes a long as a first argument.

If we refer to what has been said above, the main loop should be self-evident.

When we do the final squarings, according to the fundamental dogma on the
use of \teb{gerepile}, we keep the value of \kbd{avma} in \kbd{lbot} just {\it
before\/} the squaring, so that if it is the last one, \kbd{lbot} will indeed
be the bottom address of the garbage pile, and \kbd{gerepile} will work. Note
that it takes a completely negligible time to do this in each loop compared
to a matrix squaring.  However, when \kbd{n} is initially equal to 0, no
squaring has to be done, and we have our final result ready but we lost the
address of the bottom of the garbage pile. Hence we use the trick of copying
\kbd{y} again to the top of the stack. This is inefficient, but does the trick.
If we wanted to avoid this using only \kbd{gerepile}, the best thing to do
would be to put the instruction \kbd{lbot=avma} just before both occurrences
of the instruction \kbd{y=gadd(p2,y)}. Of course, we could also rewrite the
last block as follows:

\bprog
  /*@Ccom now square back n times */
  for (i=0; i<n; i++) y = gsqr(y);
  return gerepileupto(ltop,y);
@eprog
\noindent
because it does not matter to \teb{gerepileupto} that we have lost the address
just before the final result (note that the loop is not executed if \kbd{n}
is~0). It is safe to use \kbd{gerepileupto} here as \kbd{y} will have been
created by either \kbd{gsqr} or \kbd{gadd}, both of which are guaranteed to
return suitable objects.

\misctitle{Remarks}. As such, the program should work most of the time if
\kbd{x} is a square matrix with real or complex entries. Indeed, since
essentially the first thing that we do is to multiply by the real number~1,
the program should work for integer, real, rational, complex or quadratic
entries. This is in accordance with the behavior of transcendental functions.

Furthermore, since this program is intended to be only an illustrative
example, it has been written a little sloppily. In particular many error
checks have been omitted, and the efficiency is far from optimal. An evident
improvement would be the use of \kbd{gerepileupto} mentioned above. Another
improvement is to multiply the matrix x by the real number 1 right at the
beginning, speeding up the computation of the $L^2$-norm in many cases. These
improvements are included in the version given in Appendix~B. Still another
improvement would come from a better choice of~\kbd{n}. If the reader takes a
look at the implementation of the function \kbd{mpexp1} in the file
\kbd{basemath/trans1.c}, he can make the necessary changes himself. Finally,
there exist other algorithms of a different nature to compute the exponential
of a matrix.

\section{Adding functions to PARI}
\subsec{Nota Bene}.
%
As mentioned in the \kbd{COPYING} file, modified versions of the PARI package
can be distributed under the conditions of the GNU General Public License. If
you do modify PARI, however, it is certainly for a good reason, hence we
would like to know about it, so that everyone can benefit from it. There is
then a good chance that the modifications that you have made will be
incorporated into the next release.

(Recall the e-mail address: \kbd{pari@math.u-bordeaux.fr}, or use the mailing
lists).

Roughly four types of modifications can be made. The first type includes all
improvements to the documentation, in a broad sense. This includes correcting
typos or inacurracies of course, but also items which are not really covered
in this document, e.g.~if you happen to write a tutorial, or pieces of code
exemplifying some fine points that you think were unduly omitted.

The second type is to expand or modify the configuration routines and skeleton
files (the \kbd{Configure} script and anything in the \kbd{config/}
subdirectory) so that compilation is possible (or easier, or more efficient)
on an operating system previously not catered for. This includes discovering
and removing idiosyncrasies in the code that would hinder its portability.

The third type is to modify existing (mathematical) code, either to correct
bugs, to add new functionalities to existing functions, or to improve their
efficiency.

Finally the last type is to add new functions to PARI. We explain here how
to do this, so that in particular the new function can be called from GP.

\subsec{The calling interface from GP, parser codes}.
\label{se:gp.interface}
A \idx{parser code} is a character string describing all the GP parser
needs to know about the function prototype. It contains a sequence of the
following atoms:

\settabs\+\indent&\kbd{Dxxx}\quad&\cr
\noindent $\bullet$ Syntax requirements, used by functions like
 \kbd{for}, \kbd{sum}, etc.:
%
\+& \kbd{=} & separator \kbd{=} required at this point (between two
arguments)\cr

\noindent$\bullet$ Mandatory arguments, appearing in the same order as the
input arguments they describe:
%
\+& \kbd{G} & \kbd{GEN}\cr
\+& \kbd{\&}& \kbd{*GEN}\cr
\+& \kbd{L} & long {\rm (we implicitly identify \kbd{int} with \kbd{long})}\cr
\+& \kbd{S} & symbol (i.e.~GP identifier name). Function expects a
\kbd{*entree}\cr
\+& \kbd{V} & variable (as \kbd{S}, but rejects symbols associated to
functions)\cr
\+& \kbd{n} & variable, expects a \idx{variable number} (a \kbd{long}, not an
\kbd{*entree})\cr
\+& \kbd{I} & string containing a sequence of GP statements (a \var{seq}), %
to be processed by \kbd{lisseq}\cr
\+&&(useful for control statements)\cr
\+& \kbd{E} & string containing a {\it single} GP statement (an %
\var{expr}), to be processed by \kbd{lisexpr}\cr
\+& \kbd{r} & raw input (treated as a string without quotes). Quoted %
 args are copied as strings\cr
\+&&\quad Stops at first unquoted \kbd{')'} or \kbd{','}. Special chars can
be quoted using '\kbd{\bs}'\cr
\+&&\quad Example: \kbd{aa"b\bs n)"c} yields the string \kbd{"aab\bs{n})c"}\cr
\+& \kbd{s} & expanded string. Example: \kbd{Pi"x"2} yields \kbd{"3.142x2"}\cr
\+&& Unquoted components can be of any PARI type (converted following current
output\cr
\+&& format)\cr

\noindent$\bullet$ Optional arguments:
%
\+& \kbd{s*} & any number of strings, possibly 0 (see \kbd{s})\cr
\+& \kbd{s*p} & idem, setting $\kbd{prettyp} = 1$ (i.e.~in beautified
format)\cr
\+& \kbd{s*t} & idem, in \TeX\ format\cr
\+& \kbd{D\var{xxx}} &  argument has a default value\cr

The format to indicate a default value (atom starts with a \kbd{D}) is
``\kbd{D\var{value},\var{type},}'', where \var{type} is the code for any
mandatory atom (previous group), \var{value} is any valide GP expression
which is converted according to \var{type}, and the ending comma is
mandatory. For instance \kbd{D0,L,} stands for ``this optional argument will
be converted to a \kbd{long}, and is \kbd{0} by default''. So if the
user-given argument reads \kbd{1 + 3} at this point, \kbd{(long)4} is sent to
the function (via \tet{itos}$()$); and \kbd{(long)0} if the argument is
ommitted. The following special syntaxes are available:

\settabs\+\indent\indent&\kbd{Dxxx}\quad& optional \kbd{*GEN},&\cr
\+&\kbd{DG}& optional \kbd{GEN}, & send \kbd{NULL} if argument omitted.\cr

\+&\kbd{D\&}& optional \kbd{*GEN}, send \kbd{NULL} if argument omitted.\cr

\+&\kbd{DV}& optional \kbd{*entree}, send \kbd{NULL} if argument omitted.\cr

\+&\kbd{DI}& optional \kbd{*char}, send \kbd{NULL} if argument omitted.\cr

\+&\kbd{Dn}& optional variable number, $-1$ if omitted.\cr

\noindent$\bullet$ Automatic arguments:
%
\+& \kbd{f} &  Fake \kbd{*long}. C function requires a pointer but we
don't use the resulting \kbd{long}\cr
\+& \kbd{p} &  real precision (default \kbd{realprecision})\cr
\+& \kbd{P} &  series precision (default \kbd{seriesprecision},
 global variable \kbd{precdl} for the library)\cr

\noindent $\bullet$ Return type: \kbd{GEN} by default, otherwise the
following can appear at the start of the code string:
%
\+& \kbd{l} & return \kbd{long}\cr
\+& \kbd{v} & return \kbd{void}\cr

No more than 8 arguments can be given (syntax requirements and return types
are not considered as arguments). This is currently hardcoded but can
trivially be changed by modifying the definition of \kbd{argvec} in
\kbd{anal.c:identifier()}. This limitation should disappear in future
versions.

When the function is called under GP, the prototype is scanned and each time
an atom corresponding to a mandatory argument is met, a user-given argument
is read (GP outputs an error message it the argument was missing). Each time
an optional atom is met, a default value is inserted if the user omits the
argument. The ``automatic'' atoms fill in the argument list transparently,
supplying the current value of the corresponding variable (or a dummy
pointer).

For instance, here is how you would code the following prototypes (which
don't involve default values):
\bprog
GEN name(GEN x, GEN y, long prec)   ----> "GGp"
void name(GEN x, GEN y, long prec)  ----> "vGGp"
void name(GEN x, long y, long prec) ----> "vGLp"
long name(GEN x)                    ----> "lG"
@eprog

If you want more examples, GP gives you easy access to the parser codes
associated to all GP functions: just type \kbd{\b{h} \var{function}}. You
can then compare with the C prototypes as they stand in the code.

\misctitle{Remark}: If you need to implement complicated control statements
(probably for some improved summation functions), you'll need to know about
the \teb{entree} type, which is not documented. Check the comment before
the function list at the end of \kbd{language/init.c} and the source code
in \kbd{language/sumiter.c}. You should be able to make something of it.
\smallskip

\subsec{Coding guidelines}.
\noindent
Code your function in a file of its own, using as a guide other functions
in the PARI sources. One important thing to remember is to clean the stack
before exiting your main function (usually using \kbd{gerepile}), since
otherwise successive calls to the function will clutter the stack with
unnecessary garbage, and stack overflow will occur sooner. Also, if it
returns a \kbd{GEN} and you want it to be accessible to GP, you have to
make sure this \kbd{GEN} is suitable for \kbd{gerepileupto} (see
\secref{se:garbage}).

If error messages are to be generated in your function, use the general
error handling routine \kbd{err} (see \secref{se:err}). Recall that, apart
from the \kbd{warn} variants, this function does not return but ends with
a \kbd{longjmp} statement. As well, instead of explicit \kbd{printf}~/
\kbd{fprintf} statements, use the following encapsulated variants:

\fun{void}{pariputs}{char *s}: write \kbd{s} to the GP output stream.

\fun{void}{fprintferr}{char *s}: write \kbd{s} to the GP error stream (this
function is in fact much more versatile, see \secref{se:dbg_output}).

Declare all public functions in an appropriate header file, if you
expect somebody will access them from C.  For example, if dynamic
loading is not available, you may need to modify PARI to access these
functions, so put them in \kbd{paridecl.h}. The other functions should
be declared \kbd{static} in your file.

Your function is now ready to be used in library mode after compilation and
creation of the library. If possible, compile it as a shared library (see
the \kbd{Makefile} coming with the \kbd{matexp} example in the
distribution). It is however still inaccessible from GP.\smallskip

\subsec{Integration with GP as a shared module}

To tell GP about your function, you must do the following. First, find a
name for it. It does not have to match the one used in library mode, but
consistency is nice. It has to be a valid GP identifier, i.e.~use only
alphabetic characters, digits and the underscore character (\kbd{\_}), the
first character being alphabetic.

Then you have to figure out the correct \idx{parser code} corresponding to
the function prototype. This has been explained above
(\secref{se:gp.interface}).

Now, assuming your Operating System is supported by \tet{install}, simply
write a GP script like the following:

\bprog
install(libname, code, gpname, library)
addhelp(gpname, "some help text")
@eprog
\noindent(see \secref{se:addhelp} and~\ref{se:install}). The \idx{addhelp}
part is not mandatory, but very useful if you want others to use your
module.  \kbd{libname} is how the function is named in the library,
usually the same name as one visible from C.

Read that file from your GP session (from your \idx{preferences file} for
instance, see \secref{se:gprc}), and that's it, you can use the new
function \var{gpname} under GP (and we would very much like to hear about
it!).

\subsec{Integration the hard way}

If \tet{install} is not available for your Operating System, it's more
complicated: you have to hardcode your function in the GP binary (or
install \idx{Linux}). Here's what needs to be done: 

In the definition of \kbd{functions\_basic} (file \kbd{language/init.c}),
add your entry in exact alphabetical order by its GP name (note that digits
come before letters), in a line of the form:
\bprog
{ "gpname", V, (void*)libname, secno, "code" }
@eprog

\noindent where

\kbd{libname} is the name of your function in library mode,

\kbd{gpname} the name that you have chosen to call it under GP,

\kbd{secno} is the section number of Chapter~3 in which this function would
belong (type \kbd{?} in GP to see the list),

\kbd{V} is a number between 0 and 99. Right now, for PARI there are only two
significant values: zero means that it's possible to call the function
without argument, and non-zero means it needs at least one argument.
A binding of PARI to an external language (such as \kbd{Math::Pari}
Perl module) may actually distinguish between different non-zero values.
Better use 99 if you want a non-zero value which will not confuse anybody.

\kbd{code} is the parser code.

Once this has been done, in the file \kbd{language/helpmessages.c} add in
exact alphabetical order a short message describing the effect of your
function:
\kbd{"name(x,y,...)=short descriptive message",}

The message must be a single line, of arbitrary length. Do not use
\kbd{\bs{n}}; the necessary newlines will be inserted by GP's online help
functions. Optional arguments should be shown between braces (see the other
messages for comparison).\smallskip

Now, you can recompile GP.

\subsec{Example}.
%
A complete description could look like this:

\bprog
{
  install(bnfinit0, GD0,L,DGp, ClassGroupInit, "libpari.so");
  addhelp(ClassGroupInit, "ClassGroupInit(P,{flag=0},{data=[]}):
    compute the necessary data for ...");
}
@eprog
which means we have a function \kbd{ClassGroupInit} under GP, which calls
the library function \kbd{bnfinit0} . The function has one mandatory
argument, and possibly two more (two \kbd{'D'} in the code), plus the
current real precision. More precisely, the first argument is a \kbd{GEN},
the second one is converted to a \kbd{long} using \kbd{itos} (\kbd{0} is
passed if it is omitted), and the third one is also a \kbd{GEN}, but we
pass \kbd{NULL} if no argument was supplied by the user. This matches the C
prototype (from \kbd{paridecl.h}):
%
\bprog
  GEN bnfinit0(GEN P, long flag, GEN data, long prec)
@eprog

This function is in fact coded in \kbd{basemath/buch2.c}, and will in this
case be completely identical to the GP function \kbd{bnfinit} but GP does
not need to know about this, only that it can be found somewhere in the
shared library \kbd{libpari.so}.

\misctitle{Important note:} You see in this example that it is the
function's responsibility to correctly interpret its operands: \kbd{data =
NULL} is interpreted {\it by the function\/} as an empty vector. Note that
since \kbd{NULL} is never a valid \kbd{GEN} pointer, this trick always
enables you to distinguish between a default value and actual input: the
user could explicitly supply an empty vector!

\misctitle{Note:} If \kbd{install} were not available we would have to
modify \kbd{language/helpmessages.c}, and \kbd{language/init.c} and
recompile GP. The entry in \kbd{functions\_basic} corresponding to the
function above is actually
\bprog
{ "bnfinit", 91, (void*)bnfinit0, 6, "GD0,L,DGp" }
@eprog
\vfill\eject
