% Copyright (c) 2000  The PARI Group
%
% This file is part of the PARI/GP documentation
%
% Permission is granted to copy, distribute and/or modify this document
% under the terms of the GNU General Public License
\chapter{Programming PARI in Library Mode}

\noindent The \emph{User's Guide to Pari/GP} gives in three chapters a
general presentation of the system, of the \kbd{gp} calculator, and detailed
explanation of high level PARI routines available through the calculator. The
present manual assumes general familiarity with the contents of these
chapters and the basics of ANSI C programming, and focuses on the usage of
the PARI library. In this chapter, we introduce the general concepts of PARI
programming and describe useful general purpose functions; the following
chapters describes all public low or high-level functions, underlying or
extending the GP functions seen in Chapter 3 of the User's guide.

\section{Introduction: initializations, universal objects}
\label{se:intro4}

\noindent
To use PARI in \idx{library mode}, you must write a C program and link it to
the PARI library. See the installation guide or the Appendix to the
\emph{User's Guide to Pari/GP} on how to create and install the library and
include files. A sample Makefile is presented in Appendix~A, and a more
elaborate one in \kbd{examples/Makefile}. The best way to understand how
programming is done is to work through a complete example. We will write such
a program in~\secref{se:prog}. Before doing this, a few explanations are in
order.

First, one must explain to the outside world what kind of objects and
routines we are going to use. This is done\footnote{*}{This assumes that PARI
headers are installed in a directory which belongs to your compiler's search
path for header files. You might need to add flags like
\kbd{-I/usr/local/include} or modify \tet{C_INCLUDE_PATH}.}
with the directive

\bprog
#include <pari/pari.h>
@eprog
\noindent
In particular, this defines the fundamental type for all PARI objects: the
type \teb{GEN}, which is simply a pointer to \kbd{long}.

Before any PARI routine is called, one must initialize the system, and in
particular the PARI stack which is both a scratchboard and a repository for
computed objects. This is done with a call to the function

\fun{void}{pari_init}{size_t size, ulong maxprime}

\noindent The first argument is the number of bytes given to PARI to work
with, and the second is the upper limit on a precomputed prime number table;
\kbd{size} should not reasonably be taken below $500000$ but you may set
$\tet{maxprime} = 0$, although the system still needs to precompute all
primes up to about $2^{16}$. For lower-level variants allowing finer
control, e.g.~preventing PARI from installing its own error or signal
handlers, see~\secref{se:pari_init_tech}.

\noindent We have now at our disposal:

\item a PARI \tev{stack} containing nothing. This is a big
connected chunk of \kbd{size} bytes of memory, where all computations
take place. In large computations, intermediate results quickly
clutter up memory so some kind of garbage collecting is needed. Most
systems do garbage collecting when the memory is getting scarce, and this
slows down the performance. PARI takes a different approach, admittedly more
demanding on the programmer: you must do your own cleaning up when the
intermediate results are not needed anymore. We will see later how (and when)
this is done.

\item the following \emph{universal objects} (by definition, objects
which do not belong to the stack): the integers $0$, $1$, $-1$, $2$ and
$-2$ (respectively called \tet{gen_0}, \tet{gen_1}, \tet{gen_m1},
\tet{gen_2} and \tet{gen_m2}), the fraction $\dfrac{1}{2}$ (\tet{ghalf}).
All of these are of type \kbd{GEN}.

\item a \tev{heap} which is just a linked list of permanent
universal objects. For now, it contains exactly the ones listed above. You
will probably very rarely use the heap yourself; and if so, only as a
collection of copies of objects taken from the stack (called \idx{clone}s in
the sequel). Thus you need not bother with its internal structure, which may
change as PARI evolves. Some complex PARI functions create clones for special
garbage collecting purposes, usually destroying them when returning.

\item a table of primes (in fact of \emph{differences} between
consecutive primes), called \teb{diffptr}, of type \kbd{byteptr}
(pointer to \kbd{unsigned char}). Its use is described in
\secref{se:primetable} later. Using it directly is deprecated,
high-level iterators provide a cleaner and more flexible interface, see
\secref{se:primeiter} (such iterators use the private prime table, but extend
it dynamically).

\item access to all the built-in functions of the PARI library.
These are declared to the outside world when you include \kbd{pari.h}, but
need the above things to function properly. So if you forget the call to
\tet{pari_init}, you will get a fatal error when running your program.

\section{Important technical notes}

\subsec{Backward compatibility} The PARI function names evolved over time,
and deprecated functions are eventually deleted.  The file \kbd{pariold.h}
contains macros implementing a weak form of backward compatibility.
In particular, whenever the name of a documented function changes, a
\kbd{\#define} is added to this file so that the old name expands to the new
one (provided the prototype didn't change also).

This file is included by \kbd{pari.h}, but a large section is commented out
by default. Define \tet{PARI_OLD_NAMES} before including \kbd{pari.h} to
pollute your namespace with lots of obsolete names like
\kbd{un}\footnote{*}{For \kbd{(long)gen\_1}. Since 2004 and version 2.2.9,
typecasts are completely unnecessary in PARI programs.}: that might enable
you to compile old programs without having to modify them. The preferred way
to do that is to add \kbd{-DPARI\_OLD\_NAMES} to your compiler \kbd{CFLAGS},
so that you don't need to modify the program files themselves.

Of course, it's better to fix the program if you can!

\subsec{Types}

\noindent
Although PARI objects all have the C type \kbd{GEN}, we will freely use
the word \teb{type} to refer to PARI dynamic subtypes: \typ{INT}, \typ{REAL},
etc. The declaration
\bprog
  GEN x;
@eprog\noindent
declares a C variable of type \kbd{GEN}, but its ``value'' will be said to
have type \typ{INT}, \typ{REAL}, etc. The meaning should always be clear from
the context.

\subsec{Type recursivity}

\noindent
Conceptually, most PARI types are recursive. But the \kbd{GEN} type is a
pointer to \kbd{long}, not to \kbd{GEN}. So special macros must be used to
access \kbd{GEN}'s components. The simplest one is \tet{gel}$(V, i)$, where
\key{el} stands for \key{el}ement, to access component number $i$ of the
\kbd{GEN} $V$. This is a valid \kbd{lvalue} (may be put on the left side of
an assignment), and the following two constructions are exceedingly frequent
%
\bprog
  gel(V, i) = x;
  x = gel(V, i);
@eprog\noindent
where \kbd{x} and \kbd{V} are \kbd{GEN}s. This macro accesses and modifies
directly the components of $V$ and do not create a copy of the coefficient,
contrary to all the library \emph{functions}.

More generally, to retrieve the values of elements of lists of \dots\ of
lists of vectors we have the \tet{gmael} macros (for {\bf m}ultidimensional
{\bf a}rray {\bf el}ement). The syntax is $\key{gmael}n(V,a_1,\dots,a_n)$,
where $V$ is a \kbd{GEN}, the $a_i$ are indexes, and $n$ is an integer
between $1$ and $5$. This stands for $x[a_1][a_2]\dots[a_n]$, and returns a
\kbd{GEN}. The macros \tet{gel} (resp.~\tet{gmael}) are synonyms for
\tet{gmael1} (resp.~\kbd{gmael2}).

Finally, the macro $\tet{gcoeff}(M, i, j)$ has exactly the meaning of
\kbd{M[i,j]} in GP when \kbd{M} is a matrix. Note that due to the
implementation of \typ{MAT}s as horizontal lists of vertical vectors,
\kbd{gcoeff(x,y)} is actually equivalent to \kbd{gmael(y,x)}. One should use
\kbd{gcoeff} in matrix context, and \kbd{gmael} otherwise.

\subsec{Variations on basic functions}\label{se:low_level} In the library
syntax descriptions in Chapter~3, we have only given the basic names of the
functions. For example \kbd{gadd}$(x,y)$ assumes that $x$ and $y$ are
\kbd{GEN}s, and \emph{creates} the result $x+y$ on the PARI stack. For most
of the basic operators and functions, many other variants are available. We
give some examples for \kbd{gadd}, but the same is true for all the basic
operators, as well as for some simple common functions (a complete list
is given in Chapter~6):

\fun{GEN}{gaddgs}{GEN x, long y}

\fun{GEN}{gaddsg}{long x, GEN y}

\noindent In the following one, \kbd{z} is a preexisting \kbd{GEN} and the
result of the corresponding operation is put into~\kbd{z}. The size of the PARI
stack does not change:

\fun{void}{gaddz}{GEN x, GEN y, GEN z}

\noindent (This last form is inefficient in general and deprecated outside of
PARI kernel programming.) Low level kernel functions implement these
operators for specialized arguments and are also available: Level 0 deals
with operations at the word level (\kbd{long}s and \kbd{ulong}s), Level 1
with \typ{INT} and \typ{REAL} and Level 2 with the rest (modular arithmetic,
polynomial arithmetic and linear algebra). Here are some examples of Level
$1$ functions:

\fun{GEN}{addii}{GEN x, GEN y}: here $x$ and $y$ are \kbd{GEN}s of type
\typ{INT} (this is not checked).

\fun{GEN}{addrr}{GEN x, GEN y}: here $x$ and $y$ are \kbd{GEN}s of
type \typ{REAL} (this is not checked).

\noindent
There also exist functions \teb{addir}, \teb{addri}, \teb{mpadd} (whose
two arguments can be of type \typ{INT} or \typ{REAL}), \teb{addis} (to add a
\typ{INT} and a \kbd{long}) and so on.

The Level $1$ names are self-explanatory once you know that {\bf i} stands for a
\typ{INT}, {\bf r} for a \typ{REAL}, {\bf mp} for i or r, {\bf s} for a signed C
long integer, {\bf u} for an unsigned C long integer; finally the suffix {\bf z}
means that the result is not created on the PARI stack but assigned to a
preexisting GEN object passed as an extra argument. Chapter 6 gives a
description of these low-level functions.

Level $2$ names are more complicated, see \secref{se:level2names} for all the
gory details, and we content ourselves with a simple example used to implement
\typ{INTMOD} arithmetic:

\fun{GEN}{Fp_add}{GEN x, GEN y, GEN m}: returns the sum of $x$ and $y$ modulo
$m$. Here $x, y, m$ are \typ{INT}s (this is not checked). The operation is
more efficient if the inputs $x$, $y$ are reduced modulo $m$, but this is not
a necessary condition.

\misctitle{Important Note} These specialized functions are of course more
efficient than the generic ones, but note the hidden danger here: the types
of the objects involved (which is not checked) must be severely controlled,
e.g.~using \kbd{addii} on a \typ{FRAC} argument will cause disasters. Type
mismatches may corrupt the PARI stack, though in most cases they will just
immediately overflow the stack. Because of this, the PARI philosophy of
giving a result which is as exact as possible, enforced for generic functions
like \kbd{gadd} or \kbd{gmul}, is dropped in kernel routines of Level $1$,
where it is replaced by the much simpler rule: the result is a \typ{INT} if
and only if all arguments are integer types (\typ{INT} but also C \kbd{long}
and \kbd{ulong}) and a \typ{REAL} otherwise. For instance, multiplying a
\typ{REAL} by a \typ{INT} always yields a \typ{REAL} if you use \kbd{mulir},
where \kbd{gmul} returns the \typ{INT} \kbd{gen\_0} if the integer is $0$.

\subsec{Portability: 32-bit / 64-bit architectures}

\noindent
PARI supports both 32-bit and 64-bit based machines, but not simultaneously!
The library is compiled assuming a given architecture, and some
of the header files you include (through \kbd{pari.h}) will have been
modified to match the library.

Portable macros are defined to bypass most machine dependencies. If you want
your programs to run identically on 32-bit and 64-bit machines, you have to
use these, and not the corresponding numeric values, whenever the precise
size of your \kbd{long} integers might matter. Here are the most important
ones:

\settabs\+ -----------------------------&---------------&------------&\cr \+
& 64-bit  & 32-bit \cr\+ \tet{BITS_IN_LONG}  & 64      & 32 \cr\+
\tet{LONG_IS_64BIT} & defined & undefined \cr\+
\tet{DEFAULTPREC}   & 3       & 4 & ($\approx$ 19 decimal digits, %
see formula below) \cr\+
\tet{MEDDEFAULTPREC}& 4       & 6 & ($\approx$ 38 decimal digits) \cr\+
\tet{BIGDEFAULTPREC}& 5       & 8 & ($\approx$ 57 decimal digits) \cr
\noindent For instance, suppose you call a transcendental function, such as

\kbd{GEN \key{gexp}(GEN x, long prec)}.

\noindent The last argument \kbd{prec} is an integer $\geq 3$, corresponding
to the default floating point precision required. It is \emph{only} used if
\kbd{x} is an exact object, otherwise the relative precision is determined by
the precision of~\kbd{x}. Since the parameter \kbd{prec} sets the size of the
inexact result counted in (\kbd{long}) \emph{words} (including codewords),
the same value of \kbd{prec} will yield different results on 32-bit and
64-bit machines. Real numbers have two codewords (see~\secref{se:impl}), so
the formula for computing the bit accuracy is
$$ \tet{bit_accuracy}(\kbd{prec}) = (\kbd{prec} - 2) * \tet{BITS_IN_LONG}$$
(this is actually the definition of an inline function). The corresponding
accuracy expressed in decimal digits would be
%
$$ \kbd{bit\_accuracy(prec)} * \log(2) / \log(10).$$
%
For example if the value of \kbd{prec} is 5, the corresponding accuracy for
32-bit machines is $(5-2)*\log(2^{32})/\log(10)\approx 28$ decimal digits,
while for 64-bit machines it is $(5-2)*\log(2^{64})/\log(10)\approx 57$
decimal digits.

Thus, you must take care to change the \kbd{prec} parameter you are supplying
according to the bit size, either using the default precisions given by the
various \kbd{DEFAULTPREC}s, or by using conditional constructs of the form:
%
\bprog
#ifndef LONG_IS_64BIT
  prec = 4;
#else
  prec = 6;
#endif
@eprog
\noindent which is in this case equivalent to the statement
\kbd{prec = MEDDEFAULTPREC;}.

Note that for parity reasons, half the accuracies available on 32-bit
architectures (the odd ones) have no precise equivalents on 64-bit machines.

\subsec{Using \kbd{malloc} / \kbd{free}}
You should make use of the PARI stack as much as possible, and avoid
allocating objects using the customary functions. If you do, you should
use, or at least have a very close look at, the following wrappers:

\fun{void*}{pari_malloc}{size_t size} calls \kbd{malloc} to allocate
\kbd{size} bytes and returns a pointer to the allocated memory. If the
request fails, an error is raised. The \kbd{SIGINT} signal is blocked until
\kbd{malloc} returns, to avoid leaving the system stack in an inconsistent
state.

\fun{void*}{pari_realloc}{void* ptr, size_t size} as \tet{pari_malloc} but
calls \kbd{realloc} instead of \kbd{malloc}.

\fun{void*}{pari_calloc}{size_t size} as \tet{pari_malloc}, setting the
memory to zero.

\fun{void}{pari_free}{void* ptr} calls \kbd{free} to liberate the memory
space pointed to by \kbd{ptr}, which must have been allocated by \kbd{malloc}
(\tet{pari_malloc}) or \kbd{realloc} (\tet{pari_realloc}). The \kbd{SIGINT}
signal is blocked until \kbd{free} returns.

If you use the standard \kbd{libc} functions instead of our wrappers, then
your functions will be subtly incompatible with the \kbd{gp} calculator: when
the user tries to interrupt a computation, the calculator may crash
(if a system call is interrupted at the wrong time).

\section{Garbage collection}\label{se:garbage}\sidx{garbage collecting}

\subsec{Why and how}

\noindent
As we have seen, \kbd{pari\_init} allocates a big range of
addresses, the \tev{stack}, that are going to be used throughout. Recall
that all PARI objects are pointers. Except for a few universal objects,
they all point at some part of the stack.

The stack starts at the address \kbd{bot} and ends just before \kbd{top}. This
means that the quantity
%
$$ (\kbd{top} - \kbd{bot})\,/\,\kbd{sizeof(long)} $$
%
is (roughly) equal to the \kbd{size} argument of \kbd{pari\_init}. The PARI
stack also has a ``current stack pointer'' called \teb{avma}, which stands
for {\bf av}ailable {\bf m}emory {\bf a}ddress. These three variables are
global (declared by \kbd{pari.h}). They are of type \tet{pari_sp}, which
means \emph{pari stack pointer}.

The stack is oriented upside-down: the more recent an object, the closer to
\kbd{bot}. Accordingly, initially \kbd{avma} = \kbd{top}, and \kbd{avma} gets
\emph{decremented} as new objects are created. As its name indicates,
\kbd{avma} always points just \emph{after} the first free address on the
stack, and \kbd{(GEN)avma} is always (a pointer to) the latest created object.
When \kbd{avma} reaches \kbd{bot}, the stack overflows, aborting all
computations, and an error message is issued. To avoid this \emph{you}
need to clean up the stack from time to time, when intermediate objects are
not needed anymore. This is called ``\emph{garbage collecting}.''

We are now going to describe briefly how this is done. We will see many
concrete examples in the next subsection.

\noindent\item
First, PARI routines do their own garbage collecting, which means that
whenever a documented function from the library returns, only its result(s)
have been added to the stack, possibly up to a very small overhead
(non-documented ones may not do this). In
particular, a PARI function that does not return a \kbd{GEN} does not clutter
the stack. Thus, if your computation is small enough (e.g.~you call few PARI
routines, or most of them return \kbd{long} integers), then you do not need
to do any garbage collecting. This is probably the case in many of your
subroutines. Of course the objects that were on the stack \emph{before} the
function call are left alone. Except for the ones listed below, PARI
functions only collect their own garbage.

\noindent\item
It may happen that all objects that were created after a certain point can
be deleted~--- for instance, if the final result you need is not a
\kbd{GEN}, or if some search proved futile. Then, it is enough to record
the value of \kbd{avma} just \emph{before} the first garbage is created,
and restore it upon exit:

\bprog
pari_sp av = avma; /*@Ccom record initial avma */

garbage ...
avma = av; /*@Ccom restore it */
@eprog
\noindent All objects created in the \kbd{garbage} zone will eventually
be overwritten: they should no longer be accessed after \kbd{avma} has been
restored.

\noindent\item
If you want to destroy (i.e.~give back the memory occupied by) the
\emph{latest} PARI object on the stack (e.g.~the latest one obtained from a
function call), you can use the function\sidx{destruction}%
\vadjust{\penalty500}%discourage page break

\fun{void}{cgiv}{GEN z}

\noindent where \kbd{z} is the object you want to give back. This is
equivalent to the above where the initial \kbd{av} is computed from \kbd{z}.

\noindent\item
Unfortunately life is not so simple, and sometimes you will want
to give back accumulated garbage \emph{during} a computation without losing
recent data. We shall start with the lowest level function to get a feel for
the underlying mechanisms, we shall describe simpler variants later:

\fun{GEN}{gerepile}{pari_sp ltop, pari_sp lbot, GEN q}. This function cleans
up the stack between \kbd{ltop} and \kbd{lbot}, where $\kbd{lbot} <
\kbd{ltop}$, and returns the updated object \kbd{q}. This means:

1) we translate (copy) all the objects in the interval
$[\kbd{avma}, \kbd{lbot}[$, so that its right extremity abuts the address
\kbd{ltop}. Graphically

\vbox{\bprog
             bot           avma   lbot          ltop     top
End of stack  |-------------[++++++[-/-/-/-/-/-/-|++++++++|  Start
                free memory            garbage
@eprog
\noindent becomes:
\bprog
             bot                         avma   ltop     top
End of stack  |---------------------------[++++++[++++++++|  Start
                       free memory
@eprog
}
\noindent where \kbd{++} denote significant objects, \kbd{--} the unused part
of the stack, and \kbd{-/-} the garbage we remove.

2) The function then inspects all the PARI objects between \kbd{avma} and
\kbd{lbot} (i.e.~the ones that we want to keep and that have been translated)
and looks at every component of such an object which is not a codeword. Each
such component is a pointer to an object whose address is either

--- between \kbd{avma} and \kbd{lbot}, in which case it is suitably updated,

--- larger than or equal to \kbd{ltop}, in which case it does not change, or

--- between \kbd{lbot} and \kbd{ltop} in which case \kbd{gerepile}
raises an error (``significant pointers lost in gerepile'').

3) \key{avma} is updated (we add $\kbd{ltop} - \kbd{lbot}$ to the old value).

4) We return the (possibly updated) object \kbd{q}: if \kbd{q} initially
pointed between \kbd{avma} and \kbd{lbot}, we return the updated address, as
in~2). If not, the original address is still valid, and is returned!

As stated above, no component of the remaining objects (in particular
\kbd{q}) should belong to the erased segment [\kbd{lbot}, \kbd{ltop}[, and
this is checked within \kbd{gerepile}. But beware as well that the addresses
of the objects in the translated zone change after a call to \kbd{gerepile},
so you must not access any pointer which previously pointed into the zone
below \kbd{ltop}. If you need to recover more than one object, use the
\kbd{gerepileall} function below.

\misctitle{Remark}
As a consequence of the preceding explanation, if a PARI object is to be
relocated by \hbox{gerepile} then, apart from universal objects, the chunks
of memory used by its components should be in consecutive memory locations.
All \kbd{GEN}s created by documented PARI functions are guaranteed to satisfy
this. This is because the \kbd{gerepile} function knows only about \emph{two
connected zones}: the garbage that is erased (between \kbd{lbot} and
\kbd{ltop}) and the significant pointers that are copied and updated. If
there is garbage interspersed with your objects, disaster occurs when we try
to update them and consider the corresponding ``pointers''. In most cases of
course the said garbage is in fact a bunch of other \kbd{GEN}s, in which case
we simply waste time copying and updating them for nothing. But be wary when
you allow objects to become disconnected.

\noindent In practice this is achieved by the following programming idiom:
\bprog
  ltop = avma; garbage(); lbot = avma; q = anything();
  return gerepile(ltop, lbot, q); /*@Ccom returns the updated q */
@eprog\noindent or directly
\bprog
  ltop = avma; garbage(); lbot = avma;
  return gerepile(ltop, lbot, anything());
@eprog\noindent
Beware that
\bprog
  ltop = avma; garbage();
  return gerepile(ltop, avma, anything())
@eprog

\noindent might work, but should be frowned upon. We cannot predict whether
\kbd{avma} is evaluated after or before the call to \kbd{anything()}: it
depends on the compiler. If we are out of luck, it is \emph{after} the
call, so the result belongs to the garbage zone and the \kbd{gerepile}
statement becomes equivalent to \kbd{avma = ltop}. Thus we return a
pointer to random garbage.

\subsec{Variants}

\fun{GEN}{gerepileupto}{pari_sp ltop, GEN q}. Cleans the stack between
\kbd{ltop} and the \emph{connected} object \kbd{q} and returns \kbd{q}
updated. For this to work, \kbd{q} must have been created \emph{before} all
its components, otherwise they would belong to the garbage zone! Unless
mentioned otherwise, documented PARI functions guarantee this.

\fun{GEN}{gerepilecopy}{pari_sp ltop, GEN x}. Functionally equivalent to,
but more efficient than
\bprog
  gerepileupto(ltop, gcopy(x))
@eprog\noindent In this case, the \kbd{GEN} parameter \kbd{x} need not
satisfy any property before the garbage collection: it may be disconnected,
components created before the root, and so on. Of course, this is about
twice slower than either \tet{gerepileupto} or \tet{gerepile}, because
\kbd{x} has to be copied to a clean stack zone first. This function is a
special case of \tet{gerepileall} below, where $n=1$.

\fun{void}{gerepileall}{pari_sp ltop, int n, ...}.
To cope with complicated cases where many objects have to be preserved. The
routine expects $n$ further arguments, which are the \emph{addresses} of
the \kbd{GEN}s you want to preserve:
\bprog
  pari_sp ltop = avma;
  ...; y = ...; ... x = ...; ...;
  gerepileall(ltop, 2, &x, &y);
@eprog\noindent
It cleans up the most recent part of the
stack (between \kbd{ltop} and \kbd{avma}), updating all the \kbd{GEN}s added
to the argument list. A copy is done just before the cleaning to preserve
them, so they do not need to be connected before the call. With
\kbd{gerepilecopy}, this is the most robust of the \kbd{gerepile} functions
(the less prone to user error), hence the slowest.

\fun{void}{gerepileallsp}{pari_sp ltop, pari_sp lbot, int n, ...}.
More efficient, but trickier than \kbd{gerepileall}. Cleans the stack between
\kbd{lbot} and \kbd{ltop} and updates the \kbd{GEN}s pointed at by the
elements of \kbd{gptr} without any further copying. This is subject to the
same restrictions as \kbd{gerepile}, the only difference being that more than
one address gets updated.

\subsec{Examples}

\subsubsec{gerepile}

Let \kbd{x} and \kbd{y} be two preexisting PARI objects and suppose that we
want to compute $\kbd{x}^2 + \kbd{y}^2$. This is done using the following
program:
\bprog
  GEN x2 = gsqr(x);
  GEN y2 = gsqr(y), z = gadd(x2,y2);
@eprog\noindent
The \kbd{GEN} \kbd{z} indeed points at the desired quantity. However,
consider the stack: it contains as unnecessary garbage \kbd{x2} and \kbd{y2}.
More precisely it contains (in this order) \kbd{z}, \kbd{y2}, \kbd{x2}.
(Recall that, since the stack grows downward from the top, the most recent
object comes first.)

It is not possible to get rid of \kbd{x2}, \kbd{y2} before \kbd{z} is
computed, since they are used in the final operation. We cannot record
\kbd{avma} before \kbd{x2} is computed and restore it later, since this would
destroy \kbd{z} as well. It is not possible either to use the function
\kbd{cgiv} since \kbd{x2} and \kbd{y2} are not at the bottom of the stack and
we do not want to give back~\kbd{z}.

But using \kbd{gerepile}, we can give back the memory locations corresponding
to \kbd{x2}, \kbd{y2}, and move the object \kbd{z} upwards so that no
space is lost. Specifically:
\bprog
  pari_sp ltop = avma;  /*@Ccom remember the current top of the stack */
  GEN x2 = gsqr(x);
  GEN y2 = gsqr(y);
  pari_sp lbot = avma;  /*@Ccom the bottom of the garbage pile */
  GEN z = gadd(x2, y2); /*@Ccom z is now the last object on the stack */
  z = gerepile(ltop, lbot, z);
@eprog
\noindent Of course, the last two instructions could also have been
written more simply:
\bprog
  z = gerepile(ltop, lbot, gadd(x2,y2));
@eprog\noindent In fact \kbd{gerepileupto} is even simpler to use, because
the result of \kbd{gadd} is the last object on the stack and \kbd{gadd}
is guaranteed to return an object suitable for \kbd{gerepileupto}:
\bprog
  ltop = avma;
  z = gerepileupto(ltop, gadd(gsqr(x), gsqr(y)));
@eprog\noindent
Make sure you understand exactly what has happened before you go on!

\misctitle{Remark on assignments and gerepile} When the tree structure and
the size of the PARI objects which will appear in a computation are under
control, one may allocate sufficiently large objects at the beginning,
use assignment statements, then simply restore \kbd{avma}. Coming back to the
above example, note that \emph{if} we know that x and y are of type real
fitting into \kbd{DEFAULTPREC} words, we can program without using
\kbd{gerepile} at all:
\bprog
  z = cgetr(DEFAULTPREC); ltop = avma;
  gaffect(gadd(gsqr(x), gsqr(y)), z);
  avma = ltop;
@eprog\noindent This is often \emph{slower} than a craftily used
\kbd{gerepile} though, and certainly more cumbersome to use. As a rule,
assignment statements should generally be avoided.

\misctitle{Variations on a theme} it is often necessary to do several
\kbd{gerepile}s during a computation. However, the fewer the better. The only
condition for \kbd{gerepile} to work is that the garbage be connected. If the
computation can be arranged so that there is a minimal number of connected
pieces of garbage, then it should be done that way.

For example suppose we want to write a function of two \kbd{GEN} variables
\kbd{x} and \kbd{y} which creates the vector $\kbd{[x}^2+\kbd{y},
\kbd{y}^2+\kbd{x]}$. Without garbage collecting, one would write:
%
\bprog
  p1 = gsqr(x); p2 = gadd(p1, y);
  p3 = gsqr(y); p4 = gadd(p3, x);
  z = mkvec2(p2, p4);  /* not suitable for gerepileupto! */
@eprog\noindent
This leaves a dirty stack containing (in this order) \kbd{z}, \kbd{p4},
\kbd{p3}, \kbd{p2}, \kbd{p1}. The garbage here consists of \kbd{p1} and
\kbd{p3}, which are separated by \kbd{p2}. But if we compute \kbd{p3}
\emph{before} \kbd{p2} then the garbage becomes connected, and we get the
following program with garbage collecting:
%
\bprog
  ltop = avma; p1 = gsqr(x); p3 = gsqr(y);
  lbot = avma; z = cgetg(3, t_VEC);
  gel(z, 1) = gadd(p1,y);
  gel(z, 2) = gadd(p3,x); z = gerepile(ltop,lbot,z);
@eprog\noindent Finishing by \kbd{z = gerepileupto(ltop, z)} would be ok as
well. Beware that
\bprog
  ltop = avma; p1 = gadd(gsqr(x), y); p3 = gadd(gsqr(y), x);
  z = cgetg(3, t_VEC);
  gel(z, 1) = p1;
  gel(z, 2) = p3; z = gerepileupto(ltop,z); /*@Ccom WRONG */
@eprog\noindent
is a disaster since \kbd{p1} and \kbd{p3} are created before
\kbd{z}, so the call to \kbd{gerepileupto} overwrites them, leaving
\kbd{gel(z, 1)} and \kbd{gel(z, 2)} pointing at random data! The following
does work:
\bprog
  ltop = avma; p1 = gsqr(x); p3 = gsqr(y);
  lbot = avma; z = mkvec2(gadd(p1,y), gadd(p3,x));
  z = gerepile(ltop,lbot,z);
@eprog\noindent but is very subtly wrong in the sense that
\kbd{z = gerepileupto(ltop, z)} would \emph{not} work. The reason being
that \kbd{mkvec2} creates the root \kbd{z} of the vector \emph{after}
its arguments have been evaluated, creating the components of \kbd{z}
too early; \kbd{gerepile} does not care, but the created \kbd{z} is a time
bomb which will explode on any later \kbd{gerepileupto}.
On the other hand
\bprog
  ltop = avma; z = cgetg(3, t_VEC);
  gel(z, 1) = gadd(gsqr(x), y);
  gel(z, 2) = gadd(gsqr(y), x); z = gerepileupto(ltop,z); /*@Ccom INEFFICIENT */
@eprog\noindent
leaves the results of \kbd{gsqr(x)} and \kbd{gsqr(y)} on the stack (and
lets \kbd{gerepileupto} update them for naught). Finally, the most elegant
and efficient version (with respect to time and memory use) is as follows
\bprog
  z = cgetg(3, t_VEC);
  ltop = avma; gel(z, 1) = gerepileupto(ltop, gadd(gsqr(x), y));
  ltop = avma; gel(z, 2) = gerepileupto(ltop, gadd(gsqr(y), x));
@eprog\noindent
which avoids updating the container \kbd{z} and cleans up its components
individually, as soon as they are computed.

\misctitle{One last example} Let us compute the product of two complex
numbers $x$ and $y$, using the $3M$ method which requires 3 multiplications
instead of the obvious 4. Let $z = x*y$, and set $x = x_r + i*x_i$ and
similarly for $y$ and $z$. We compute $p_1 = x_r*y_r$, $p_2=x_i*y_i$,
$p_3=(x_r+x_i)*(y_r+y_i)$, and then we have $z_r=p_1-p_2$,
$z_i=p_3-(p_1+p_2)$. The program is as follows:
%
\bprog
ltop = avma;
p1 = gmul(gel(x,1), gel(y,1));
p2 = gmul(gel(x,2), gel(y,2));
p3 = gmul(gadd(gel(x,1), gel(x,2)), gadd(gel(y,1), gel(y,2)));
p4 = gadd(p1,p2);
lbot = avma; z = cgetg(3, t_COMPLEX);
gel(z, 1) = gsub(p1,p2);
gel(z, 2) = gsub(p3,p4); z = gerepile(ltop,lbot,z);
@eprog

\misctitle{Exercise} Write a function which multiplies a matrix by a column
vector. Hint: start with a \kbd{cgetg} of the result, and use \kbd{gerepile}
whenever a coefficient of the result vector is computed. You can look at the
answer in \kbd{src/basemath/RgV.c:RgM\_RgC\_mul()}.

\subsubsec{gerepileall}

Let us now see why we may need the \kbd{gerepileall} variants. Although it
is not an infrequent occurrence, we do not give a specific example but a
general one: suppose that we want to do a computation (usually inside a
larger function) producing more than one PARI object as a result, say two for
instance. Then even if we set up the work properly, before cleaning up we
have a stack which has the desired results \kbd{z1}, \kbd{z2} (say), and
then connected garbage from lbot to ltop. If we write
\bprog
  z1 = gerepile(ltop, lbot, z1);
@eprog\noindent
then the stack is cleaned, the pointers fixed up, but we have lost the
address of \kbd{z2}. This is where we need the \idx{gerepileall}
function:
\bprog
  gerepileall(ltop, 2, &z1, &z2)
@eprog
\noindent copies \kbd{z1} and \kbd{z2} to new locations, cleans the stack
from \kbd{ltop} to the old \kbd{avma}, and updates the pointers \kbd{z1} and
\kbd{z2}. Here we do not assume anything about the stack: the garbage can be
disconnected and \kbd{z1}, \kbd{z2} need not be at the bottom of the stack.
If all of these assumptions are in fact satisfied, then we can call
\kbd{gerepilemanysp} instead, which is usually faster since we do not need
the initial copy (on the other hand, it is less cache friendly).

A most important usage is ``random'' garbage collection during loops
whose size requirements we cannot (or do not bother to) control in advance:
\bprog
  pari_sp av = avma;
  GEN x, y;
  while (...)
  {
    garbage(); x = anything();
    garbage(); y = anything(); garbage();
    if (gc_needed(av,1)) /*@Ccom memory is running low (half spent since entry) */
      gerepileall(av, 2, &x, &y);
  }
@eprog
\noindent Here we assume that only \kbd{x} and \kbd{y} are needed from one
iteration to the next. As it would be costly to call gerepile once for each
iteration, we only do it when it seems to have become necessary.

More precisely, the macro \tet{stack_lim}\kbd{(av,$n$)} denotes an address
where $2^{n-1} / (2^{n-1}+1)$ of the remaining stack space since reference
point \kbd{av} is exhausted ($1/2$ for $n=1$, $2/3$ for $n=2$). The test
\tet{gc_needed}\kbd{(av,$n$)} becomes true whenever \kbd{avma} drops below
that address.

\subsec{Comments}

First, \kbd{gerepile} has turned out to be a flexible and fast garbage
collector for number-theoretic computations, which compares favorably with
more sophisticated methods used in other systems. Our benchmarks indicate
that the price paid for using \kbd{gerepile} and \kbd{gerepile}-related
copies, when properly used, is usually less than 1\% of the total
running time, which is quite acceptable!

Second, it is of course harder on the programmer, and quite error-prone
if you do not stick to a consistent PARI programming style. If all seems
lost, just use \tet{gerepilecopy} (or \tet{gerepileall}) to fix up the stack
for you. You can always optimize later when you have sorted out exactly which
routines are crucial and what objects need to be preserved and their usual
sizes.

\smallskip If you followed us this far, congratulations, and rejoice: the
rest is much easier.

\section{Creation of PARI objects, assignments, conversions}

\subsec{Creation of PARI objects}\sidx{creation}
The basic function which creates a PARI object is

\fun{GEN}{cgetg}{long l, long t}
$l$ specifies the number of longwords to be allocated to the
object, and $t$ is the type of the object, in symbolic
form (see \secref{se:impl} for the list of these). The precise effect of
this function is as follows: it first creates on the PARI \emph{stack} a
chunk of memory of size \kbd{length} longwords, and saves the address of the
chunk which it will in the end return. If the stack has been used up, a
message to the effect that ``the PARI stack overflows'' is printed,
and an error raised. Otherwise, it sets the type and length of the PARI object.
In effect, it fills its first codeword (\kbd{z[0]}). Many PARI
objects also have a second codeword (types \typ{INT}, \typ{REAL},
\typ{PADIC}, \typ{POL}, and \typ{SER}). In case you want to produce one of
those from scratch, which should be exceedingly rare, \emph{it is your
responsibility to fill this second codeword}, either explicitly (using the
macros described in \secref{se:impl}), or implicitly using an assignment
statement (using \kbd{gaffect}).

Note that the length argument $l$ is predetermined for a number of types:
3 for types \typ{INTMOD}, \typ{FRAC}, \typ{COMPLEX}, \typ{POLMOD},
\typ{RFRAC}, 4 for type \typ{QUAD} and \typ{QFI}, and 5 for type \typ{PADIC}
and \typ{QFR}. However for the sake of efficiency,
\kbd{cgetg} does not check this: disasters will occur if you give an incorrect
length for those types.

\misctitle{Notes} 1)  The main use of this function is create efficiently
a constant object, or to prepare for later assignments (see
\secref{se:assign}). Most of the time you will use \kbd{GEN} objects as they
are created and returned by PARI functions. In this case you do not need to
use \kbd{cgetg} to create space to hold them.

\noindent 2) For the creation of leaves, i.e.~\typ{INT} or \typ{REAL},

\fun{GEN}{cgeti}{long length}

\fun{GEN}{cgetr}{long length}

\noindent should be used instead of \kbd{cgetg(length, t\_INT)} and
\kbd{cgetg(length, t\_REAL)} respectively. Finally

\fun{GEN}{cgetc}{long prec}

\noindent creates a \typ{COMPLEX} whose real and imaginary part are
\typ{REAL}s allocated by \kbd{cgetr(prec)}.

\misctitle{Examples} 1) Both \kbd{z = cgeti(DEFAULTPREC)} and
\kbd{cgetg(DEFAULTPREC, t\_INT)} create a \typ{INT} whose ``precision'' is
\kbd{bit\_accuracy(DEFAULTPREC)} = 64. This means \kbd{z} can hold rational
integers of absolute value less than $2^{64}$. Note that in both cases, the
second codeword is \emph{not} filled. Of course we could use numerical
values, e.g.~\kbd{cgeti(4)}, but this would have different meanings on
different machines as \kbd{bit\_accuracy(4)} equals 64 on 32-bit machines,
but 128 on 64-bit machines.

\noindent 2) The following creates a \emph{complex number} whose real and
imaginary parts can hold real numbers of precision
$\kbd{bit\_accuracy(MEDDEFAULTPREC)} = 96\hbox{ bits:}$
%
\bprog
  z = cgetg(3, t_COMPLEX);
  gel(z, 1) = cgetr(MEDDEFAULTPREC);
  gel(z, 2) = cgetr(MEDDEFAULTPREC);
@eprog\noindent
or simply \kbd{z = cgetc(MEDDEFAULTPREC)}.

\noindent 3) To create a matrix object for $4\times 3$ matrices:
%
\bprog
  z = cgetg(4, t_MAT);
  for(i=1; i<4; i++) gel(z, i) = cgetg(5, t_COL);
@eprog\noindent
or simply \kbd{z = zeromatcopy(4, 3)}, which further initializes all entries
to \kbd{gen\_0}.

These last two examples illustrate the fact that since PARI types are
recursive, all the branches of the tree must be created. The function
\teb{cgetg} creates only the ``root'', and other calls to \kbd{cgetg} must be
made to produce the whole tree. For matrices, a common mistake is to think
that \kbd{z = cgetg(4, t\_MAT)} (for example) creates the root of the
matrix: one needs also to create the column vectors of the matrix (obviously,
since we specified only one dimension in the first \kbd{cgetg}!). This is
because a matrix is really just a row vector of column vectors (hence a
priori not a basic type), but it has been given a special type number so that
operations with matrices become possible.

Finally, to facilitate input of constant objects when speed is not paramount,
there are four \tet{varargs} functions:

\fun{GEN}{mkintn}{long n, ...}
returns the non-negative \typ{INT} whose development in base $2^{32}$
is given by the following $n$ 32bit-words (\kbd{unsigned int}).
\bprog
  mkintn(3, a2, a1, a0);
@eprog
\noindent returns $a_2 2^{64} + a_1 2^{32} + a_0$.

\fun{GEN}{mkpoln}{long n, ...}
Returns the \typ{POL} whose $n$ coefficients (\kbd{GEN}) follow, in order of
decreasing degree.
\bprog
  mkpoln(3, gen_1, gen_2, gen_0);
@eprog
\noindent returns the polynomial $X^2 + 2X$ (in variable $0$, use
\tet{setvarn} if you want other variable numbers). Beware that $n$ is the
number of coefficients, hence \emph{one more} than the degree.

\fun{GEN}{mkvecn}{long n, ...}
returns the \typ{VEC} whose $n$ coefficients (\kbd{GEN}) follow.

\fun{GEN}{mkcoln}{long n, ...}
returns the \typ{COL} whose $n$ coefficients (\kbd{GEN}) follow.

\misctitle{Warning} Contrary to the policy of general PARI functions, the
latter three functions do \emph{not} copy their arguments, nor do they produce
an object a priori suitable for \tet{gerepileupto}. For instance
\bprog
  /*@Ccom gerepile-safe: components are universal objects */
  z = mkvecn(3, gen_1, gen_0, gen_2);

  /*@Ccom not OK for gerepileupto: stoi(3) creates component before root */
  z = mkvecn(3, stoi(3), gen_0, gen_2);

  /*@Ccom NO! First vector component \kbd{x} is destroyed */
  x = gclone(gen_1);
  z = mkvecn(3, x, gen_0, gen_2);
  gunclone(x);
@eprog

\noindent The following function is also available as a special case of
\tet{mkintn}:

\fun{GEN}{uu32toi}{ulong a, ulong b}

Returns the \kbd{GEN} equal to $2^{32} a + b$, \emph{assuming} that
$a,b < 2^{32}$. This does not depend on \kbd{sizeof(long)}: the behavior is
as above on both $32$ and $64$-bit machines.

\subsec{Sizes}

\fun{long}{gsizeword}{GEN x} returns the total number of \B-bit words occupied
by the tree representing~\kbd{x}.

\fun{long}{gsizebyte}{GEN x} returns the total number of bytes occupied
by the tree representing~\kbd{x}, i.e.~\kbd{gsizeword(x)} multiplied by
\kbd{sizeof(long)}. This is normally useless since PARI functions use
a number of \emph{words} as input for lengths and precisions.

\subsec{Assignments}
Firstly, if \kbd{x} and \kbd{y} are both declared as \kbd{GEN} (i.e.~pointers
to something), the ordinary C assignment \kbd{y = x} makes perfect sense: we
are just moving a pointer around. However, physically modifying either
\kbd{x} or \kbd{y} (for instance, \kbd{x[1] = 0}) also changes the other
one, which is usually not desirable. \label{se:assign}

\misctitle{Very important note} Using the functions described in this
paragraph is inefficient and often awkward: one of the \tet{gerepile}
functions (see~\secref{se:garbage}) should be preferred. See the paragraph
end for one exception to this rule.

\noindent
The general PARI \idx{assignment} function is the function \teb{gaffect} with
the following syntax:

\fun{void}{gaffect}{GEN x, GEN y}

\noindent
Its effect is to assign the PARI object \kbd{x} into the \emph{preexisting}
object \kbd{y}. Both \kbd{x} and \kbd{y} must be \emph{scalar} types. For
convenience, vector or matrices of scalar types are also allowed.

This copies the whole structure of \kbd{x} into \kbd{y} so many conditions
must be met for the assignment to be possible. For instance it is allowed to
assign a \typ{INT} into a \typ{REAL}, but the converse is forbidden. For
that, you must use the truncation or rounding function of your choice,
e.g.\kbd{mpfloor}.

It can also happen that \kbd{y} is not large enough or does not have the proper
tree structure to receive the object \kbd{x}. For instance, let \kbd{y} the zero
integer with length equal to 2; then \kbd{y} is too small to accommodate any
non-zero \typ{INT}. In general common sense tells you what is possible,
keeping in mind the PARI philosophy which says that if it makes sense it is
valid. For instance, the assignment of an imprecise object into a precise one
does \emph{not} make sense. However, a change in precision of imprecise
objects is allowed, even if it \emph{increases} its accuracy: we complement
the ``mantissa'' with infinitely many $0$ digits in this case. (Mantissa
between quotes, because this is not restricted to \typ{REAL}s, it also
applies for $p$-adics for instance.)

All functions ending in ``\kbd{z}'' such as \teb{gaddz}
(see~\secref{se:low_level}) implicitly use this function. In fact what they
exactly do is record {\teb{avma}} (see~\secref{se:garbage}), perform the
required operation, \teb{gaffect} the result to the last operand, then
restore the initial \kbd{avma}.

You can assign ordinary C long integers into a PARI object (not necessarily
of type \typ{INT}) using

\fun{void}{gaffsg}{long s, GEN y}

\misctitle{Note} Due to the requirements mentioned above, it is usually
a bad idea to use \tet{gaffect} statements. There is one exception: for simple
objects (e.g.~leaves) whose size is controlled, they can be easier to use than
\kbd{gerepile}, and about as efficient.

\misctitle{Coercion} It is often useful to coerce an inexact object to a
given precision. For instance at the beginning of a routine where precision
can be kept to a minimum; otherwise the precision of the input is used in all
subsequent computations, which is inefficient if the latter is known to
thousands of digits. One may use the \kbd{gaffect} function for this, but it
is easier and more efficient to call

\fun{GEN}{gtofp}{GEN x, long prec} converts the complex number~\kbd{x}
(\typ{INT}, \typ{REAL}, \typ{FRAC}, \typ{QUAD} or \typ{COMPLEX}) to either
a \typ{REAL} or \typ{COMPLEX} whose components are \typ{REAL} of length
\kbd{prec}.

\subsec{Copy} It is also very useful to \idx{copy} a PARI object, not
just by moving around a pointer as in the \kbd{y = x} example, but by
creating a copy of the whole tree structure, without pre-allocating a
possibly complicated \kbd{y} to use with \kbd{gaffect}. The function which
does this is called \teb{gcopy}. Its syntax is:

\fun{GEN}{gcopy}{GEN x}

\noindent and the effect is to create a new copy of x on the PARI stack.

Sometimes, on the contrary, a quick copy of the skeleton of \kbd{x} is
enough, leaving pointers to the original data in \kbd{x} for the sake of
speed instead of making a full recursive copy. Use
\fun{GEN}{shallowcopy}{GEN x} for this. Note that the result is not suitable
for \tet{gerepileupto} !

Make sure at this point that you understand the difference between \kbd{y =
x}, \kbd{y = gcopy(x)}, \kbd{y = shallowcopy(x)} and \kbd{gaffect(x,y)}.

\subsec{Clones}\sidx{clone}\label{se:clone}
Sometimes, it is more efficient to create a \emph{persistent} copy of a PARI
object. This is not created on the stack but on the heap, hence unaffected by
\tet{gerepile} and friends. The function which does this is called
\teb{gclone}. Its syntax is:

\fun{GEN}{gclone}{GEN x}

A clone can be removed from the heap (thus destroyed) using

\fun{void}{gunclone}{GEN x}

\noindent No PARI object should keep references to a clone which has been
destroyed!

\subsec{Conversions}\sidx{conversions}
The following functions convert C objects to PARI objects (creating them on
the stack as usual):

\fun{GEN}{stoi}{long s}: C \kbd{long} integer  (``small'') to \typ{INT}.

\fun{GEN}{dbltor}{double s}: C \kbd{double} to \typ{REAL}. The accuracy of
the result is 19 decimal digits, i.e.~a type \typ{REAL} of length
\kbd{DEFAULTPREC}, although on 32-bit machines only 16 of them are
significant.

\noindent We also have the converse functions:

\fun{long}{itos}{GEN x}: \kbd{x} must be of type \typ{INT},

\fun{double}{rtodbl}{GEN x}: \kbd{x} must be of type \typ{REAL},

\noindent as well as the more general ones:

\fun{long}{gtolong}{GEN x},

\fun{double}{gtodouble}{GEN x}.

\section{Implementation of the PARI types}
\label{se:impl}

\noindent
We now go through each type and explain its implementation. Let \kbd{z} be a
\kbd{GEN}, pointing at a PARI object. In the following paragraphs, we will
constantly mix two points of view: on the one hand, \kbd{z} is treated as the
C pointer it is, on the other, as PARI's handle on some mathematical entity,
so we will shamelessly write $\kbd{z} \ne 0$ to indicate that the
\emph{value} thus represented is nonzero (in which case the
\emph{pointer}~\kbd{z} is certainly non-\kbd{NULL}). We offer no apologies
for this style. In fact, you had better feel comfortable juggling both views
simultaneously in your mind if you want to write correct PARI programs.

Common to all the types is the first codeword \kbd{z[0]}, which we do not
have to worry about since this is taken care of by \kbd{cgetg}. Its precise
structure depends on the machine you are using, but it always contains the
following data: the \emph{internal type number}\sidx{type number} attached
to the symbolic type name, the \emph{length} of the root in longwords, and a
technical bit which indicates whether the object is a clone or not (see
\secref{se:clone}). This last one is used by \kbd{gp} for internal garbage
collecting, you will not have to worry about it.

Some types have a second codeword, different for each type, which we will
soon describe as we will shortly consider each of them in turn.

\noindent The first codeword is handled through the following \emph{macros}:

\fun{long}{typ}{GEN z} returns the type number of \kbd{z}.

\fun{void}{settyp}{GEN z, long n} sets the type number of \kbd{z} to
\kbd{n} (you should not have to use this function if you use \kbd{cgetg}).

\fun{long}{lg}{GEN z} returns the length (in longwords) of the root of \kbd{z}.

\fun{long}{setlg}{GEN z, long l} sets the length of \kbd{z} to \kbd{l} (you
should not have to use this function if you use \kbd{cgetg}; however, see
an advanced example in \secref{se:prog}).

\fun{long}{isclone}{GEN z} is \kbd{z} a clone?

\fun{void}{setisclone}{GEN z} sets the \emph{clone} bit.

\fun{void}{unsetisclone}{GEN z} clears the \emph{clone} bit.

\misctitle{Important remark} For the sake of efficiency, none of the
codeword-handling macros check the types of their arguments even when there
are stringent restrictions on their use. It is trivial to create invalid
objects, or corrupt one of the ``universal constants'' (e.g. setting the sign
of \kbd{gen\_0} to $1$), and they usually provide negligible savings.
Use higher level functions whenever possible.

\misctitle{Remark} The clone bit is there so that \kbd{gunclone} can check
it is deleting an object which was allocated by \kbd{gclone}. Miscellaneous
vector entries are often cloned by \kbd{gp} so that a GP statement like
\kbd{v[1] = x} does not involve copying the whole of \kbd{v}: the component
\kbd{v[1]} is deleted if its clone bit is set, and is replaced by a clone of
\kbd{x}. Don't set/unset yourself the clone bit unless you know what you are
doing: in particular \emph{never} set the clone bit of a vector component
when the said vector is scheduled to be uncloned. Hackish code may abuse the
clone bit to tag objects for reasons unrelated to the above instead of using
proper data structures. Don't do that.

\subsec{Type \typ{INT} (integer)}
\sidx{integer}\kbdsidx{t_INT}this type has a second codeword \kbd{z[1]} which
contains the following information:

the sign of \kbd{z}: coded as $1$, $0$ or $-1$ if $\kbd{z} > 0$, $\kbd{z} = 0$,
$\kbd{z} < 0$ respectively.

the \emph{effective length} of \kbd{z}, i.e.~the total number of significant
longwords. This means the following: apart from the integer 0, every integer
is ``normalized'', meaning that the most significant mantissa longword is
non-zero. However, the integer may have been created with a longer length.
Hence the ``length'' which is in \kbd{z[0]} can be larger than the
``effective length'' which is in \kbd{z[1]}.

\noindent This information is handled using the following macros:

\fun{long}{signe}{GEN z} returns the sign of \kbd{z}.

\fun{void}{setsigne}{GEN z, long s} sets the sign of \kbd{z} to \kbd{s}.

\fun{long}{lgefint}{GEN z} returns the \idx{effective length} of \kbd{z}.

\fun{void}{setlgefint}{GEN z, long l} sets the effective length
of \kbd{z} to \kbd{l}.

The integer 0 can be recognized either by its sign being~0, or by its
effective length being equal to~2. Now assume that $\kbd{z} \ne 0$, and let
$$ |z| = \sum_{i = 0}^n z_i B^i,
  \quad\text{where}~z_n\ne 0~\text{and}~B = 2^{\kbd{BITS\_IN\_LONG}}.
$$
With these notations, $n$ is \kbd{lgefint(z) - 3}, and the mantissa of
$\kbd{z}$ may be manipulated via the following interface:

\fun{GEN}{int_MSW}{GEN z} returns a pointer to the most significant word of
\kbd{z}, $z_n$.

\fun{GEN}{int_LSW}{GEN z} returns a pointer to the least significant word of
\kbd{z}, $z_0$.

\fun{GEN}{int_W}{GEN z, long i} returns the $i$-th significant word of
\kbd{z}, $z_i$. Accessing the $i$-th significant word for $i > n$
yields unpredictable results.

\fun{GEN}{int_W_lg}{GEN z, long i, long lz} returns the $i$-th significant
word of \kbd{z}, $z_i$, assuming \kbd{lgefint(z)} is \kbd{lz} ($= n + 3$).
Accessing the $i$-th significant word for $i > n$ yields unpredictable
results.

\fun{GEN}{int_precW}{GEN z} returns the previous (less significant) word of
\kbd{z}, $z_{i-1}$ assuming \kbd{z} points to $z_i$.

\fun{GEN}{int_nextW}{GEN z} returns the next (more significant) word of \kbd{z},
$z_{i+1}$ assuming \kbd{z} points to $z_i$.

Unnormalized integers, such that $z_n$ is possibly $0$, are explicitly
forbidden. To enforce this, one may write an arbitrary mantissa then call

\fun{void}{int_normalize}{GEN z, long known0}

\noindent normalizes in place a non-negative integer (such that $z_n$ is
possibly $0$), assuming at least the first \kbd{known0} words are zero.

\noindent For instance a binary \kbd{and} could be implemented in the
following way:
\bprog
GEN AND(GEN x, GEN y) {
  long i, lx, ly, lout;
  long *xp, *yp, *outp; /* mantissa pointers */
  GEN out;

  if (!signe(x) || !signe(y)) return gen_0;
  lx = lgefint(x); xp = int_LSW(x);
  ly = lgefint(y); yp = int_LSW(y); lout = min(lx,ly); /* > 2 */

  out = cgeti(lout); out[1] = evalsigne(1) | evallgefint(lout);
  outp = int_LSW(out);
  for (i=2; i < lout; i++)
  {
    *outp = (*xp) & (*yp);
    outp  = int_nextW(outp);
    xp    = int_nextW(xp);
    yp    = int_nextW(yp);
  }
  if ( !*int_MSW(out) ) out = int_normalize(out, 1);
  return out;
}
@eprog

\noindent This low-level interface is mandatory in order to write portable
code since PARI can be compiled using various multiprecision kernels, for
instance the native one or GNU MP, with incompatible internal structures
(for one thing, the mantissa is oriented in different directions).

\subsec{Type \typ{REAL} (real number)}
\kbdsidx{t_REAL}\sidx{real number}this type has a second codeword z[1] which
also encodes its sign, obtained or set using the same functions as for a
\typ{INT}, and a binary exponent. This exponent is handled using the
following macros:

\fun{long}{expo}{GEN z} returns the exponent of \kbd{z}.
This is defined even when \kbd{z} is equal to zero.

\fun{void}{setexpo}{GEN z, long e} sets the exponent of \kbd{z} to \kbd{e}.

\noindent Note the functions:

\fun{long}{gexpo}{GEN z} which tries to return an exponent for \kbd{z},
even if \kbd{z} is not a real number.

\fun{long}{gsigne}{GEN z} which returns a sign for \kbd{z}, even when
\kbd{z} is neither real nor integer (a rational number for instance).

The real zero is characterized by having its sign equal to 0. If \kbd{z} is
not equal to~0, then it is represented as $2^e M$, where $e$ is the exponent,
and $M\in [1, 2[$ is the mantissa of $z$, whose digits are stored in
$\kbd{z[2]},\dots, \kbd{z[lg(z)-1]}$.

More precisely, let $m$ be the integer (\kbd{z[2]},\dots, \kbd{z[lg(z)-1]})
in base \kbd{2\pow BITS\_IN\_LONG}; here, \kbd{z[2]} is the most significant
longword and is normalized, i.e.~its most significant bit is~1. Then we have
$M := m / 2^{\kbd{bit\_accuracy(lg(z))} - 1 - \kbd{expo}(z)}$.

\fun{GEN}{mantissa_real}{GEN z, long *e} returns the mantissa $m$ of $z$, and
sets \kbd{*e} to the exponent $\kbd{bit\_accuracy(lg(z))}-1-\kbd{expo}(z)$,
so that $z = m / 2^e$.

Thus, the real number $3.5$ to accuracy \kbd{bit\_accuracy(lg(z))} is
represented as \kbd{z[0]} (encoding $\kbd{type} = \typ{REAL}$, \kbd{lg(z)}),
\kbd{z[1]} (encoding $\kbd{sign} = 1$, $\kbd{expo} = 1$), $\kbd{z[2]} =
\kbd{0xe0000000}$, $\kbd{z[3]} =\dots = \kbd{z[lg(z)-1]} = \kbd{0x0}$.

\subsec{Type \typ{INTMOD}}\kbdsidx{t_INTMOD}
\kbd{z[1]} points to the modulus, and \kbd{z[2]} at the number representing
the class \kbd{z}. Both are separate \kbd{GEN} objects, and both must be
\typ{INT}s, satisfying the inequality $0 \le \kbd{z[2]} < \kbd{z[1]}$.

\subsec{Type \typ{FRAC} (rational number)}%
\kbdsidx{t_FRAC}\sidx{rational number}
\kbd{z[1]} points to the numerator $n$, and \kbd{z[2]} to the denominator
$d$. Both must be of type \typ{INT} such that $n\neq 0$, $d > 0$ and
$(n,d) = 1$.

\subsec{Type \typ{FFELT} (finite field element)}%
\kbdsidx{t_FFELT}\sidx{finite field element} (Experimental)

Components of this type should normally not be accessed directly. Instead,
finite field elements should be created using \kbd{ffgen}.

\noindent The second codeword \kbd{z[1]} determines the storage format of the
element, among

\item \tet{t_FF_FpXQ}: \kbd{A=z[2]} and \kbd{T=z[3]} are \kbd{FpX},
\kbd{p=z[4]} is a \typ{INT}, where $p$ is a prime number, $T$ is irreducible
modulo $p$, and $\deg A < \deg T$.
This represents the element $A\pmod{T}$ in $\F_p[X]/T$.

\item \tet{t_FF_Flxq}: \kbd{A=z[2]} and \kbd{T=z[3]} are \kbd{Flx},
\kbd{l=z[4]} is a \typ{INT}, where $l$ is a prime number, $T$ is irreducible
modulo $l$, and $\deg A < \deg T$ This represents the element $A\pmod{T}$ in
$\F_l[X]/T$.

\item \tet{t_FF_F2xq}: \kbd{A=z[2]} and \kbd{T=z[3]} are \kbd{F2x},
\kbd{l=z[4]} is the \typ{INT} $2$, $T$ is irreducible modulo $2$, and
$\deg A < \deg T$. This represents the element $A\pmod{T}$ in $\F_2[X]/T$.

\subsec{Type \typ{COMPLEX} (complex number)}%
\kbdsidx{t_COMPLEX}\sidx{complex number}
\kbd{z[1]} points to the real part, and \kbd{z[2]} to the imaginary part.
The components \kbd{z[1]} and \kbd{z[2]} must be of type
\typ{INT}, \typ{REAL} or \typ{FRAC}. For historical reasons \typ{INTMOD}
and \typ{PADIC} are also allowed (the latter for $p = 2$ or
congruent to 3 mod 4 only), but one should rather use the more general
\typ{POLMOD} construction.

\subsec{Type \typ{PADIC} ($p$-adic numbers)}%
\sidx{p-adic number}\kbdsidx{t_PADIC} this type has a second codeword
\kbd{z[1]} which contains the following information: the $p$-adic precision
(the exponent of $p$ modulo which the $p$-adic unit corresponding to
\kbd{z} is defined if \kbd{z} is not~0), i.e.~one less than the number of
significant $p$-adic digits, and the exponent of \kbd{z}. This information
can be handled using the following functions:

\fun{long}{precp}{GEN z} returns the $p$-adic precision of \kbd{z}. This is
$0$ if $\kbd{z} = 0$.

\fun{void}{setprecp}{GEN z, long l} sets the $p$-adic precision of \kbd{z}
to \kbd{l}.

\fun{long}{valp}{GEN z} returns the $p$-adic valuation of \kbd{z}
(i.e. the exponent). This is defined even if \kbd{z} is equal to~0.

\fun{void}{setvalp}{GEN z, long e} sets the $p$-adic valuation of \kbd{z}
to \kbd{e}.

In addition to this codeword, \kbd{z[2]} points to the prime $p$,
\kbd{z[3]} points to $p^{\text{precp(z)}}$, and \kbd{z[4]} points to
a\typ{INT} representing the $p$-adic unit attached to \kbd{z} modulo
\kbd{z[3]} (and to zero if \kbd{z} is zero). To summarize, if $z\neq
0$, we have the equality:
$$ \kbd{z} = p^{\text{valp(z)}} * (\kbd{z[4]} + O(\kbd{z[3]})),\quad
\text{where}\quad \kbd{z[3]} = O(p^{\text{precp(z)}}). $$

\subsec{Type \typ{QUAD} (quadratic number)}
\sidx{quadratic number}\kbdsidx{t_QUAD}\kbd{z[1]} points to the canonical
polynomial $P$ defining the quadratic field (as output by \tet{quadpoly}),
\kbd{z[2]} to the ``real part'' and \kbd{z[3]} to the ``imaginary part''. The
latter are of type \typ{INT}, \typ{FRAC}, \typ{INTMOD}, or \typ{PADIC} and
are to be taken as the coefficients of \kbd{z} with respect to the canonical
basis $(1,X)$ of $\Q[X]/(P(X))$. Exact complex numbers may be implemented as
quadratics, but \typ{COMPLEX} is in general more versatile (\typ{REAL}
components are allowed) and more efficient.

Operations involving a \typ{QUAD} and \typ{COMPLEX} are implemented by
converting the \typ{QUAD} to a \typ{REAL} (or \typ{COMPLEX} with \typ{REAL}
components) to the accuracy of the \typ{COMPLEX}. As a consequence,
operations between \typ{QUAD} and \emph{exact} \typ{COMPLEX}s are not allowed.

\subsec{Type \typ{POLMOD} (polmod)}\kbdsidx{t_POLMOD}\sidx{polmod}
as for \typ{INTMOD}s, \kbd{z[1]} points to the modulus, and \kbd{z[2]}
to a polynomial representing the class of~\kbd{z}. Both must be of type
\typ{POL} in the same variable, satisfying the inequality $\deg \kbd{z[2]}
< \deg \kbd{z[1]}$. However, \kbd{z[2]} is allowed to be a simplification
of such a polynomial, e.g.~a scalar. This is tricky considering the
hierarchical structure of the variables; in particular, a polynomial in
variable of \emph{lesser} priority (see \secref{se:vars}) than the
modulus variable is valid, since it is considered as the constant term of
a polynomial of degree 0 in the correct variable. On the other hand a
variable of \emph{greater} priority is not acceptable.

\subsec{Type \typ{POL} (polynomial)}\kbdsidx{t_POL}\sidx{polynomial} this
type has a second codeword. It contains a ``\emph{sign}'': 0 if the
polynomial is equal to~0, and 1 if not (see however the important remark
below) and a \emph{variable number} (e.g.~0 for $x$, 1 for $y$, etc\dots).

\noindent These data can be handled with the following macros: \teb{signe}
and \teb{setsigne} as for \typ{INT} and \typ{REAL},

\fun{long}{varn}{GEN z} returns the variable number of the object \kbd{z},

\fun{void}{setvarn}{GEN z, long v} sets the variable number of \kbd{z} to
\kbd{v}.

The variable numbers encode the relative priorities of variables, we will
give more details in \secref{se:vars}. Note also the function
\fun{long}{gvar}{GEN z} which tries to return a \idx{variable number} for
\kbd{z}, even if \kbd{z} is not a polynomial or power series. The variable
number of a scalar type is set by definition equal to \tet{NO_VARIABLE},
which has lower priority than any other variable number.

The components \kbd{z[2]}, \kbd{z[3]},\dots \kbd{z[lg(z)-1]} point to the
coefficients of the polynomial \emph{in ascending order}, with \kbd{z[2]}
being the constant term and so on.

For a \typ{POL} of non-zero sign, \tet{degpol}, \tet{leading_coeff},
\tet{constant_coeff}, return its degree, and a pointer to the leading,
resp. constant, coefficient with respect to the main variable. Note that no
copy is made on the PARI stack so the returned value is not safe for a basic
\kbd{gerepile} call. Applied to any other type than \typ{POL}, the result is
unspecified. Those three functions are still defined when the sign is $0$,
see \secref{se:accessors} and \secref{se:polynomials}.

\fun{long}{degree}{GEN x} returns the degree of \kbd{x} with respect to its
main variable even when \kbd{x} is not a polynomial (a rational function for
instance). By convention, the degree of a zero polynomial is~$-1$.

\misctitle{Important remark}
The leading coefficient of a \typ{POL} may be equal to zero:

\item it is not allowed to be an exact rational $0$, such as \tet{gen_0};

\item an exact non-rational $0$, like \kbd{Mod(0,2)}, is possible for
constant polynomials, i.e. of length $3$ and no other coefficient: this
carries information about the base ring for the polynomial;

\item an inexact $0$, like \kbd{0.E-38} or \kbd{O(3\pow 5)}, is always
possible. Inexact zeroes do not correspond to an actual $0$, but to a
very small coefficient according to some metric; we keep them to give
information on how much cancellation occurred in previous computations.

A polynomial disobeying any of these rules is an invalid \emph{unnormalized}
object. We advise \emph{not} to use low-level constructions to build a
\typ{POL} coefficient by coefficient, such as
\bprog
  GEN T = cgetg(4, t_POL);
  T[1] = evalvarn(0);
  gel(T, 2) = x;
  gel(T, 3) = y;
@eprog\noindent But if you do and it is not clear whether the result will be
normalized, call

\fun{GEN}{normalizepol}{GEN x} applied to an unnormalized \typ{POL}~\kbd{x}
(with all coefficients correctly set except that \kbd{leading\_term(x)} might
be zero), normalizes \kbd{x} correctly in place and returns~\kbd{x}. This
functions sets \kbd{signe} (to $0$ or $1$) properly.

\misctitle{Caveat} A consequence of the remark above is that zero polynomials
are characterized by the fact that their sign is~0. It is in general
incorrect to check whether \kbd{lg(x)} is $2$ or \kbd{degpol(x)} $< 0$,
although both tests are valid when the coefficient types are under control:
for instance, when they are guaranteed to be \typ{INT}s or \typ{FRAC}s.
The same remark applies to \typ{SER}s.

\subsec{Type \typ{SER} (power series)}
\kbdsidx{t_SER}\sidx{power series}This type also has a second codeword, which
encodes a ``\emph{sign}'', i.e.~0 if the power series is 0, and 1 if not, a
\emph{variable number} as for polynomials, and an \emph{exponent}. This
information can be handled with the following functions: \teb{signe},
\teb{setsigne}, \teb{varn}, \teb{setvarn} as for polynomials, and \teb{valp},
\teb{setvalp} for the exponent as for $p$-adic numbers. Beware: do \emph{not}
use \teb{expo} and \teb{setexpo} on power series.

The coefficients \kbd{z[2]}, \kbd{z[3]},\dots \kbd{z[lg(z)-1]} point to
the coefficients of \kbd{z} in ascending order. As for polynomials
(see remark there), the sign of a \typ{SER} is $0$ if and only all
its coefficients are equal to $0$. (The leading coefficient cannot be an
integer $0$.) A series whose coefficients are integers equal to zero
is represented as $O(x^n)$ (\kbd{zeroser}$(\var{vx},n)$).
A series whose coefficients are exact zeroes, but not all of
them integers  (e.g. an \typ{INTMOD} such as \kbd{Mod(0,2)}) is
represented as $z*x^{n-1} +O(x^n)$, where $z$ is the $0$ of the
base ring, as per \tet{RgX_get_0}.

Note that the exponent of a power series can be negative, i.e.~we are
then dealing with a Laurent series (with a finite number of negative
terms).

\subsec{Type \typ{RFRAC} (rational function)}%
\kbdsidx{t_RFRAC}\sidx{rational function} \kbd{z[1]} points to the
numerator $n$, and \kbd{z[2]} on the denominator $d$. The denominator must be
of type \typ{POL}, with variable of higher priority than the numerator. The
numerator $n$ is not an exact $0$ and $(n,d) = 1$ (see \tet{gred_rfac2}).

\subsec{Type \typ{QFR} (indefinite binary quadratic form)}%
\kbdsidx{t_QFR}\sidx{indefinite binary quadratic form} \kbd{z[1]},
\kbd{z[2]}, \kbd{z[3]} point to the three coefficients of the form and are of
type \typ{INT}. \kbd{z[4]} is Shanks's distance function, and must be of type
\typ{REAL}.

\subsec{Type \typ{QFI} (definite binary quadratic form)}%
\kbdsidx{t_QFI}\sidx{definite binary quadratic form} \kbd{z[1]}, \kbd{z[2]},
\kbd{z[3]} point to the three coefficients of the form. All three are of type
\typ{INT}.

\subsec{Type \typ{VEC} and \typ{COL} (vector)}%
\kbdsidx{t_VEC}\kbdsidx{t_COL}\sidx{row vector}\sidx{column vector}
\kbd{z[1]}, \kbd{z[2]},\dots \kbd{z[lg(z)-1]} point to the components of the
vector.

\subsec{Type \typ{MAT} (matrix)}\kbdsidx{t_MAT}\sidx{matrix} \kbd{z[1]},
\kbd{z[2]},\dots \kbd{z[lg(z)-1]} point to the column vectors of \kbd{z},
i.e.~they must be of type \typ{COL} and of the same length.

\subsec{Type \typ{VECSMALL} (vector of small integers)}\kbdsidx{t_VECSMALL}
\kbd{z[1]}, \kbd{z[2]},\dots \kbd{z[lg(z)-1]} are ordinary signed long
integers. This type is used instead of a \typ{VEC} of \typ{INT}s for
efficiency reasons, for instance to implement efficiently permutations,
polynomial arithmetic and linear algebra over small finite fields, etc.

\subsec{Type \typ{STR} (character string)}%
\kbdsidx{t_STR}\sidx{character string}

\fun{char *}{GSTR}{z} (= \kbd{(z+1)}) points to the first character of the
(\kbd{NULL}-terminated) string.

\subsec{Type \typ{ERROR} (error context)}\kbdsidx{t_ERROR}\sidx{error context}
This type holds error messages, as well as details about the error, as
returned by the exception handling system. The second codeword \kbd{z[1]}
contains the error type (an \kbd{int}, as passed to \tet{pari_err}). The
subsequent words \kbd{z[2]},\dots \kbd{z[lg(z)-1]} are \kbd{GEN}s containing
additional data, depending on the error type.

\subsec{Type \typ{CLOSURE} (closure)}\kbdsidx{t_CLOSURE}\sidx{closure}
This type holds GP functions and closures, in compiled form. The internal detail
of this type is subject to change each time the GP language evolves. Hence we
do not describe it here and refer to the Developer's Guide.  However
functions to create or to evaluate \typ{CLOSURE}s are documented in
\secref{se:closure}.

\fun{long}{closure_arity}{GEN C} returns the arity of the \typ{CLOSURE}.

\fun{long}{closure_is_variadic}{GEN C} returns $1$ if the closure \kbd{C} is
variadic, $0$ else.

\subsec{Type \typ{INFINITY} (infinity)}\kbdsidx{t_INFINITY}\sidx{infinity}

This type has a single \typ{INT} component, which is either $1$ or $-1$,
corresponding to $+\infty$ and $-\infty$ respectively.

\fun{GEN}{mkmoo}{} returns $-\infty$

\fun{GEN}{mkoo}{} returns $\infty$

\fun{long}{inf_get_sign}{GEN x} returns $1$ if $x$ is $+\infty$, and $-1$
if $x$ is $-\infty$.

\subsec{Type \typ{LIST} (list)}\kbdsidx{t_LIST}\sidx{list}
this type was introduced for specific \kbd{gp} use and is rather inefficient
compared to a straightforward linked list implementation (it requires more
memory, as well as many unnecessary copies). Hence we do not describe it
here and refer to the Developer's Guide.

\misctitle{Implementation note} For the types including an exponent (or a
valuation), we actually store a biased non-negative exponent (bit-ORing the
biased exponent to the codeword), obtained by adding a constant to the true
exponent: either \kbd{HIGHEXPOBIT} (for \typ{REAL}) or \kbd{HIGHVALPBIT} (for
\typ{PADIC} and \typ{SER}). Of course, this is encapsulated by the
exponent/valuation-handling macros and needs not concern the library user.

\section{PARI variables}\label{se:vars}
\subsec{Multivariate objects}
\sidx{variable (priority)}

\noindent We now consider variables and formal computations. As we have seen
in \secref{se:impl}, the codewords for types \typ{POL} and \typ{SER} encode a
``variable number''. This is an integer, ranging from $0$ to \kbd{MAXVARN}.
Relative priorities may be ascertained using

\fun{int}{varncmp}{long v, long w}

\noindent which is $>0$, $=0$, $<0$ whenever $v$ has lower, resp.~same,
resp.~higher priority than $w$.

The way an object is considered in formal computations depends entirely on
its ``principal variable number'' which is given by the function

\fun{long}{gvar}{GEN z}

\noindent which returns a \idx{variable number} for \kbd{z}, even if \kbd{z}
is not a polynomial or power series. The variable number of a scalar type is
set by definition equal to \tet{NO_VARIABLE} which has lower priority than any
valid variable number. The variable number of a recursive type which is not a
polynomial or power series is the variable number with highest priority among
its components. But for polynomials and power series only the ``outermost''
number counts (we directly access $\tet{varn}(x)$ in the codewords): the
representation is not symmetrical at all.

Under \kbd{gp}, one needs not worry too much since the interpreter defines
the variables as it sees them\footnote{*}{The first time a given identifier
is read by the GP parser a new variable is created, and it is assigned a
strictly lower priority than any variable in use at this point. On startup,
before any user input has taken place, 'x' is defined in this way and has
initially maximal priority (and variable number $0$).}
%
and do the right thing with the polynomials produced.

But in library mode, they are tricky objects if you intend to build
polynomials yourself (and not just let PARI functions produce them, which is
less efficient). For instance, it does not make sense to have a variable
number occur in the components of a polynomial whose main variable has a
lower priority, even though PARI cannot prevent you from doing it.

\subsec{Creating variables} A basic difficulty is to ``create'' a variable.
Some initializations are needed before you can use a given integer $v$ as a
variable number.

Initially, this is done for $0$ and $1$ (the variables \kbd{x} and
\kbd{y} under \kbd{gp}), and $2,\dots,9$ (printed as \kbd{t2}, \dots
\kbd{t9}), with decreasing priority.

\subsubsec{User variables}\sidx{variable (user)} When the program starts,
\kbd{x} (number~$0$) and \kbd{y} (number~$1$) are the only available
variables, numbers $2$ to $9$ (decreasing priority) are reserved for building
polynomials with predictable priorities.

To define further ones, you may use

\fun{GEN}{varhigher}{const char *s}

\fun{GEN}{varlower}{const char *s}

to recover a monomial of degree $1$ in a new variable, which is guaranteed
to have higer (resp.~lower) priority than all existing ones at the
time of the function call. The variable is printed as $s$, but is not part
of GP's interpreter: it is not a symbol bound to a value.

On the other hand

\fun{long}{fetch_user_var}{char *s}: inspects the user variable whose name is
the string pointed to by \kbd{s}, creating it if needed, and returns its
variable number.
\bprog
long v = fetch_user_var("y");
GEN gy = pol_x(v);
@eprog\noindent
The function raises an exception if the name is already in use for an
\tet{install}ed or built-in function, or an alias. This function
is mostly useless since it returns a variable with unpredictable priority.
Don't use it to create new variables.

\misctitle{Caveat} You can use \tet{gp_read_str}
(see~\secref{se:gp_read_str}) to execute a GP command and create GP
variables on the fly as needed:
\bprog
GEN gy = gp_read_str("'y"); /*@Ccom returns \kbd{pol\_x}($v$), for some $v$ */
long v = varn(gy);
@eprog\noindent
But please note the quote \kbd{'y} in the above. Using \kbd{gp\_read\_str("y")}
might work, but is dangerous, especially when programming functions to
be used under \kbd{gp}. The latter reads the value of \kbd{y}, as
\emph{currently} known by the \kbd{gp} interpreter, possibly creating it
in the process. But if \kbd{y} has been modified by previous \kbd{gp}
commands (e.g.~\kbd {y = 1}), then the value of \kbd{gy} is not what you
expected it to be and corresponds instead to the current value of the
\kbd{gp} variable (e.g.~\kbd{gen\_1}).

\fun{GEN}{fetch_var_value}{long v} returns a shallow copy of the current
value of the variable numbered $v$. Returns \kbd{NULL} if that variable
number is unknown to the interpreter,  e.g. it is a user variable. Note
that this may not be the same as \kbd{pol\_x(v)} if assignments have been
performed in the interpreter.

\subsubsec{Temporary variables}\sidx{variable (temporary)}
You can create temporary variables using

\fun{long}{fetch_var}{}
returns a new variable with \emph{lower} priority than any variable currently
in use.

\fun{long}{fetch_var_higher}{}
returns a new variable with \emph{higher} priority than any variable
currently in use.

\noindent
After the statement \kbd{v = fetch\_var()}, you can use
\kbd{pol\_1(v)} and \kbd{pol\_x(v)}. The variables created in this way have no
identifier assigned to them though, and are printed as
\kbd{t\text{number}}. You can assign a name to a temporary variable, after
creating it, by calling the function

\fun{void}{name_var}{long n, char *s}

\noindent after which the output machinery will use the name \kbd{s} to
represent the variable number~\kbd{n}. The GP parser will \emph{not}
recognize it by that name, however, and calling this on a variable known
to~\kbd{gp} raises an error. Temporary variables are meant to be used as free
variables to build polynomials and power series, and you should never assign
values or functions to them as you would do with variables under~\kbd{gp}.
For that, you need a user variable.

All objects created by \kbd{fetch\_var} are on the heap and not on the stack,
thus they are not subject to standard garbage collecting (they are not
destroyed by a \kbd{gerepile} or \kbd{avma = ltop} statement). When you do
not need a variable number anymore, you can delete it using

\fun{long}{delete_var}{}

\noindent which deletes the \emph{latest} temporary variable created and
returns the variable number of the previous one (or simply returns 0 if none
remain). Of course you should make sure that
the deleted variable does not appear anywhere in the objects you use later
on. Here is an example:

\bprog
  long first = fetch_var();
  long n1 = fetch_var();
  long n2 = fetch_var(); /*@Ccom prepare three variables for internal use */
  ...
  /*@Ccom delete all variables before leaving */
  do { num = delete_var(); } while (num && num <= first);
@eprog\noindent
The (dangerous) statement

\bprog
  while (delete_var()) /*@Ccom empty */;
@eprog\noindent
removes all temporary variables in use.

\subsec{Comparing variables}

Let us go back to \kbd{varncmp}. There is an interesting corner case, when
one of the compared variables (from \kbd{gvar}, say) is \kbd{NO\_VARIABLE}.
In this case, \kbd{varncmp} declares it has lower priority than any other
variable; of course, comparing \kbd{NO\_VARIABLE} with itself yields
$0$ (same priority);

In addition to \kbd{varncmp} we have

\fun{long}{varnmax}{long v, long w} given two variable numbers (possibly
\kbd{NO\_VARIABLE}), returns the variable with the highest priority.
This function always returns a valid variable number unless it is comparing
\kbd{NO\_VARIABLE} to itself.

\fun{long}{varnmin}{long x, long y} given two variable numbers (possibly
\kbd{NO\_VARIABLE}), returns the variable with the lowest priority. Note
that when comparing a true variable with \kbd{NO\_VARIABLE}, this function
returns \kbd{NO\_VARIABLE}, which is not a valid variable number.

\section{Input and output}

\noindent
Two important aspects have not yet been explained which are specific to
library mode: input and output of PARI objects.

\subsec{Input}

\noindent
For \idx{input}, PARI provides several powerful high level functions
which enable you to input your objects as if you were under \kbd{gp}. In fact,
it \emph{is} essentially the GP syntactical parser.

There are two similar functions available to parse a string:

\fun{GEN}{gp_read_str}{const char *s}\label{se:gp_read_str}

\fun{GEN}{gp_read_str_multiline}{const char *s, char *last}

\noindent
Both functions read the whole string \kbd{s}. The function
\kbd{gp\_read\_str} ignores newlines: it assumes that the input is one
expression and returns the result of this expression.

The function \kbd{gp\_read\_str\_multiline} processes the text in the
same way as the GP command \tet{read}: newlines are significant and can
be used to separate expressions.
The return value is that of the last non-empty expression evaluated.

In \kbd{gp\_read\_str\_multiline}, if \kbd{last} is non-\kbd{NULL},
then \kbd{*last} receives the last character from the \emph{filtered}
input: this can be used to check if the last character was a semi-colon
(to hide the output in interactive usage). If (and only if) the
input contains no statements, then \kbd{*last} is set to \kbd{0}.

For both functions, \kbd{gp}'s metacommands \emph{are} recognized.

\misctitle{Note} The obsolete form

\fun{GEN}{readseq}{char *t}

still exists for backward compatibility (assumes filtered input, without
spaces or comments). Don't use it.

To read a \kbd{GEN} from a file, you can use the simpler interface

\fun{GEN}{gp_read_stream}{FILE *file}

\noindent which reads a character string of arbitrary length from the stream
\kbd{file} (up to the first complete expression sequence), applies
\kbd{gp\_read\_str} to it, and returns the resulting \kbd{GEN}. This way, you
do not have to worry about allocating buffers to hold the string. To
interactively input an expression, use \kbd{gp\_read\_stream(stdin)}.

Finally, you can read in a whole file, as in GP's \tet{read} statement

\fun{GEN}{gp_read_file}{char *name}

\noindent As usual, the return value is that of the last non-empty expression
evaluated. There is one technical exception: if \kbd{name} is a \emph{binary}
file (from \tet{writebin}) containing more than one object, a \typ{VEC}
containing them all is returned. This is because binary objects bypass the
parser, hence reading them has no useful side effect.

\subsec{Output to screen or file, output to string}\sidx{output}

General output functions return nothing but print a character string as a
side effect. Low level routines are available to write on PARI output stream
\tet{pari_outfile} (\tet{stdout} by default):

\fun{void}{pari_putc}{char c}: write character \kbd{c} to the output stream.

\fun{void}{pari_puts}{char *s}: write \kbd{s} to the output stream.

\fun{void}{pari_flush}{}: flush output stream; most streams are buffered by
default, this command makes sure that all characters output so are actually
written.

\fun{void}{pari_printf}{const char *fmt, ...}: the most versatile such
function. \kbd{fmt} is a character string similar to the one
\tet{printf} uses. In there, \kbd{\%} characters have a special meaning, and
describe how to print the remaining operands. In addition to the standard
format types (see the GP function \tet{printf}), you can use the \emph{length
modifier}~\kbd{P} (for PARI of course!) to specify that an argument is a
\kbd{GEN}. For instance, the following are valid conversions for a \kbd{GEN}
argument
\bprog
    %Ps     @com convert to \kbd{char*} (will print an arbitrary \kbd{GEN})
    %P.10s  @com convert to \kbd{char*}, truncated to 10 chars
    %P.2f   @com convert to floating point format with 2 decimals
    %P4d    @com convert to integer, field width at least 4

    pari_printf("x[%d] = %Ps is not invertible!\n", i, gel(x,i));
@eprog\noindent
Here \kbd{i} is an \kbd{int}, \kbd{x} a \kbd{GEN} which is not a leaf
(presumably a vector, or a polynomial) and this would insert the value of its
$i$-th \kbd{GEN} component: \kbd{gel(x,i)}.

\noindent Simple but useful variants to \kbd{pari\_printf} are

\fun{void}{output}{GEN x} prints \kbd{x} in raw format, followed by a
newline and a buffer flush. This is more or less equivalent to
\bprog
    pari_printf("%Ps\n", x);
    pari_flush();
@eprog

\fun{void}{outmat}{GEN x} as above except if $x$ is a \typ{MAT}, in which
case a multi-line display is used to display the matrix. This is prettier for
small dimensions, but quickly becomes unreadable and cannot be pasted and
reused for input. If all entries of $x$ are small integers, you may use the
recursive features of \kbd{\%Pd} and obtain the same (or better) effect with
\bprog
    pari_printf("%Pd\n", x);
    pari_flush();
@eprog\noindent
A variant like \kbd{"\%5Pd"} would improve alignment by imposing
5 chars for each coefficient. Similarly if all entries are to be converted to
floats, a format like \kbd{"\%5.1Pf"} could be useful.


These functions write on (PARI's idea of) standard output, and must be used
if you want your functions to interact nicely with \kbd{gp}. In most
programs, this is not a concern and it is more flexible to write to an
explicit \kbd{FILE*}, or to recover a character string:

\fun{void}{pari_fprintf}{FILE *file, const char *fmt, ...} writes the
remaining arguments to stream \kbd{file} according to the format
specification \kbd{fmt}.

\fun{char*}{pari_sprintf}{const char *fmt, ...} produces a string from the
remaining arguments, according to the PARI format \kbd{fmt} (see \tet{printf}).
This is the \kbd{libpari} equivalent of \tet{Strprintf}, and returns a
\kbd{malloc}'ed string, which must be freed by the caller. Note that contrary
to the analogous \tet{sprintf} in the \kbd{libc} you do not provide a buffer
(leading to all kinds of buffer overflow concerns); the function provided is
actually closer to the GNU extension \kbd{asprintf}, although the latter has
a different interface.

Simple variants of \tet{pari_sprintf} convert a \kbd{GEN} to a
\kbd{malloc}'ed ASCII string, which you must still \kbd{free} after use:

\fun{char*}{GENtostr}{GEN x}, using the current default output format
(\kbd{prettymat} by default).

\fun{char*}{GENtoTeXstr}{GEN x}, suitable for inclusion in a \TeX\ file.


Note that we have \tet{va_list} analogs of the functions of \kbd{printf} type
seen so far:

\fun{void}{pari_vprintf}{const char *fmt, va_list ap}

\fun{void}{pari_vfprintf}{FILE *file, const char *fmt, va_list ap}

\fun{char*}{pari_vsprintf}{const char *fmt, va_list ap}

\subsec{Errors}\sidx{error}\kbdsidx{e_MISC}

\noindent
If you want your functions to issue error messages, you can use the general
error handling routine \tet{pari_err}. The basic syntax is
%
\bprog
  pari_err(e_MISC, "error message");
@eprog\noindent
This prints the corresponding error message and exit the program (in
library mode; go back to the \kbd{gp} prompt otherwise).\label{se:err} You can
also use it in the more versatile guise
\bprog
  pari_err(e_MISC, format, ...);
@eprog\noindent
where \kbd{format} describes the format to use to write the remaining
operands, as in the \tet{pari_printf} function. For instance:
\bprog
  pari_err(e_MISC, "x[%d] = %Ps is not invertible!", i, gel(x,i));
@eprog\noindent
The simple syntax seen above is just a special case with a constant format
and no remaining arguments. The general syntax is

\fun{void}{pari_err}{numerr,...}

\noindent where \kbd{numerr} is a codeword which specifies the error class
and what to do with the remaining arguments and what message to print.
For instance, if $x$ is a \kbd{GEN} with internal type \typ{STR}, say,
\kbd{pari\_err(e\_TYPE,"extgcd", $x$)} prints the message:
\bprog
    ***   incorrect type in extgcd (t_STR),
@eprog\noindent See \secref{se:errors} for details. In the libpari code
itself, the general-purpose \kbd{e\_MISC} is used sparingly: it is so
flexible that the corresponding error contexts (\typ{ERROR}) become hard to
use reliably. Other more rigid error types are generally more useful: for
instance the error context attached to the \kbd{e\_TYPE} exception above is
precisely documented and contains \kbd{"extgcd"} and $x$ (not only its type)
as readily available components.

\subsec{Warnings}

\noindent To issue a warning, use

\fun{void}{pari_warn}{warnerr,...}
In that case, of course, we do \emph{not} abort the computation, just print
the requested message and go on. The basic example is
%
\bprog
    pari_warn(warner, "Strategy 1 failed. Trying strategy 2")
@eprog\noindent
which is the exact equivalent of \kbd{pari\_err(e\_MISC,...)} except that
you certainly do not want to stop the program at this point, just inform the
user that something important has occurred; in particular, this output would be
suitably highlighted under \kbd{gp}, whereas a simple \kbd{printf} would not.

The valid \emph{warning} keywords are \tet{warner} (general), \tet{warnprec}
(increasing precision), \tet{warnmem} (garbage collecting) and \tet{warnfile}
(error in file operation), used as follows:
\bprog
    pari_warn(warnprec, "bnfinit", newprec);
    pari_warn(warnmem,  "bnfinit");
    pari_warn(warnfile, "close", "afile");  /* error when closing "afile" */
@eprog

\subsec{Debugging output}\sidx{debugging}\sidx{format}\label{se:dbg_output}

For debugging output, you can use the standard output
functions, \tet{output} and \tet{pari_printf} mainly. Corresponding to the
\kbd{gp} metacommand \kbd{\b x}, you can also output the \idx{hexadecimal
tree} attached to an object:

\fun{void}{dbgGEN}{GEN x, long nb = -1}, displays the recursive structure of
\kbd{x}. If $\kbd{nb} = -1$, the full structure is printed, otherwise
the leaves (non-recursive components) are truncated to \kbd{nb} words.

\noindent The function \tet{output} is vital under debuggers, since none of
them knows how to print PARI objects by default. Seasoned PARI developers
add the following \kbd{gdb} macro to their \kbd{.gdbinit}:
\bprog
  define i
    call output((GEN)$arg0)
  end
@eprog\noindent
Typing \kbd{i x} at a breakpoint in \kbd{gdb} then prints the value of the
\kbd{GEN} \kbd{x} (provided the optimizer has not put it into a register, but
it is rarely a good idea to debug optimized code).

\noindent
The global variables \teb{DEBUGLEVEL} and \teb{DEBUGMEM} (corresponding
to the default \teb{debug} and \teb{debugmem})
are used throughout the PARI code to govern the amount of diagnostic and
debugging output, depending on their values. You can use them to debug your
own functions, especially if you \tet{install} the latter under \kbd{gp}.

\fun{void}{dbg_pari_heap}{void} print debugging statements about the PARI
stack, heap, and number of variables used. Corresponds to \kbd{\bs s}
under gp.

\subsec{Timers and timing output}

\noindent To handle timings in a reentrant way, PARI defines a dedicated data
type, \tet{pari_timer}, together with the following methods:

\fun{void}{timer_start}{pari_timer *T} start (or reset) a timer.

\fun{long}{timer_delay}{pari_timer *T} returns the number of milliseconds
elapsed since the timer was last reset. Resets the timer as a side effect.

\fun{long}{timer_get}{pari_timer *T} returns the number of milliseconds
elapsed since the timer was last reset. Does \emph{not} reset the timer.

\fun{long}{timer_printf}{pari_timer *T, char *format,...} This diagnostics
function is equivalent to the following code
\bprog
  err_printf("Time ")
  ... prints remaining arguments according to format ...
  err_printf(": %ld", timer_delay(T));
@eprog\noindent Resets the timer as a side effect.

\noindent They are used as follows:
\bprog
  pari_timer T;
  timer_start(&T); /* initialize timer */
  ...
  printf("Total time: %ldms\n", timer_delay(&T));
@eprog\noindent
or
\bprog
  pari_timer T;
  timer_start(&T);
  for (i = 1; i < 10; i++) {
    ...
    timer_printf(&T, "for i = %ld (L[i] = %Ps)", i, gel(L,i));
  }
@eprog

The following functions provided the same functionality, in a
non-reentrant way, and are now deprecated.

\fun{long}{timer}{void}

\fun{long}{timer2}{void}

\fun{void}{msgtimer}{const char *format, ...}

The following function implements \kbd{gp}'s timer and should not be used in
libpari programs:
\fun{long}{gettime}{void} equivalent to \tet{timer_delay}$(T)$ attached to a
private timer $T$.

\section{Iterators, Numerical integration, Sums, Products}
\subsec{Iterators}
Since it is easier to program directly simple loops in library mode, some GP
iterators are mainly useful for GP programming. Here are the others:

\item \tet{fordiv} is a trivial iteration over a list produced by
\tet{divisors}.

\item \tet{forell} and \tet{forsubgroup} are currently not implemented as an
iterator but as a procedure with callbacks.

\fun{void}{forell}{void *E, long fun(void*, GEN), GEN a, GEN b}
goes through the same curves as \tet{forell(ell,a,b,)}, calling
\tet{fun(E, ell)} for each curve \kbd{ell}, stopping if \kbd{fun} returns a
non-zero value.

\fun{void}{forsubgroup}{void *E, long fun(void*, GEN), GEN G, GEN B}
goes through the same subgroups as \tet{forsubgroup(H = G, B,)}, calling
\tet{fun(E, H)} for each subgroup $H$, stopping if \kbd{fun} returns a
non-zero value.

\item \tet{forprime}, for which we refer you to the next subsection.

\item \tet{forcomposite}, we provide an iterator over composite integers:

\fun{int}{forcomposite}{forcomposite_t *T, GEN a, GEN b} initialize an
iterator $T$ over composite integers in $[a,b]$; over composites $\geq a$ if
$b = \kbd{NULL}$. Return $0$ if the range is known to be empty from the start
(as if $b < a$ or $b < 0$), and return $1$ otherwise.

\fun{GEN}{forcomposite_next}{forcomposite_t *T} returns the next composite in
the range, assuming that $T$ was initialized by \tet{forcomposite_init}.

\item \tet{forvec}, for which we provide a convenient iterator. To
initialize the analog of \kbd{forvec(X = v, ..., flag)}, call

\fun{int}{forvec_init}{forvec_t *T, GEN v, long flag}
initialize an iterator $T$ over the vectors generated by
\kbd{forvec(X = $v$,..., flag)}. This returns $0$ if this vector list is
empty, and $1$ otherwise.

\fun{GEN}{forvec_next}{forvec_t *T} returns the next element in the
\kbd{forvec} sequence, or \kbd{NULL} if we are done. The return value must be
used immediately or copied since the next call to the iterator destroys it:
the relevant vector is updated in place. The iterator works hard to not
use up PARI stack, and is more efficient when all lower bounds in the
initialization vector $v$ are integers. In that case, the cost is linear in
the number of tuples enumerated, and you can expect to run over more than
$10^9$ tuples per minute. If speed is critical and all integers involved
would fit in $C$ \kbd{long}s, write a simple direct backtracking algorithm
yourself.

\item \tet{forpart} is a variant of \kbd{forvec} which iterates over
  partitions. See the documentation of the \kbd{forpart} GP function for
  details. This function is available as a loop with callbacks:

  \fun{void}{forpart}{void *data, long (*call)(void*,GEN), long k, GEN a, GEN n}

  \noindent It is also available as an iterator:

  \fun{void}{forpart_init}{forpart_t *T, long k, GEN a, GEN n} initializes an
  iterator over the partitions of $k$, with length restricted by $n$,
  and components restricted by $a$, either of which can be set to \kbd{NULL}
  to run without restriction.

  \fun{GEN}{forpart_next}{forpart_t *T} returns the next partition, or
  \kbd{NULL} when all partitions have been exhausted.

  \fun{GEN}{forpart_prev}{forpart_t *T}returns the previous partition, or
  \kbd{NULL} when all partitions have been exhausted.

  You may \emph{not} mix calls to \tet{forpart_next} and \tet{forpart_prev}:
  the first one called determines the ordering used to iterate over the
  partitions; you can not go back since the \tet{forpart_t} structure is used
  in incompatible ways.

\subsec{Iterating over primes}\label{se:primeiter}

The library provides a high-level iterator, which stores its (private) data
in a \kbd{struct} \tet{forprime_t} and runs over arbitrary ranges of primes,
without ever overflowing.

The iterator has two flavors, one providing the successive primes as
\kbd{ulong}s, the other as \kbd{GEN}. They are initialized as follows,
where we expect to run over primes $\geq a$ and $\leq b$:

\fun{int}{forprime_init}{forprime_t *T, GEN a, GEN b} for the \kbd{GEN}
variant, where $b = \kbd{NULL}$ means $+\infty$.

\fun{int}{u_forprime_init}{forprime_t *T, ulong a, ulong b} for the
\kbd{ulong} variant, where $b = \kbd{ULONG\_MAX}$ means we will run through
all primes representable in a \kbd{ulong} type.

Both variant return $1$ on success, and $0$ if the iterator would run over an
empty interval (if $a > b$, for instance). They allocate the \tet{forprime_t}
data structure on the PARI stack.

\noindent The successive primes are then obtained using

\fun{GEN}{forprime_next}{forprime_t *T}, returns \kbd{NULL} if no more primes
are available in the interval.

\fun{ulong}{u_forprime_next}{forprime_t *T}, returns $0$ if no more primes
are available in the interval.

These two functions leave alone the PARI stack, and write their state
information in the preallocated \tet{forprime_t} struct. The typical usage is
thus:
\bprog
  forprime_t T;
  GEN p;
  pari_sp av = avma, av2;

  forprime_init(&T, gen_2, stoi(1000));
  av2 = avma;
  while ( (p = forprime_next(&T)) )
  {
    ...
    if ( prime_is_OK(p) ) break;
    avma = av2; /* delete garbage accumulated in this iteration */
  }
  avma = av; /* delete all */
@eprog\noindent Of course, the final \kbd{avma = av} could be replaced
by a \kbd{gerepile} call. Beware that swapping the
\kbd{av2 = avma} and \tet{forprime_init} call would be incorrect: the
first \kbd{avma = av2} would delete the \tet{forprime_t} structure!

\subsec{Numerical analysis}

Numerical routines code a function (to be integrated, summed, zeroed, etc.)
with two parameters named
\bprog
  void *E;
  GEN (*eval)(void*, GEN)
@eprog\noindent
The second is meant to contain all auxiliary data needed by your function.
The first is such that \kbd{eval(x, E)} returns your function evaluated at
\kbd{x}. For instance, one may code the family of functions
$f_t: x \to (x+t)^2$ via
\bprog
GEN fun(void *t, GEN x) { return gsqr(gadd(x, (GEN)t)); }
@eprog\noindent
One can then integrate $f_1$ between $a$ and $b$ with the call
\bprog
intnum((void*)stoi(1), &fun, a, b, NULL, prec);
@eprog\noindent
Since you can set \kbd{E} to a pointer to any \kbd{struct} (typecast to
\kbd{void*}) the above mechanism handles arbitrary functions. For simple
functions without extra parameters, you may set \kbd{E = NULL} and ignore
that argument in your function definition.

\section{Catching exceptions}

\subsec{Basic use}

PARI provides a mechanism to trap exceptions generated via \kbd{pari\_err}
using the \tet{pari_CATCH} construction. The basic usage is as follows
\bprog
 pari_CATCH(err_code) {
   @com recovery branch
 }
 pari_TRY {
   @com main branch
 }
 pari_ENDCATCH
@eprog\noindent This fragment executes the main branch, then the recovery
branch \emph{if} exception \kbd{err\_code} is thrown, e.g. \kbd{e\_TYPE}.
See \secref{se:errors} for the description of all error classes.
The special error code \tet{CATCH_ALL} is available to catch all errors.

One can replace the \tet{pari_TRY} keyword by \tet{pari_RETRY}, in which case
once the recovery branch is run, we run the main branch again, still catching
the same exceptions.

\misctitle{Restrictions}

\item Such constructs can be nested without adverse effect, the innermost
handler catching the exception.

\item It is \emph{valid} to leave either branch using \tet{pari_err}.

\item It is \emph{invalid} to use C flow control instructions (\kbd{break},
\kbd{continue}, \kbd{return}) to directly leave either branch without seeing
the \tet{pari_ENDCATCH} keyword. This would leave an invalid structure in the
exception handler stack, and the next exception would crash.

\item In order to leave using \kbd{break}, \kbd{continue} or \kbd{return},
one must precede the keyword by a call to

\fun{void}{pari_CATCH_reset}{} disable the current handler, allowing to leave
without adverse effect.

\subsec{Advanced use}

In the recovery branch, the exception context can be examined via the
following helper routines:

\fun{GEN}{pari_err_last}{} returns the exception context, as a \typ{ERROR}.
The exception $E$ returned by \tet{pari_err_last} can be rethrown, using
\bprog
  pari_err(0, E);
@eprog

\fun{long}{err_get_num}{GEN E} returns the error symbolic name. E.g
\kbd{e\_TYPE}.

\fun{GEN}{err_get_compo}{GEN E, long i} error $i$-th component, as documented
in \secref{se:errors}.

\noindent For instance
\bprog
 pari_CATCH(CATCH_ALL) { /* catch everything */
    GEN x, E = pari_err_last();
    long code = err_get_num(E);
    if (code != e_INV) pari_err(0, E); /* unexpected error, rethrow */
    x = err_get_compo(E, 2);
    /* e_INV has two components, 1: function name 2: non-invertible x */
    if (typ(x) != t_INTMOD) pari_err(0, E); /* unexpected type, rethrow */
    pari_CATCH_reset();
    return x; /* leave ! */
    @com @dots
 } pari_TRY {
   @com main branch
 }
 pari_ENDCATCH
@eprog

\section{A complete program}
\label{se:prog}

\noindent
Now that the preliminaries are out of the way, the best way to learn how to
use the library mode is to study a detailed example. We want to write a
program which computes the gcd of two integers, together with the Bezout
coefficients. We shall use the standard quadratic algorithm which is not
optimal but is not too far from the one used in the PARI function
\teb{bezout}.

Let $x,y$ two integers and initially
$ \pmatrix{s_x & s_y \cr t_x & t_y } =
  \pmatrix{1 & 0 \cr 0 & 1}$, so that
$$ \pmatrix{s_x & s_y \cr
            t_x & t_y }
   \pmatrix{x \cr y } =
   \pmatrix{x \cr y }.
$$
To apply the ordinary Euclidean algorithm to the right hand side,
multiply the system from the left by
$ \pmatrix{0 & 1 \cr 1 & -q }$,
with $q = \kbd{floor}(x / y)$. Iterate until $y = 0$ in the right hand side,
then the first line of the system reads
$$ s_x x + s_y y = \gcd(x,y).$$
In practice, there is no need to update $s_y$ and $t_y$ since
$\gcd(x,y)$ and $s_x$ are enough to recover $s_y$. The following program
is now straightforward. A couple of new functions appear in there, whose
description can be found in the technical reference manual in Chapter 5,
but whose meaning should be clear from their name and the context.

This program can be found in \kbd{examples/extgcd.c} together with a
proper \kbd{Makefile}. You may ignore the first comment
\bprog
/*
GP;install("extgcd", "GG&&", "gcdex", "./libextgcd.so");
*/
@eprog\noindent which instruments the program so that \kbd{gp2c-run extgcd.c}
can import the \kbd{extgcd()} routine into an instance of the \kbd{gp}
interpreter (under the name \kbd{gcdex}). See the \kbd{gp2c} manual for
details.
\newpage

\bprogfile{../examples/extgcd.c}

\noindent For simplicity, the inner loop does not include any garbage
collection, hence memory use is quadratic in the size of the inputs instead
of linear. Here is a better version of that loop:
\bprog
  pari_sp av = avma;
  ...
  while (!gequal0(b))
  {
    GEN r, q = dvmdii(a, b, &r), v = vx;

    vx = subii(ux, mulii(q, vx));
    ux = v; a = b; b = r;
    if (gc_needed(av,1))
      gerepileall(av, 4, &a, &b, &ux, &vx);
  }
@eprog
\newpage
