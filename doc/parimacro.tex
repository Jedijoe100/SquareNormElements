\catcode`\@=11
\newif\ifgphelp
\long\def\@ifundef#1#2#3{\expandafter\ifx\csname
  #1\endcsname\relax#2\else#3\fi}
\@ifundef{fromgphelp}{\gphelpfalse}{\gphelptrue}

\ifgphelp
  \magnification=\magstep1
  \baselineskip=13pt
\else
  \magnification=\magstephalf
\fi

\newread\std
\openin\std=paricfg.tex
\ifeof\std
  \openin\std=doc/paricfg.tex
  \ifeof\std
    \message{paricfg.tex not found. You should run Configure.}
% use default values
    \def\vers{2.0.x}
    \def\miscdir{/usr/local/lib/pari}
    \def\includedir{/usr/local/include/pari}
    \def\libdir{/usr/local/lib}
    \def\wwwsite{\kbd{http://hasse.mathematik.tu-muenchen.de/ntsw/pari/}}
  \else
    \input doc/paricfg.tex
  \fi
\else
  \input paricfg.tex
\fi

\font\chaptertitlefont=cmr12 scaled \magstep1
\font\captertitlebf=cmbx10 scaled \magstep2
\font\sectiontitlebf=cmbx12
\font\seventt=cmtt8 scaled 875
\scriptfont\ttfam=\seventt % we should really set the \hyphenchar etc first
\parskip=6pt plus 3pt minus 1.5pt
\overfullrule=0pt

%
% CROSS REFERENCING & INDEX
%
\newif\ifsecondpass
\newwrite\out
\newwrite\aux

\ifgphelp % disable most of it
  \def\sidx#1{}
  \def\tocwrite#1{}
  \def\label#1{}
  \def\ref#1{[{\bf ??}]}
\else
\def\typeout#1{\immediate\write\out{#1}}
\def\@namedef#1{\expandafter\def\csname #1\endcsname}
\def\newlabel#1#2{\@ifundef{r@#1}{}{\message{Label `#1' multiply
  defined}}\global\@namedef{r@#1}{#2}}

\openin\std=users.std
\ifeof\std
  \secondpassfalse
  \typeout{FIRST PASS}
  \csname newwrite\endcsname\toc
  \csname newwrite\endcsname\index
  \openout\toc=\jobname.toc
  \openout\index=\jobname.idx
  \let\condwrite=\write
\else
  \secondpasstrue
  \typeout{SECOND PASS}
  \let\immediate\relax
  \def\condwrite#1#2{}
\fi
% index
\def\sidx#1{{\condwrite\index{!#1!\the\pageno!}}}
%cf TeXBook Exercice 21.10
\def\tocwrite#1{{\let\the=0\edef\next{\condwrite\toc{#1}}\next}}
% don't use \jobname here, macros may be used by tutorial.tex
\openin\std=users.aux
\ifeof\std
  \message{No aux file.}
\else
  \input users.aux
\fi
\openin\std=users.std

% \ref, \label
\openout\aux=\jobname.aux

\ifx\inputlineno\undefined
  \let\on@line\empty
\else
  \def\on@line{ on input line \the\inputlineno}
\fi
\def\@errundef#1{\typeout{Reference `#1' on page \the\pageno \space
undefined\on@line}}

\def\@car#1#2\@nil{#1}
\def\@cdr#1#2\@nil{#2}

\def\@ref{\expandafter\@cdr\@temp \@nil\null}
\def\@cref{\expandafter\@car\@temp \@nil\null}

\def\label#1{\immediate\write\aux{\string
  \newlabel{#1}{{\the\chapno}{\currentlabel}}}}
\def\ref#1{\@ifundef{r@#1}
  {{\bf ??}\@errundef{#1}}
  {\edef\@temp{\csname r@#1\endcsname}%
    \def\lbl{\@ref}\def\chp{\@cref}%
    \ifx\chp{\the\chapno}\lbl\else\chp.\lbl\fi}}
\fi

\def\idx#1{\sidx{#1}#1}
\def\secref#1{Section~\ref{#1}}

%
% SECTIONS
%
\newcount\appno
\newcount\chapno
\newcount\secno
\newcount\subsecno
\newcount\subsubsecno

\def\newpage{\hbox{}\vfill\eject}

\def\myhook{}
\def\title#1#2{%
  \ifodd\pageno\else\newpage\fi
  \tocwrite{#1 #2\string\dotfill\the\pageno}
  \ifgphelp\else
     \centerline{\chaptertitlefont #1}\medskip
  \fi
  \centerline{\let\bf\chaptertitlebf \chaptertitlefont #2}\vskip1cm}
\def\sectitle#1{%
  \tocwrite{{\sevenrm\hskip0.5cm#1\string\dotfill\the\pageno}}
  \vskip 0pt plus 54pt\penalty-600 % good break
  \vskip 24pt plus -45pt minus 9pt
  \leftline{\myhook{\sectiontitlebf #1}.}\def\myhook{}
  \penalty10000 % impossible break
  \smallskip}% whatever follows will add a \parskip

\def\subsectitle#1{%
  \vskip 0pt plus 45pt\penalty-300
  \vskip 6pt plus -42pt minus 3pt
  \noindent\myhook{\bf{#1}}\def\myhook{}}

\def\subsubsectitle#1{% a \parskip is being added anyway by \noindent
  \par\vskip 0pt plus 39pt\penalty-200
  \vskip 0pt plus -37pt minus 1.5pt
  \noindent\myhook{\bf{#1}}\def\myhook{}}

\ifgphelp
  \def\maketitle#1{#1}
\else
  \def\maketitle#1{
  \ifnum\chapno=0
    \currentlabel. #1%
  \else
    \number\chapno.\currentlabel\ #1%
  \fi}
\fi

\def\misctitle#1{\par\vskip 0pt plus 66pt\penalty-400
  \vskip 3pt plus -64pt minus 1.5pt\noindent{\bf #1}}

% Chapter headings now occupy two lines in the manual (still only one in
% INSTALL.tex and gphelp-extracted bits).  Syntax requirement: After calling
% \chapter{...} and possibly \label{...}  etc, an empty line _must_ follow
% before the first paragraph of text or section heading begins. [GN]
\def\chapter#1#2\par{
  \secno=0\global\advance\chapno by 1
  \title{Chapter \number\chapno:}{#1}#2\noindent\ignorespaces}
\def\appendix#1\par{
  \chapno=0
  \secno=0\global\advance\appno by 1
  \title{Appendix %
  \ifcase\appno\or A\or B\or C\or D\or E\or F\or G\or H\or I\or J\or K\fi:}%
  {#1}\noindent\ignorespaces}
\def\section#1{
  \subsecno=0\advance\secno by 1
  \gdef\currentlabel{\number\secno}
  \sectitle{\maketitle{#1}}}
\def\subsec#1{
  \subsubsecno=0\advance\subsecno by 1
  \gdef\currentlabel{\number\secno.\number\subsecno}
  \subsectitle{\maketitle{#1}}}
\def\subsubsec#1{
  \advance\subsubsecno by 1
  \gdef\currentlabel{\number\secno.\number\subsecno.\number\subsubsecno}
  \subsubsectitle{\maketitle{#1}}}

\def\unix{\strut\hbox to 0pt{\hss {UNIX: }\indent}}
\def\unixx{\noindent\strut\hbox to 0pt{\hss {UNIX: }}}
\def\emacs{\strut\hbox to 0pt{\hss {EMACS: }\indent}}

\def\sectionunix#1{\def\myhook{\unixx{}}\section{#1}}
\def\unixsubsecunix#1{\def\myhook{\unix{}}\subsec{#1}}
\def\subsecidxunix#1{\def\myhook{\unix{}}\subsecidx{#1}}
\def\subsubsecunix#1{\def\myhook{\unix{}}\subsubsec{#1}}
\def\subsubsecidxunix#1{\def\myhook{\unix{}}\subsubsecidx{#1}}

\def\subseckbd#1{\subsec{\kbd{#1}}}
\def\subsecidx#1{\subsec{#1}\sidx{#1}}
\def\subsubsecidx#1{\subsubsec{#1}\sidx{#1}}
%
% General purpose
%
\def\begintitle{
  \begingroup\nopagenumbers
  \font\mine=cmb10 scaled 1893
  \hbox{}
}
\def\authors{
  \centerline{C. Batut, K. Belabas, D. Bernardi, H. Cohen, M. Olivier}
  \vskip 1.truecm
  \centerline{Laboratoire A2X, U.M.R. 9936 du C.N.R.S.}
  \centerline{Universit\'e Bordeaux I, 351 Cours de la Lib\'eration}
  \centerline{33405 TALENCE Cedex, FRANCE}
  \centerline{\tt e-mail: pari@math.u-bordeaux.fr}
  \vskip 2.truecm
  \centerline{Home Page:}
  \centerline{\wwwsite}
  \vskip .5truecm
  \centerline{Primary {\tt ftp} site:}
  \centerline{\tt ftp://megrez.math.u-bordeaux.fr/pub/pari/}
  \vskip 1.truecm
}
\def\endtitle{\newpage\endgroup}

\newfam\euffam
\font\teneuf=eufm10
\textfont\euffam=\teneuf
\def\goth#1{{\fam\euffam#1}}
\def\d{\goth d}
\def\f{\goth f}
\def\p{{\bf p}}
\def\kbd#1{{\tt #1}}
\def\key#1{{\bf #1}}
\def\var#1{\hbox{\it #1\/}}
\def\fl{\var{f\kern0pt lag}}
\def\synx#1#2#3{\sidx{#3}The library syntax is $\key{#1}({#2})$}
\def\syn#1#2{\synx{#1}{#2}{#1}}
\def\synt#1#2{\syn{#1}{\hbox{\tt #2}}}

\def\bs{{\char'134}}
\def\obr{{\char'173}}
\def\cbr{{\char'175}}
\def\pow{\^{}\hskip0pt}
\def\til{\raise-0.3em\hbox{\~{}}}
\def\b#1{{\tt \bs#1}}
\def\mod{\,{\rm mod}\,}
\def\text#1{{\rm#1}}
\def\teb#1{{\bf\idx{#1}}}
\def\tet#1{{\tt\idx{#1}}}
\def\dfrac#1#2{{{#1}\over{#2}}}
\def\binom#1#2{\pmatrix{{#1}\cr{#2}}}
\def\Bbb#1{{\bf #1}}
\def\typ#1{\kbd{t\_\kern1pt#1}}
\def\idxtyp#1{\sidx{t\string\_#1}}
\def\Z{\Bbb Z}
\def\Q{\Bbb Q}
\def\F{\Bbb F}
\def\R{\Bbb R}
\def\C{\Bbb C}
\def\dotfill{\leaders\hbox to 3truemm{\hfil.\hfil}\hfill}

{\catcode`\^=13\obeyspaces
\gdef\bprog{\par\begingroup%
\catcode`\^=13\def^{{\pow}}%
\tt\obeylines\leavevmode\parskip=0pt plus 1pt%
\obeyspaces\def {\ }\def\h{\hbox{}}%
\interlinepenalty2000\clubpenalty9000\widowpenalty9000%
\h}
% variant of same, for use with tabbing.  Tabbing doesn't build paragraphs,
% the lines go directly into the vlist -- so we must add our own \parskip. [GN]
\gdef\bprogtabs{\par\vskip\parskip\begingroup%
\catcode`\^=13\def^{{\pow}}
\obeyspaces\def {\ }\def\h{\hbox{}}%
\tt\settabs}}
% both to be ended with:
\def\eprog{\endgroup\par}
\catcode`\@=12
