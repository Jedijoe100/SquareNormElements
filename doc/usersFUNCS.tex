% $Id$
% Copyright (c) 2000  The PARI Group
%
% This file is part of the PARI/GP documentation
%
% Permission is granted to copy, distribute and/or modify this document
% under the terms of the GNU General Public License
\chapter{Functions and Operations Available in PARI and GP}
\label{se:functions}

The functions and operators available in PARI and in the GP/PARI calculator
are numerous and ever-expanding. Here is a description of the ones available
in version \vers. It should be noted that many of these functions accept
quite different types as arguments, but others are more restricted. The list
of acceptable types will be given for each function or class of functions.
Except when stated otherwise, it is understood that a function or operation
which should make natural sense is legal. In this chapter, we will describe
the functions according to a rough classification. The general entry looks
something like:

\key{foo}$(x,\{\fl=0\})$: short description.

The library syntax is \kbd{GEN foo(GEN x, long fl = 0)}.

\noindent
This means that the GP function \kbd{foo} has one mandatory argument $x$, and
an optional one, $\fl$, whose default value is 0. (The $\{\}$ should not be
typed, it is just a convenient notation we will use throughout to denote
optional arguments.) That is, you can type \kbd{foo(x,2)}, or \kbd{foo(x)},
which is then understood to mean \kbd{foo(x,0)}. As well, a comma or closing
parenthesis, where an optional argument should have been, signals to GP it
should use the default. Thus, the syntax \kbd{foo(x,)} is also accepted as a
synonym for our last expression. When a function has more than one optional
argument, the argument list is filled with user supplied values, in order.
When none are left, the defaults are used instead. Thus, assuming that
\kbd{foo}'s prototype had been
$$\hbox{%
\key{foo}$(\{x=1\},\{y=2\},\{z=3\})$,%
}$$
typing in \kbd{foo(6,4)} would give
you \kbd{foo(6,4,3)}. In the rare case when you want to set some far away
argument, and leave the defaults in between as they stand, you can use the
``empty arg'' trick alluded to above: \kbd{foo(6,,1)} would yield
\kbd{foo(6,2,1)}. By the way, \kbd{foo()} by itself yields
\kbd{foo(1,2,3)} as was to be expected.

In this rather special case of a function having no mandatory argument, you
can even omit the $()$: a standalone \kbd{foo} would be enough (though we
do not recommend it for your scripts, for the sake of clarity). In defining
GP syntax, we strove to put optional arguments at the end of the argument
list (of course, since they would not make sense otherwise), and in order of
decreasing usefulness so that, most of the time, you will be able to ignore
them.

Finally, an optional argument (between braces) followed by a star, like
$\{\var{x}\}*$, means that any number of such arguments (possibly none) can
be given. This is in particular used by the various \kbd{print} routines.

\misctitle{Flags} A \tev{flag} is an argument which, rather than conveying
actual information to the routine, instructs it to change its default
behavior, e.g.~return more or less information. All such
flags are optional, and will be called \fl\ in the function descriptions to
follow. There are two different kind of flags

\item generic: all valid values for the flag are individually
described (``If \fl\ is equal to $1$, then\dots'').

\item binary:\sidx{binary flag} use customary binary notation as a
compact way to represent many toggles with just one integer. Let
$(p_0,\dots,p_n)$ be a list of switches (i.e.~of properties which take either
the value $0$ or~$1$), the number $2^3 + 2^5 = 40$ means that $p_3$ and $p_5$
are set (that is, set to $1$), and none of the others are (that is, they
are set to $0$). This is announced as ``The binary digits of $\fl$ mean 1:
$p_0$, 2: $p_1$, 4: $p_2$'', and so on, using the available consecutive
powers of~$2$.

\misctitle{Mnemonics for flags} Numeric flags as mentioned above are
obscure, error-prone, and quite rigid: should the authors
want to adopt a new flag numbering scheme (for instance when noticing
flags with the same meaning but different numeric values across a set of
routines), it would break backward compatibility. The only advantage of
explicit numeric values is that they are fast to type, so their use is only
advised when using the calculator \kbd{gp}.

As an alternative, one can replace a numeric flag by a character string
containing symbolic identifiers. For a generic flag, the mnemonic
corresponding to the numeric identifier is given after it as in

\bprog
fun(x, {flag = 0} ):

  If flag is equal to 1 = AGM, use an agm formula ...
@eprog\noindent
which means that one can use indifferently \kbd{fun($x$, 1)} or
\kbd{fun($x$, "AGM")}.

For a binary flag, mnemonics corresponding to the various toggles are given
after each of them. They can be negated by prepending \kbd{no\_} to the
mnemonic, or by removing such a prefix. These toggles are grouped together
using any punctuation character (such as ',' or ';'). For instance (taken
from description of $\tet{ploth}(X=a,b,\var{expr},\{\fl=0\},\{n=0\})$)

\centerline{Binary digits of flags mean: $1=\kbd{Parametric}$,
$2=\kbd{Recursive}$, \dots}

\noindent so that, instead of $1$, one could use the mnemonic
\kbd{"Parametric; no\_Recursive"}, or simply \kbd{"Parametric"} since
\kbd{Recursive} is unset by default (default value of $\fl$ is $0$,
i.e.~everything unset). People used to the bit-or notation in languages like
C may also use the form \kbd{"Parametric | no\_Recursive"}.

\misctitle{Pointers} \varsidx{pointer} If a parameter in the function
prototype is prefixed with a \& sign, as in

\key{foo}$(x,\&e)$

\noindent it means that, besides the normal return value, the function may
assign a value to $e$ as a side effect. When passing the argument, the \&
sign has to be typed in explicitly. As of version \vers, this \tev{pointer}
argument is optional for all documented functions, hence the \& will always
appear between brackets as in \kbd{Z\_issquare}$(x,\{\&e\})$.

\misctitle{About library programming}
The \var{library} function \kbd{foo}, as defined at the beginning of this
section, is seen to have two mandatory arguments, $x$ and \fl: no function
seen in the present chapter has been implemented so as to accept a variable
number of arguments, so all arguments are mandatory when programming with the
library (usually, variants are provided corresponding to the various flag values).
We include an \kbd{= default value} token in the prototype to signal how a missing
argument should be encoded. Most of the time, it will be a \kbd{NULL} pointer, or
-1 for a variable number. Refer to the \emph{User's Guide to the PARI library}
for general background and details.

\section{Standard monadic or dyadic operators}

\subseckbd{+$/$-} The expressions \kbd{+}$x$ and \kbd{-}$x$ refer
to monadic operators (the first does nothing, the second negates $x$).

The library syntax is \fun{GEN}{gneg}{GEN x} for \kbd{-}$x$.

\subseckbd{+} The expression $x$ \kbd{+} $y$ is the \idx{sum} of $x$ and $y$.
Addition between a scalar type $x$ and a \typ{COL} or \typ{MAT} $y$ returns
respectively $[y[1] + x, y[2],\dots]$ and $y + x \text{Id}$. Other additions
between a scalar type and a vector or a matrix, or between vector/matrices of
incompatible sizes are forbidden.

The library syntax is \fun{GEN}{gadd}{GEN x, GEN y}.

\subseckbd{-} The expression $x$ \kbd{-} $y$ is the \idx{difference} of $x$
and $y$. Subtraction between a scalar type $x$ and a \typ{COL} or \typ{MAT}
$y$ returns respectively $[y[1] - x, y[2],\dots]$ and $y - x \text{Id}$.
Other subtractions between a scalar type and a vector or a matrix, or
between vector/matrices of incompatible sizes are forbidden.

The library syntax is \fun{GEN}{gsub}{GEN x, GEN y} for $x$ \kbd{-} $y$.

\subseckbd{*} The expression $x$ \kbd{*} $y$ is the \idx{product} of $x$
and $y$. Among the prominent impossibilities are multiplication between
vector/matrices of incompatible sizes, between a \typ{INTMOD} or \typ{PADIC}
Restricted to scalars, \kbd{*} is commutative; because of vector and matrix
operations, it is not commutative in general.

Multiplication between two \typ{VEC}s or two \typ{COL}s is not
allowed; to take the \idx{scalar product} of two vectors of the same length,
transpose one of the vectors (using the operator \kbd{\til} or the function
\kbd{mattranspose}, see \secref{se:linear_algebra}) and multiply a line vector
by a column vector:
\bprog
? a = [1,2,3];
? a * a
  ***   at top-level: a*a
  ***                  ^--
  *** _*_: forbidden multiplication t_VEC * t_VEC.
? a * a~
%2 = 14
@eprog

If $x,y$ are binary quadratic forms, compose them; see also
\kbd{qfbnucomp} and \kbd{qfbnupow}. If $x,y$ are \typ{VECSMALL} of the same
length, understand them as permutations and compose them.

The library syntax is \fun{GEN}{gmul}{GEN x, GEN y} for $x$ \kbd{*} $y$.
Also available is \fun{GEN}{gsqr}{GEN x} for $x$ \kbd{*} $x$.

\subseckbd{/} The expression $x$ \kbd{/} $y$ is the \idx{quotient} of $x$
and $y$. In addition to the impossibilities for multiplication, note that if
the divisor is a matrix, it must be an invertible square matrix, and in that
case the result is $x*y^{-1}$. Furthermore note that the result is as exact
as possible: in particular, division of two integers always gives a rational
number (which may be an integer if the quotient is exact) and \emph{not} the
Euclidean quotient (see $x$ \kbd{\bs} $y$ for that), and similarly the
quotient of two polynomials is a rational function in general. To obtain the
approximate real value of the quotient of two integers, add \kbd{0.} to the
result; to obtain the approximate $p$-adic value of the quotient of two
integers, add \kbd{O(p\pow k)} to the result; finally, to obtain the
\idx{Taylor series} expansion of the quotient of two polynomials, add
\kbd{O(X\pow k)} to the result or use the \kbd{taylor} function
(see \secref{se:taylor}). \label{se:gdiv}

The library syntax is \fun{GEN}{gdiv}{GEN x, GEN y} for $x$ \kbd{/} $y$.

\subseckbd{\bs} The expression \kbd{$x$ \bs\ $y$} is the \idx{Euclidean
quotient} of $x$ and $y$. If $y$ is a real scalar, this is defined as
\kbd{floor($x$/$y$)} if $y > 0$, and \kbd{ceil($x$/$y$)} if $y < 0$ and
the division is not exact. Hence the remainder \kbd{$x$ - ($x$\bs$y$)*$y$}
is in $[0, |y|[$.

Note that when $y$ is an integer and $x$ a polynomial, $y$ is first promoted
to a polynomial of degree $0$. When $x$ is a vector or matrix, the operator
is applied componentwise.

The library syntax is \fun{GEN}{gdivent}{GEN x, GEN y}
for $x$ \kbd{\bs} $y$.

\subseckbd{\bs/} The expression $x$ \b{/} $y$ evaluates to the rounded
\idx{Euclidean quotient} of $x$ and $y$. This is the same as \kbd{$x$ \bs\ $y$}
except for scalar division: the quotient is such that the corresponding
remainder is smallest in absolute value and in case of a tie the quotient
closest to $+\infty$ is chosen (hence the remainder would belong to
$]{-}|y|/2, |y|/2]$).

When $x$ is a vector or matrix, the operator is applied componentwise.

The library syntax is \fun{GEN}{gdivround}{GEN x, GEN y}
for $x$ \b{/} $y$.

\subseckbd{\%} The expression \kbd{$x$ \% $y$} evaluates to the modular
\idx{Euclidean remainder} of $x$ and $y$, which we now define. When $x$ or $y$
is a non-integral real number, \kbd{$x$\%$y$} is defined as \kbd{$x$ -
($x$\bs$y$)*$y$}. Otherwise, if $y$ is an integer, this is the smallest
non-negative integer congruent to $x$ modulo $y$. (This actually coincides with
the previous definition if and only if $x$ is an integer.) If $y$ is a
polynomial, this is the polynomial of smallest degree congruent to $x$ modulo
$y$. For instance:
\bprog
? (1/2) % 3
%1 = 2
? 0.5 % 3
%2 = 0.5000000000000000000000000000
? (1/2) % 3.0
%3 = 1/2
@eprog
Note that when $y$ is an integer and $x$ a polynomial, $y$ is first promoted
to a polynomial of degree $0$. When $x$ is a vector or matrix, the operator
is applied componentwise.

The library syntax is \fun{GEN}{gmod}{GEN x, GEN y}
for $x$ \kbd{\%} $y$.

\subseckbd{\pow} The expression $x\hbox{\kbd{\pow}}n$ is \idx{powering}.
If the exponent is an integer, then exact operations are performed using
binary (left-shift) powering techniques. In particular, in this case $x$
cannot be a vector or matrix unless it is a square matrix (invertible
if the exponent is negative). If $x$ is a $p$-adic number, its
precision will increase if $v_p(n) > 0$. Powering a binary quadratic form
(types \typ{QFI} and \typ{QFR}) returns a reduced representative of the
class, provided the input is reduced. In particular, $x\hbox{\kbd{\pow}}1$ is
identical to $x$.

PARI is able to rewrite the multiplication $x * x$ of two \emph{identical}
objects as $x^2$, or $\kbd{sqr}(x)$. Here, identical means the operands are
two different labels referencing the same chunk of memory; no equality test
is performed. This is no longer true when more than two arguments are
involved.

If the exponent is not of type integer, this is treated as a transcendental
function (see \secref{se:trans}), and in particular has the effect of
componentwise powering on vector or matrices.

As an exception, if the exponent is a rational number $p/q$ and $x$ an
integer modulo a prime or a $p$-adic number, return a solution $y$ of
$y^q=x^p$ if it exists. Currently, $q$ must not have large prime factors.
Beware that
\bprog
? Mod(7,19)^(1/2)
%1 = Mod(11, 19) /* is any square root */
? sqrt(Mod(7,19))
%2 = Mod(8, 19)  /* is the smallest square root */
? Mod(7,19)^(3/5)
%3 = Mod(1, 19)
? %3^(5/3)
%4 = Mod(1, 19)  /* Mod(7,19) is just another cubic root */
@eprog

If the exponent is a negative integer, an \idx{inverse} must be computed.
For non-invertible \typ{INTMOD}, this will fail and implicitly exhibit a
non trivial factor of the modulus:
\bprog
? Mod(4,6)^(-1)
  ***   at top-level: Mod(4,6)^(-1)
  ***                         ^-----
  *** _^_: impossible inverse modulo: Mod(2, 6).
@eprog\noindent
(Here, a factor 2 is obtained directly. In general, take the gcd of the
representative and the modulus.) This is most useful when performing
complicated operations modulo an integer $N$ whose factorization is
unknown. Either the computation succeeds and all is well, or a factor $d$
is discovered and the computation may be restarted modulo $d$ or $N/d$.

For non-invertible \typ{POLMOD}, this will fail without exhibiting a
factor.
\bprog
? Mod(x^2, x^3-x)^(-1)
  ***   at top-level: Mod(x^2,x^3-x)^(-1)
  ***                               ^-----
  *** _^_: non-invertible polynomial in RgXQ_inv.
? a = Mod(3,4)*y^3 + Mod(1,4); b = y^6+y^5+y^4+y^3+y^2+y+1;
? Mod(a, b)^(-1);
  ***   at top-level: Mod(a,b)^(-1)
  ***                         ^-----
  *** _^_: impossible inverse modulo: Mod(0, 4).
@eprog\noindent
In fact the latter polynomial is invertible, but the algorithm used
(subresultant) assumes the base ring is a domain. If it is not the case,
as here for $\Z/4\Z$, a result will be correct but chances are an error
will occur first. In this specific case, one should work with $2$-adics.
In general, one can try the following approach
\bprog
? inversemod(a, b) =
{ my(m);
  m = polsylvestermatrix(polrecip(a), polrecip(b));
  m = matinverseimage(m, matid(#m)[,1]);
  Polrev( vecextract(m, Str("..", poldegree(b))), variable(b) )
}
? inversemod(a,b)
%2 = Mod(2,4)*y^5 + Mod(3,4)*y^3 + Mod(1,4)*y^2 + Mod(3,4)*y + Mod(2,4)
@eprog\noindent
This is not guaranteed to work either since it must invert pivots. See
\secref{se:linear_algebra}.

The library syntax is \fun{GEN}{gpow}{GEN x, GEN n, long prec}
for $x\hbox{\kbd{\pow}}n$.

%SECTION: operators

\subsec{Comparison and Boolean operators}\sidx{Boolean operators} The six
standard \idx{comparison operators} \kbd{<=}, \kbd{<}, \kbd{>=}, \kbd{>},
\kbd{==}, \kbd{!=} are available in GP. The result is 1 if the comparison is
true, 0 if it is false. The operator \kbd{==} is quite liberal : for
instance, the integer 0, a 0 polynomial, and a vector with 0 entries are all
tested equal.

The extra operator \kbd{===} tests whether two objects are identical and is
much stricter than \kbd{==} : objects of different type or length are never
identical.


For the purpose of comparison, \typ{STR} objects are strictly larger than any
other non-string type; two \typ{STR} objects are compared using the standard
lexicographic order.

GP accepts the following synonyms for some of the above functions:
\kbd{|} and \kbd{\&} are accepted as\sidx{bitwise and}\sidx{bitwise or}
synonyms of \kbd{||} and \kbd{\&\&} respectively. Also, \kbd{<>} is accepted
as a synonym for \kbd{!=}. On the other hand, \kbd{=} is definitely
\emph{not} a synonym for \kbd{==}: it is the assignment statement. (We do not
use the customary C operators for bitwise \kbd{and} or bitwise \kbd{or}; use
\tet{bitand} or \tet{bitor}.)

The standard boolean operators \kbd{||} (\idx{inclusive or}), \kbd{\&\&}
(\idx{and})\sidx{or} and \kbd{!} (\idx{not}) are also available.

\section{Conversions and similar elementary functions or commands}
\label{se:conversion}

\noindent
Many of the conversion functions are rounding or truncating operations. In
this case, if the argument is a rational function, the result is the
Euclidean quotient of the numerator by the denominator, and if the argument
is a vector or a matrix, the operation is done componentwise. This will not
be restated for every function.

%SECTION: conversions

\section{Transcendental functions}\label{se:trans}

Since the values of transcendental functions cannot be exactly represented,
these functions will always return an inexact object: a real number,
a complex number, a $p$-adic number or a power series.  All these objects
have a certain finite precision.

As a general rule, which of course in some cases may have exceptions,
transcendental functions operate in the following way:

\item If the argument is either a real number or an inexact complex number
(like \kbd{1.0 + I} or \kbd{Pi*I} but not \kbd{2 - 3*I}), then the
computation is done with the precision of the argument.
In the example below, we see that changing the precision to $50$ digits does
not matter, because $x$ only had a precision of $19$ digits.
\bprog
? \p 15
   realprecision = 19 significant digits (15 digits displayed)
? x = Pi/4
%1 = 0.785398163397448
? \p 50
   realprecision = 57 significant digits (50 digits displayed)
? sin(x)
%2 = 0.7071067811865475244
@eprog

Note that even if the argument is real, the result may be complex
(e.g.~$\text{acos}(2.0)$ or $\text{acosh}(0.0)$). See each individual
function help for the definition of the branch cuts and choice of principal
value.

\item If the argument is either an integer, a rational, an exact complex
number or a quadratic number, it is first converted to a real
or complex number using the current \idx{precision} held in the default
\tet{realprecision}.  This precision (the number of decimal digits) can be
changed using \b{p} or \kbd{default(realprecision,...)}).
After this conversion, the computation proceeds as above for real or complex
arguments.

In library mode, the \kbd{realprecision} does not matter; instead the
precision is taken from the \kbd{prec} parameter which every transcendental
function has.  As in \kbd{gp}, this \kbd{prec} is not used when the
argument to a function is already inexact.
Note that the argument \var{prec} stands for the length in words of a real
number, including codewords. Hence we must have $\var{prec} \geq 3$.

Some accuracies attainable on 32-bit machines cannot be attained
on 64-bit machines for parity reasons. For example the default \kbd{gp} accuracy
is 28 decimal digits on 32-bit machines, corresponding to \var{prec} having
the value 5, but this cannot be attained on 64-bit machines.

\item If the argument is a polmod (representing an algebraic number),
then the function is evaluated for every possible complex embedding of that
algebraic number.  A column vector of results is returned, with one component
for each complex embedding.  Therefore, the number of components equals
the degree of the \typ{POLMOD} modulus.

\item If the argument is an intmod or a $p$-adic, at present only a
few functions like \kbd{sqrt} (square root), \kbd{sqr} (square), \kbd{log},
\kbd{exp}, powering, \kbd{teichmuller} (Teichm\"uller character) and
\kbd{agm} (arithmetic-geometric mean) are implemented.

Note that in the case of a $2$-adic number, $\kbd{sqr}(x)$ may not be
identical to $x*x$: for example if $x = 1+O(2^5)$ and $y = 1+O(2^5)$ then
$x*y = 1+O(2^5)$ while $\kbd{sqr}(x) = 1+O(2^6)$. Here, $x * x$ yields the
same result as $\kbd{sqr}(x)$ since the two operands are known to be
\emph{identical}. The same statement holds true for $p$-adics raised to the
power $n$, where $v_p(n) > 0$.

\misctitle{Remark} If we wanted to be strictly consistent with
the PARI philosophy, we should have $x*y = (4 \mod 8)$ and $\kbd{sqr}(x) =
(4 \mod 32)$ when both $x$ and $y$ are congruent to $2$ modulo $4$.
However, since intmod is an exact object, PARI assumes that the modulus
must not change, and the result is hence $(0\, \mod\, 4)$ in both cases. On
the other hand, $p$-adics are not exact objects, hence are treated
differently.

\item If the argument is a polynomial, a power series or a rational function,
it is, if necessary, first converted to a power series using the current
series precision, held in the default \tet{seriesprecision}. This precision
(the number of significant terms) can be changed using \b{ps} or
\kbd{default(seriesprecision,...)}. Then the Taylor series expansion of the
function around $X=0$ (where $X$ is the main variable) is computed to a
number of terms depending on the number of terms of the argument and the
function being computed.

Under \kbd{gp} this again is transparent to the user. When programming in
library mode, however, it is \emph{strongly} advised to perform an explicit
conversion to a power series first, as in \kbd{x = gtoser(x, seriesprec)},
where the number of significant terms \kbd{seriesprec} can be specified
explicitly. If you do not do this, a global variable \kbd{precdl} is used
instead, to convert polynomials and rational functions to a power series with
a reasonable number of terms; tampering with the value of this global
variable is \emph{deprecated} and strongly discouraged.


\item If the argument is a vector or a matrix, the result is the
componentwise evaluation of the function. In particular, transcendental
functions on square matrices, which are not implemented in the present
version \vers, will have a different name if they are implemented some day.

\subseckbd{\pow} If $y$ is not of type integer, \kbd{x\pow y} has the same
effect as \kbd{exp(y*log(x))}. It can be applied to $p$-adic numbers as well
as to the more usual types.\sidx{powering}

The library syntax is \fun{GEN}{gpow}{GEN x, GEN n, long prec}
for $x\hbox{\kbd{\pow}}n$.

%SECTION: transcendental

\section{Arithmetic functions}\label{se:arithmetic}

These functions are by definition functions whose natural domain of
definition is either $\Z$ (or $\Z_{>0}$). The way these functions are used is
completely different from transcendental functions in that there are no
automatic type conversions: in general only integers are accepted as
arguments.

\subsec{Arithmetic functions and factorization}
All arithmetic functions in the narrow sense of the word~--- Euler's
totient\sidx{Euler totient function} function, the \idx{Moebius} function,
the sums over divisors or powers of divisors etc.--- call, after trial
division by small primes, the same versatile factoring machinery described
under \kbd{factorint}. It includes \idx{Shanks SQUFOF}, \idx{Pollard Rho},
\idx{ECM} and \idx{MPQS} stages, and has an early exit option for the
functions \teb{moebius} and (the integer function underlying)
\teb{issquarefree}. This machinery relies on a fairly strong
probabilistic primality test, see \kbd{ispseudoprime}, but you may also set
\bprog
  default(factor_proven, 1)
@eprog\noindent to ensure that all tentative factorizations are fully proven.
This should not slow down PARI too much, unless prime numbers with
hundreds of decimal digits occur frequently in your application.

\subsec{Orders in finite groups and Discrete Logarithm functions}
\label{se:DLfun}

The following functions compute the order of an element in a finite group:
\kbd{ellorder} (the rational points on an elliptic curve defined over a
finite field), \kbd{fforder} (the multiplicative group of a finite field),
\kbd{znorder} (the invertible elements in $\Z/n\Z$). The following functions
compute discrete logarithms in the same groups (whenever this is meaningful)
\kbd{elllog}, \kbd{fflog}, \kbd{znlog}.

All such functions allow an optional argument specifying an integer
$N$, representing the order of the group. (The \emph{order} functions also
allows any non-zero multiple of the order, with a minor loss of efficiency.)
The meaning of that optional argument is as follows and depends on its type

\item \typ{INT}: the integer $N$,

\item \typ{MAT}: the factorization \kbd{fa = factor($N$)},

\item \typ{VEC}: this is the preferred format and provides both the
integer $N$ and its factorization in a two-component vector \kbd{[$N$,
fa]}.

When the group is fixed and many orders or discrete logarithms will be
computed, it is much more efficient to initialize this data once and for all
and pass it to the relevant functions, as in
\bprog
? p = nextprime(10^40);
? v = [p-1, factor(p-1)]; \\ data for discrete log & order computations
? znorder(Mod(2,p), v)
%3 = 500000000000000000000000000028
? g = znprimroot(p);
? znlog(2, g, v)
%5 = 543038070904014908801878611374
@eprog
\bigskip

%SECTION: number_theoretical

\section{Functions related to elliptic curves}

We have implemented a number of functions which are useful for number
theorists working on elliptic curves. We always use \idx{Tate}'s notations.
The functions assume that the curve is given by a general Weierstrass
model\sidx{Weierstrass equation}
$$
  y^2+a_1xy+a_3y=x^3+a_2x^2+a_4x+a_6,
$$
where a priori the $a_i$ can be of any scalar type. This curve can be
considered as a five-component vector \kbd{E=[a1,a2,a3,a4,a6]}. Points on
\kbd{E} are represented as two-component vectors \kbd{[x,y]}, except for the
point at infinity, i.e.~the identity element of the group law, represented by
the one-component vector \kbd{[0]}.

  It is useful to have at one's disposal more information. This is given by
the function \tet{ellinit}, which initializes and returns an
\tev{ell} structure by default. If a specific flag is added, a
shortened \tev{smallell} is returned, which is much
faster to compute but contains less information. The following \idx{member
functions} are available to deal with the output of \kbd{ellinit},
both \var{ell} and \var{smallell}:
\settabs\+xxxxxxxxxxxxxxxxxx&: &\cr

\+ \kbd{a1}--\kbd{a6}, \kbd{b2}--\kbd{b8}, \kbd{c4}--\kbd{c6} &: &
coefficients of the elliptic curve.\cr

\+ \tet{area} &: &  volume of the complex lattice defining $E$.\cr

\+ \tet{disc} &: & discriminant of the curve.\cr

\+ \tet{j}    &: & $j$-invariant of the curve.\cr

\+ \tet{omega}&: & $[\omega_1,\omega_2]$, periods forming a basis of
the complex lattice defining $E$ ($\omega_1$ is the\cr

\+            &   & real period, and $\omega_1/\omega_2$ belongs to
Poincar\'e's half-plane).\cr

\+ \tet{eta}  &: & quasi-periods $[\eta_1, \eta_2]$, such that
$\eta_1\omega_2-\eta_2\omega_1=2i\pi$.\cr

\+ \tet{roots}&: & roots of the associated Weierstrass equation.\cr

\+ \tet{tate} &: & $[u^2,u,v]$ in the notation of Tate.\cr

\+ \tet{w} &: & Mestre's $w$ (this is technical).\cr

\noindent
{\bf Warning:} as for the orientation of the basis of the period lattice,
beware that many sources use the inverse convention where $\omega_2/\omega_1$
has positive imaginary part and our $\omega_2$ is the negative of theirs. Our
convention $\tau = \omega_1/\omega_2$  ensures that the action of $\text{PSL}_2$ is the natural
one:
$$[a,b;c,d]\cdot\tau = (a\tau+b)/(c\tau+d)
  = (a \omega_1 + b\omega_2)/(c\omega_1 + d\omega_2),$$
instead of a twisted one. (Our $tau$ is $-1/\tau$ in the above inverse
convention.)

The member functions \kbd{area}, \kbd{eta} and \kbd{omega} are only available
for curves over $\Q$. Conversely, \kbd{tate} and \kbd{w} are only available
for curves defined over $\Q_p$. The use of member functions is best described
by an example:
\bprog
  ? E = ellinit([0,1]); \\@com The curve $y^2 = x^3 + 0.x + 1$
  ? E.a6
  %2 = 1
  ? E.c6
  %3 = -864
  ? E.disc
  %4 = -432
@eprog
\smallskip

Some functions, in particular those relative to height computations (see
\kbd{ellheight}) require also that the curve be in minimal Weierstrass
form, which is duly stressed in their description below. This is achieved by
the function \kbd{ellminimalmodel}. \emph{Using a non-minimal model in such a
routine will yield a wrong result!}

All functions related to elliptic curves share the prefix \kbd{ell}, and the
precise curve we are interested in is always the first argument, in either
one of the three formats discussed above, unless otherwise specified. The
requirements are given as the \emph{minimal} ones: any richer structure may
replace the ones requested. For instance, in functions which have no use for
the extra information given by an \tev{ell} structure, the curve can be given
either as a five-component vector, as a \var{smallell}, or as an \var{ell};
if a \var{smallell} is requested, an \var{ell} may equally be given.


A few routines --- namely \tet{ellgenerators}, \tet{ellidentify},
\tet{ellsearch}, \tet{forell} --- require the optional package \tet{elldata}
(John Cremona's database) to be installed. The function \tet{ellinit} will
also allow alternative inputs, e.g.~\kbd{ellinit("11a1")}. Functions using
this package need to load chunks of a large database in memory and require
at least 2MB stack to avoid stack overflows.

%SECTION: elliptic_curves

\section{Functions related to general number fields}

In this section can be found functions which are used almost exclusively for
working in general number fields. Other less specific functions can be found
in the next section on polynomials. Functions related to quadratic number
fields are found in section \secref{se:arithmetic} (Arithmetic functions).

\subsec{Number field structures}

Let $K = \Q[X] / (T)$ a number field, $\Z_K$ its ring of integers, $T\in\Z[X]$
is monic. Three basic number field structures can be associated to $K$ in
GP:

\item $\tev{nf}$ denotes a number field, i.e.~a data structure output by
\tet{nfinit}. This contains the basic arithmetic data associated to the
number field: signature, maximal order (given by a basis \kbd{nf.zk}),
discriminant, defining polynomial $T$, etc.

\item $\tev{bnf}$ denotes a ``Buchmann's number field'', i.e.~a
data structure output by \tet{bnfinit}. This contains
$\var{nf}$ and the deeper invariants of the field: units $U(K)$, class group
$\Cl(K)$, as well as technical data required to solve the two associated
discrete logarithm problems.

\item $\tev{bnr}$ denotes a ``ray number field'', i.e.~a data structure
output by \kbd{bnrinit}, corresponding to the ray class group structure of
the field, for some modulus $f$. It contains a \var{bnf}, the modulus
$f$, the ray class group $\Cl_f(K)$ and data associated to
the discrete logarithm problem therein.

\subsec{Algebraic numbers and ideals}

\noindent An \tev{algebraic number} belonging to $K = \Q[X]/(T)$ is given as

\item a \typ{INT}, \typ{FRAC} or \typ{POL} (implicitly modulo $T$), or

\item a \typ{POLMOD} (modulo $T$), or

\item a \typ{COL}~\kbd{v} of dimension $N = [K:\Q]$, representing
the element in terms of the computed integral basis, as
\kbd{sum(i = 1, N,~v[i] * nf.zk[i])}. Note that a \typ{VEC}
will not be recognized.
\medskip

\noindent An \tev{ideal} is given in any of the following ways:

\item an algebraic number in one of the above forms, defining a principal ideal.

\item a prime ideal, i.e.~a 5-component vector in the format output by
\kbd{idealprimedec} or \kbd{idealfactor}.

\item a \typ{MAT}, square and in Hermite Normal Form (or at least
upper triangular with non-negative coefficients), whose columns represent a
$\Z$-basis of the ideal.

One may use \kbd{idealhnf} to convert any ideal to the last (preferred) format.

\item an \emph{extended ideal} \sidx{ideal (extended)} is a 2-component
vector $[I, t]$, where $I$ is an ideal as above and $t$ is an algebraic
number, representing the ideal $(t)I$. This is useful whenever \tet{idealred}
is involved, implicitly working in the ideal class group, while keeping track
of principal ideals. Ideal operations suitably update the principal part
when it makes sense (in a multiplicative context), e.g.~using \kbd{idealmul}
on $[I,t]$, $[J,u]$, we obtain $[IJ, tu]$. When it does not make sense, the
extended part is silently discarded, e.g.~using \kbd{idealadd} with the above
input produces $I+J$.

The ``principal part'' $t$ in an extended ideal may be
represented in any of the above forms, and \emph{also} as a factorization
matrix (in terms of number field elements, not ideals!), possibly the empty
matrix \kbd{[;]} representing $1$. In the latter case, elements stay in
factored form, or \tev{famat} for \emph{fa}ctorization \emph{mat}rix, which
is a convenient way to avoid coefficient explosion. To recover the
conventional expanded form, try \tet{nffactorback}; but many functions
already accept \var{famat}s as input, for instance \tet{ideallog}, so
expanding huge elements should never be necessary.

\subsec{Finite abelian groups}

A finite abelian group $G$ in user-readable format is given by its Smith
Normal Form as a pair $[h,d]$ or triple $[h,d,g]$.
Here $h$ is the cardinality of $G$, $(d_i)$ is the vector of elementary
divisors, and $(g_i)$ is a vector of generators. In short,
$G = \oplus_{i\leq n} (\Z/d_i\Z) g_i$, with $d_n \mid \dots \mid d_2 \mid d_1$
and $\prod d_i = h$. This information can also be retrieved as
$G.\kbd{no}$, $G.\kbd{cyc}$ and $G.\kbd{gen}$.

\item a \tev{character} on the abelian group
$\oplus (\Z/d_i\Z) g_i$
is given by a row vector $\chi = [a_1,\ldots,a_n]$ such that
$\chi(\prod g_i^{n_i}) = \exp(2i\pi\sum a_i n_i / d_i)$.

\item given such a structure, a \tev{subgroup} $H$ is input as a square
matrix in HNF, whose columns express generators of $H$ on the given generators
$g_i$. Note that the determinant of that matrix is equal to the index $(G:H)$.

\subsec{Relative extensions}

We now have a look at data structures associated to relative extensions
of number fields $L/K$, and to projective $\Z_K$-modules. When defining a
relative extension $L/K$, the $\var{nf}$ associated to the base field $K$
must be defined by a variable having a lower priority (see
\secref{se:priority}) than the variable defining the extension. For example,
you may use the variable name $y$ to define the base field $K$, and $x$ to
define the relative extension $L/K$.

\subsubsec{Basic definitions}\label{se:ZKmodules}

\item $\tev{rnf}$ denotes a relative number field, i.e.~a data structure
output by \kbd{rnfinit}, associated to the extension $L/K$. The \var{nf}
associated to be base field $K$ is \kbd{rnf.nf}.

\item A \emph{relative matrix} is an $m\times n$ matrix whose entries are
elements of $K$, in any form. Its $m$ columns $A_j$ represent elements
in $K^n$.

\item An \tev{ideal list} is a row vector of fractional ideals of the number
field $\var{nf}$.

\item A \tev{pseudo-matrix} is a 2-component row vector $(A,I)$ where $A$
is a relative $m\times n$ matrix and $I$ an ideal list of length $n$. If $I =
\{{\Bbb a}_1,\dots, {\Bbb a}_n\}$ and the columns of $A$ are $(A_1,\dots,
A_n)$, this data defines the torsion-free (projective) $\Z_K$-module ${\Bbb
a}_1 A_1\oplus {\Bbb a}_n A_n$.

\item An \tev{integral pseudo-matrix} is a 3-component row vector w$(A,I,J)$
where $A = (a_{i,j})$ is an $m\times n$ relative matrix and $I = ({\Bbb
b}_1,\dots, {\Bbb b}_m)$, $J = ({\Bbb a}_1,\dots, {\Bbb a}_n)$ are ideal
lists, such that $a_{i,j} \in {\Bbb b}_i {\Bbb a}_j^{-1}$ for all $i,j$. This
data defines two abstract projective $\Z_K$-modules $N = {\Bbb
a_1}\omega_1\oplus \cdots\oplus {\Bbb a_n}\omega_n $ in $K^n$, $P = {\Bbb
b_1}\eta_1\oplus \cdots\oplus {\Bbb b_m}\eta_m$ in $K^m$, and a $\Z_K$-linear
map $f:N\to P$ given by
$$ f(\sum \alpha_j\omega_j) = \sum_i \Big(a_{i,j}\alpha_j\Big) \eta_i.$$
This data defines the $\Z_K$-module $M = P/f(N)$.

\item Any \emph{projective} $\Z_K$-module\varsidx{projective module} $M$
of finite type in $K^m$ can be given by a pseudo matrix $(A,I)$.

\item An arbitrary $\Z_K$ modules of finite type in $K^m$, with non-trivial
torsion, is given by an integral pseudo-matrix $(A,I,J)$

\subsubsec{Pseudo-bases, determinant}

\item The pair $(A,I)$ is a \tev{pseudo-basis} of the module it
generates if the ${\Bbb a_j}$ are non-zero, and the $A_j$ are $K$-linearly
independent. We call $n$ the \emph{size} of the pseudo-basis. If $A$ is a
relative matrix, the latter condition means it is square with non-zero
determinant; we say that it is in Hermite Normal Form\sidx{Hermite normal
form} (HNF) if it is upper triangular and all the elements of the diagonal
are equal to 1.

\item For instance, the relative integer basis \kbd{rnf.zk} is a pseudo-basis
$(A,I)$ of $\Z_L$, where $A = \kbd{rnf.zk[1]}$ is a vector of elements of $L$,
which are $K$-linearly independent. Most \var{rnf} routines return and handle
$\Z_K$-modules contained in $L$ (e.g.~$\Z_L$-ideals) via a pseudo-basis
$(A',I')$, where $A'$ is a relative matrix representing a vector of elements of
$L$ in terms of the fixed basis \kbd{rnf.zk[1]}

\item The \emph{determinant} of a pseudo-basis $(A,I)$ is the ideal
equal to the product of the determinant of $A$ by all the ideals of $I$. The
determinant of a pseudo-matrix is the determinant of any pseudo-basis of the
module it generates.

\subsec{Class field theory}\label{se:CFT}

A $\tev{modulus}$, in the sense of class field theory, is a divisor supported
on the non-complex places of $K$. In PARI terms, this means either an
ordinary ideal $I$ as above (no Archimedean component), or a pair $[I,a]$,
where $a$ is a vector with $r_1$ $\{0,1\}$-components, corresponding to the
infinite part of the divisor. More precisely, the $i$-th component of $a$
corresponds to the real embedding associated to the $i$-th real root of
\kbd{K.roots}. (That ordering is not canonical, but well defined once a
defining polynomial for $K$ is chosen.) For instance, \kbd{[1, [1,1]]} is a
modulus for a real quadratic field, allowing ramification at any of the two
places at infinity, and nowhere else.

A \tev{bid} or ``big ideal'' is a structure output by \kbd{idealstar}
needed to compute in $(\Z_K/I)^*$, where $I$ is a modulus in the above sense.
It is a finite abelian group as described above, supplemented by
technical data needed to solve discrete log problems.

Finally we explain how to input ray number fields (or \var{bnr}), using class
field theory. These are defined by a triple $A$, $B$, $C$, where the
defining set $[A,B,C]$ can have any of the following forms: $[\var{bnr}]$,
$[\var{bnr},\var{subgroup}]$, $[\var{bnf},\var{mod}]$,
$[\var{bnf},\var{mod},\var{subgroup}]$. The last two forms are kept for
backward compatibility, but no longer serve any real purpose (see example
below); no newly written function will accept them.

\item $\var{bnf}$ is as output by \kbd{bnfinit}, where units are mandatory
unless the modulus is trivial; \var{bnr} is as output by \kbd{bnrinit}. This
is the ground field $K$.

\item \emph{mod} is a modulus $\goth{f}$, as described above.

\item \emph{subgroup} a subgroup of the ray class group modulo $\goth{f}$ of
$K$. As described above, this is input as a square matrix expressing
generators of a subgroup of the ray class group \kbd{\var{bnr}.clgp} on the
given generators.

The corresponding \var{bnr} is the subfield of the ray class field of $K$
modulo $\goth{f}$, fixed by the given subgroup.

\bprog
  ? K = bnfinit(y^2+1);
  ? bnr = bnrinit(K, 13)
  ? %.clgp
  %3 = [36, [12, 3]]
  ? bnrdisc(bnr); \\ discriminant of the full ray class field
  ? bnrdisc(bnr, [3,1;0,1]); \\ discriminant of cyclic cubic extension of K
@eprog\noindent
We could have written directly
\bprog
  ? bnrdisc(K, 13);
  ? bnrdisc(K, 13, [3,1;0,1]);
@eprog\noindent
avoiding one \tet{bnrinit}, but this would actually be slower since the
\kbd{bnrinit} is called internally anyway. And now twice!

\subsec{General use}

All the functions which are specific to relative extensions, number fields,
Buchmann's number fields, Buchmann's number rays, share the prefix \kbd{rnf},
\kbd{nf}, \kbd{bnf}, \kbd{bnr} respectively. They take as first argument a
number field of that precise type, respectively output by \kbd{rnfinit},
\kbd{nfinit}, \kbd{bnfinit}, and \kbd{bnrinit}.

However, and even though it may not be specified in the descriptions of the
functions below, it is permissible, if the function expects a $\var{nf}$, to
use a $\var{bnf}$ instead, which contains much more information. On the other
hand, if the function requires a \kbd{bnf}, it will \emph{not} launch
\kbd{bnfinit} for you, which is a costly operation. Instead, it will give you
a specific error message. In short, the types
$$ \kbd{nf} \leq \kbd{bnf} \leq \kbd{bnr}$$
are ordered, each function requires a minimal type to work properly, but you
may always substitute a larger type.

The data types corresponding to the structures described above are rather
complicated. Thus, as we already have seen it with elliptic curves, GP
provides ``member functions'' to retrieve data from these structures (once
they have been initialized of course). The relevant types of number fields
are indicated between parentheses: \smallskip

\sidx{member functions}
\settabs\+xxxxxxx&(\var{bnr},x&\var{bnf},x&nf\hskip2pt&)x&: &\cr
\+\tet{bid}    &(\var{bnr}&&&)&: & bid ideal structure.\cr

\+\tet{bnf}    &(\var{bnr},& \var{bnf}&&)&: & Buchmann's number field.\cr

\+\tet{clgp}  &(\var{bnr},& \var{bnf}&&)&: & classgroup. This one admits the
following three subclasses:\cr

\+      \quad \tet{cyc} &&&&&: & \quad cyclic decomposition
 (SNF)\sidx{Smith normal form}.\cr

\+      \quad \kbd{gen}\sidx{gen (member function)} &&&&&: &
 \quad generators.\cr

\+      \quad \tet{no}  &&&&&: & \quad number of elements.\cr

\+\tet{diff}  &(\var{bnr},& \var{bnf},& \var{nf}&)&: & the different ideal.\cr

\+\tet{codiff}&(\var{bnr},& \var{bnf},& \var{nf}&)&: & the codifferent
(inverse of the different in the ideal group).\cr

\+\tet{disc} &(\var{bnr},& \var{bnf},& \var{nf}&)&: & discriminant.\cr

\+\tet{fu}   &(\var{bnr},& \var{bnf}&&)&: & \idx{fundamental units}.\cr

\+\tet{index}   &(\var{bnr},& \var{bnf},& \var{nf}&)&: &
 \idx{index} of the power order in the ring of integers.\cr

\+\tet{mod}   &(\var{bnr}&&&)&: & modulus.\cr

\+\tet{nf}   &(\var{bnr},& \var{bnf},& \var{nf}&)&: & number field.\cr

\+\tet{pol}   &(\var{bnr},& \var{bnf},& \var{nf}&)&: & defining polynomial.\cr

\+\tet{r1} &(\var{bnr},& \var{bnf},& \var{nf}&)&: & the number
of real embeddings.\cr

\+\tet{r2} &(\var{bnr},& \var{bnf},& \var{nf}&)&: & the number
of pairs of complex embeddings.\cr

\+\tet{reg}  &(\var{bnr},& \var{bnf}&&)&: & regulator.\cr

\+\tet{roots}&(\var{bnr},& \var{bnf},& \var{nf}&)&: & roots of the
polynomial generating the field.\cr

\+\tet{sign} &(\var{bnr},& \var{bnf},& \var{nf}&)&: & signature $[r1,r2]$.\cr

\+\tet{t2}   &(\var{bnr},& \var{bnf},& \var{nf}&)&: & the $T_2$ matrix (see
\kbd{nfinit}).\cr

\+\tet{tu}   &(\var{bnr},& \var{bnf}&&)&: & a generator for the torsion
units.\cr

\+\tet{zk}   &(\var{bnr},& \var{bnf},& \var{nf}&)&: & integral basis, i.e.~a
$\Z$-basis of the maximal order.\cr

\+\tet{zkst}   &(\var{bnr}&&&)&: & structure of $(\Z_K/m)^*$.\cr

\misctitle{Deprecated} The following member functions are still available, but deprecated and should not be used in new scripts :
\+\tet{futu} &(\var{bnr},& \var{bnf},&&)&: &
 $[u_1,...,u_r,w]$, $(u_i)$ is a vector of fundamental units,\cr
\+&&&&&& $w$ generates the torsion units.\cr

\+\tet{tufu} &(\var{bnr},& \var{bnf},&&)&: &
 $[w,u_1,...,u_r]$, $(u_i)$ is a vector of fundamental units,\cr
\+&&&&&& $w$ generates the torsion units.\cr


  For instance, assume that $\var{bnf} = \kbd{bnfinit}(\var{pol})$, for some
polynomial. Then \kbd{\var{bnf}.clgp} retrieves the class group, and
\kbd{\var{bnf}.clgp.no} the class number. If we had set $\var{bnf} =
\kbd{nfinit}(\var{pol})$, both would have output an error message. All these
functions are completely recursive, thus for instance
\kbd{\var{bnr}.bnf.nf.zk} will yield the maximal order of \var{bnr}, which
you could get directly with a simple \kbd{\var{bnr}.zk}.

\subsec{Class group, units, and the GRH}\label{se:GRHbnf}

Some of the functions starting with \kbd{bnf} are implementations of the
sub-exponential algorithms for finding class and unit groups under \idx{GRH},
due to Hafner-McCurley, \idx{Buchmann} and Cohen-Diaz-Olivier. The general
call to the functions concerning class groups of general number fields
(i.e.~excluding \kbd{quadclassunit}) involves a polynomial $P$ and a
technical vector
$$\var{tech} = [c_1, c_2, \var{nrpid} ],$$
where the parameters are to be understood as follows:

$P$ is the defining polynomial for the number field, which must be in
$\Z[X]$, irreducible and monic. In fact, if you supply a non-monic polynomial
at this point, \kbd{gp} issues a warning, then \emph{transforms your
polynomial} so that it becomes monic. The \kbd{nfinit} routine
will return a different result in this case: instead of \kbd{res}, you get a
vector \kbd{[res,Mod(a,Q)]}, where \kbd{Mod(a,Q) = Mod(X,P)} gives the change
of variables. In all other routines, the variable change is simply lost.

The \var{tech} interface is obsolete and you should not tamper with
these parameters. Indeed, from version 2.4.0 on,

\item the results are always rigorous under \idx{GRH} (before that version,
they relied on a heuristic strengthening, hence the need for overrides).

\item the influence of these parameters on execution time and stack size is
marginal. They \emph{can} be useful to fine-tune and experiment with the
\kbd{bnfinit} code, but you will be better off modifying all tuning
parameters in the C code (there are many more than just those three).
We nevertheless describe it for completeness.

The numbers $c_1 \leq c_2$ are positive real numbers. For $i = 1,2$, let $B_i
= c_i(\log |d_K|)^2$, and denote by $S(B)$ the set of maximal ideals of $K$
whose norm is less than $B$. We want $S(B_1)$ to generate $\Cl(K)$ and hope
that $S(B_2)$ can be \emph{proven} to generate $\Cl(K)$.

More precisely, $S(B_1)$ is a factorbase used to compute a tentative
$\Cl(K)$ by generators and relations. We then check explicitly, using
essentially \kbd{bnfisprincipal}, that the elements of $S(B_2)$ belong to the
span of $S(B_1)$. Under the assumption that $S(B_2)$ generates $\Cl(K)$, we
are done. User-supplied $c_i$ are only used to compute initial guesses for
the bounds $B_i$, and the algorithm increases them until one can \emph{prove}
under GRH that $S(B_2)$ generates $\Cl(K)$. A uniform result of Bach says
that $c_2 = 12$ is always suitable, but this bound is very pessimistic and a
direct algorithm due to Belabas-Diaz-Friedman is used to check the condition,
assuming GRH. The default values are $c_1 = c_2 = 0.3$.

$\var{nrpid}$ is the maximal number of small norm relations associated to each
ideal in the factor base. Set it to $0$ to disable the search for small norm
relations. Otherwise, reasonable values are between 4 and 20. The default is
4.

\misctitle{Warning} Make sure you understand the above! By default, most of
the \kbd{bnf} routines depend on the correctness of the GRH. In particular,
any of the class number, class group structure, class group generators,
regulator and fundamental units may be wrong, independently of each other.
Any result computed from such a \kbd{bnf} may be wrong. The only guarantee is
that the units given generate a subgroup of finite index in the full unit
group. You must use \kbd{bnfcertify} to certify the computations
unconditionally.

\misctitle{Remarks}

You do not need to supply the technical parameters (under the library you
still need to send at least an empty vector, coded as \kbd{NULL}). However,
should you choose to set some of them, they \emph{must} be given in the
requested order. For example, if you want to specify a given value of
\var{nrpid}, you must give some values as well for $c_1$ and $c_2$, and provide
a vector $[c_1,c_2,\var{nrpid}]$.

Note also that you can use an $\var{nf}$ instead of $P$, which avoids
recomputing the integral basis and analogous quantities.

\smallskip

%SECTION: number_fields

\section{Polynomials and power series}

We group here all functions which are specific to polynomials or power
series. Many other functions which can be applied on these objects are
described in the other sections. Also, some of the functions described here
can be applied to other types.

%SECTION: polynomials

\section{Vectors, matrices, linear algebra and sets}
\label{se:linear_algebra}

Note that most linear algebra functions operating on subspaces defined by
generating sets (such as \tet{mathnf}, \tet{qflll}, etc.) take matrices as
arguments. As usual, the generating vectors are taken to be the
\emph{columns} of the given matrix.

Since PARI does not have a strong typing system, scalars live in
unspecified commutative base rings. It is very difficult to write
robust linear algebra routines in such a general setting. We thus
assume that the base ring is a domain and work over its field of
fractions. If the base ring is \emph{not} a domain, one gets an error as soon
as a non-zero pivot turns out to be non-invertible. Some functions,
e.g.~\kbd{mathnf} or \kbd{mathnfmod}, specifically assume that the base ring is
$\Z$.

%SECTION: linear_algebra

\section{Sums, products, integrals and similar functions}
\label{se:sums}

Although the \kbd{gp} calculator is programmable, it is useful to have
a number of preprogrammed loops, including sums, products, and a certain
number of recursions. Also, a number of functions from numerical analysis
like numerical integration and summation of series will be described here.

One of the parameters in these loops must be the control variable, hence a
simple variable name. In the descriptions, the letter $X$ will always denote
any simple variable name, and represents the formal parameter used in the
function. The expression to be summed, integrated, etc. is any legal PARI
expression, including of course expressions using loops.

\misctitle{Library mode}
Since it is easier to program directly the loops in library mode, these
functions are mainly useful for GP programming. On the other hand, numerical
routines code a function (to be integrated, summed, etc.) with two parameters
named
\bprog
  GEN (*eval)(void*,GEN)
  void *E;  \\ context: eval(E, x) must evaluate your function at x.
@eprog\noindent
see the Libpari manual for details.

\misctitle{Numerical integration}\sidx{numerical integration}
Starting with version 2.2.9 the ``double exponential'' univariate
integration method is implemented in \tet{intnum} and its variants. Romberg
integration is still available under the name \kbd{intnumromb}, but
superseded. It is possible to compute numerically integrals to thousands of
decimal places in reasonable time, as long as the integrand is regular. It is
also reasonable to compute numerically integrals in several variables,
although more than two becomes lengthy. The integration domain may be
non-compact, and the integrand may have reasonable singularities at
endpoints. To use \kbd{intnum}, you must split the integral into a sum
of subintegrals where the function has no singularities except at the
endpoints. Polynomials in logarithms are not considered singular, and
neglecting these logs, singularities are assumed to be algebraic (asymptotic
to $C(x-a)^{-\alpha}$ for some $\alpha > -1$ when $x$ is
close to $a$), or to correspond to simple discontinuities of some (higher)
derivative of the function. For instance, the point $0$ is a singularity of
$\text{abs}(x)$.

See also the discrete summation methods below, sharing the prefix \kbd{sum}.

%SECTION: sums

\section{Plotting functions}

  Although plotting is not even a side purpose of PARI, a number of plotting
functions are provided. Moreover, a lot of people
\footnote{*}{Among these, special thanks go to Klaus-Peter Nischke who
suggested the recursive plotting and forking/resizing stuff the graphical
window, and Ilya Zakharevich who rewrote the graphic code from scratch
implementing many new primitives (splines, clipping). Nils Skoruppa and Bill
Allombert wrote the \tet{Qt} and \tet{fltk} graphic drivers respectively.}
suggested ideas or submitted patches for this section of the code. There are
three types of graphic functions.

\subsec{High-level plotting functions} (all the functions starting with
\kbd{ploth}) in which the user has little to do but explain what type of plot
he wants, and whose syntax is similar to the one used in the preceding
section.

\subsec{Low-level plotting functions} (called \var{rectplot} functions,
sharing the prefix \kbd{plot}), where every drawing primitive (point, line,
box, etc.) is specified by the user. These low-level functions work as
follows. You have at your disposal 16 virtual windows which are filled
independently, and can then be physically ORed on a single window at
user-defined positions. These windows are numbered from 0 to 15, and must be
initialized before being used by the function \kbd{plotinit}, which specifies
the height and width of the virtual window (called a \var{rectwindow} in the
sequel). At all times, a virtual cursor (initialized at $[0,0]$) is associated
to the window, and its current value can be obtained using the function
\kbd{plotcursor}.

A number of primitive graphic objects (called \var{rect} objects) can then
be drawn in these windows, using a default color associated to that window
(which can be changed using the \kbd{plotcolor} function) and only the part
of the object which is inside the window will be drawn, with the exception of
polygons and strings which are drawn entirely. The ones sharing the prefix
\kbd{plotr} draw relatively to the current position of the virtual cursor,
the others use absolute coordinates. Those having the prefix \kbd{plotrecth}
put in the rectwindow a large batch of rect objects corresponding to the
output of the related \kbd{ploth} function.

   Finally, the actual physical drawing is done using \kbd{plotdraw}. The
rectwindows are preserved so that further drawings using the same windows at
different positions or different windows can be done without extra work. To
erase a window, use \kbd{plotkill}. It is not possible to partially erase a
window: erase it completely, initialize it again, then fill it with the
graphic objects that you want to keep.

   In addition to initializing the window, you may use a scaled window to
avoid unnecessary conversions. For this, use \kbd{plotscale}. As long as this
function is not called, the scaling is simply the number of pixels, the
origin being at the upper left and the $y$-coordinates going downwards.

   Plotting functions are platform independent, but a number of graphical
drivers are available for screen output: X11-windows (hence also for GUI's
based on X11 such as Openwindows and Motif), and the Qt and FLTK graphical
libraries. The physical window opened by \kbd{plotdraw} or any of the
\kbd{ploth*} functions is completely separated from \kbd{gp} (technically, a
\kbd{fork} is done, and the non-graphical memory is immediately freed in the
child process), which means you can go on working in the current \kbd{gp}
session, without having to kill the window first. This window can be closed,
enlarged or reduced using the standard window manager functions. No zooming
procedure is implemented though (yet).

\subsec{Functions for PostScript output} in the same way that \kbd{printtex} allows you to have a \TeX\ output
corresponding to printed results, the functions starting with \kbd{ps} allow
you to have \tet{PostScript} output of the plots. This will not be identical
with the screen output, but sufficiently close. Note that you can use
PostScript output even if you do not have the plotting routines enabled. The
PostScript output is written in a file whose name is derived from the
\tet{psfile} default (\kbd{./pari.ps} if you did not tamper with it). Each
time a new PostScript output is asked for, the PostScript output is appended
to that file. Hence you probably want to remove this file, or change the
value of \kbd{psfile}, in between plots. On the other hand, in this manner,
as many plots as desired can be kept in a single file. \smallskip

\subsec{Library mode} \emph{None of the graphic functions are available
within the PARI library, you must be under \kbd{gp} to use them}. The reason
for that is that you really should not use PARI for heavy-duty graphical work,
there are better specialized alternatives around. This whole set of routines
was only meant as a convenient, but simple-minded, visual aid. If you really
insist on using these in your program (we warned you), the source
(\kbd{plot*.c}) should be readable enough for you to achieve something.

%SECTION: graphic

\section{Programming in GP: control statements}
\sidx{programming}\label{se:programming}

  A number of control statements are available in GP. They are simpler and
have a syntax slightly different from their C counterparts, but are quite
powerful enough to write any kind of program. Some of them are specific to
GP, since they are made for number theorists. As usual, $X$ will denote any
simple variable name, and \var{seq} will always denote a sequence of
expressions, including the empty sequence.

\misctitle{Caveat} In constructs like
\bprog
    for (X = a,b, seq)
@eprog\noindent
the variable \kbd{X} is lexically scoped to the loop, leading to possibly
unexpected behavior:
\bprog
    n = 5;
    for (n = 1, 10,
      if (something_nice(), break);
    );
    \\ @com at this point \kbd{n} is 5 !
@eprog\noindent
If the sequence \kbd{seq} modifies the loop index, then the loop
is modified accordingly:
\bprog
    ? for (n = 1, 10, n += 2; print(n))
    3
    6
    9
    12
@eprog

%SECTION: programming/control

\section{Programming in GP: other specific functions}
\label{se:gp_program}

  In addition to the general PARI functions, it is necessary to have some
functions which will be of use specifically for \kbd{gp}, though a few of these can
be accessed under library mode. Before we start describing these, we recall
the difference between \emph{strings} and \emph{keywords} (see
\secref{se:strings}): the latter don't get expanded at all, and you can type
them without any enclosing quotes. The former are dynamic objects, where
everything outside quotes gets immediately expanded.

%SECTION: programming/specific

\section{GP defaults}
\label{se:gp_defaults} This section documents the GP defaults

%SECTION: default

\vfill\eject
