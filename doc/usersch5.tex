% $Id$
% Copyright (c) 2000  The PARI Group
%
% This file is part of the PARI/GP documentation
%
% Permission is granted to copy, distribute and/or modify this document
% under the terms of the GNU General Public License
\chapter{Technical Reference Guide for Low-Level Functions}

In this chapter, we describe all public low-level functions of the
PARI library. These essentially include functions for handling all the PARI
types. Higher level functions, such as arithmetic or transcendental
functions, are described in Chapter~3 of the GP user's manual.

Many other undocumented functions can be found throughout the source code.
These private functions are more efficient than the library wrappers, but
sloppier on argument checking and damage control. Use them at your own risk!

\misctitle{Important advice}: generic routines eventually call lower level
functions. Optimize your algorithms first, not overhead and conversion costs
between PARI routines. For generic operations, use generic routines first,
don't waste time looking for the most specialized one available unless you
identify a genuine bottleneck. The PARI source code is part of the
documentation; look for inspiration there.\smallskip

As in the previous chapter, we let \B\ abbreviate \tet{BITS_IN_LONG}. The
type \kbd{long} denotes a \B-bit signed long integer. The type \tet{ulong} is
defined as \kbd{unsigned long}. The word \emph{stack} always refer to the
PARI stack, allocated through an initial \kbd{pari\_init} call.
Refer to Chapters 1--2 and~4 for general background.
\kbdsidx{BIL}
\section{Handling \kbd{GEN}s}
\noindent Almost all these functions are either macros or inlined. Unless
mentioned otherwise, they do not evaluate their arguments twice. Most of them
are specific to a set of types, although no consistency checks are made:
e.g.~one may access the \kbd{sign} of a \typ{PADIC}, but the result is
meaningless.

\subsec{Read type-dependent information}

\fun{long}{typ}{GEN x} returns the type number of~\kbd{x}. The header files
included through \kbd{pari.h} define symbolic constants for the \kbd{GEN}
types: \typ{INT} etc. Never use their actual numerical values. E.g to determine
whether \kbd{x} is a \typ{INT}, simply check
\bprog
  if (typ(x) == t_INT) { }
@eprog\noindent
The types are internally ordered and this simplifies the implementation of
commutative binary operations (e.g addition, gcd). Avoid using the ordering
directly, as it may change in the future; use type grouping macros
instead (\secref{se:typegroup}).

\fun{long}{lg}{GEN x} returns the length of~\kbd{x} in \B-bit words.

\fun{long}{lgefint}{GEN x} returns the effective length of the \typ{INT} \kbd{x}
in \B-bit words.

\fun{long}{signe}{GEN x} returns the sign ($-1$, 0 or 1) of~\kbd{x}. Can be
used for \typ{INT}, \typ{REAL}, \typ{POL} and \typ{SER} (for the last two
types, only 0 or 1 are possible).

\fun{long}{gsigne}{GEN x} same as \kbd{signe}, but also valid for \typ{FRAC}
(and marginally less efficient for the other types). Raise a type error if
\kbd{typ(x)} is not among those three.

\fun{long}{expi}{GEN x} returns the binary exponent of the real number equal
to the \typ{INT}~\kbd{x}. This is a special case of \kbd{gexpo}.

\fun{long}{expo}{GEN x} returns the binary exponent of the
\typ{REAL}~\kbd{x}.

\fun{long}{gexpo}{GEN x} same as \kbd{expo}, but also valid when \kbd{x}
is not a \typ{REAL} (returns the largest exponent found among the components
of \kbd{x}). When \kbd{x} is an exact~0, this returns
\hbox{\kbd{-HIGHEXPOBIT}}, which is lower than any valid exponent.

\fun{long}{valp}{GEN x} returns the 16-bit $p$-adic valuation (for
a \typ{PADIC}) or $X$-adic valuation (for a \typ{SER}, taken with respect to
the main variable) of~\kbd{x}.

\fun{long}{precp}{GEN x} returns the precision of the \typ{PADIC}~\kbd{x}.

\fun{long}{varn}{GEN x} returns the variable number of the
\typ{POL} or \typ{SER}~\kbd{x} (between 0 and \kbd{MAXVARN}).

\fun{long}{gvar}{GEN x} returns the main variable number when any variable
at all occurs in the composite object~\kbd{x} (the smallest variable number
which occurs), and \kbd{BIGINT} otherwise.

\fun{long}{degpol}{GEN x} returns the degree of \typ{POL}~\kbd{x},
\emph{assuming} its leading coefficient is non-zero (an exact $0$ is
impossible, but an inexact $0$ is allowed). By convention the degree of an
exact $0$ polynomial is $-1$. If the leading coefficient of \kbd{x} is $0$,
the result is undefined.

\fun{int}{precision}{GEN x} If \kbd{x} is of type \typ{REAL}, returns the
precision of~\kbd{x} (the length of \kbd{x} in \B-bit words if \kbd{x} is not
zero, and a reasonable quantity obtained from the exponent of \kbd{x} if
\kbd{x} is numerically equal to zero). If \kbd{x} is of type \typ{COMPLEX},
returns the minimum of the precisions of the real and imaginary part.
Otherwise, returns~0 (which stands in fact for infinite precision).

\fun{int}{gprecision}{GEN x} as \kbd{precision} for scalars; returns the
lowest precision encountered among the components otherwise.

\fun{long}{sizedigit}{GEN x} returns 0 if \kbd{x} is exactly~0. Otherwise,
returns \kbd{\key{gexpo}(x)} multiplied by $\log_{10}(2)$. This gives a crude
estimate for the maximal number of decimal digits of the components
of~\kbd{x}.

\subsec{Eval type-dependent information}.
These routines convert type-dependant information to bitmask to fill the
codewords of \kbd{GEN} objects (see \secref{se:impl}). E.g for a
\typ{REAL}~\kbd{z}:
\bprog
  z[1] = evalsigne(-1) | evalexpo(2)
@eprog
Compatible components of a codeword for a given type can be OR-ed as above.

\fun{ulong}{evaltyp}{long x} convert type~\kbd{x} to bitmask (first
codeword of all \kbd{GEN}s)

\fun{long}{evallg}{long x} convert length~\kbd{x} to bitmask (first
codeword of all \kbd{GEN}s). Raise overflow error if \kbd{x} is so large that
the corresponding length cannot be represented

\fun{long}{_evallg}{long x} as \kbd{evallg} \emph{without} the overflow
check.

\fun{ulong}{evalvarn}{long x} convert variable number~\kbd{x} to bitmask
(second codeword of \typ{POL} and \typ{SER})

\fun{long}{evalsigne}{long x} convert sign~\kbd{x} (in $-1,0,1$) to bitmask
(second codeword of \typ{INT}, \typ{REAL}, \typ{POL}, \typ{SER})

\fun{long}{evalprecp}{long x} convert $p$-adic ($X$-adic) precision~\kbd{x}
to bitmask (second codeword of \typ{PADIC}, \typ{SER})

\fun{long}{evalvalp}{long x} convert $p$-adic ($X$-adic) valuation~\kbd{x} to
bitmask (second codeword of \typ{PADIC}, \typ{SER}). Raise overflow error if
\kbd{x} is so large that the corresponding valuation cannot be represented

\fun{long}{_evalvalp}{long x} same as \kbd{evalvalp} \emph{without} the
overflow check.

\fun{long}{evalexpo}{long x} convert exponent~\kbd{x} to bitmask (second
codeword of \typ{REAL}). Raise overflow error if \kbd{x} is so
large that the corresponding exponent cannot be represented

\fun{long}{_evalexpo}{long x} same as \kbd{evalexpo} \emph{without} the
overflow check.

\fun{long}{evallgefint}{long x} convert effective length~\kbd{x} to bitmask
(second codeword \typ{INT}). This should be less or equal than the length
of the \typ{INT}, hence there is no overflow check for the effective length.

\fun{long}{evallgeflist}{long x} convert effective length~\kbd{x} to bitmask
(second codeword \typ{LIST}). This should be less or equal than the length of
the \typ{LIST}, hence there is no overflow check for the effective length.

\subsec{Set type-dependent information}.
Use these macros with extreme care since usually the corresponding
information is set otherwise, and the components and further codeword fields
(which are left unchanged) may not be compatible with the new information.

\fun{void}{settyp}{GEN x, long s} sets the type number of~\kbd{x} to~\kbd{s}.

\fun{void}{setlg}{GEN x, long s} sets the length of~\kbd{x} to~\kbd{s}. This
is an efficient way of truncating vectors, matrices or polynomials.

\fun{void}{setlgefint}{GEN x, long s} sets the effective length
of the \typ{INT} \kbd{x} to~\kbd{s}. The number \kbd{s} must be less than or
equal to the length of~\kbd{x}.

\fun{void}{setsigne}{GEN x, long s} sets the sign of~\kbd{x} to~\kbd{s}.
If \kbd{x} is a \typ{INT} or \typ{REAL}, \kbd{s} must be equal to $-1$, 0
or~1, and if \kbd{x} is a \typ{POL} or \typ{SER}, \kbd{s} must be equal to 0
or~1.

\fun{void}{setexpo}{GEN x, long s} sets the binary exponent of the
\typ{REAL}~\kbd{x} to \kbd{s}. The value \kbd{s} must be a 24-bit signed
number.

\fun{void}{setvalp}{GEN x, long s} sets the $p$-adic or $X$-adic valuation
of~\kbd{x} to~\kbd{s}, if \kbd{x} is a \typ{PADIC} or a \typ{SER},
respectively.

\fun{void}{setprecp}{GEN x, long s} sets the $p$-adic precision of the
\typ{PADIC}~\kbd{x} to~\kbd{s}.

\fun{void}{setvarn}{GEN x, long s} sets the variable number of the \typ{POL}
or \typ{SER}~\kbd{x} to~\kbd{s} (where $0\le \kbd{s}\le\kbd{MAXVARN}$).

\subsec{Type groups}\label{se:typegroup}.
In the following macros, \kbd{t} denotes the type of a \kbd{GEN}. 
Some of these macros may evaluate their argument twice. Always use them as in
\bprog
  long tx = typ(x);
  if (is_intreal_t(tx)) { }
@eprog

\fun{int}{is_recursive_t}{long t} \kbd{true} iff \kbd{t} is a recursive
type (the recursive types are \typ{INT}, \typ{REAL}, \typ{STR} or
\typ{VECSMALL}).

\fun{int}{is_intreal_t}{long t} \kbd{true} iff \kbd{t} is \typ{INT}
or \typ{REAL}.

\fun{int}{is_rational_t}{long t} \kbd{true} iff \kbd{t} is \typ{INT}
or \typ{FRAC}.

\fun{int}{is_vec_t}{long t} \kbd{true} iff \kbd{t} is \typ{VEC}
or \typ{COL}.

\fun{int}{is_matvec_t}{long t} \kbd{true} iff \kbd{t} is \typ{MAT}, \typ{VEC}
or \typ{COL}.

\fun{int}{is_scalar_t}{long t} \kbd{true} iff \kbd{t} is a scalar, i.e
a \typ{INT},
\typ{REAL},
\typ{INTMOD},
\typ{FRAC},
\typ{COMPLEX},
\typ{PADIC},
\typ{QUAD},
or
\typ{POLMOD}.

\fun{int}{is_extscalar_t}{long t} \kbd{true} iff \kbd{t} is a scalar (see
\kbd{is\_scalar\_t}) or \kbd{t} is \typ{POL}.

\fun{int}{is_const_t}{long t} \kbd{true} iff \kbd{t} is a scalar which is not
\typ{POLMOD}.

\subsec{Accessors and components}. The first two functions return \kbd{GEN}
components as copies on the stack:

\fun{GEN}{compo}{GEN x, long n} creates a copy of the \kbd{n}-th true
component (i.e.\ not counting the codewords) of the object~\kbd{x}.

\fun{GEN}{truecoeff}{GEN x, long n} creates a copy of the coefficient of
degree~\kbd{n} of~\kbd{x} if \kbd{x} is a scalar, \typ{POL} or \typ{SER},
and otherwise of the \kbd{n}-th component of~\kbd{x}.
\smallskip

\noindent On the contrary, the following routines return the address of a
\kbd{GEN} component. No copy is made on the stack:

\fun{GEN}{constant_term}{GEN x} returns the address the constant term of
\typ{POL}~\kbd{x}. By convention, a $0$ polynomial (whose \kbd{sign} is $0$)
has \kbd{gzero} constant term.

\fun{GEN}{leading_term}{GEN x} returns the address the leading term of
\typ{POL}~\kbd{x}. This may be an inexact $0$.

\fun{GEN}{gcoeff}{GEN x, long i, long j} returns the address of the 
\kbd{x[i,j]} entry of \typ{MAT}~\kbd{x}, i.e.~the coefficient at row~\kbd{i}
and column~\kbd{j}.

\fun{GEN}{gmael}{GEN x, long i, long j} returns the address of the 
\kbd{x[i][j]} entry of~\kbd{x}. (\kbd{mael} stands for multidimensional array
element.)

\fun{GEN}{gmael2}{GEN A, long x1, long x2} is an alias for \kbd{gmael}.
Similar macros \tet{gmael3}, \tet{gmael4}, \tet{gmael5} are available.

\section{Handling the PARI stack}

\subsec{Allocating memory on the stack}

\fun{GEN}{cgetg}{long n, long t} allocates memory on the stack for
an object of length \kbd{n} and type~\kbd{t}, and initializes its first
codeword.

\fun{GEN}{cgeti}{long n} allocates memory on the stack for a \typ{INT}
of length~\kbd{n}, and initializes its first codeword. Identical to
\kbd{cgetg(n,\typ{INT})}.

\fun{GEN}{cgetr}{long n} allocates memory on the stack for a \typ{REAL}
of length~\kbd{n}, and initializes its first codeword. Identical to
\kbd{cgetg(n,\typ{REAL})}.

\fun{GEN}{cgetc}{long n} allocates memory on the stack for a
\typ{COMPLEX}, whose real and imaginary parts are \typ{REAL}s
of length~\kbd{n}.

\fun{GEN}{cgetp}{GEN x} creates space sufficient to hold the
\typ{PADIC}~\kbd{x}, and sets the prime $p$ and the $p$-adic precision to
those of~\kbd{x}, but does not copy (the $p$-adic unit or zero representative
and the modulus of)~\kbd{x}.

\fun{GEN}{new_chunk}{size\_t n} allocates a \kbd{GEN} with $n$ components,
\emph{without} filling the required code words. This is the low-level
constructor underlying \kbd{cgetg}, which calls \kbd{new\_chunk} then sets
the first code word. It works by simply returning the address
\kbd{((GEN)avma) - n}, after checking that it is larger than \kbd{(GEN)bot}.

\fun{char*}{stackmalloc}{size\_t n} allocates memory on the stack for $n$
chars (\emph{not} $n$ \kbd{GEN}s). This is faster than using \kbd{malloc},
and easier to use in most situations when temporary storage is needed. In
particular there is no need to \kbd{free} individually all variables thus
allocated: a simple \kbd{avma = oldavma} might be enough. On the other hand,
beware that this is not permanent independant storage, but part of the stack.

\noindent Objects allocated through these last two functions cannot be
\kbd{gerepile}'d. They are not valid \kbd{GEN}s since they have no PARI type.

\subsec{Garbage collection}.
See \secref{se:garbage} for a detailed explanation and many examples.

\fun{void}{cgiv}{GEN x} frees object \kbd{x} if it is the last created on the
stack (otherwise nothing happens).

\fun{GEN}{gerepile}{pari\_sp p, pari\_sp q, GEN x} general garbage collector
for the stack.

\fun{void}{gerepileall}{pari\_sp av, int n, ...} cleans up the stack from
\kbd{av} on (i.e from \kbd{avma} to \kbd{av}), preserving the \kbd{n} objects
which follow in the argument list (of type \kbd{GEN*}). E.g:
\kbd{gerepileall(av, 2, \&x, \&y)} preserves \kbd{x} and \kbd{y}.

\fun{void}{gerepileallsp}{pari\_sp av, pari\_sp ltop, int n, ...}
cleans up the stack between \kbd{av} and \kbd{ltop}, updating
the \kbd{n} elements which follow \kbd{n} in the argument list (of type
\kbd{GEN*}). Check that the elements of \kbd{g} have no component between
\kbd{av} and \kbd{ltop}, and assumes that no garbage is present between
\kbd{avma} and \kbd{ltop}. Analogous to (but faster than) \kbd{gerepileall}
otherwise.

\fun{GEN}{gerepilecopy}{pari\_sp av, GEN x} cleans up the stack  from
\kbd{av} on, preserving the object \kbd{x}. Special case of \kbd{gerepileall}
(case $\kbd{n} = 1$), except that the routine returns the preserved \kbd{GEN} 
instead of updating its adress through a pointer.

\fun{void}{gerepilemany}{pari\_sp av, GEN* g[], int n} alternative interface
to \kbd{gerepileall}

\fun{void}{gerepilemanysp}{pari\_sp av, pari\_sp ltop, GEN* g[], int n}
alternative interface to \kbd{gerepileallsp}.

\fun{void}{gerepilecoeffs}{pari\_sp av, GEN x, int n} cleans up the stack
from \kbd{av} on, preserving \kbd{x[0]}, \dots, \kbd{x[n-1]} (which are
\kbd{GEN}s).

\fun{void}{gerepilecoeffssp}{pari\_sp av, pari\_sp ltop, GEN x, int n}
cleans up the stack from \kbd{av} to \kbd{ltop}, preserving \kbd{x[0]},
\dots, \kbd{x[n-1]} (which are \kbd{GEN}s). Same assumptions as in
\kbd{gerepilemanysp}, of which this is a variant. For instance
\bprog
  z = cgetg(3, t_COMPLEX);
  av = avma; garbage(); ltop = avma;
  z[1] = fun1();
  z[2] = fun2();
  gerepilecoeffssp(av, ltop, z + 1, 2);
  return z;
@eprog\noindent
cleans up the garbage between \kbd{av} and \kbd{ltop}, and connects \kbd{z}
and its two components. This is marginally more efficient than the standard
\bprog
  av = avma; garbage(); ltop = avma;
  z = cgetg(3, t_COMPLEX);
  z[1] = fun1();
  z[2] = fun2(); return gerepile(av, ltop, z);
@eprog\noindent

\fun{GEN}{gerepileupto}{pari\_sp av, GEN q} analogous to (but faster than)
\kbd{gerepilecopy}. Assumes that \kbd{q} is connected and that its root was
created before any component.

\fun{GEN}{gerepileuptoint}{pari\_sp av, GEN q} analogous to (but faster than)
\kbd{gerepileupto}. Assumes further that \kbd{q} is a \typ{INT}. The
length and effective length of the resulting \typ{INT} are equal.

\fun{GEN}{gerepileuptoleaf}{pari\_sp av, GEN q} analogous to (but faster than)
\kbd{gerepileupto}. Assumes further that \kbd{q} is a leaf, i.e a
non-recursive type (\kbd{is\_recursive\_t(typ(q))} is non-zero). Contrary to
\kbd{gerepileuptoint}, \kbd{gerepileuptoleaf} leaves length and effective
length of a \typ{INT} unchanged.

\fun{void}{stackdummy}{GEN x, long n} inhibits a memory area with respect to
\kbd{gerepile}. The memory area starting at \kbd{x} is declared to be a
non-recursive type of length \kbd{n}. Thus gerepile will not inspect the zone,
at most copy it. To be used in the following situation:
\bprog
  av0 = avma; z = cgetg(t_VEC, 3);
  z[1] = HUGE(); av = avma; garbage(); ltop = avma;
  z[2] = HUGE(); n = av - ltop; stackdummy((GEN)ltop, n);
@eprog\noindent
Compared to the orthodox
\bprog
  z[2] = (long)gerepile(av, ltop, (GEN)z[2]);
@eprog\noindent
or even more wasteful
\bprog
  z = gerepilecopy(av0, z);
@eprog\noindent
we temporarily lose \kbd{n} words but save a costly \kbd{gerepile}. (In
principle, a garbage collection higher up the call chain should reclaim this
later anyway.) Without the \kbd{stackdummy}, if the $[\kbd{av}, \kbd{ltop}]$
zone is arbitrary (not even valid \kbd{GEN}s as could happen after direct
truncation via \kbd{setlg}), we would leave dangerous data in the middle
of~\kbd{z}, which would be a problem for a later
\bprog
  gerepile(..., ... , z);
@eprog\noindent
And even if it were made of valid \kbd{GEN}s, inhibiting the area makes sure
\kbd{gerepile} will not inspect their components, saving time.

\subsec{Copies and clones}

\fun{GEN}{gclone}{GEN x} creates a new permanent copy of the object \kbd{x}
on the heap. The \emph{clone bit} of the result is set.

\fun{void}{gunclone}{GEN x} delete the clone~\kbd{x} (created by \kbd{gclone}).
Fatal error if~\kbd{x} not a clone.

\fun{GEN}{gcopy}{GEN x} creates a new copy of the object~\kbd{x} on the
stack. Subobjects not belonging to the stack may not be copied (only the
pointer would be): this occurs for \kbd{x[1]} if \kbd{x} is a \typ{INTMOD},
\typ{POLMOD} or \typ{QUAD}, for \kbd{x[2]} if \kbd{x} is a \typ{PADIC}. This
happens for instance when one initializes a sequence of operation from a
clone.

\fun{int}{isonstack}{GEN x} \kbd{true} iff \kbd{x} belongs to the stack. This
is a macro whose argument is evaluated several times.

\fun{void}{copyifstack}{GEN x, GEN y} sets \kbd{y = gcopy(x)} if
\kbd{x} belongs to the stack, and \kbd{y = x} otherwise. This macro evaluates
its arguments once, contrary to
\bprog
  y = isonstack(x)? gcopy(x): x;
@eprog

\fun{void}{icopyifstack}{GEN x, GEN y} as \kbd{copyifstack} assuming \kbd{x}
is a \typ{INT}.

\fun{GEN}{forcecopy}{GEN x} same as \key{gcopy} except that all subobjects
are copied onto the stack.

\fun{long}{taille}{GEN x} returns the total number of \B-bit words occupied
by the tree representing~\kbd{x}.

\section{Level 0 kernel (operations on ulongs)}

\subsec{Micro-kernel}.
Level 0 operations simulate basic operations of the 68020 processor on which
PARI was originally implemented. They need ``global'' \kbd{ulong} variables
\kbd{overflow} (which will contain only 0 or 1) and \kbd{hiremainder} to
function properly. However, for certain architectures these are replaced with
local variables for efficiency; and the `functions' mentioned below are
really chunks of inlined assembler code. So, a routine using one of these
lowest-level functions where the description mentions either
\kbd{hiremainder} or \kbd{overflow} must declare the corresponding
\bprog
  LOCAL_HIREMAINDER;
  LOCAL_OVERFLOW;
@eprog\noindent
in a declaration block. Variables \kbd{hiremainder} and \kbd{overflow} then
become available in the enclosing block. For instance a loop over the powers
of an \kbd{ulong}~\kbd{p} protected from overflows could read
\bprog
 while (pk < lim)
 {
   LOCAL_HIREMAINDER;
   ...
   pk = mulll(pk, p); if (hiremainder) break;
 } 
@eprog

\fun{ulong}{addll}{ulong x, ulong y} adds \kbd{x} and \kbd{y}, returns the
lower \B\ bits and puts the carry bit into \kbd{overflow}.

\fun{ulong}{addllx}{ulong x, ulong y} adds \kbd{overflow} to the sum of the
\kbd{x} and \kbd{y}, returns the lower \B\ bits and puts the carry bit into
\kbd{overflow}.

\fun{ulong}{subll}{ulong x, ulong y} subtracts \kbd{x} and \kbd{y}, returns
the lower \B\ bits and put the carry (borrow) bit into \kbd{overflow}.

\fun{ulong}{subllx}{ulong x, ulong y} subtracts \kbd{overflow} from the
difference of \kbd{x} and \kbd{y}, returns the lower \B\ bits and puts the
carry (borrow) bit into \kbd{overflow}.

\fun{ulong}{shiftl}{ulong x, ulong y} shifts \kbd{x} left by \kbd{y} bits,
returns the lower \B\ bits and stores the high-order \B\ bits into
\kbd{hiremainder}. We must have $1\le\kbd{y}\le\B$; in particular,
$\kbd{y}\neq 0$. If \kbd{y} does not satisfy this condition, the result is
undefined.

\fun{ulong}{shiftlr}{ulong x, ulong y} shifts \kbd{x << \B} right
by \kbd{y} bits, returns the higher \B\ bits and stores the low-order
\B\ bits into \kbd{hiremainder}. We must have $1\le\kbd{y}\le\B$; in
particular, $\kbd{y}\neq 0$. If \kbd{y} does not satisfy this condition, the
result is undefined.

\fun{int}{bfffo}{ulong x} returns the number of leading zero bits in \kbd{x}.
That is, the number of bit positions by which it would have to be shifted
left until its leftmost bit first becomes equal to~1, which can be between 0
and $\B-1$ for nonzero \kbd{x}. When \kbd{x} is~0, the result is undefined.

\fun{ulong}{mulll}{ulong x, ulong y} multiplies \kbd{x} by \kbd{y}, returns
the lower \B\ bits and stores the high-order \B\ bits into \kbd{hiremainder}.

\fun{ulong}{addmul}{ulong x, ulong y} adds \kbd{hiremainder} to the product
of \kbd{x} and \kbd{y}, returns the lower \B\ bits and stores the high-order
\B\ bits into \kbd{hiremainder}.

\fun{ulong}{divll}{ulong x, ulong y} returns the Euclidean quotient of
(\kbd{hiremainder << \B})${}+{}$\kbd{x} by \kbd{y} and stores the remainder
into \kbd{hiremainder}. An error occurs if the quotient cannot be represented
by an \kbd{ulong}, i.e.~if initially $\kbd{hiremainder}\ge\kbd{y}$.

\subsec{Modular kernel}.
The following routines are not part of the level 0 kernel per se, but
implement modular operations on words in terms of the above. They are written
so that no overflow may occur. Let $m \geq 1$ be the modulus; all operands
representing classes modulo $m$ are assumed to belong to $[0,m-1[$. The
result may be wrong for a number of reasons otherwise: it may not be reduced,
overflow can occur, etc.

\fun{ulong}{Fl_add}{ulong x, ulong y, ulong m} returns the smallest
positive representative of $x + y$ modulo $m$.

\fun{ulong}{Fl_sub}{ulong x, ulong y, ulong m} returns the smallest
positive representative of $x - y$ modulo $m$.

\fun{ulong}{Fl_mul}{ulong x, ulong y, ulong m} returns the smallest positive
representative of $x y$ modulo $m$.

\fun{ulong}{Fl_inv}{ulong x, ulong m} returns the smallest
positive representative of $x^{-1}$ modulo $m$. If $x$ is not invertible
mod~$m$, return $0$.

\fun{ulong}{Fl_div}{ulong x, ulong y, ulong m} returns the smallest
positive representative of $x y^{-1}$ modulo $m$. If $y$ is not invertible
mod $m$, return $0$.

\fun{ulong}{Fl_pow}{ulong x, ulong n, ulong m} returns the smallest
positive representative of $x^n$ modulo $m$.

\fun{ulong}{Fl_sqrt}{ulong x, ulong p} returns the square root of \kbd{x}
modulo \kbd{p} (smallest positive representative). Assumes \kbd{p} to be
prime, and \kbd{x} to be a square modulo \kbd{p}.

\fun{ulong}{Fl_gener}{ulong p} returns a primitive root modulo \kbd{p},
assuming \kbd{p} is prime.

\fun{ulong}{Fl_gener_fact}{ulong p, GEN fa} as \kbd{Fl\_gener}, assuming
\kbd{fa} is the factorization of $\kbd{p} - 1$. (As output by \kbd{factor}.)

\fun{long}{krouu}{ulong x, ulong y} returns the Kronecker symbol $(x|y)$,
i.e.$-1$, $0$ or $1$. Assumes \kbd{y} is non-zero. If \kbd{y} is prime, this
is the Legendre symbol.

\section{Level 1 kernel (operations on longs, integers and reals)}

\misctitle{Note:} Many functions consist of an elementary operation,
immediately followed by an assignment statement. They will be introduced as
in the following example:

\fun{GEN}{gadd[z]}{GEN x, GEN y[, GEN z]} followed by the explicit
description of the function

\kbd{GEN \key{gadd}(GEN x, GEN y)}

\noindent which creates its result on the stack, returning a \kbd{GEN} pointer
to it, and the parts in brackets indicate that there exists also a function

\kbd{void \key{gaddz}(GEN x, GEN y, GEN z)}

\noindent which assigns its result to the pre-existing object
\kbd{z}, leaving the stack unchanged. All such functions are obtained using
macros (see the file \kbd{paricom.h}), hence you can easily extend the list.
These assignment variants are inefficient; don't use them.

\subsec{Creation}

\fun{GEN}{cgeti}{long n} allocates memory on the PARI stack for a \typ{INT}
of length~\kbd{n}, and initializes its first codeword. Identical to
\kbd{cgetg(n,\typ{INT})}.

\fun{GEN}{cgetr}{long n} allocates memory on the PARI stack for a \typ{REAL}
of length~\kbd{n}, and initializes its first codeword. Identical to
\kbd{cgetg(n,\typ{REAL})}.

\fun{GEN}{cgetc}{long n} allocates memory on the PARI stack for a
\typ{COMPLEX}, whose real and imaginary parts are \typ{REAL}s
of length~\kbd{n}.

\fun{GEN}{realun}{long prec} create a \typ{REAL} equal to $1$ to \kbd{prec}
words of accuracy.

\fun{GEN}{realzero_bit}{long bit} create a \typ{REAL} equal to $0$ with
exponent $-\kbd{bit}$.

\fun{GEN}{realzero}{long prec} is a shorthand for
\bprog
  realzero_bit( -bit_accuracy(prec) )
@eprog

\subsec{Assignment}.
In this section, the \kbd{z} argument in the \kbd{z}-functions must be of type
\typ{INT} or~\typ{REAL}.

\fun{void}{mpaff}{GEN x, GEN z} assigns \kbd{x} into~\kbd{z} (where \kbd{x}
and \kbd{z} are \typ{INT} or \typ{REAL}).
Assumes that $\kbd{lg(z)} > 2$.

\fun{void}{affii}{GEN x, GEN z} assigns the \typ{INT} \kbd{x} into the
\typ{INT}~\kbd{z}.

\fun{void}{affir}{GEN x, GEN z} assigns the \typ{INT} \kbd{x} into the
\typ{REAL}~\kbd{z}. Assumes that $\kbd{lg(z)} > 2$.

\fun{void}{affiz}{GEN x, GEN z} assigns \typ{INT}~\kbd{x} into \typ{INT} or
\typ{REAL}~\kbd{z}. Assumes that $\kbd{lg(z)} > 2$.

\fun{void}{affsi}{long s, GEN z} assigns the \kbd{long}~\kbd{s} into the
\typ{INT}~\kbd{z}. Assumes that $\kbd{lg(z)} > 2$.

\fun{void}{affsr}{long s, GEN z} assigns the \kbd{long}~\kbd{s} into the
\typ{REAL}~\kbd{z}. Assumes that $\kbd{lg(z)} > 2$.

\fun{void}{affsz}{long s, GEN z} assigns the \kbd{long}~\kbd{s} into the
\typ{INT} or \typ{REAL}~\kbd{z}. Assumes that $\kbd{lg(z)} > 2$.

\fun{void}{affui}{ulong u, GEN z} assigns the \kbd{ulong}~\kbd{u} into the
\typ{INT}~\kbd{z}. Assumes that $\kbd{lg(z)} > 2$.

\fun{void}{affur}{ulong u, GEN z} assigns the \kbd{ulong}~\kbd{u} into the
\typ{REAL}~\kbd{z}. Assumes that $\kbd{lg(z)} > 2$.

\fun{void}{affrr}{GEN x, GEN z} assigns the \typ{REAL}~\kbd{x} into the
\typ{REAL}~\kbd{z}.

\noindent The function \kbd{affrs} and \kbd{affri} do not exist. So don't use
them.

\subsec{Copy}

\fun{GEN}{icopy}{GEN x} copy relevant words of the \typ{INT}~\kbd{x} on the
stack: the length and effective length of the copy are equal.

\fun{GEN}{rcopy}{GEN x} copy the \typ{REAL}~\kbd{x} on the stack.

\fun{GEN}{mpcopy}{GEN x} copy the \typ{INT} or \typ{REAL}~\kbd{x} on the
stack. Contrary to \kbd{icopy}, \kbd{mpcopy} preserves the original length
of a \typ{INT}.

\subsec{Conversions}

\fun{GEN}{itor}{GEN x, long prec} converts the \typ{INT}~\kbd{x} to a
\typ{REAL} of length \kbd{prec} and return the latter.
Assumes that $\kbd{prec} > 2$.

\fun{long}{itos}{GEN x} converts the \typ{INT}~\kbd{x} to a \kbd{long} if
possible, otherwise raise an exception.

\fun{long}{itos_or_0}{GEN x} converts the \typ{INT}~\kbd{x} to a \kbd{long} if
possible, otherwise return $0$.

\fun{ulong}{itou}{GEN x} converts the \typ{INT}~\kbd{|x|} to an \kbd{ulong} if
possible, otherwise raise an exception.

\fun{long}{itou_or_0}{GEN x} converts the \typ{INT}~\kbd{|x|} to an
\kbd{ulong} if possible, otherwise return $0$.

\fun{GEN}{stoi}{long s} creates the \typ{INT} corresponding to the
\kbd{long}~\kbd{s}.

\fun{GEN}{stor}{long s, long prec} converts the \kbd{long}~\kbd{s} into a
\typ{REAL} of length \kbd{prec} and return the latter. Assumes that 
$\kbd{prec} > 2$.

\fun{GEN}{utoi}{ulong s} converts the \kbd{ulong}~\kbd{s} into a \typ{INT}
and return the latter.

\fun{GEN}{utor}{ulong s, long prec} converts the \kbd{ulong}~\kbd{s} into a
\typ{REAL} of length \kbd{prec} and return the latter. Assumes that
$\kbd{prec} > 2$.

\fun{GEN}{rtor}{GEN x, long prec} converts the \typ{REAL}~\kbd{x} to a
\typ{REAL} of length \kbd{prec} and return the latter. If 
$\kbd{prec} < \kbd{lg(x)}$, round properly. If $\kbd{prec} > \kbd{lg(x)}$,
padd with zeroes. Assumes that $\kbd{prec} > 2$.

\noindent The following function is also available as a special case of
\tet{coefs_to_int}:

\fun{GEN}{u2toi}{ulong a, ulong b}

Returns the \kbd{GEN} equal to $2^{32} a + b$, \emph{assuming} that
$a,b < 2^{32}$. This does not depend on \kbd{sizeof(long)}: the behaviour is
as above on both $32$ and $64$-bit machines.

\subsec{Integer parts}

\fun{GEN}{ceilr}{GEN x} smallest integer larger or equal
to the \typ{REAL}~\kbd{x} (i.e.~the \kbd{ceil} function).

\fun{GEN}{floorr}{GEN x} largest integer smaller or equal to the
\typ{REAL}~\kbd{x} (i.e.~the \kbd{floor} function).

\fun{GEN}{roundr}{GEN x} rounds the \typ{REAL} \kbd{x} to the nearest integer
(towards~$+\infty$).

\fun{GEN}{truncr}{GEN x} truncates the \typ{REAL}~\kbd{x} (not the same as
\kbd{floorr} if \kbd{x} is and negative).

\fun{GEN}{mpceil[z]}{GEN x[, GEN z]}
as \kbd{ceilr} except that \kbd{x} may be a \typ{INT}.

\fun{GEN}{ceil_safe}{GEN x}, \kbd{x} being a real number (not necessarily a
\typ{REAL}) returns an integer which is larger than any possible incarnation
of \kbd{x}. (Recall that a \typ{REAL} represents an interval of possible
values.)

\fun{GEN}{mpfloor[z]}{GEN x[, GEN z]}
as \kbd{floorr} except that \kbd{x} may be a \typ{INT}.

\fun{GEN}{mpround[z]}{GEN x[, GEN z]}
as \kbd{roundr} except that \kbd{x} may be a \typ{INT}.

\fun{GEN}{mptrunc[z]}{GEN x[, GEN z]}
as \kbd{truncr} except that \kbd{x} may be a \typ{INT}.

\fun{GEN}{diviiround}{GEN x, GEN y} if \kbd{x} and \kbd{y} are \typ{INT}s,
returns the quotient $\kbd{x}/\kbd{y}$ of \kbd{x} and~\kbd{y}, rounded to
the nearest integer. If $\kbd{x}/\kbd{y}$ falls exactly halfway between
two consecutive integers, then it is rounded towards~$+\infty$ (as for
\tet{roundr}).

\subsec{Valuation and shift}

\fun{long}{vals}{long s} 2-adic valuation of the \kbd{long}~\kbd{s}. Returns
$-1$ if \kbd{s} is equal to 0.

\fun{long}{vali}{GEN x} 2-adic valuation of the \typ{INT}~\kbd{x}. Returns $-1$
if \kbd{x} is equal to 0.

\fun{GEN}{mpshift[z]}{GEN x, long n[, GEN z]} shifts the~\typ{INT} or
\typ{REAL} \kbd{x} by~\kbd{n}. If \kbd{n} is positive, this is a left shift,
i.e.~multiplication by $2^{\kbd{n}}$. If \kbd{n} is negative, it is a right
shift by~$-\kbd{n}$, which amounts to the truncation of the quotient of \kbd{x}
by~$2^{-\kbd{n}}$.

\fun{GEN}{shifti}{GEN x, long n} shifts the \typ{INT}~\kbd{x} by~\kbd{n}.

\fun{GEN}{shiftr}{GEN x, long n} shifts the \typ{REAL}~\kbd{x} by~\kbd{n}.

\fun{long}{pvaluation}{GEN x, GEN p, GEN *r} applied to non-zero \typ{INT}s
\kbd{x} and~\kbd{p}, returns the highest exponent $e$ such that $\kbd{p}^{e}$
divides~\kbd{x}. If \kbd{r} is non-\kbd{NULL}, creates the quotient
$\kbd{x}/\kbd{p}^{e}$ and returns its address in~\kbd{*r}. In particular, if
\kbd{p} is a prime, this returns the valuation at \kbd{p} of~\kbd{x}, and
\kbd{*r} yields the prime-to-\kbd{p} part of~\kbd{x}.

\fun{long}{svaluation}{ulong x, ulong p, ulong *r} as \kbd{pvaluation},
except the inputs/outputs are now \kbd{ulong}, and $\kbd{r} = \kbd{NULL}$ is
not allowed.

\subsec{Generic unary operators}. Let ``\op'' be a unary operation among

\op=\key{neg}: negation ($-$\kbd{x}).


\op=\key{abs}: absolute value ($|\kbd{x}|$).

\noindent The names and prototypes of the low-level functions corresponding
to \op\ are as follows. The result is of the same type as~\kbd{x}.

\funno{GEN}{mp\op}{GEN x} creates the result of \op\ applied to the
\typ{INT} or \typ{REAL}~\kbd{x}.

\funno{GEN}{\op i}{GEN x} creates the result of \op\ applied to the
\typ{INT}~\kbd{x}.

\funno{GEN}{\op r}{GEN x} creates the result of \op\ applied to the
\typ{REAL}~\kbd{x}.

\funno{GEN}{mp\op z}{GEN x, GEN z} assigns the result of applying \op\ to the
\typ{INT} or \typ{REAL}~\kbd{x} into the \typ{INT} or \typ{REAL}~\kbd{z}.

\misctitle{Remark:} it has not been considered useful to include
functions {\tt void \op sz(long,GEN)}, {\tt void \op iz(GEN,GEN)} and
{\tt void \op rz(GEN, GEN)}.
\smallskip

\subsec{Comparison operators}

\fun{int}{mpcmp}{GEN x, GEN y} compares the \typ{INT} or \typ{REAL}~\kbd{x}
to the \typ{INT} or \typ{REAL}~\kbd{y}. The result is the sign of
$\kbd{x}-\kbd{y}$.

\fun{int}{cmpii}{GEN x, GEN y} compares the \typ{INT} \kbd{x} to the
\typ{INT}~\kbd{y}.

\fun{int}{cmpir}{GEN x, GEN y} compares the \typ{INT} \kbd{x} to the
\typ{REAL}~\kbd{y}.

\fun{int}{cmpis}{GEN x, long s} compares the \typ{INT}~\kbd{x} to the
\kbd{long}~\kbd{s}.

\fun{int}{cmpsi}{long s, GEN x} compares the \kbd{long}~\kbd{s} to the
\typ{INT}~\kbd{x}.

\fun{int}{cmpsr}{long s, GEN x} compares the \kbd{long}~\kbd{s} to the
\typ{REAL}~\kbd{x}.

\fun{int}{cmpri}{GEN x, GEN y} compares the \typ{REAL}~\kbd{x} to the
\typ{INT}~\kbd{y}.

\fun{int}{cmprr}{GEN x, GEN y} compares the \typ{REAL}~\kbd{x} to the
\typ{REAL}~\kbd{y}.

\fun{int}{cmprs}{GEN x, long s} compares the \typ{REAL}~\kbd{x} to the
\kbd{long}~\kbd{s}.

\fun{int}{egalii}{GEN x, GEN y} compares the \typ{INT}s \kbd{x} and~\kbd{y}.
The result is $1$ if $\kbd{x} = \kbd{y}$, $0$ otherwise.

\fun{int}{absi_cmp}{GEN x, GEN y} compares the \typ{INT}s \kbd{x} and~\kbd{y}.
The result is the sign of $|\kbd{x}| - |\kbd{y}|$.

\fun{int}{absi_equal}{GEN x, GEN y} compares the \typ{INT}s \kbd{x}
and~\kbd{y}. The result is $1$ if $|\kbd{x}| = |\kbd{y}|$, $0$ otherwise.

\fun{int}{absr_cmp}{GEN x, GEN y} compares the \typ{REAL}s \kbd{x} and~\kbd{y}.
The result is the sign of $|\kbd{x}| - |\kbd{y}|$.

\subsec{Generic binary operators}. Let ``\op'' be a binary operation among

\op=\key{add}: addition (\kbd{x + y}). The result is a \typ{REAL} unless both
\kbd{x} and \kbd{y} are \typ{INT}s (or longs).

\op=\key{sub}: subtraction (\kbd{x - y}). The result is a \typ{REAL} unless both
\kbd{x} and \kbd{y} are \typ{INT} (or longs).

\op=\key{mul}: multiplication (\kbd{x * y}). The result is a \typ{REAL}
unless both \kbd{x} and \kbd{y} are \typ{INT}s (or longs), \emph{or} if
\kbd{x} or \kbd{y} is an exact $0$.

\op=\key{div}: division (\kbd{x / y}). In the case where \kbd{x} and \kbd{y}
are both \typ{INT}s or longs, the result is the Euclidean quotient, where the
remainder has the same sign as the dividend~\kbd{x}. It is the ordinary
division otherwise. If one of \kbd{x} or \kbd{y} is a \typ{REAL}, the result
is a \typ{REAL} unless \kbd{x} is an exact $0$. A division-by-$0$ error
occurs if \kbd{y} is equal to $0$.

\op=\key{rem}: remainder (``\kbd{x \% y}''). This operation is defined only
when \kbd{x} and \kbd{y} are longs or \typ{INT}. The result is the Euclidean
remainder corresponding to \kbd{div},~i.e. its sign is that of the
dividend~\kbd{x}. The result is always a \typ{INT}.

\op=\key{mod}: true remainder (\kbd{x \% y}). This operation is defined only
when \kbd{x} and \kbd{y} are longs or \typ{INT}s. The result is the true
Euclidean remainder, i.e.~non-negative and less than the absolute value
of~\kbd{y}.

\noindent The names and prototypes of the low-level functions corresponding
to \op\ are as follows. In this section, the \kbd{z} argument in the
\kbd{z}-functions must be of type \typ{INT} or~\typ{REAL}. \typ{INT} is only
allowed when no `r' appears in the argument code (no \typ{REAL} operand is
involved).

\funno{GEN}{mp\op[z]}{GEN x, GEN y[, GEN z]} applies \op\ to
the \typ{INT} or \typ{REAL} \kbd{x} and~\kbd{y}.

\funno{GEN}{\op si[z]}{long s, GEN x[, GEN z]} applies \op\ to the
\kbd{long}~\kbd{s} and the \typ{INT}~\kbd{x}.

\funno{GEN}{\op sr[z]}{long s, GEN x[, GEN z]} applies \op\ to the
\kbd{long}~\kbd{s} and the \typ{REAL}~\kbd{x}.

\funno{GEN}{\op ss[z]}{long s, long t[, GEN z]} applies \op\ to the longs
\kbd{s} and~\kbd{t}.

\funno{GEN}{\op ii[z]}{GEN x, GEN y[, GEN z]} applies \op\ to the
\typ{INT}s \kbd{x} and~\kbd{y}.

\funno{GEN}{\op ir[z]}{GEN x, GEN y[, GEN z]} applies \op\ to the
\typ{INT} \kbd{x} and the \typ{REAL}~\kbd{y}.

\funno{GEN}{\op is[z]}{GEN x, long s[, GEN z]} applies \op\ to the
\typ{INT}~\kbd{x} and the \kbd{long}~\kbd{s}.

\funno{GEN}{\op ri[z]}{GEN x, GEN y[, GEN z]} applies \op\ to the
\typ{REAL}~\kbd{x} and the \typ{INT}~\kbd{y}.

\funno{GEN}{\op rr[z]}{GEN x, GEN y[, GEN z]} applies \op\ to the
\typ{REAL}s~\kbd{x} and~\kbd{y}.

\funno{GEN}{\op rs[z]}{GEN x, long s[, GEN z]} applies \op\ to the
\typ{REAL}~\kbd{x} and the \kbd{long}~\kbd{s}.

\noindent Some miscellaneous routines whose meaning should be clear from
their names:

\fun{GEN}{muluu}{ulong x, ulong y}

\fun{GEN}{mului}{ulong x, GEN y}

\fun{GEN}{muliu}{GEN x, ulong y}

\fun{GEN}{sqri}{GEN x} squares the \typ{INT}~\kbd{x}

\fun{GEN}{centermodii}{GEN x, GEN y, GEN y2}, given 
\typ{INT}s \kbd{x}, \kbd{y}, returns $z$ congruent to \kbd{x} modulo \kbd{y},
such that $-\kbd{y}/2 \leq z < \kbd{y}/2$. Assumes that \kbd{y2 = shifti(y,
-1)}. the representative of ssquares the \typ{INT}~\kbd{x}

\subsec{Modulo to longs}. The following variants of \kbd{modii} do not
clutter the stack:

\fun{long}{smodis}{GEN x, long y} computes the true Euclidean
remainder of the \typ{INT}~\kbd{x} by the \kbd{long}~\kbd{y}. This is the
non-negative remainder, not the one whose sign is the sign of \kbd{x}
as in the \kbd{div} functions.

\fun{long}{smodsi}{long x, GEN y} computes the true Euclidean
remainder of the \kbd{long}~\kbd{x} by a \typ{INT}~\kbd{y}.

\fun{long}{smodss}{long x, long y} computes the true Euclidean
remainder of the \kbd{long}~\kbd{x} by a \typ{long}~\kbd{y}.

\fun{ulong}{umodiu}{GEN x, ulong y} computes the true Euclidean
remainder of the \typ{INT}~\kbd{x} by the \kbd{ulong}~\kbd{y}.

\fun{ulong}{umodui}{ulong x, GEN y} computes the true Euclidean
remainder of the \kbd{ulong}~\kbd{x} by the \typ{INT}~\kbd{|y|}.

The routine \tet{smodsi} does not exist, since it would not always be
defined: for a \emph{signed} \kbd{x}, its result \kbd{x + |y|} would
in general not fit into a \kbd{long}. Use either \kbd{umodui} or
\kbd{modsi}.

\subsec{Exact division and divisibility}

\fun{void}{diviiexact}{GEN x, GEN y} returns the Euclidean quotient
$\kbd{x} / \kbd{y}$, assuming $\kbd{y}$ divides $\kbd{x}$. Uses Jebelean
algorithm (Jebelean-Krandick bidirectional exact division is not
implemented).

\fun{void}{diviuexact}{GEN x, ulong y} returns the Euclidean quotient
$|\kbd{x}| / \kbd{y}$ (note the absolue value!), assuming $\kbd{y}$ divides
$\kbd{x}$ and $\kbd{y}$ is \emph{odd}.

\fun{int}{dvdii}{GEN x, GEN y} if the \typ{INT}~\kbd{y} divides the
\typ{INT}~\kbd{x}, returns 1 (true), otherwise returns 0 (false).

\fun{int}{dvdiiz}{GEN x, GEN y, GEN z} if the \typ{INT}~\kbd{y} divides the
\typ{INT}~\kbd{x}, assigns the quotient to the \typ{INT}~\kbd{z} and returns 1
(true), otherwise returns 0 (false).

\fun{int}{dvdisz}{GEN x, GEN y, GEN z} if the \typ{long}~\kbd{y} divides the
\typ{INT}~\kbd{x}, assigns the quotient to the \typ{INT}~\kbd{z} and returns 1
(true), otherwise returns 0 (false).

\subsec{Division with remainder}. The following functions return two objects,
unless specifically asked for only one of them~--- a quotient and a remainder.
The quotient is returned and the remainder is returned through the variable
whose address is passed as the \kbd{r} argument. The term \emph{true
Euclidean remainder} refers to the non-negative one (\kbd{mod}), and
\emph{Euclidean remainder} by itself to the one with the same sign as the
dividend (\kbd{rem}). All \kbd{GEN}s, whether returned directly or through a
pointer, are created on the stack.

\fun{GEN}{dvmdii}{GEN x, GEN y, GEN *r} returns the Euclidean quotient of the
\typ{INT}~\kbd{x} by a \typ{INT}~\kbd{y} and puts the remainder
into~\kbd{*r}. If \kbd{r} is equal to \kbd{NULL}, the remainder is not
created, and if \kbd{r} is equal to  \kbd{ONLY\_REM}, only the remainder is
created and returned. In the generic case, the remainder is created after the
quotient and can be disposed of individually with a \kbd{cgiv(r)}. The
remainder is always of the sign of the dividend~\kbd{x}. If the remainder
is $0$ set \kbd{r = gzero}.

\fun{void}{dvmdiiz}{GEN x, GEN y, GEN z, GEN t} assigns the Euclidean
quotient of the \typ{INT}s \kbd{x} and \kbd{y} into the \typ{INT} or
\typ{REAL}~\kbd{z}, and the Euclidean remainder into the \typ{INT} or
\typ{REAL}~\kbd{t}.

\noindent Analogous routines \tet{dvmdis}\kbd{[z]}, \tet{dvmdsi}\kbd{[z]},
\tet{dvmdss}\kbd{[z]} are available, where \kbd{s} denotes a \kbd{long}
argument. But the following routines are in general more flexible:

\fun{long}{sdivss_rem}{long s, long t, long *r} computes the Euclidean
quotient and remainder of the longs \kbd{s} and~\kbd{t}. Puts the remainder
into \kbd{*r}, and returns the quotient. The remainder is of the sign of the
dividend~\kbd{s}, and has strictly smaller absolute value than~\kbd{t}.

\fun{long}{sdivsi_rem}{long s, GEN x, long *r} computes the Euclidean
quotient and remainder of the \kbd{long}~\kbd{s} by the \typ{INT}~\kbd{x}. As
\kbd{sdivss\_rem} otherwise.

\fun{GEN}{divis_rem}{GEN x, long s, long *r} computes the Euclidean quotient
and remainder of the \typ{INT}~\kbd{x} by the \kbd{long}~\kbd{s}. As
\kbd{sdivss\_rem} otherwise.

\fun{GEN}{diviu_rem}{GEN x, ulong s, long *r} computes the Euclidean quotient
and remainder of the \typ{INT}~\kbd{x} by the \kbd{ulong}~\kbd{s}. As
\kbd{sdivss\_rem} otherwise.

\fun{GEN}{divsi_rem}{long s, GEN y, long *r} computes the Euclidean quotient
and remainder of the \typ{long}~\kbd{s} by the \kbd{GEN}~\kbd{y}. As
\kbd{sdivss\_rem} otherwise.

\fun{GEN}{divss_rem}{long x, long y, long *r} computes the Euclidean quotient
and remainder of the \typ{long}~\kbd{x} by the \kbd{long}~\kbd{y}. As
\kbd{sdivss\_rem} otherwise.
\smallskip
\fun{GEN}{truedvmdii}{GEN x, GEN y, GEN *r}, as \kbd{dvmdii} but with a
non-negative remainder.

\subsec{Pseudo-random integers}

\fun{long}{random_bits}{long k} returns a random $0 \leq x < 2^k$. Assumes
that $0 \leq k < 31$.

\fun{long}{pari_rand31}{long k} as \kbd{random\_bits} with $k = 31$.

\fun{GEN}{randomi}{GEN n} returns a random \typ{INT} between $0$ and $\kbd{n}
- 1$. The result is pasted from successive calls to \kbd{pari\_rand31}.

\subsec{Modular operations}. In this subsection, all \kbd{GEN}s are \typ{INT}.

\fun{ulong}{Fp_powu}{GEN x, ulong n, GEN m} raises \kbd{x} to the \kbd{n}-th
power modulo \kbd{p} (smallest non-negative residue).

\fun{GEN}{Fp_pow}{GEN x, GEN n, GEN m} returns $\kbd{x}^\kbd{n}$
modulo \kbd{p} (smallest non-negative residue).

\fun{GEN}{Fp_inv}{GEN a, GEN m} returns an inverse of \kbd{a} modulo \kbd{m}
(smallest non-negative residue). Raise an error if \kbd{a} is not invertible.

\fun{GEN}{Fp_invsafe}{GEN a, GEN m} as \kbd{Fp\_inv}, but return
\kbd{NULL} if \kbd{a} is not invertible.

\fun{int}{invmod}{GEN a, GEN m, GEN *g},  return $1$ if \kbd{a}
modulo \kbd{m} is invertible, else return $0$ and set
$\kbd{g} = \gcd(\kbd{a},\kbd{m})$.

\fun{GEN}{Fp_sqrt}{GEN x, GEN p} returns a square root of \kbd{x} modulo
\kbd{p} (the smallest non-negative residue), where \kbd{x}, \kbd{p} are
\typ{INT}s, and \kbd{p} is assumed to be prime. Return \kbd{NULL}
if \kbd{x} is not a quadratic residue modulo \kbd{p}.

\fun{GEN}{Fp_sqrtn}{GEN x, GEN n, GEN p, GEN *zn} returns an \kbd{n}-th
root of $\kbd{x}$ modulo \kbd{p} (smallest non-negative residue), where
\kbd{x}, \kbd{n}, \kbd{p} are \typ{INT}s, and \kbd{p} is assumed to be prime.
Return \kbd{NULL} if \kbd{x} is not an \kbd{n}-th power residue. Otherwise,
if \kbd{zn} is non-\kbd{NULL} set it to a primitive \kbd{n}-th root of $1$.

\fun{long}{kross}{long x, long y} returns the Kronecker symbol $(x|y)$,
i.e.$-1$, $0$ or $1$. If \kbd{y} is prime, this is the Legendre symbol.
(Contrary to \kbd{krouu}, \kbd{kross} also supports $\kbd{y} = 0$)

\fun{long}{krois}{GEN x, long y} returns the Kronecker symbol $(x|y)$
of \typ{INT}~x and \kbd{long}~\kbd{y}. As \kbd{kross} otherwise.

\fun{long}{krosi}{long x, GEN y} returns the Kronecker symbol $(x|y)$
of \kbd{long}~x and \typ{INT}~\kbd{y}. As \kbd{kross} otherwise.

\fun{long}{kronecker}{GEN x, GEN y} returns the Kronecker symbol $(x|y)$
of \typ{INT}s~x and~\kbd{y}. As \kbd{kross} otherwise.

\fun{GEN}{Fp_gener}{GEN p} returns a primitive root modulo \kbd{p}, assuming
\kbd{p} is prime.

\fun{GEN}{Fp_gener_fact}{GEN p, GEN fa} as \kbd{Fp\_gener}, assuming
\kbd{fa} is the factorization of $\kbd{p} - 1$.

\subsec{Miscellaneous functions}

\fun{void}{addumului}{ulong a, ulong b, GEN x} return $a + b|X|$.

\fun{long}{cgcd}{long x, long y}, returns the GCD of the \typ{long}s \kbd{x}
and \kbd{y}.

\fun{long}{cbezout}{long a,long b, long *u,long *v}, returns the GCD
$d$ of \kbd{a} and \kbd{b} and sets \kbd{u}, \kbd{v} to the Bezout coefficients
such that $\kbd{au} + \kbd{bv} = d$.

\fun{GEN}{bezout}{GEN a,GEN b, GEN *u,GEN *v}, returns the GCD $d$ of
\typ{INT}s \kbd{a} and \kbd{b} and sets \kbd{u}, \kbd{v} to the Bezout
coefficients such that $\kbd{au} + \kbd{bv} = d$.

\fun{GEN}{gcdii}{GEN x, GEN y}, returns the GCD of the \typ{INT}s \kbd{x} and
\kbd{y}.

\fun{GEN}{lcmii}{GEN x, GEN y}, returns the LCM of the \typ{INT}s \kbd{x} and
\kbd{y}.

\fun{long}{maxss}{long x, long y}, return the largest of \kbd{x} and \kbd{y}.

\fun{long}{minss}{long x, long y}, return the smallest of \kbd{x} and \kbd{y}.

\fun{void}{rdivii}{GEN x, GEN y, long prec}, assuming \kbd{x} and \kbd{y}
are both of type \typ{INT}, return the quotient x/y as a \typ{REAL} of
precision \kbd{prec}.

\fun{void}{rdivis}{GEN x, long y, long prec}, assuming \kbd{x}
is of type \typ{INT}, return the quotient x/y as a \typ{REAL} of
precision \kbd{prec}.

\fun{void}{rdivsi}{long x, GEN y, long prec}, assuming \kbd{y}
is of type \typ{INT}, return the quotient x/y as a \typ{REAL} of
precision \kbd{prec}.

\fun{void}{rdivss}{long x, long y, long prec}, return the quotient x/y as a
\typ{REAL} of precision \kbd{prec}.

%%%%%%%%
\section{Level 2 kernel (modular arithmetic)}

\noindent These routines implement univariate polynomial arithmetic and
linear algebra over finite fields, in fact over finite rings of the form
$(\Z/p\Z)[X]/(T)$, where $p$ is not necessarily prime and $T\in(\Z/p\Z)[X]$ is
possibly reducible; and finite extensions thereof. All this can be emulated
with \typ{INTMOD} and \typ{POLMOD} coefficients and using generic routines,
at a considerable loss of efficiency. Also, some specialized routines are
available that have no obvious generic equivalent.

\subsec{Naming scheme}. A function name is built in the following way:
$A_1\kbd{\_}\dots\kbd{\_}A_n\var{fun}$ for an operation \var{fun} with $n$
arguments of class $A_1$,\dots, $A_n$. A class name is given by a base ring
followed by a number of code letters. Base rings are among

  \kbd{Fl}: $\Z/l\Z$ where $l < 2^{\B}$ is not necessarily prime. Implemented
            using \kbd{ulong}s

  \kbd{Fp}: $\Z/p\Z$ where $p$ is a \typ{INT}, not necessarily prime.
Implemented as \typ{INT}s $z$, $0 \leq z < p$.

  \kbd{Fq}: $\Z[X]/(p,T(X))$, $p$ a \typ{INT}, $T$ a \typ{POL} with \kbd{Fp}
coefficients or \kbd{NULL} (in which case no reduction modulo \kbd{T} is
performed). Implemented as \typ{POL}s $z$ with \kbd{Fp} coefficients,
$\deg(z) < \deg \kbd{T}$.

  \kbd{Z}:  the integers $\Z$, implemented as \typ{INT}s.

  \kbd{z}:  the integers $\Z$, implemented using \kbd{long}s

  \kbd{Q}:  the rational numbers $\Q$, implemented as \typ{INT}s and
\typ{FRAC}s.

  \kbd{R}:  an arbitrary (commutative) ring, whose elements can be
\kbd{gadd}-ed, \kbd{gmul}-ed, etc.

\noindent Possible letters are:

  \kbd{X}: polynomial in $X$ (\typ{POL} in a fixed variable), e.g. \kbd{FpX}
           means $\Z/p\Z[X]$

  \kbd{Y}: polynomial in $Y\neq X$. E.g. \kbd{FpXY} means $((\Z/p\Z)[Y])[X]$

  \kbd{V}: vector (\typ{VEC} or \typ{COL}). E.g. \kbd{ZV} means $\Z^k$ for
           some $k$.

  \kbd{C}: column vector, e.g. \kbd{ZC} means $\Z^k$ for some $k$. (Used to
clarify the meaning of \kbd{V} when ambiguous.)

  \kbd{M}: matrix (\typ{MAT}). E.g. \kbd{QM} means a matrix with rational entries

  \kbd{Q}: representative (\typ{POL}) of a class in a polynomial quotient
ring. E.g.~an \kbd{FpXQ} belongs to $(\Z/p\Z)[X]/(T(X))$, \kbd{FpXQV} means a
vector of such elements, etc.

  \kbd{x}, \kbd{m}, \kbd{v}, \kbd{c}, \kbd{q}: as their uppercase
counterpart, but coefficient arrays are implemented using \typ{VECSMALL}s.

\noindent Omitting the letter means the argument is a scalar in the base
ring. Standard functions \var{fun} are 

  \kbd{add}: add

  \kbd{sub}: subtract

  \kbd{mul}: multiply

  \kbd{sqr}: square

  \kbd{div}: divide (Euclidean quotient)

  \kbd{rem}: Euclidean remainder 

  \kbd{divrem}: return Euclidean quotient, store remainder in a pointer
argument.
  
  \kbd{gcd}: GCD

  \kbd{extgcd}: return GCD, store Bezout coefficients in pointer arguments

  \kbd{pow}: exponentiate

  \kbd{compo}: composition

%  \kbd{red}: reduction to some normal form (canonical representative)

\subsec{\kbd{ZX}, \kbd{ZM}} A \kbd{ZV} (resp.~a~\kbd{ZM}, resp.~a~\kbd{ZX})
is a \typ{VEC} or \typ{COL} (resp.~\typ{MAT}, resp.~\typ{POL}) with \typ{INT}
coefficients.

%GEN     ZX_caract(GEN A, GEN B, long v);
%GEN     ZX_disc(GEN x);
%int     ZX_is_squarefree(GEN x);
%GEN     ZX_QX_resultant(GEN A, GEN B);
%GEN     ZX_resultant(GEN A, GEN B);
%GEN     ZX_s_add(GEN y,long x);
%GEN     QX_invmod(GEN A, GEN B);
%long    ZX_valuation(GEN x, GEN *Z);
%
%GEN     ZM_inv(GEN M, GEN dM);
%GEN     QM_inv(GEN M, GEN dM);

\subsec{\kbd{FpX}}. Let \kbd{p} an understood \typ{INT}, to be given in
the function arguments; an \kbd{Fp} object is a \typ{INT} belonging to $[0,
\kbd{p}-1]$, an \kbd{FpX} is a \typ{POL} in a fixed variable whose
coefficients are \kbd{Fp} objects. Unless mentionned otherwise, all outputs
in this section are \kbd{FpX}s. All operations are understood to take place
in $(\Z/\kbd{p}\Z)[X]$.

\subsubsec{Basic operations}. In what follows \kbd{p} is always a \typ{INT},
not necessarily prime.

\fun{GEN}{FpX}{GEN z, GEN p}, \kbd{z} a \kbd{ZX}. Returns \kbd{z * Mod(1,p)},
normalized (hence with \typ{INTMOD} coefficients).

\fun{GEN}{FpX_red}{GEN z, GEN p}, \kbd{z} a \kbd{ZX}, returns \kbd{lift(z *
Mod(1,p))}, normalized.

\fun{GEN}{FpXV_red}{GEN z, GEN p}, \kbd{z} a \typ{VEC} of \kbd{ZX}. Applies
\kbd{FpX\_red} componentwise and returns the result (and we obtain a vector
of \kbd{FpX}s).

\fun{GEN}{FpXX_red}{GEN z, GEN p}, \kbd{z} a \typ{POL} whose coefficients are
either \kbd{ZX}s or \typ{INT}s. Returns the \typ{POL} equal to \kbd{z} with
all components reduced modulo \kbd{p}.

\noindent Now, except for \kbd{p}, the operands and outputs are all \kbd{FpX}
objects. Results are undefined on other inputs.

\fun{GEN}{FpX_add}{GEN x,GEN y, GEN p} adds \kbd{x} and \kbd{y}. If \kbd{p}
is \kbd{NULL}, the result is not reduced mod \kbd{p} (ordinary \typ{INT}s)

\fun{GEN}{FpX_neg}{GEN x,GEN p} returns $-\kbd{x}$.

\fun{GEN}{FpX_sub}{GEN x,GEN y,GEN p} subtracts \kbd{y} from \kbd{x}. If
\kbd{p} is \kbd{NULL}, the result is not reduced mod \kbd{p} (ordinary
\typ{INT}s)

\fun{GEN}{FpX_mul}{GEN x,GEN y,GEN p} multiplies \kbd{x} and \kbd{y}. If
\kbd{p} is \kbd{NULL}, the result is not reduced mod \kbd{p} (ordinary
\typ{INT}s)

\fun{GEN}{FpX_sqr}{GEN x,GEN p} returms $\kbd{x}^2$. If \kbd{p} is
\kbd{NULL}, the result is not reduced mod \kbd{p} (ordinary \typ{INT}s)

\fun{GEN}{FpX_divrem}{GEN x, GEN y, GEN p, GEN *pr} returns the quotient
of \kbd{x} by \kbd{y}, and sets \kbd{pr} to the remainder.

\fun{GEN}{FpX_div}{GEN x, GEN y, GEN p} returns the quotient of \kbd{x} by
\kbd{y}.

\fun{GEN}{FpX_rem}{GEN x, GEN y, GEN p} returns the remainder \kbd{x} mod
\kbd{y}

\fun{GEN}{FpX_gcd}{GEN x, GEN y, GEN p} returns GCD(\kbd{x},\kbd{y}).

\fun{GEN}{FpX_extgcd}{GEN x, GEN y, GEN p, GEN *u, GEN *v} returns
$d = \text{GCD}(\kbd{x},\kbd{y})$, and sets \kbd{*u}, \kbd{*v} to the Bezout
coefficients such that $\kbd{*ux} + \kbd{*vy} = d$.

\fun{GEN}{FpX_center}{GEN z, GEN p} returns the polynomial whose coefficient
belong to the symmetric residue system (clean version of \kbd{centermod},
which assumes the coefficients already belong to $[0,\kbd{p}-1]$). 

\subsubsec{Miscellaneous operations}

\fun{GEN}{FpX_normalize}{GEN z, GEN p} divides the \kbd{FpX}~\kbd{z} by its
leading coefficient. If the latter is~$1$, \kbd{z} itself is returned, not a
copy. If not, the inverse remains uncollected on the stack.


\fun{GEN}{FpX_Fp_add}{GEN y, GEN x, GEN p} add the \kbd{Fp}~\kbd{x} to the
\kbd{FpX}~\kbd{y}.

\fun{GEN}{FpX_Fp_mul}{GEN y, GEN x, GEN p} multiplies the \kbd{FpX}~\kbd{y}
by the \kbd{Fp}~\kbd{x}.

\fun{GEN}{FpX_rescale}{GEN P, GEN h, GEN p} returns $h^{\deg(P)} P(x/h)$.
\kbd{P} is an \kbd{FpX} and \kbd{h} is a non-zero \kbd{Fp} (the routine would
work with any non-zero \typ{INT} but is not efficient in this case).

\fun{GEN}{FpX_eval}{GEN x, GEN y, GEN p} evaluates the \kbd{FpX}~\kbd{x}
at the \kbd{Fp}~\kbd{y}. The result is an~\kbd{Fp}.

\fun{GEN}{FpXV_FpV_innerprod}{GEN V, GEN W, GEN p} returns 
$\sum_{i=1}^{n} \kbd{V[i] * W[i]}$, where \kbd{V} is a vector of $n$
\kbd{FpX}, and \kbd{W} is a vector of at least $n$ \kbd{INT}s.

\fun{GEN}{FpXV_prod}{GEN V, GEN p}, \kbd{V} being a vector of \kbd{FpX},
returns their product.

\fun{GEN}{FpV_roots_to_pol}{GEN V, GEN p, long v}, \kbd{V} being a vector
of \kbd{INT}s, returns the monic \kbd{FpX}
$\prod_i (\kbd{polx[v]} - \kbd{V[i]})$.

\fun{GEN}{FpX_chinese_coprime}{GEN x,GEN y, GEN Tx,GEN Ty, GEN Tz, GEN p}
returns an \kbd{FpX}, congruent to \kbd{x} mod \kbd{Tx} and to \kbd{y} mod
\kbd{Ty}. Assumes \kbd{Tx} and \kbd{Ty} are coprime, and \kbd{Tz = Tx * Ty}
or \kbd{NULL} (in which case it is computed within).

\fun{GEN}{FpV_polint}{GEN x, GEN y, GEN p} returns the \kbd{FpX}
interpolation polynomial with value \kbd{y[i]} at \kbd{x[i]}. Assumes lengths
are the same, components are \typ{INT}s, and the \kbd{x[i]} are distinct
modulo \kbd{p}.

\fun{long}{FpX_is_squarefree}{GEN f, GEN p} returns $1$ if the
\kbd{FpX}~\kbd{f} is squarefree, $0$ otherwise.

\fun{long}{FpX_is_irred}{GEN f, GEN p} returns $1$ if the \kbd{FpX}~\kbd{f}
is irreducible, $0$ otherwise. Assumes that \kbd{p} is prime. If~\kbd{f} has
few factors, \kbd{FpX\_nbfact(f,p) == 1} is much faster.

\fun{long}{FpX_is_totally_split}{GEN f, GEN p} returns $1$ if the
\kbd{FpX}~\kbd{f} splits into a product of distinct linear factors, $0$
otherwise. Assumes that \kbd{p} is prime.

\fun{GEN}{FpX_factor}{GEN f, GEN p}, factors the \kbd{FpX}~\kbd{f}. Assumes
that \kbd{p} is prime. The returned value \kbd{v} has two components:
\kbd{v[1]} is a vector of distinct irreducible (\kbd{FpX}) factors, and
\kbd{v[2]} is a \typ{VECSMALL} of corresponding exponents. The order
of the factors is deterministic (the computation is not).

\fun{long}{FpX_nbfact}{GEN f, GEN p}, assuming the \kbd{FpX}~f is squarefree,
returns the number of its irreducible factors. Assumes that \kbd{p} is prime.

\fun{long}{FpX_degfact}{GEN f, GEN p}, as \kbd{FpX\_factor}, but the
degrees of the irreducible factors are returned instead of the factors
themselves (as a \typ{VECSMALL}). Assumes that \kbd{p} is prime.

\fun{long}{FpX_nbroots}{GEN f, GEN p} returns the number of distinct
roots in \kbd{\Z/p\Z} of the \kbd{FpX}~\kbd{f}. Assumes that \kbd{p} is prime.

\fun{GEN}{FpX_roots}{GEN f, GEN p} returns the roots in \kbd{\Z/p\Z} of
the \kbd{FpX}~\kbd{f} (without multiplicity, as a vector of \kbd{Fp}s).
Assumes that \kbd{p} is prime.

\fun{GEN}{FpX_rand}{long d, long v, GEN p} returns a random \kbd{FpX}
in variable \kbd{v}, of degree less than~\kbd{d}.

\fun{GEN}{FpY_FpXY_resultant}{GEN a, GEN b, GEN p}, \kbd{a} a \typ{POL} of
\typ{INT}s (say in variable $Y$), \kbd{b} a \typ{POL} (say in variable $X$)
whose coefficients are either \typ{POL}s in $\Z[Y]$ or \typ{INT}s.
Returns $\text{Res}_Y(a, b)$, which is an \kbd{FpX}.

\subsec{\kbd{FpXQ}, \kbd{Fq}}. Let \kbd{p} a \typ{INT} and \kbd{T} an
\kbd{FpX} for \kbd{p}, both to be given in the function arguments; an \kbd{FpXQ}
object is an \kbd{FpX} whose degree is strictly less than the degree of
\kbd{T}. An \kbd{Fq} is either an \kbd{FpXQ} or an \kbd{Fp}. Both represent
a class in $(\Z/\kbd{p}\Z)[X] / (T)$, in which all operations below take
place. In addition, \kbd{Fq} routines also allow $\kbd{T} = \kbd{NULL}$, in
which case no reduction mod \kbd{T} is performed on the result.

For efficiency, the routines in this section may leave small unused objects
behind on the stack (their output is still suitable for \kbd{gerepileupto}).
Besides \kbd{T} and \kbd{p}, arguments are either \kbd{FpXQ} or \kbd{Fq}
depending on the function name. (All \kbd{Fq} routines accept \kbd{FpXQ}s by
definition, not the other way round.)

\fun{GEN}{Fq_red}{GEN x, GEN T, GEN p}, \kbd{x} a \kbd{ZX} or \typ{INT},
reduce it to an \kbd{Fq} ($\kbd{T} = \kbd{NULL}$ is allowed iff \kbd{x} is a
\typ{INT}).

\fun{GEN}{FqX_red}{GEN x, GEN T, GEN p}, \kbd{x} a \typ{POL} 
whose coefficients are \kbd{ZX}s or \typ{INT}s, reduce them to \kbd{Fq}s. (If
$\kbd{T} = \kbd{NULL}$, as \kbd{FpXX\_red(x, p)}.)

\fun{GEN}{FqV_red}{GEN x, GEN T, GEN p}, \kbd{x} a vector of \kbd{ZX}s or
\typ{INT}s, reduce them to \kbd{Fq}s. (If $\kbd{T} = \kbd{NULL}$, only
reduce components mod \kbd{p} to \kbd{FpX}s or \kbd{Fp}s.)

\fun{GEN}{FpXQ_mul}{GEN y, GEN x, GEN T,GEN p}

\fun{GEN}{FpXQ_sqr}{GEN y, GEN T, GEN p}

\fun{GEN}{FpXQ_div}{GEN x, GEN y, GEN T,GEN p}

\fun{GEN}{FpXQ_inv}{GEN x, GEN T, GEN p} computes the inverse of \kbd{x}

\fun{GEN}{FpXQ_invsafe}{GEN x,GEN T,GEN p}, as \kbd{FpXQ\_inv}, returning
\kbd{NULL} if \kbd{x} is not invertible.

\fun{GEN}{FpXQ_pow}{GEN x, GEN n, GEN T, GEN p} computes $\kbd{x}^\kbd{n}$.

\fun{GEN}{Fq_add}{GEN x, GEN y, GEN T/*unused*/, GEN p}

\fun{GEN}{Fq_sub}{GEN x, GEN y, GEN T/*unused*/, GEN p}

\fun{GEN}{Fq_mul}{GEN x, GEN y, GEN T, GEN p}

\fun{GEN}{Fq_neg}{GEN x, GEN T, GEN p}

\fun{GEN}{Fq_neg_inv}{GEN x, GEN T, GEN p} computes $-\kbd{x}^{-1}$

\fun{GEN}{Fq_inv}{GEN x, GEN pol, GEN p} computes $\kbd{x}^{-1}$, raising an
error if \kbd{x} is not invertible.

\fun{GEN}{Fq_invsafe}{GEN x, GEN pol, GEN p} as \kbd{Fq\_inv}, but returns
\kbd{NULL} if \kbd{x} is not invertible.

\fun{GEN}{Fq_pow}{GEN x, GEN n, GEN pol, GEN p} returns $\kbd{x}^\kbd{n}$.

\fun{GEN}{FpXQ_charpoly}{GEN x, GEN T, GEN p} returns the characteristic
polynomial of \kbd{x}

\fun{GEN}{FpXQ_minpoly}{GEN x, GEN T, GEN p} returns the minimal polynomial
of \kbd{x}

\fun{GEN}{FpXQ_powers}{GEN x, long n, GEN T, GEN p} returns $[\kbd{x}^0,
\dots, \kbd{x}^\kbd{n}]$ as a \typ{VEC} of \kbd{FpXQ}s.

\fun{GEN}{FpX_FpXQ_compo}{GEN f,GEN x,GEN T,GEN p} returns
$\kbd{f}(\kbd{x})$.

\fun{GEN}{FpX_FpXQV_compo}{GEN f,GEN V,GEN T,GEN p} returns
$\kbd{f}(\kbd{x})$, assuming that \kbd{V} was computed by
$\kbd{FpXQ\_powers}(\kbd{x}, n, \kbd{T}, \kbd{p})$ for some $n > \sqrt{\deg
\kbd{f}}$.

\subsec{\kbd{FpXQX}, \kbd{FqX}}.
Contrary to what the name implies, an \kbd{FpXQX} is a \typ{POL} whose
coefficients are \kbd{Fq}s. So the only difference between \kbd{FqX} and
\kbd{FpXQX} routines is that $\kbd{T} = \kbd{NULL}$ is not allowed in the
latter. (It was thought more useful to allow \typ{INT} components than to
enforce strict consistency, which would not imply any efficiency gain.)

\subsubsec{Basic operations}

\fun{GEN}{FqX_mul}{GEN x, GEN y, GEN T, GEN p}

\fun{GEN}{FqX_Fq_mul}{GEN P, GEN U, GEN T, GEN p} multiplies the
\kbd{FqX}~\kbd{y} by the \kbd{Fq}~\kbd{x}.

\fun{GEN}{FqX_normalize}{GEN z, GEN T, GEN p} divides the \kbd{FqX}~\kbd{z}
by its leading term.

\fun{GEN}{FqX_sqr}{GEN x, GEN T, GEN p}

\fun{GEN}{FqX_divrem}{GEN x, GEN y, GEN T, GEN p, GEN *z}

\fun{GEN}{FqX_div}{GEN x, GEN y, GEN T, GEN p}

\fun{GEN}{FqX_rem}{GEN x, GEN y, GEN T, GEN p}

\fun{GEN}{FqX_gcd}{GEN P, GEN Q, GEN T, GEN p}

\fun{GEN}{FpXQX_red}{GEN z, GEN T, GEN p} \kbd{z} a \typ{POL} whose
coefficients are \kbd{ZX}s or \typ{INT}s, reduce them to \kbd{FpXQ}s.

\fun{GEN}{FpXQX_mul}{GEN x, GEN y, GEN T, GEN p}

\fun{GEN}{FpXQX_sqr}{GEN x, GEN T, GEN p}

\fun{GEN}{FpXQX_divrem}{GEN x, GEN y, GEN T, GEN p, GEN *pr}

\fun{GEN}{FpXQX_extgcd}{GEN x, GEN y, GEN T, GEN p, GEN *ptu, GEN *ptv}

\fun{GEN}{FpXQYQ_pow}{GEN x, GEN n, GEN S, GEN T, GEN p}, \kbd{x} and
\kbd{T} being \kbd{FpXQX}s, returns $\kbd{x}^\kbd{n}$ modulo \kbd{S}.

\fun{GEN}{FpXQXV_prod}{GEN V, GEN T, GEN p}, \kbd{V} being a vector of
\kbd{FpXQX}, returns their product.

\fun{GEN}{FqV_roots_to_pol}{GEN V, GEN T, GEN p, long v},
\kbd{V} being a vector of \kbd{Fq}s, returns the monic \kbd{FqX}
$\prod_i (\kbd{polx[v]} - \kbd{V[i]})$.

\subsubsec{Miscellaneous operations}

\fun{long}{FqX_is_squarefree}{GEN P, GEN T, GEN p}

\fun{GEN}{FqX_factor}{GEN x, GEN T, GEN p} same output convention as
\kbd{FpX\_factor}. Assumes \kbd{p} is prime and \kbd{T} irreducible
in $\F_p[X]$.

\fun{GEN}{Fp_factor_irred}{GEN P, GEN T, GEN p}. Assumes \kbd{p} is prime and
\kbd{T} irreducible in $\F_p[X]$. \kbd{P} being an \emph{irreducible}
\kbd{FpX}, factors it over the finite field $\F_p[Y]/(T(Y))$ and returns the
vector of irreducible \kbd{FqX}s factors (the exponents, being all equal to
$1$, are not included).

\fun{GEN}{Fp_isom}{GEN P, GEN Q, GEN p}. Assumes \kbd{p} is prime,
\kbd{P}, \kbd{Q} are \kbd{ZX}s, both irreducible mod \kbd{p}, and 
$\deg(P) \mid \deg Q$. Outputs a monomorphism between $\F_p[X]/(P)$ and
$\F_p[X]/(Q)$, as a polynomial $R$ such that $\kbd{Q} \mid \kbd{P}(R)$ in
$\F_p[X]$. If \kbd{P} and \kbd{Q} have the same degree, it is of course an
isomorphism.

\fun{GEN}{Fp_inv_isom}{GEN S, GEN T, GEN p}. Assumes \kbd{p} is prime,
\kbd{T} a \kbd{ZX}, which is irreducible modulo \kbd{p}, \kbd{S} a
\kbd{ZX} representing an automorphism of $\F_q := \F_p[X]/(\kbd{T})$.
($\kbd{S}(X)$ is the image of $X$ by the automorphism.) Returns the
inverse automorphism of \kbd{S}, in the same format, i.e.~an \kbd{FpX}~$H$
such that $H(\kbd{S}) \equiv X$ modulo $(\kbd{T}, \kbd{p})$.

\fun{long}{FqX_nbfact}{GEN u, GEN T, GEN p}.
Assumes \kbd{p} is prime and \kbd{T} irreducible in $\F_p[X]$.

\fun{long}{FqX_nbroots}{GEN f, GEN T, GEN p}
Assumes \kbd{p} is prime and \kbd{T} irreducible in $\F_p[X]$.


\subsec{\kbd{FpV}, \kbd{FpM}, \kbd{FqM}}. A \kbd{ZV} (resp.~a~\kbd{ZM}) is a
\typ{VEC} or \typ{COL} (resp.~\typ{MAT}) with \typ{INT} coefficients. An
\kbd{FpV} or \kbd{FpM}, with respect to a given \typ{INT}~\kbd{p}, is the
same with \kbd{Fp} coordinates; operations are understood over $\Z/p\Z$. An
\kbd{FqM} is a matrix with \kbd{Fq} coefficients (with respect to given
\kbd{T}, \kbd{p}), not necessarily reduced (i.e arbitrary \typ{INT}s and
\kbd{ZX}s in the same variable as \kbd{T}).

\subsubsec{Basic operations}

\fun{GEN}{FpV}{GEN z, GEN p}, \kbd{z} a \kbd{ZV}. Returns \kbd{z * Mod(1,p)},
hence with \typ{INTMOD} coefficients.

\fun{GEN}{FpM}{GEN z, GEN p}, \kbd{z} a \kbd{ZM}. Returns \kbd{z * Mod(1,p)},
hence with \typ{INTMOD} coefficients.

\fun{GEN}{FpV_red}{GEN z, GEN p}, \kbd{z} a \kbd{ZV}. Returns \kbd{lift(z *
Mod(1,p))}, which is an \kbd{FpV}

\fun{GEN}{FpM_red}{GEN z, GEN p}, \kbd{z} a \kbd{ZM}. Returns \kbd{lift(z *
Mod(1,p))}, which is an \kbd{FpM}.

\fun{GEN}{FpM_mul}{GEN x, GEN y, GEN p} multiplies the two \kbd{ZM}s~\kbd{x}
and \kbd{y} (assumed to have compatible dimensions), and reduce modulo
\kbd{p} to obtain an \kbd{FpM}.

\fun{GEN}{FpM_FpV_mul}{GEN x, GEN y, GEN p} multiplies the \kbd{ZM}~\kbd{x}
by the \kbd{ZV}~\kbd{y} (assumed to have compatible dimensions), and reduce
modulo \kbd{p} to obtain an \kbd{FpV}.

\fun{GEN}{FpC_FpV_mul}{GEN x, GEN y, GEN p} multiplies the \kbd{ZV}~\kbd{x}
(seen as a column vector) by the \kbd{ZV}~\kbd{y} (seen as a row vector,
assumed to have compatible dimensions), and reduce modulo \kbd{p} to obtain
an \kbd{FpM}.

\subsubsec{\kbd{Fp}-linear algebra}. The implementations are not
asymptotically efficient ($O(n^3)$ standard algorithms).

\fun{GEN}{FpM_deplin}{GEN x, GEN p} returns a non-trivial kernel vector,
or \kbd{NULL} if none exist.

\fun{GEN}{FpM_image}{GEN x, GEN p} as \kbd{image}

\fun{GEN}{FpM_intersect}{GEN x, GEN y, GEN p} as \kbd{intersect}

\fun{GEN}{FpM_inv}{GEN x, GEN p} returns the inverse of \kbd{x}, or
\kbd{NULL} if \kbd{x} is not invertible.

\fun{GEN}{FpM_invimage}{GEN m, GEN v, GEN p} as \kbd{inverseimage}

\fun{GEN}{FpM_ker}{GEN x, GEN p} as \kbd{ker} as \kbd{ker}

\fun{long}{FpM_rank}{GEN x, GEN p} as \kbd{rank}

\fun{GEN}{FpM_indexrank}{GEN x, GEN p} as \kbd{indexrank} but returns a 
\typ{VECSMALL} 

\fun{GEN}{FpM_suppl}{GEN x, GEN p} as \kbd{suppl}

\subsubsec{\kbd{Fq}-linear algebra}

\fun{GEN}{FqM_gauss}{GEN a, GEN b, GEN T, GEN p} as \kbd{gauss}

\fun{GEN}{FqM_ker}{GEN x, GEN T, GEN p} as \kbd{ker}

\fun{GEN}{FqM_suppl}{GEN x, GEN T, GEN p} as \kbd{suppl}


\subsec{\kbd{Flx}} Let \kbd{p} an understood \kbd{ulong}, to be given the 
the function arguments; an \kbd{Fl} is an \kbd{ulong} belonging to
$[0,\kbd{p}-1]$, an \kbd{Flx}~\kbd{z} is a \typ{VECSMALL} representing a
polynomial with small integer coefficients. Specifically \kbd{z[0]}
is the usual codeword, \kbd{z[1] = evalvarn($v$)} for some variable $v$,
then the coefficients by increasing degree. An \kbd{FlxX} is a \typ{POL}
whose coefficients are \kbd{Flx}s.

\noindent In the following, an argument called \kbd{sv} is of the form
\kbd{evalvarn}$(v)$ for some variable number~$v$.

\subsubsec{Basic operations}

\fun{GEN}{Flx_red}{GEN z, ulong p} converts from \kbd{zx} with
non-negative coefficients to \kbd{Flx} (by reducing them mod \kbd{p}).

\fun{GEN}{Flx_add}{GEN x, GEN y, ulong p}

\fun{GEN}{Flx_neg}{GEN x, ulong p}

\fun{GEN}{Flx_neg_inplace}{GEN x, ulong p}, same as \kbd{Flx\_neg}, in place
(\kbd{x} is destroyed).

\fun{GEN}{Flx_sub}{GEN x, GEN y, ulong p}

\fun{GEN}{Flx_mul}{GEN x, GEN y, ulong p}

\fun{GEN}{Flx_sqr}{GEN x, ulong p}

\fun{GEN}{Flx_divrem}{GEN x, GEN y, ulong p, GEN *pr}

\fun{GEN}{Flx_rem}{GEN x, GEN y, ulong p}

\fun{GEN}{Flx_deriv}{GEN z, ulong p}

\fun{GEN}{Flx_gcd}{GEN a, GEN b, ulong p}

\fun{GEN}{Flx_gcd_i}{GEN a, GEN b, ulong p}, same as \kbd{Flx\_gcd} without
collecting garbage.

\fun{GEN}{Flx_extgcd}{GEN a, GEN b, ulong p, GEN *ptu, GEN *ptv}

\fun{GEN}{Flx_pow}{GEN x, long n, ulong p}


\subsubsec{Miscellaneous operations}
\fun{GEN}{Flx_normalize}{GEN z, ulong p}, as \kbd{FpX\_normalize}.

\fun{GEN}{Flx_Fl_mul}{GEN y, ulong x, ulong p}

\fun{GEN}{Flx_recip}{GEN x}, returns the reciprocal polynomial 

\fun{ulong}{Flx_resultant}{GEN a, GEN b, ulong p}, returns the resultant
of \kbd{a} and \kbd{b}

\fun{ulong}{Flx_extresultant}{GEN a, GEN b, ulong p, GEN *ptU, GEN *ptV}
returns the resultant and sets Bezout coefficients (if the resultant is 0,
the latter are not set).

\fun{GEN}{Flx_invmontgomery}{GEN T, ulong p}, returns the Montgomery inverse
of \kbd{T}, i.e.~\kbd{truncate(x / polrecip(T)+O(x\pow n)}. Assumes
$\kbd{T}(0) \neq 0$.

\fun{GEN}{Flx_rem_montgomery}{GEN x, GEN mg, GEN T, ulong p}, returns
\kbd{x} modulo \kbd{T}, where \kbd{mg} is the Montgomery inverse of \kbd{T}.

\fun{GEN}{Flx_renormalize}{GEN x, long l}, as \kbd{normalizepol}, where
$\kbd{l} = \kbd{lg(x)}$, in place.

\fun{GEN}{Flx_shift}{GEN T, ulong n, ulong p}, returns \kbd{T} multiplied by $\kbd{x}^n$.

\fun{long}{Flx_valuation}{GEN x} returns the valuation of \kbd{x}, i.e. the
multiplicity of the $0$ root.

\fun{GEN}{FlxYqQ_pow}{GEN x, GEN n, GEN S, GEN T, ulong p}, as
\kbd{FpXQYQ\_pow}.


\fun{GEN}{Flx_div_by_X_x}{GEN a, ulong x, ulong p, ulong *rem}, divides
\kbd{Flx}~\kbd{a} by $X - \kbd{x}$, and sets \kbd{rem} to the remainder ($ = 
\kbd{a}(\kbd{x})$).

\fun{ulong}{Flx_eval}{GEN x, ulong y, ulong p}, as \kbd{FpX\_eval}.

\fun{GEN}{FlxV_Flv_innerprod}{GEN V, GEN W, ulong p}, as
\kbd{FpXV\_FpV\_innerprod}.

\fun{int}{Flx_is_squarefree}{GEN z, ulong p}

\fun{long}{Flx_nbfact}{GEN z, ulong p}, as \kbd{FpX\_nbfact}.

\fun{long}{Flx_nbroots}{GEN f, ulong p}, as \kbd{FpX\_nbroots}.

\fun{GEN}{Flv_polint}{GEN x, GEN y, ulong p, long sv} as \kbd{FpV\_polint},
returning an \kbd{Flx} in variable $v$.

\fun{GEN}{Flv_roots_to_pol}{GEN a, ulong p, long sv} as
\kbd{FpV\_roots\_to\_pol} returning an \kbd{Flx} in variable $v$.

\subsec{\kbd{Flxq}}. See \kbd{FpXQ} operations.

\fun{GEN}{Flxq_mul}{GEN y, GEN x, GEN pol, ulong p}

\fun{GEN}{Flxq_sqr}{GEN y, GEN pol, ulong p}

\fun{GEN}{Flxq_inv}{GEN x, GEN T, ulong p}

\fun{GEN}{Flxq_invsafe}{GEN x, GEN T, ulong p}

\fun{GEN}{Flxq_pow}{GEN x, GEN n, GEN pol, ulong p}

\fun{GEN}{Flxq_powers}{GEN x, long l, GEN T, ulong p}

\subsec{\kbd{FlxqX}}. See \kbd{FpXQX} operations.

\fun{GEN}{FlxqX_mul}{GEN x, GEN y, GEN T, ulong p}

\fun{GEN}{FlxqX_Flxq_mul}{GEN P, GEN U, GEN T, ulong p}

\fun{GEN}{FlxqX_normalize}{GEN z, GEN T, ulong p}

\fun{GEN}{FlxqX_sqr}{GEN x, GEN T, ulong p}

\fun{GEN}{FlxqX_divrem}{GEN x, GEN y, GEN T, ulong p, GEN *pr}

\fun{GEN}{FlxqX_red}{GEN z, GEN T, ulong p}

\fun{GEN}{FlxqXQ_pow}{GEN x, GEN n, GEN S, GEN T, ulong p}

\subsec{\kbd{Flv}, \kbd{Flm}}. See \kbd{FpV}, \kbd{FpM} operations.

\fun{GEN}{Flm_Flv_mul}{GEN x, GEN y, ulong p}

\fun{GEN}{Flm_deplin}{GEN x, ulong p}

\fun{GEN}{Flm_indexrank}{GEN x, ulong p}

\fun{GEN}{Flm_inv}{GEN x, ulong p}

\fun{GEN}{Flm_ker}{GEN x, ulong p}

\fun{GEN}{Flm_ker_sp}{GEN x, ulong p, long deplin}, as \kbd{Flm\_ker}, in
place (destroys~\kbd{x}).

\fun{GEN}{Flm_mul}{GEN x, GEN y, ulong p}

\subsec{Conversions involving single precision objects}

\subsubsec{To single precision}

\fun{GEN}{ZX_Flx}{GEN x, ulong p} reduce \kbd{ZX}~\kbd{x} modulo \kbd{p}
(yielding an \kbd{Flx}).

\fun{GEN}{ZV_Flv}{GEN x, ulong p} reduce \kbd{ZV}~\kbd{x} modulo \kbd{p}
(yielding an \kbd{Flv}).

\fun{GEN}{ZXV_FlxV}{GEN v, ulong p}, as \kbd{ZX\_Flx}, repeatedly called on
the vector's coefficients.

\fun{GEN}{ZXX_FlxX}{GEN B, ulong p, long v}, as \kbd{ZX\_Flx}, repeatedly
called on the polynomial's coefficients.

\fun{GEN}{ZM_Flm}{GEN x, ulong p} reduce \kbd{ZM}~\kbd{x} modulo \kbd{p}
(yielding an \kbd{Flm}).

\fun{GEN}{ZV_zv}{GEN z}, converts coefficients using \kbd{itos}

\fun{GEN}{ZM_zm}{GEN z}, converts coefficients using \kbd{itos}

\subsubsec{From single precision}

\fun{GEN}{Flx_ZX}{GEN z}, converts to \kbd{ZX} (\typ{POL} of non-negative
\typ{INT}s in this case)

\fun{GEN}{Flx_ZX_inplace}{GEN z}, same as \kbd{Flx\_ZX}, in place (\kbd{z} is
destroyed).

\fun{GEN}{FlxX_ZXX}{GEN B}, converts an \kbd{FlxX} to a polynomial with
\kbd{ZX} or \kbd{gzero} coefficients (repeated calls to \kbd{Flx\_ZX}).

\fun{GEN}{zx_ZX}{GEN z}, as \kbd{Flx\_ZX}, without assuming coefficients are
non-negative.

\fun{GEN}{Flv_ZC}{GEN z}, converts to \kbd{ZC} (\typ{COL} of non-negative
\typ{INT}s in this case)

\fun{GEN}{Flv_ZV}{GEN z}, converts to \kbd{ZV} (\typ{VEC} of non-negative
\typ{INT}s in this case)

\fun{GEN}{Flm_ZM}{GEN z}, converts to \kbd{ZM} (\typ{MAT} with non-negative
\typ{INT}s coefficients in this case)

\fun{GEN}{zv_ZC}{GEN z} as \kbd{Flv\_ZC}, without assuming coefficients are
non-negative.

\fun{GEN}{zv_ZV}{GEN z} as \kbd{Flv\_ZV}, without assuming coefficients are
non-negative.

\fun{GEN}{zm_ZM}{GEN z} as \kbd{Flm\_ZM}, without assuming coefficients are
non-negative.

\subsubsec{Mixed precision linear algebra} Assumes dimensions are compatible.
Multiply a multiprecision object by a single-precision one.

\fun{GEN}{RM_zc_mul}{GEN x, GEN y} 

\fun{GEN}{RM_zm_mul}{GEN x, GEN y}

\fun{GEN}{RV_zc_mul}{GEN x, GEN y}

\fun{GEN}{RV_zm_mul}{GEN x, GEN y}

\fun{GEN}{ZM_zc_mul}{GEN x, GEN y}

\fun{GEN}{ZM_zm_mul}{GEN x, GEN y}

\subsubsec{Miscellaneous} 

\fun{GEN}{zero_Flx}{long sv} returns a zero \kbd{Flx} in variable $v$.

\fun{GEN}{polx_Flx}{long sv} returns the variable $v$ as degree~1~\kbd{Flx}.

\fun{GEN}{Fl_Flx}{ulong x, long sv} converts from scalar to degree~$1$
polynomial in variable $v$.

\fun{GEN}{Flx_Flv_lg}{GEN x, long n} converts from \kbd{Flx} to \kbd{Flv}
with \kbd{n} components (assumed larger than the number of coefficients of
\kbd{x}).

\fun{GEN}{Flv_Flx}{GEN x, long sv} converts from vector (coefficient array)
to (normalized) polynomial in variable $v$.

\fun{GEN}{Flm_FlxV}{GEN x, long sv} converts the colums of \kbd{Flm}~\kbd{x}
to an array of \kbd{Flx} (repeated calls to \kbd{Flv\_Flx}).

\fun{GEN}{Flm_FlxX}{GEN x, long sv,long w} converts the colums of
\kbd{Flm}~\kbd{x} to the coefficient of an \kbd{FlxX}, and normalize the
result.

\fun{GEN}{FlxV_Flm}{GEN v, long n} reverse \kbd{Flm\_FlxV}, to obtain
an \kbd{Flm} with \kbd{n} rows (repeated calls to \kbd{Flx\_Flv\_lg}).

\fun{GEN}{FlxX_Flm}{GEN v, long n} reverse \kbd{Flm\_FlxX}, to obtain
an \kbd{Flm} with \kbd{n} rows (repeated calls to \kbd{Flx\_Flv\_lg}).

\section{Operations on general PARI objects}

\subsec{Assignment}

\fun{void}{gaffsg}{long s, GEN x} assigns the \kbd{long}~\kbd{s} into the
object~\kbd{x}.

\fun{void}{gaffect}{GEN x, GEN y} assigns the object \kbd{x} into the
object~\kbd{y}.

\subsec{Conversions}

\subsubsec{Scalars}

\fun{double}{rtodbl}{GEN x} applied to a \typ{REAL}~\kbd{x},
converts \kbd{x} into a \kbd{double} if possible.

\fun{GEN}{dbltor}{double x} converts the \kbd{double} \kbd{x} into a \typ{REAL}.

\fun{double}{gtodouble}{GEN x} if \kbd{x} is a real number (not necessarily
a~\typ{REAL}), converts \kbd{x} into a \kbd{double} if possible.

\fun{long}{gtolong}{GEN x} if \kbd{x} is an integer (not necessarily
a~\typ{INT}), converts \kbd{x} into a \kbd{long} if possible.

\fun{GEN}{quadtoc}{GEN x, long l} applied to a quadratic number~\kbd{x} (type
\typ{QUAD}), converts \kbd{x} into a \typ{REAL} or \typ{COMPLEX} depending on
the sign of the discriminant of~\kbd{x}, to precision \hbox{\kbd{l} \B-bit}
words.% forbid line brk at hyphen here [GN]

\fun{GEN}{gcvtop}{GEN x, GEN p, long l} converts \kbd{x} into a \typ{PADIC}
\kbd{p}-adic number of precision~\kbd{l}.

\subsubsec{Modular objects}

\fun{GEN}{gmodulcp}{GEN x, GEN y} creates the object \kbd{\key{Mod}(x,y)} on
the PARI stack, where \kbd{x} and \kbd{y} are either both \typ{INT}s, and the
result is a \typ{INTMOD}, or \kbd{x} is a scalar or a \typ{POL} and \kbd{y} a
\typ{POL}, and the result is a \typ{POLMOD}.

\fun{GEN}{gmodulgs}{GEN x, long y} same as \key{gmodulcp} except \kbd{y} is a
\kbd{long}.

\fun{GEN}{gmodulss}{long x, long y} same as \key{gmodulcp} except both
\kbd{x} and \kbd{y} are \kbd{long}s.

\fun{GEN}{gmodulo}{GEN x, GEN y} same as \key{gmodulcp} except that the
modulus \kbd{y} is cloned onto the heap, not copied onto the PARI stack.

\subsubsec{Other types}

\fun{GEN}{greffe}{GEN x, long l, int use\_stack} applied to a
\typ{POL}~\kbd{x}, creates a \typ{SER} of length~\kbd{l} starting
with~\kbd{x}, but without actually copying the coefficients, just the
pointers. If \kbd{use\_stack} is $0$, this is created through malloc, and
must be freed after use. Intended for internal use only.

\fun{GEN}{gtopoly}{GEN x, long v} converts or truncates the object~\kbd{x}
into a \typ{POL} with main variable number~\kbd{v}. A common application
would be the conversion of coefficient vectors.

\fun{GEN}{gtopolyrev}{GEN x, long v} converts or truncates the object~\kbd{x}
into a \typ{POL} with main variable number~\kbd{v}, but vectors are
converted in reverse order.

\fun{GEN}{gtoser}{GEN x, long v} converts the object~\kbd{x} into a \typ{SER}
with main variable number~\kbd{v}.

\fun{GEN}{gtocol}{GEN x} converts the object~\kbd{x} into a \typ{COL}

\fun{GEN}{gtomat}{GEN x} converts the object~\kbd{x} into a \typ{MAT}.

\fun{GEN}{gtovec}{GEN x} converts the object~\kbd{x} into a \typ{VEC}.

\fun{GEN}{gtovecsmall}{GEN x} converts the object~\kbd{x} into a
\typ{VECSMALL}.

\fun{GEN}{normalize}{GEN x} applied to an unnormalized \typ{SER}~\kbd{x}
(i.e.~type \typ{SER} with all coefficients correctly set except that \kbd{x[2]}
might be zero), normalizes \kbd{x} correctly in place. Returns~\kbd{x}.
For internal use.

\fun{GEN}{normalizepol}{GEN x} applied to an unnormalized \typ{POL}~\kbd{x}
(with all coefficients correctly set except that \kbd{x[2]} might be zero),
normalizes \kbd{x} correctly in place and returns~\kbd{x}. For internal use.

\subsec{Constructors}

\fun{GEN}{zeropadic}{GEN p, long n} creates a $0$ \typ{PADIC} equal to
$O(\kbd{p}^\kbd{n})$.

\fun{GEN}{zeroser}{long v, long n} creates a $0$ \typ{SER} in variable
\kbd{v} equal to $O(X^\kbd{n})$.

% dans j, pas dans i

\fun{GEN}{scalarser}{GEN x, long v, long prec} creates a constant \typ{SER}
in variable \kbd{v} and precision \kbd{prec}, whose constant coefficient is
(a copy of) \kbd{x}, in other words $\kbd{x} + O(\kbd{v}^\kbd{prec})$.
Assumes that \kbd{x} is non-zero.

\fun{GEN}{zeropol}{long v} creates a $0$ \typ{POL} in variable \kbd{v}.

\fun{GEN}{scalarpol}{GEN x, long v} creates a constant \typ{POL} in variable
\kbd{v}, whose constant coefficient is (a copy of) \kbd{x}.


\fun{GEN}{zerocol}{long n} creates a \typ{COL} with \kbd{n} components set to 
\kbd{gzero}.

\fun{GEN}{zerovec}{long n} creates a \typ{VEC} with \kbd{n} components set to 
\kbd{gzero}.

\fun{GEN}{vec_ei}{long n, long i} creates a \typ{VEC} with \kbd{n} components
set to \kbd{gzero}, but the \kbd{i}-th one which us set to \kbd{gun}
(\kbd{i}-th vector in the canonical basis).

\fun{GEN}{vec_Cei}{long n, long i} creates a \typ{VEC} with \kbd{n} components
set to \kbd{gzero}, but the \kbd{i}-th one which us set to \kbd{C}. \emph{No
copy is made}, so if \kbd{C} is not a permanent object, the result is not
suitable for \kbd{gerepileupto}.

\fun{GEN}{zeromat}{long m, long n} creates a \typ{MAT} with \kbd{m} x \kbd{n}
components set to \kbd{gzero}.

\smallskip

\fun{GEN}{coefs_to_int}{long n, ...} returns the non-negative \typ{INT} whose
development in base $2^{32}$ is given by the following $n$ words
(\kbd{unsigned long}). It is assumed that all such arguments are less than
$2^{32}$ (the actual \kbd{sizeof(long)} is irrelevant, the behaviour is also
as above on $64$-bit machines).
\bprog
  coefs_to_int(3, a2, a1, a0);
@eprog
\noindent returns $a_2 2^{64} + a_1 2^{32} + a_0$.

\fun{GEN}{coefs_to_pol}{long n, ...} Returns the \typ{POL} whose $n$
coefficients (\kbd{GEN}) follow, in order of decreasing degree.
\bprog
  coefs_to_pol(3, gun, gdeux, gzero);
@eprog
\noindent returns the polynomial $X^2 + 2X$ (in variable $0$, use
\tet{setvarn} if you want other variable numbers). Beware that $n$ is the
number of coefficients, hence \emph{one more} than the degree.

\fun{GEN}{coefs_to_vec}{long n, ...} returns the \typ{VEC} whose $n$
coefficients (\kbd{GEN}) follow.

\fun{GEN}{coefs_to_col}{long n, ...} returns the \typ{COL} whose $n$
coefficients (\kbd{GEN}) follow.

\misctitle{Warning}: Contrary to the policy of general PARI functions, the
latter three functions do \emph{not} copy their arguments, nor do they produce
an object a priori suitable for \tet{gerepileupto}.

\subsec{Integer parts}

\fun{GEN}{gfloor}{GEN x} creates the floor of~\kbd{x}, i.e.\ the (true)
integral part.

\fun{GEN}{gfrac}{GEN x} creates the fractional part of~\kbd{x}, i.e.\ \kbd{x}
minus the floor of~\kbd{x}.

\fun{GEN}{gceil}{GEN x} creates the ceiling of~\kbd{x}.

\fun{GEN}{ground}{GEN x} rounds towards~$+\infty$ the components of \kbd{x}
to the nearest integers.

\fun{GEN}{grndtoi}{GEN x, long *e} same as \kbd{ground}, but in addition sets
\kbd{*e} to the binary exponent of $x - \kbd{ground}(x)$. If this is
positive, all significant bits are lost. This kind of situation raises an
error message in \key{ground} but not in \key{grndtoi}.

\fun{GEN}{gtrunc}{GEN x} truncates~\kbd{x}. This is the false integer part
if \kbd{x} is a real number (i.e.~the unique integer closest to \kbd{x} among
those between 0 and~\kbd{x}). If \kbd{x} is a \typ{SER}, it is truncated
to a \typ{POL}; if \kbd{x} is a \typ{RFRAC}, this takes the polynomial part.

\fun{GEN}{gcvtoi}{GEN x, long *e} same as \key{grndtoi} except that
rounding is replaced by truncation.

\subsec{Valuation and shift}

\fun{GEN}{gshift[z]}{GEN x, long n[, GEN z]} yields the result of shifting
(the components of) \kbd{x} left by \kbd{n} (if \kbd{n} is non-negative)
or right by $-\kbd{n}$ (if \kbd{n} is negative). Applies only to \typ{INT}
and vectors/matrices of such. For other types, it is simply multiplication
by~$2^{\kbd{n}}$.

\fun{GEN}{gmul2n[z]}{GEN x, long n[, GEN z]} yields the product of \kbd{x}
and~$2^{\kbd{n}}$. This is different from \kbd{gshift} when \kbd{n} is negative
and \kbd{x} is a \typ{INT}: \key{gshift} truncates, while \key{gmul2n}
creates a fraction if necessary.

\fun{long}{ggval}{GEN x, GEN p} returns the greatest exponent~$e$ such that
$\kbd{p}^e$ divides~\kbd{x}, when this makes sense.

\fun{long}{gval}{GEN x, long v} returns the highest power of the variable
number \kbd{v} dividing the \typ{POL}~\kbd{x}.

\fun{long}{polvaluation}{GEN P, GEN *z} returns the valuation $v$ of the
\typ{POL}~\kbd{P} with respect to its main variable $X$. Check whether
coefficients are $0$ using \kbd{gcmp0}. If \kbd{z} is non-\kbd{NULL}, set it
to $\kbd{P} / X^v$.

\fun{long}{polvaluation_inexact}{GEN P, GEN *z} as \kbd{polvaluation}
but use \kbd{isexactzero} instead of \kbd{gcmp0}.

\fun{long}{ZX_valuation}{GEN P, GEN *z} as \kbd{polvaluation}, but assumes
\kbd{P} has \typ{INT} coefficients.

\subsec{Comparison operators}

\fun{int}{isexactzero}{GEN x} returns 1 (true) if \kbd{x} is exactly equal
to~0, 0~(false) otherwise. Note that many PARI functions will return a
pointer to \key{gzero} when they are aware that the result they return is an
exact zero, so it is almost always faster to test for pointer equality first,
and call \key{isexactzero} (or \key{gcmp0}) only when the first test fails.

\fun{int}{gcmp0}{GEN x} returns 1 (true) if \kbd{x} is equal to~0, 0~(false)
otherwise.

\fun{int}{gcmp1}{GEN x} returns 1 (true) if \kbd{x} is equal to~1, 0~(false)
otherwise.

\fun{int}{gcmp_1}{GEN x} returns 1 (true) if \kbd{x} is equal to~$-1$,
0~(false) otherwise.

\fun{long}{gcmp}{GEN x, GEN y} comparison of \kbd{x} with \kbd{y} (returns
the sign of $\kbd{x}-\kbd{y}$).

\fun{long}{gcmpsg}{long s, GEN x} comparison of the \kbd{long}~\kbd{s}
with~\kbd{x}.

\fun{long}{gcmpgs}{GEN x, long s} comparison of \kbd{x} with the
\kbd{long}~\kbd{s}.

\fun{long}{lexcmp}{GEN x, GEN y} comparison of \kbd{x} with \kbd{y} for the
lexicographic ordering.

\fun{long}{gegal}{GEN x, GEN y} returns 1 (true) if \kbd{x} is equal
to~\kbd{y}, 0~otherwise. A priori, this makes sense only if \kbd{x} and
\kbd{y} have the same type. When the types are different, a \kbd{true} result
means that \kbd{x - y} was successfully computed and found equal to $0$
(by \kbd{gcmp0}). In particular
\bprog
  gegal(cgetg(1, t_VEC), gzero)
@eprog\noindent is true, and the relation is not transitive. E.g.~an empty
\typ{COL} and an empty \typ{VEC} are not equal but are both equal to
\kbd{gzero}.

\fun{long}{gegalsg}{long s, GEN x} returns 1 (true) if the \kbd{long}~\kbd{s}
is equal to~\kbd{x}, 0~otherwise.

\fun{long}{gegalgs}{GEN x, long s} returns 1 (true) if \kbd{x} is equal to
the \kbd{long}~\kbd{s}, 0~otherwise.

\fun{long}{iscomplex}{GEN x} returns 1 (true) if \kbd{x} is a complex number
(of component types embeddable into the reals) but is not itself real, 0~if
\kbd{x} is a real (not necessarily of type \typ{REAL}), or raises an error if
\kbd{x} is not embeddable into the complex numbers.

\fun{long}{ismonome}{GEN x} returns 1 (true) if \kbd{x} is a non-zero
monomial in its main variable, 0~otherwise.

\subsec{Generic unary operators}

\funno{GEN}{gneg[\key{z}]}{GEN x[, GEN z]} yields $-\kbd{x}$.

\funno{GEN}{gabs[\key{z}]}{GEN x[, GEN z]} yields $|\kbd{x}|$.

\fun{GEN}{gsqr}{GEN x} creates the square of~\kbd{x}.

\fun{GEN}{ginv}{GEN x} creates the inverse of~\kbd{x}.

\subsec{Divisibility, Euclidean division}

\fun{GEN}{gdivexact}{GEN x, GEN y} returns the quotient $\kbd{x} / \kbd{y}$,
assuming $\kbd{y}$ divides $\kbd{x}$.

\fun{int}{gdvd}{GEN x, GEN y}  returns 1 (true) if \kbd{y} divides~\kbd{x},
0~otherwise.

\fun{GEN}{gdiventres}{GEN x, GEN y} creates a 2-component vertical
vector whose components are the true Euclidean quotient and remainder
of \kbd{x} and~\kbd{y}.

\fun{GEN}{gdivent[z]}{GEN x, GEN y[, GEN z]} yields the true Euclidean
quotient of \kbd{x} and the \typ{INT} or \typ{POL}~\kbd{y}.

\fun{GEN}{gdiventsg[z]}{long s, GEN y[, GEN z]}, as \kbd{gdivent}
except that \kbd{x} is a \kbd{long}.

\fun{GEN}{gdiventgs[z]}{GEN x, long s[, GEN z]}, as \kbd{gdivent}
except that \kbd{y} is a \kbd{long}.

\fun{GEN}{gmod[z]}{GEN x, GEN y[, GEN z]} yields the true remainder of \kbd{x}
modulo the \typ{INT} or \typ{POL}~\kbd{y}. A \typ{REAL} or \typ{FRAC} \kbd{y}
is also allowed, in which case the remainder is the unique real $r$ such that
$0 \leq r < |\kbd{y}|$ and $\kbd{y} = q\kbd{x} + r$ for some (in fact unique)
integer $q$.

\fun{GEN}{gmodsg[z]}{long s, GEN y[, GEN z]} as \kbd{gmod}, except \kbd{x} is
a \kbd{long}.

\fun{GEN}{gmodgs[z]}{GEN x, long s[, GEN z]} as \kbd{gmod}, except \kbd{y} is
a \kbd{long}.

\fun{GEN}{gdivmod}{GEN x, GEN y, GEN *r} If \kbd{r} is not equal to
\kbd{NULL} or \kbd{ONLY\_REM}, creates the (false) Euclidean quotient of
\kbd{x} and~\kbd{y}, and puts (the address of) the remainder into~\kbd{*r}.
If \kbd{r} is equal to \kbd{NULL}, do not create the remainder, and if
\kbd{r} is equal to \kbd{ONLY\_REM}, create and output only the remainder.
The remainder is created after the quotient and can be disposed of
individually with a \kbd{cgiv(r)}.

\fun{GEN}{poldivrem}{GEN x, GEN y, GEN *r} same as \key{gdivmod} but
specifically for \typ{POL}s~\kbd{x} and~\kbd{y}, not necessarily in the same
variable. Either of \kbd{x} and \kbd{y} may also be scalars (treated as
polynomials of degree $0$)

\fun{GEN}{poldivrem_i}{GEN x, GEN y, GEN *r, long v} same as \key{poldivrem}
for \typ{POL}s \kbd{x} and \kbd{y} in the same variable \kbd{v}

\fun{GEN}{gdeuc}{GEN x, GEN y} creates the Euclidean quotient of the
\typ{POL}s~\kbd{x} and~\kbd{y}. Either of \kbd{x} and \kbd{y} may also be
scalars (treated as polynomials of degree $0$)

\fun{GEN}{grem}{GEN x, GEN y} creates the Euclidean remainder of the
\typ{POL}~\kbd{x} divided by the \typ{POL}~\kbd{y}.

\fun{GEN}{gdivround}{GEN x, GEN y} if \kbd{x} and \kbd{y} are \typ{INT},
as \kbd{diviiround}. Operate componentwise if \kbd{x} is
a \typ{COL}, \typ{VEC} or \typ{MAT}. Otherwise as \key{gdivent}.

\fun{GEN}{centermod_i}{GEN x, GEN y, GEN y2}, as \kbd{centermodii},
componentwise.

\fun{GEN}{centermod}{GEN x, GEN y}, as \kbd{centermod\_i}, except that
\kbd{y2} is computed (and left on the stack for efficiency).

\fun{GEN}{ginvmod}{GEN x, GEN y} creates the inverse of \kbd{x} modulo \kbd{y}
when it exists. \kbd{y} must be of type \typ{INT} (in which case \kbd{x} is
of type \typ{INT}) or \typ{POL} (in which case \kbd{x} is either a scalar
type or a \typ{POL}).

\subsec{GCD, content and primitive part}

\fun{GEN}{subres}{GEN x, GEN y} creates the resultant of the \typ{POL}s
\kbd{x} and~\kbd{y} computed using the subresultant algorithm. Either of
\kbd{x} and \kbd{y} may also be scalars (treated as polynomials of degree
$0$)

\fun{GEN}{ggcd}{GEN x, GEN y} creates the GCD of \kbd{x} and~\kbd{y}.

\fun{GEN}{glcm}{GEN x, GEN y} creates the LCM of \kbd{x} and~\kbd{y}.

\fun{GEN}{gbezout}{GEN x,GEN y, GEN *u,GEN *v} creates the GCD of \kbd{x}
and~\kbd{y}, and puts (the addresses of) objects $u$ and~$v$ such that
$u\kbd{x}+v\kbd{y}=\gcd(\kbd{x},\kbd{y})$ into \kbd{*u} and~\kbd{*v}.

\fun{GEN}{bezoutpol}{GEN a,GEN b, GEN *u,GEN *v}, returns the GCD $d$ of
\typ{INT}s \kbd{a} and \kbd{b} and sets \kbd{u}, \kbd{v} to the Bezout
coefficients such that $\kbd{au} + \kbd{bv} = d$.

\fun{GEN}{content}{GEN x} creates the GCD of all the components of~\kbd{x}.

\fun{GEN}{primitive_part}{GEN x, GEN *c}, sets \kbd{c} to \kbd{content(x)}
and returns the primitive part \kbd{x} / \kbd{c}.

\fun{GEN}{primpart}{GEN x} as \kbd{primitive\_part} but the content is lost.
(For efficiency, the content remains on the stack.)

\subsec{Generic binary operators}. Let ``\op'' be a binary operation among

\op=\key{add}: addition (\kbd{x + y}).

\op=\key{sub}: subtraction (\kbd{x - y}). 

\op=\key{mul}: multiplication (\kbd{x * y}).

\op=\key{div}: division (\kbd{x / y}).

\op=\key{max}: maximum (\kbd{max(x, y)})

\op=\key{min}: minimum (\kbd{min(x, y)})

\noindent The names and prototypes of the functions corresponding
to \op\ are as follows:

\funno{GEN}{g\op[z]}{GEN x, GEN y[, GEN z]}

\funno{GEN}{g\op gs[z]}{GEN x, long s[, GEN z]}

\funno{GEN}{g\op sg[z]}{long s, GEN y[, GEN z]}

\fun{GEN}{gpow}{GEN x, GEN y, long l} creates $\kbd{x}^{\kbd{y}}$. If
\kbd{y} is a \typ{INT}, return \kbd{powgi(x,y)} (the precision \kbd{l} is not
taken into account). Otherwise, the result is $\exp(\kbd{y}*\log(\kbd{x}))$
computed to precision~\kbd{l}.

\fun{GEN}{gpowgs}{GEN x, long n} creates $\kbd{x}^{\kbd{n}}$ using
binary powering.

\fun{GEN}{powgi}{GEN x, GEN y} creates $\kbd{x}^{\kbd{y}}$, where \kbd{y} is a
\typ{INT}, using left-shift binary powering.


\fun{GEN}{gsubst}{GEN x, long v, GEN y} substitutes the object \kbd{y}
into~\kbd{x} for the variable number~\kbd{v}.

\vfill\eject
