\chapter{Technical Reference Guide for Low-Level Functions}
\def\B{\kbd{BIL}}
\def\op{{\it op\/}}
\def\fun#1#2#3{\noindent{\tt #1 \key{#2}(#3)}:}

In this chapter, we give a description all public low-level functions of the
PARI system. These essentially include functions for handling all the PARI
types. Higher level functions, such as arithmetic or transcendental
functions, are described fully in Chapter~3 of this manual.

Many other undocumented functions can be found throughout the source code.
These private functions are more efficient than the library functions that
call them, but much sloppier on argument checking and damage control. Use
them at your own risk!

\section{Level 0 kernel (operations on unsigned longs)}

\noindent
For the non-68k versions, we need level 0 operations simulating basic
operations of the 68020 processor (on which PARI was originally
implemented). The type \tet{ulong} is defined in the file \kbd{parigen.h} as
\kbd{unsigned long}. Note that in the prototypes below a \kbd{ulong} is
sometimes implicitly typecast to \kbd{int} or \kbd{long}.

The global \kbd{ulong} variables \kbd{overflow} (which will contain
only 0 or 1) and \kbd{hiremainder} used to be declared in the file
\kbd{pariinl.h}. However, for certain architectures they are no longer
needed, and/or have been replaced with local variables for efficiency;
and the `functions' mentioned below are really chunks of assembler code which
will be inlined at each invocation by the compiler. If you really need to
use these lowest-level operations directly, make sure you know your way
through the PARI kernel sources, and understand the architecture dependencies.

To make the following descriptions valid both for 32-bit and 64-bit
machines, we will set \B\ to be equal to 32 (resp.~64), an abbreviation of
\kbd{BITS\_IN\_LONG}, which is what is actually used in the source code.

\fun{int}{addll}{int x, int y} adds the ulongs \kbd{x} and \kbd{y},
returns the lower \B\ bits and puts the carry bit into \kbd{overflow}.

\fun{int}{addllx}{int x, int y} adds \kbd{overflow} to the sum of the
ulongs \kbd{x} and \kbd{y}, returns the lower \B\ bits and puts the
carry bit into \kbd{overflow}.

\fun{int}{subll}{int x, int y} subtracts the ulongs \kbd{x} and \kbd{y},
returns the lower \B\ bits and put the carry (borrow) bit into \kbd{overflow}.

\fun{int}{subllx}{int x, int y} subtracts \kbd{overflow} from the
difference of the ulongs \kbd{x} and \kbd{y}, returns the lower \B\ bits
and puts the carry (borrow) bit into \kbd{overflow}.

\fun{int}{shiftl}{ulong x, ulong y} shifts the ulong \kbd{x} left by \kbd{y}
bits, returns the lower \B\ bits and stores the high-order \B\ bits into
\kbd{hiremainder}. We must have $1\le\kbd{y}\le\B$. In particular, \kbd{y}
must be non-zero; the caller is responsible for testing this.

\fun{int}{shiftlr}{ulong x, ulong y} shifts the ulong \kbd{x << \B} right
by \kbd{y} bits, returns the higher \B\ bits and stores the low-order
\B\ bits into \kbd{hiremainder}. We must have $1\le\kbd{y}\le\B$. In
particular, \kbd{y} must be non-zero.

% break ugly ff ligatures in bfffo
\fun{int}{b{}f{}f{}fo}{ulong x} returns the number of leading zero bits in the
ulong \kbd{x} (i.e. the number of bit positions by which it would have to be
shifted left until its leftmost bit first becomes equal to~1, which can be
between 0 and $\B-1$ for nonzero \kbd{x}). When \kbd{x} is~0, \B\ is returned.

\fun{int}{mulll}{ulong x, ulong y} multiplies the ulong \kbd{x} by the ulong
\kbd{y}, returns the lower \B\ bits and stores the high-order \B\ bits into
\kbd{hiremainder}.

\fun{int}{addmul}{ulong x, ulong y} adds \kbd{hiremainder} to the product
of the ulongs \kbd{x} and \kbd{y}, returns the lower \B\ bits and stores the
high-order \B\ bits into \kbd{hiremainder}.

\fun{int}{divll}{ulong x, ulong y} returns the Euclidean quotient of
(\kbd{hiremainder << \B})${}+{}$\kbd{x} and the ulong divisor \kbd{y} and
stores the remainder into \kbd{hiremainder}. An error occurs if the quotient
cannot be represented by a ulong, i.e.~if $\kbd{hiremainder}\ge\kbd{y}$
initially.

\section{Level 1 kernel (operations on longs, integers and reals)}

\noindent
In this section as elsewhere, \kbd{long} denotes a \B-bit signed C-integer,
``integer'' denotes a PARI multiprecise integer (type \typ{INT}), ``real''
denotes a PARI multiprecise real (type \typ{REAL}). Refer to Chapters
1--2 and~4 for general background.

\misctitle{Note:} Many functions consist of an elementary operation,
immediately followed by an assignment statement. All such functions are
obtained using macros (see the file \kbd{paricom.h}), hence you can easily
extend the list. Below, they will be introduced like in the following
example:

\fun{GEN}{gadd[z]}{GEN x, GEN y[, GEN z]} followed by the explicit
description of the function

\kbd{GEN \key{gadd}(GEN x, GEN y)}

\noindent which creates its result on the stack, returning a \kbd{GEN} pointer
to it, and the parts in brackets indicate that there exists also a function

\kbd{void \key{gaddz}(GEN x, GEN y, GEN z)}

\noindent which assigns its result to the pre-existing object
\kbd{z}, leaving the stack unchanged.

\subsec{Basic unit and subunit handling functions}

\fun{long}{typ}{GEN x} returns the type number of~\kbd{x}. (The header files
included through \kbd{pari.h} will give you access to the symbolic constants
\typ{INT} etc., so you should never need to know the actual numerical values.)

\fun{long}{lg}{GEN x} returns the length of~\kbd{x} in \B-bit words.

\fun{long}{lgef}{GEN x} returns the effective length of the polynomial \kbd{x}
in \B-bit words.

\fun{long}{lgefint}{GEN x} returns the effective length of the integer \kbd{x}
in \B-bit words.

\fun{long}{signe}{GEN x} returns the sign ($-1$, 0 or 1) of~\kbd{x}. Can be
used for integers, reals, polynomials and power series (for the last two
types, only 0 or 1 are possible).

\fun{long}{gsigne}{GEN x} same as \kbd{signe}, but also valid for rational
numbers (and marginally less efficient for the other types).

\fun{long}{expo}{GEN x} returns the unbiased binary exponent of the real
number~\kbd{x}.

\fun{long}{gexpo}{GEN x} same as \kbd{expo}, but also valid when \kbd{x}
is not a real number. When \kbd{x} is an exact~0, this returns
\hbox{\kbd{-HIGHEXPOBIT}}.

\fun{long}{expi}{GEN x} returns the binary exponent of the real number equal
to the integer~\kbd{x}. This is a special case of \kbd{gexpo} above, covering
the case where \kbd{x} is of type~\typ{INT}.

\fun{long}{valp}{GEN x} returns the unbiased 16-bit $p$-adic valuation (for
a $p$-adic) or $X$-adic valuation (for a power series, taken with respect
to the main variable) of~\kbd{x}.

\fun{long}{precp}{GEN x} returns the precision of the $p$-adic~\kbd{x}.

\fun{long}{varn}{GEN x} returns the variable number of \kbd{x} (between 0 and
\kbd{MAXVARN}). Should be used only for polynomials and power series.

\fun{long}{gvar}{(GEN x)} returns the main variable number when any variable
at all occurs in the composite object~\kbd{x} (the smallest variable number
which occurs), and \kbd{BIGINT} otherwise.

\fun{void}{settyp}{GEN x, long s} sets the type number of~\kbd{x} to~\kbd{s}.
This should be used with extreme care since usually the type is set
otherwise, and the components and further codeword fields (which are left
unchanged) may not match the PARI conventions for the new type.

\fun{void}{setlg}{GEN x, long s} sets the length of~\kbd{x} to~\kbd{s}. Again
this should be used with extreme care since usually the length is set
otherwise, and increasing the length joins previously unrelated memory words
to the root node of~\kbd{x}. This is, however, an extremely efficient way of
truncating vectors or polynomials.

\fun{void}{setlgef}{GEN x, long s} sets the effective length of \kbd{x}
to~\kbd{s}, where \kbd{x} is a polynomial. The number \kbd{s} must be less
than or equal to the length of~\kbd{x}.

\fun{void}{setlgefint}{GEN x, long s} sets the effective length
of the integer \kbd{x} to~\kbd{s}. The number \kbd{s} must be less than or
equal to the length of~\kbd{x}.

\fun{void}{setsigne}{GEN x, long s} sets the sign of~\kbd{x} to~\kbd{s}.
If \kbd{x} is an integer or real, \kbd{s} must be equal to $-1$, 0 or~1,
and if \kbd{x} is a polynomial or a power series, \kbd{s} must be equal to
0 or~1.

\fun{void}{setexpo}{GEN x, long s} sets the binary exponent of the real
number~\kbd{x} to \kbd{s}, after adding the appropriate bias. The unbiased
value \kbd{s} must be a 24-bit signed number.

\fun{void}{setvalp}{GEN x, long s} sets the $p$-adic or $X$-adic valuation
of~\kbd{x} to~\kbd{s}, if \kbd{x} is a $p$-adic or a power series,
respectively.

\fun{void}{setprecp}{GEN x, long s} sets the $p$-adic precision of the
$p$-adic number~\kbd{x} to~\kbd{s}.

\fun{void}{setvarn}{GEN x, long s} sets the variable number of the polynomial
or power series~\kbd{x} to~\kbd{s} (where $0\le \kbd{s}\le\kbd{MAXVARN}$).


\subsec{Memory allocation on the PARI stack}

\fun{GEN}{cgetg}{long n, long t} allocates memory on the PARI stack for
an object of length \kbd{n} and type~\kbd{t}, and initializes its first
codeword.

\fun{GEN}{cgeti}{long n} allocates memory on the PARI stack for an
integer of length~\kbd{n}, and initializes its first codeword. Identical to
{\tt cgetg(n,\typ{INT})}.

\fun{GEN}{cgetr}{long n} allocates memory on the PARI stack for a real
of length~\kbd{n}, and initializes its first codeword. Identical to
{\tt cgetg(n,\typ{REAL})}.

\fun{void}{cgiv}{GEN x} frees object \kbd{x} if it is the last created on the
PARI stack (otherwise disaster occurs).

\fun{GEN}{gerepile}{long p, long q, GEN x} general garbage collector
for the PARI stack. See \secref{se:garbage} for a detailed explanation and
many examples.

\subsec{Assignments, conversions and integer parts}

\fun{void}{mpaff}{GEN x, GEN z} assigns \kbd{x} into~\kbd{z} (where
\kbd{x} and \kbd{z} are integers or reals).

\fun{void}{affsz}{long s, GEN z} assigns the long \kbd{s} into the integer or
real~\kbd{z}.

\fun{void}{affsi}{long s, GEN z} assigns the long \kbd{s} into the
integer~\kbd{z}.

\fun{void}{affsr}{long s, GEN z} assigns the long \kbd{s} into the
real~\kbd{z}.

\fun{void}{affii}{GEN x, GEN z} assigns the integer \kbd{x} into the
integer~\kbd{z}.

\fun{void}{affir}{GEN x, GEN z} assigns the integer \kbd{x} into the
real~\kbd{z}.

\fun{void}{affrs}{GEN x, long s} assigns the real \kbd{x} into the
long~\kbd{s}\dots not. This is a forbidden assignment in PARI, so an error
message is issued.

\fun{void}{affri}{GEN x, GEN z} assigns the real \kbd{x} into the
integer~\kbd{z}\dots no it doesn't. This is a forbidden assignment in PARI,
so an error message is issued.

\fun{void}{affrr}{GEN x, GEN z} assigns the real \kbd{x} into the real~\kbd{z}.
\smallskip

\fun{GEN}{stoi}{long s} creates the PARI integer corresponding to the
long~\kbd{s}.

\fun{long}{itos}{GEN x} converts the PARI integer \kbd{x} to a C long (if
possible, otherwise an error message is issued).
\smallskip

\fun{GEN}{mptrunc[z]}{GEN x[, GEN z]} truncates the integer or real~\kbd{x}
(not the same as the integer part if \kbd{x} is non-integer and negative).

\fun{GEN}{mpent[z]}{GEN x[, GEN z]} true integer part of the integer or
real~\kbd{x} (i.e.~the \kbd{floor} function).

\subsec{Valuation and shift}

\fun{long}{vals}{long s} 2-adic valuation of the long~\kbd{s}. Returns $-1$
if \kbd{s} is equal to 0, with no error.

\fun{long}{vali}{GEN x} 2-adic valuation of the integer~\kbd{x}. Returns $-1$
if \kbd{s} is equal to 0, with no error.

\fun{GEN}{mpshift[z]}{GEN x, long n[, GEN z]} shifts the real or
integer \kbd{x} by~\kbd{n}. If \kbd{n} is positive, this is a left shift,
i.e.~multiplication by $2^{\kbd{n}}$. If \kbd{n} is negative, it is a right
shift by~$-\kbd{n}$, which amounts to the truncation of the quotient of \kbd{x}
by~$2^{-\kbd{n}}$.

\fun{GEN}{shifts}{long s, long n} converts the long \kbd{s} into a PARI
integer and shifts the value by~\kbd{n}.

\fun{GEN}{shifti}{GEN x, long n} shifts the integer~\kbd{x} by~\kbd{n}.

\fun{GEN}{shiftr}{GEN x, long n} shifts the real~\kbd{x} by~\kbd{n}.

\subsec{Unary operations}

\noindent
Let ``\op'' be some unary operation of type \kbd{GEN (*)(GEN)}. The names and
prototypes of the low-level functions corresponding to \op\ will be as follows.

\fun{GEN}{mp\op}{GEN x} creates the result of \op\ applied to the integer
or real~\kbd{x}.

\fun{GEN}{\op s}{long s} creates the result of \op\ applied to the
long~\kbd{s}.

\fun{GEN}{\op i}{GEN x} creates the result of \op\ applied to the
integer~\kbd{x}.

\fun{GEN}{\op r}{GEN x} creates the result of \op\ applied to the real~\kbd{x}.

\fun{GEN}{mp\op z}{GEN x, GEN z} assigns the result of applying \op\ to the
integer or real~\kbd{x} into the integer or real \kbd{z}.

\misctitle{Remark:} it has not been considered useful to include the
functions {\tt void \op sz(long,GEN)}, {\tt void \op iz(GEN,GEN)} and
{\tt void \op rz(GEN, GEN)}.
\smallskip

\noindent The above prototype schemes apply to the following operators:

\op=\key{neg}: negation ($-$\kbd{x}). The result is of the same type
as~\kbd{x}.

\op=\key{abs}: absolute value ($|\kbd{x}|$). The result is of the same type
as~\kbd{x}.

\noindent In addition, there exist the following special unary functions with
assignment:

\fun{void}{mpinvz}{GEN x, GEN z} assigns the inverse of the integer or
real \kbd{x} into the real~\kbd{z}. The inverse is computed as a quotient
of real numbers, not as a Euclidean division.

\fun{void}{mpinvsr}{long s, GEN z} assigns the inverse of the long \kbd{s}
into the real~\kbd{z}.

\fun{void}{mpinvir}{GEN x, GEN z} assigns the inverse of the integer \kbd{x}
into the real~\kbd{z}.

\fun{void}{mpinvrr}{GEN x, GEN z} assigns the inverse of the real \kbd{x} into
the real~\kbd{z}.

\subsec{Comparison operators}

\fun{long}{mpcmp}{GEN x, GEN y} compares the integer or real \kbd{x} to the
integer or real~\kbd{y}. The result is the sign of $\kbd{x}-\kbd{y}$.

\fun{long}{cmpss}{long s, long t} returns the sign of $\kbd{s}-\kbd{t}$.

\fun{long}{cmpsi}{long s, GEN x} compares the long \kbd{s} to the
integer~\kbd{x}.

\fun{long}{cmpsr}{long s, GEN x} compares the long \kbd{s} to the real~\kbd{x}.

\fun{long}{cmpis}{GEN x, long s} compares the integer \kbd{x} to the
long~\kbd{s}.

\fun{long}{cmpii}{GEN x, GEN y} compares the integer \kbd{x} to the
integer~\kbd{y}.

\fun{long}{cmpir}{GEN x, GEN y} compares the integer \kbd{x} to the
real~\kbd{y}.

\fun{long}{cmprs}{GEN x, long s} compares the real \kbd{x} to the
long~\kbd{s}.

\fun{long}{cmpri}{GEN x, GEN y} compares the real \kbd{x} to the
integer~\kbd{y}.

\fun{long}{cmprr}{GEN x, GEN y} compares the real \kbd{x} to the real~\kbd{y}.

\subsec{Binary operations}

\noindent
Let ``\op'' be some operation of type \kbd{GEN (*)(GEN,GEN)}. The names and
prototypes of the low-level functions corresponding to \op\ will be as follows.
In this section, the \kbd{z} argument in the \kbd{z}-functions must be of type
\typ{INT} or~\typ{REAL}.

\fun{GEN}{mp\op[z]}{GEN x, GEN y[, GEN z]} applies \op\ to
the integer-or-reals \kbd{x} and~\kbd{y}.

\fun{GEN}{\op ss[z]}{long s, long t[, GEN z]} applies \op\ to the longs
\kbd{s} and~\kbd{t}.

\fun{GEN}{\op si[z]}{long s, GEN x[, GEN z]} applies \op\ to the long \kbd{s}
and the integer~\kbd{x}.

\fun{GEN}{\op sr[z]}{long s, GEN x[, GEN z]} applies \op\ to the long \kbd{s}
and the real~\kbd{x}.

\fun{GEN}{\op is[z]}{GEN x, long s[, GEN z]} applies \op\ to the
integer \kbd{x} and the long~\kbd{s}.

\fun{GEN}{\op ii[z]}{GEN x, GEN y[, GEN z]} applies \op\ to the
integers \kbd{x} and~\kbd{y}.

\fun{GEN}{\op ir[z]}{GEN x, GEN y[, GEN z]} applies \op\ to the
integer \kbd{x} and the real~\kbd{y}.

\fun{GEN}{\op rs[z]}{GEN x, long s[, GEN z]} applies \op\ to the real \kbd{x}
and the long~\kbd{s}.

\fun{GEN}{\op ri[z]}{GEN x, GEN y[, GEN z]} applies \op\ to the real \kbd{x}
and the integer~\kbd{y}.

\fun{GEN}{\op rr[z]}{GEN x, GEN y[, GEN z]} applies \op\ to the reals \kbd{x}
and~\kbd{y}.
\smallskip
\noindent Each of the above can be used with the following operators.

\op=\key{add}: addition (\kbd{x + y}). The result is real unless both \kbd{x}
and \kbd{y} are integers (or longs).

\op=\key{sub}: subtraction (\kbd{x - y}). The result is real unless both
\kbd{x} and \kbd{y} are integers (or longs).

\op=\key{mul}: multiplication (\kbd{x * y}). The result is real unless both
\kbd{x} and \kbd{y} are integers (or longs), OR if \kbd{x} or \kbd{y} is the
integer or long zero.

\op=\key{div}: division (\kbd{x / y}). In the case where \kbd{x} and \kbd{y}
are both integers or longs, the result is the Euclidean quotient, where the
remainder has the same sign as the dividend~\kbd{x}. If one of \kbd{x} or
\kbd{y} is real, the result is real unless \kbd{x} is the integer or long
zero. A division-by-zero error occurs if \kbd{y} is equal to zero.

\op=\key{res}: remainder (``\kbd{x \% y}''). This operation is defined only
when \kbd{x} and \kbd{y} are longs or integers. The result is the Euclidean
remainder corresponding to \kbd{div},~i.e. its sign is that of the
dividend~\kbd{x}. The result is always an integer.

\op=\key{mod}: remainder (\kbd{x \% y}). This operation is defined only when
\kbd{x} and \kbd{y} are longs or integers. The result is the true Euclidean
remainder, i.e.~non-negative and less than the absolute value of~\kbd{y}.

\subsec{Division with remainder}: the following functions return two objects,
unless specifically asked for only one of them~--- a quotient and a remainder.
The remainder will be created on the stack, and a \kbd{GEN} pointer to this
object will be returned through the variable whose address is passed as the
\kbd{r} argument.

\fun{GEN}{dvmdss}{long s, long t, GEN *r} creates the Euclidean
quotient and remainder of the longs \kbd{s} and~\kbd{t}. If \kbd{r} is not
\kbd{NULL} or \kbd{ONLY\_REM}, this puts the remainder into \kbd{*r},
and returns the quotient. If \kbd{r} is equal to \kbd{NULL}, only the
quotient is returned. If \kbd{r} is equal to \kbd{ONLY\_REM}, the remainder
is returned instead of the quotient. In the generic case, the remainder is
created after the quotient and can be disposed of individually with a
\kbd{cgiv(r)}. The remainder is always of the sign of the dividend~\kbd{s}.

\fun{GEN}{dvmdsi}{long s, GEN x, GEN *r} creates the Euclidean
quotient and remainder of the long \kbd{s} by the integer~\kbd{x}.
Obeys the same conventions with respect to~\kbd{r}.

\fun{GEN}{dvmdis}{GEN x, long s, GEN *r} create the Euclidean
quotient and remainder of the integer x by the long~s.

\fun{GEN}{dvmdii}{GEN x, GEN y, GEN *r} returns the Euclidean quotient
of the integer \kbd{x} by the integer \kbd{y} and puts the remainder
into~\kbd{*r}. If \kbd{r} is equal to \kbd{NULL}, the remainder is not
created, and if \kbd{r} is equal to  \kbd{ONLY\_REM}, only the remainder
is created and returned. In the generic case, the remainder is created
after the quotient and can be disposed of individually with a \kbd{cgiv(r)}.
The remainder is always of the sign of the dividend~\kbd{x}.

\fun{GEN}{truedvmdii}{GEN x, GEN y, GEN *r}, as \kbd{dvmdii} but with a
non-negative remainder.

\fun{void}{mpdvmdz}{GEN x, GEN y, GEN z, GEN *r} assigns the Euclidean
quotient of the integers \kbd{x} and \kbd{y} into the integer or real~\kbd{z},
putting the remainder into~\kbd{*r} (unless \kbd{r} is equal to \kbd{NULL} or
\kbd{ONLY\_REM} as above).

\fun{void}{dvmdssz}{long s, long t, GEN z, GEN *r} assigns the Euclidean
quotient of the longs \kbd{s} and \kbd{t} into the integer or real~\kbd{z},
putting the remainder into~\kbd{*r} (unless \kbd{r} is equal to \kbd{NULL} or
\kbd{ONLY\_REM} as above).

\fun{void}{dvmdsiz}{long s, GEN x, GEN z, GEN *r} assigns the Euclidean
quotient of the long \kbd{s} and the integer \kbd{x} into the integer or
real~\kbd{z}, putting the remainder into \kbd{*r} (unless \kbd{r} is equal
to \kbd{NULL} or \kbd{ONLY\_REM} as above).

\fun{void}{dvmdisz}{GEN x, long s, GEN z, GEN *r} assigns the Euclidean
quotient of the integer \kbd{x} and the long \kbd{s} into the integer or
real~\kbd{z}, putting the remainder into~\kbd{*r} (unless \kbd{r} is equal
to \kbd{NULL} or \kbd{ONLY\_REM} as above).

\fun{void}{dvmdiiz}{GEN x, GEN y, GEN z, GEN *r} assigns the Euclidean
quotient of the integers \kbd{x} and \kbd{y} into the integer or real~\kbd{z},
putting the address of the remainder into~\kbd{*r} (unless \kbd{r} is equal
to \kbd{NULL} or \kbd{ONLY\_REM} as above).

\subsec{Miscellaneous functions}

\fun{void}{addsii}{long s, GEN x, GEN z} assigns the sum of the long \kbd{s}
and the integer \kbd{x} into the integer~\kbd{z} (essentially identical to
\kbd{addsiz} except that \kbd{z} is specifically an integer).

\fun{long}{divise}{GEN x, GEN y} if the integer \kbd{y} divides the
integer~\kbd{x}, returns 1 (true), otherwise returns 0 (false).

\fun{long}{divisii}{GEN x, long s, GEN z} assigns the Euclidean quotient of
the integer \kbd{x} and the long \kbd{s} into the integer \kbd{z}, and returns
the remainder as a long.

\fun{long}{mpdivis}{GEN x, GEN y, GEN z} if the integer \kbd{y} divides the
integer~\kbd{x}, assigns the quotient to the integer~\kbd{z} and returns
1 (true), otherwise returns 0 (false).

\fun{void}{mulsii}{long s, GEN x, GEN z} assigns the product of the long
\kbd{s} and the integer \kbd{x} into the integer~\kbd{z} (essentially
dentical to \kbd{mulsiz} except that \kbd{z} is specifically an integer).

\section{Level 2 kernel (operations on general PARI objects)}

\noindent The functions available to handle subunits are the following.

\fun{GEN}{compo}{GEN x, long n} creates a copy of the \kbd{n}-th true
component (i.e.\ not counting the codewords) of the object~\kbd{x}.

\fun{GEN}{truecoeff}{GEN x, long n} creates a copy of the coefficient of
degree~\kbd{n} of~\kbd{x} if \kbd{x} is a scalar, polynomial or power series,
and otherwise of the \kbd{n}-th component of~\kbd{x}.

\noindent % borderline case -- looks better like this [GN]
The remaining two are macros, NOT functions (see \secref{se:typecast} for a
detailed explanation):

\fun{long}{coeff}{GEN x, long i, long j} applied to a matrix \kbd{x} (type
\typ{MAT}), this gives the address of the coefficient at row \kbd{i} and
column~\kbd{j} of~\kbd{x}.

\fun{long}{mael$n$}{GEN x, long $a_1$, ..., long $a_n$} stands for
\kbd{x[$a_1$][$a_2$]...[$a_n$]}, where $2\le n \le 5$, with all the
necessary typecasts.

\subsec{Copying and conversion}

\fun{GEN}{cgetp}{GEN x} creates space sufficient to hold the $p$-adic~\kbd{x},
and sets the prime $p$ and the $p$-adic precision to those of~\kbd{x}, but
does not copy (the $p$-adic unit or zero representative and the modulus
of)~\kbd{x}.

\fun{GEN}{gcopy}{GEN x} creates a new copy of the object~\kbd{x} on the PARI
stack. For permanent subobjects, only the pointer is copied.

\fun{GEN}{forcecopy}{GEN x} same as \key{copy} except that even permanent
subobjects are copied onto the stack.

\fun{long}{taille}{GEN x} returns the total number of \B-bit words occupied
by the tree representing~\kbd{x}.

\fun{GEN}{gclone}{GEN x} creates a new permanent copy of the object \kbd{x}
on the heap.

\fun{GEN}{greffe}{GEN x, long l, int use\_stack} applied to a
polynomial~\kbd{x} (type \typ{POL}), creates a power series (type \typ{SER})
of length~\kbd{l} starting with~\kbd{x}, but without actually copying the
coefficients, just the pointers. If \kbd{use\_stack} is zero, this is created
through malloc, and must be freed after use. Intended for internal use only.

\fun{double}{rtodbl}{GEN x} applied to a real~\kbd{x} (type \typ{REAL}),
converts \kbd{x} into a C double if possible.

\fun{GEN}{dbltor}{double x} converts the C double \kbd{x} into a PARI real.

\fun{double}{gtodouble}{GEN x} if \kbd{x} is a real number (but not
necessarily of type \typ{REAL}), converts \kbd{x} into a C double if possible.

\fun{long}{gtolong}{GEN x} if \kbd{x} is an integer (not a C long,
but not necessarily of type \typ{INT}), converts \kbd{x} into a C long
if possible.

\fun{GEN}{gtopoly}{GEN x, long v} converts or truncates the object~\kbd{x}
into a polynomial with main variable number~\kbd{v}. A common application
would be the conversion of coefficient vectors.

\fun{GEN}{gtopolyrev}{GEN x, long v} converts or truncates the object~\kbd{x}
into a polynomial with main variable number~\kbd{v}, but vectors are converted
in reverse order.

\fun{GEN}{gtoser}{GEN x, long v} converts the object~\kbd{x} into a power
series with main variable number~\kbd{v}.

\fun{GEN}{gtovec}{GEN x} converts the object~\kbd{x} into a (row) vector.

\fun{GEN}{co8}{GEN x, long l} applied to a quadratic number~\kbd{x}
(type \typ{QUAD}), converts \kbd{x} into a real or complex number
depending on the sign of the discriminant of~\kbd{x}, to precision
\hbox{\kbd{l} \B-bit} words.% absolutely forbid line brk at hyphen here [GN]

\fun{GEN}{gcvtop}{GEN x, GEN p, long l} converts \kbd{x} into a \kbd{p}-adic
number of precision~\kbd{l}.

\fun{GEN}{gmodulcp}{GEN x, GEN y} creates the object \kbd{\key{Mod}(x,y)}
on the PARI stack, where \kbd{x} and \kbd{y} are either both integers, and
the result is an integermod (type \typ{INTMOD}), or \kbd{x} is a scalar or
a polynomial and \kbd{y} a polynomial, and the result is a polymod
(type \typ{POLMOD}).

\fun{GEN}{gmodulgs}{GEN x, long y} same as \key{gmodulcp} except \kbd{y} is a
\kbd{long}.

\fun{GEN}{gmodulss}{long x, long y} same as \key{gmodulcp} except both \kbd{x}
and \kbd{y} are \kbd{long}s.

\fun{GEN}{gmodulo}{GEN x, GEN y} same as \key{gmodulcp} except that the
modulus \kbd{y} is copied onto the heap and not onto the PARI stack.

\fun{long}{gexpo}{GEN x} returns the binary exponent of \kbd{x} or the maximal
binary exponent of the coefficients of~\kbd{x}. Returns
\hbox{\kbd{-HIGHEXPOBIT}} if \kbd{x} has no components or is an exact zero.

\fun{long}{gsize}{GEN x} returns 0 if \kbd{x} is exactly~0. Otherwise,
returns \kbd{\key{gexpo}(x)} multiplied by $\log_{10}(2)$. This gives a
crude estimate for the maximal number of decimal digits of the components
of~\kbd{x}.

\fun{long}{gsigne}{GEN x} returns the sign of~\kbd{x} ($-1$, 0 or 1) when
\kbd{x} is an integer, real or (irreducible or reducible) fraction. Raises
an error for all other types.

\fun{long}{gvar}{GEN x} returns the main variable of~\kbd{x}. If no component
of~\kbd{x} is a polynomial or power series, this returns \kbd{BIGINT}.

\fun{int}{precision}{GEN x} If \kbd{x} is of type \typ{REAL}, returns the
precision of~\kbd{x} (the length of \kbd{x} in \B-bit words if \kbd{x} is
not zero, and a reasonable quantity obtained from the exponent of \kbd{x}
if \kbd{x} is numerically equal to zero). If \kbd{x} is of type \typ{COMPLEX},
returns the minimum of the precisions of the real and imaginary part.
Otherwise, returns~0 (which stands in fact for infinite precision).

\subsec{Comparison operators and valuations}

\fun{int}{gcmp0}{GEN x} returns 1 (true) if \kbd{x} is equal to~0, 0~(false)
otherwise.

\fun{int}{isexactzero}{GEN x} returns 1 (true) if \kbd{x} is exactly equal
to~0, 0~(false) otherwise. Note that many PARI functions will return a
pointer to \key{gzero} when they are aware that the result they return is
an exact zero, so it is almost always faster to test for pointer equality
first, and call \key{isexactzero} (or \key{gcmp0}) only when the first
test fails.

\fun{int}{gcmp1}{GEN x} returns 1 (true) if \kbd{x} is equal to~1, 0~(false)
otherwise.

\fun{int}{gcmp\_1}{GEN x} returns 1 (true) if \kbd{x} is equal to~$-1$,
0~(false) otherwise.

\fun{long}{gcmp}{GEN x, GEN y} comparison of \kbd{x} with \kbd{y} (returns
the sign of $\kbd{x}-\kbd{y}$).

\fun{long}{gcmpsg}{long s, GEN x} comparison of the long \kbd{s} with~\kbd{x}.

\fun{long}{gcmpgs}{GEN x, long s} comparison of \kbd{x} with the long~\kbd{s}.

\fun{long}{lexcmp}{GEN x, GEN y} comparison of \kbd{x} with \kbd{y} for the
lexicographic ordering.

\fun{long}{gegal}{GEN x, GEN y} returns 1 (true) if \kbd{x} is equal
to~\kbd{y}, 0~otherwise.

\fun{long}{gegalsg}{long s, GEN x} returns 1 (true) if the long \kbd{s} is
equal to~\kbd{x}, 0~otherwise.

\fun{long}{gegalgs}{GEN x, long s} returns 1 (true) if \kbd{x} is equal to
the long~\kbd{s}, 0~otherwise.

\fun{long}{iscomplex}{GEN x} returns 1 (true) if \kbd{x} is a complex number
(of component types embeddable into the reals) but is not itself real, 0~if
\kbd{x} is a real (not necessarily of type \typ{REAL}), or raises an error
if \kbd{x} is not embeddable into the complex numbers.

\fun{long}{ismonome}{GEN x} returns 1 (true) if \kbd{x} is a non-zero monomial
in its main variable, 0~otherwise.

\fun{long}{ggval}{GEN x, GEN p} returns the greatest exponent~$e$ such that
$\kbd{p}^e$ divides~\kbd{x}, when this makes sense.

\fun{long}{gval}{GEN x, long v} returns the highest power of the variable
number \kbd{v} dividing the polynomial~\kbd{x}.

\fun{int}{pvaluation}{GEN x, GEN p, GEN *r} applied to non-zero integers
\kbd{x} and~\kbd{p}, returns the highest exponent $e$ such that
$\kbd{p}^{e}$ divides~\kbd{x}, creates the quotient $\kbd{x}/\kbd{p}^{e}$
and returns its address in~\kbd{*r}.
In particular, if \kbd{p} is a prime, this returns the valuation at \kbd{p}
of~\kbd{x}, and \kbd{*r} will obtain the prime-to-\kbd{p} part of~\kbd{x}.

\subsec{Assignment statements}

\fun{void}{gaffsg}{long s, GEN x} assigns the long \kbd{s} into the
object~\kbd{x}.

\fun{void}{gaffect}{GEN x, GEN y} assigns the object \kbd{x} into the
object~\kbd{y}.

\subsec{Unary operators}

\fun{GEN}{gneg[\key{z}]}{GEN x[, GEN z]} yields $-\kbd{x}$.

\fun{GEN}{gabs[\key{z}]}{GEN x[, GEN z]} yields $|\kbd{x}|$.

\fun{GEN}{gsqr}{GEN x} creates the square of~\kbd{x}.

\fun{GEN}{ginv}{GEN x} creates the inverse of~\kbd{x}.

\fun{GEN}{gfloor}{GEN x} creates the floor of~\kbd{x}, i.e.\ the (true)
integral part.

\fun{GEN}{gfrac}{GEN x} creates the fractional part of~\kbd{x}, i.e.\ \kbd{x}
minus the floor of~\kbd{x}.

\fun{GEN}{gceil}{GEN x} creates the ceiling of~\kbd{x}.

\fun{GEN}{ground}{GEN x} rounds the components of \kbd{x} to the nearest
integers. Exact half-integers are rounded towards~$+\infty$.

\fun{GEN}{grndtoi}{GEN x, long *e} same as \key{round}, but in addition puts
minus the number of significant binary bits left after rounding into~\kbd{*e}.
If \kbd{*e} is positive, all significant bits have been lost. This kind of
situation raises an error message in \key{ground} but not in \key{grndtoi}.

\fun{GEN}{gtrunc}{GEN x} truncates~\kbd{x}. This is the (false) integer part
if \kbd{x} is an integer (i.e.~the unique integer closest to \kbd{x} among
those between 0 and~\kbd{x}). If \kbd{x} is a series, it will be truncated
to a polynomial; if \kbd{x} is a rational function, this takes the
polynomial part.

\fun{GEN}{gcvtoi}{GEN x, long *e} same as \key{grndtoi} except that
rounding is replaced by truncation.

\fun{GEN}{gred[z]}{GEN x[, GEN z]} reduces \kbd{x} to lowest terms if \kbd{x}
is a fraction or rational function (types \typ{FRAC}, \typ{FRACN},
\typ{RFRAC} and \typ{RFRACN}), otherwise creates a copy of~\kbd{x}.

\fun{GEN}{content}{GEN x} creates the GCD of all the components of~\kbd{x}.

\fun{GEN}{normalize}{GEN x} applied to an unnormalized power series~\kbd{x}
(i.e.~type \typ{SER} with all coefficients correctly set except that \kbd{x[2]}
might be zero), normalizes \kbd{x} correctly in place. Returns~\kbd{x}.
For internal use.

\fun{GEN}{normalizepol}{GEN x} applied to an unnormalized polynomial~\kbd{x}
(i.e.~type \typ{POL} with all coefficients correctly set except that \kbd{x[2]}
might be zero), normalizes \kbd{x} correctly in place and returns~\kbd{x}.
For internal use.

\subsec{Binary operators}

\fun{GEN}{gmax[z]}{GEN x, GEN y[, GEN z]} yields the maximum of the objects
\kbd{x} and~\kbd{y} if they can be compared.

\fun{GEN}{gmaxsg[z]}{long s, GEN x[, GEN z]} yields the maximum of the long
\kbd{s} and the object~\kbd{x}.

\fun{GEN}{gmaxgs[z]}{GEN x, long s[, GEN z]} yields the maximum of the object
\kbd{x} and the long~\kbd{s}.

\fun{GEN}{gmin[z]}{GEN x, GEN y[, GEN z]} yields the minimum of the objects
\kbd{x} and~\kbd{y} if they can be compared.

\fun{GEN}{gminsg[z]}{long s, GEN x[, GEN z]} yields the minimum of the long
\kbd{s} and the object~\kbd{x}.

\fun{GEN}{gmings[z]}{GEN x, long s[, GEN z]} yields the minimum of the object
\kbd{x} and the long~\kbd{s}.

\fun{GEN}{gadd[z]}{GEN x, GEN y[, GEN z]} yields the sum of the objects \kbd{x}
and~\kbd{y}.

\fun{GEN}{gaddsg[z]}{long s, GEN x[, GEN z]} yields the sum of the long \kbd{s}
and the object~\kbd{x}.

\fun{GEN}{gaddgs[z]}{GEN x, long s[, GEN z]} yields the sum of the object
\kbd{x} and the long~\kbd{s}.

\fun{GEN}{gsub[z]}{GEN x, GEN y[, GEN z]} yields the difference of the objects
\kbd{x} and~\kbd{y}.

\fun{GEN}{gsubgs[z]}{GEN x, long s[, GEN z]} yields the difference of the
object \kbd{x} and the long~\kbd{s}.

\fun{GEN}{gsubsg[z]}{long s, GEN x[, GEN z]} yields the difference of the
long \kbd{s} and the object~\kbd{x}.

\fun{GEN}{gmul[z]}{GEN x, GEN y[, GEN z]} yields the product of the objects
\kbd{x} and~\kbd{y}.

\fun{GEN}{gmulsg[z]}{long s, GEN x[, GEN z]} yields the product of the long
\kbd{s} with the object~\kbd{x}.

\fun{GEN}{gmulgs[z]}{GEN x, long s[, GEN z]} yields the product of the object
\kbd{x} with the long~\kbd{s}.

\fun{GEN}{gshift[z]}{GEN x, long n[, GEN z]} yields the result of shifting
(the components of) \kbd{x} left by \kbd{n} (if \kbd{n} is non-negative)
or right by $-\kbd{n}$ (if \kbd{n} is negative).
Applies only to integers, reals and vectors/matrices of such. For other
types, it is simply multiplication by~$2^{\kbd{n}}$.

\fun{GEN}{gmul2n[z]}{GEN x, long n[, GEN z]} yields the product of \kbd{x}
and~$2^{\kbd{n}}$. This is different from \kbd{gshift} when \kbd{n} is negative
and \kbd{x} is of type \typ{INT}: \key{gshift} truncates, while \key{gmul2n}
creates a fraction if necessary.

\fun{GEN}{gdiv[z]}{GEN x, GEN y[, GEN z]} yields the quotient of the objects
\kbd{x} and~\kbd{y}.

\fun{GEN}{gdivgs[z]}{GEN x, long s[, GEN z]} yields the quotient of the object
\kbd{x} and the long~\kbd{s}.

\fun{GEN}{gdivsg[z]}{long s, GEN x[, GEN z]} yields the quotient of the long
\kbd{s} and the object~\kbd{x}.

\fun{GEN}{gdivent[z]}{GEN x, GEN y[, GEN z]} yields the true Euclidean
quotient of \kbd{x} and the integer or polynomial~\kbd{y}.

\fun{GEN}{gdiventsg[z]}{long s, GEN x[, GEN z]} yields the true Euclidean
quotient of the long \kbd{s} by the integer~\kbd{x}.

\fun{GEN}{gdiventgs[z]}{GEN x, long s[, GEN z]} yields the true Euclidean
quotient of the integer \kbd{x} by the long~\kbd{s}.

\fun{GEN}{gdiventres}{GEN x, GEN y} creates a 2-component vertical
vector whose components are the true Euclidean quotient and remainder
of \kbd{x} and~\kbd{y}.

\fun{GEN}{gdivmod}{GEN x, GEN y, GEN *r} If \kbd{r} is not equal to
\kbd{NULL} or \kbd{ONLY\_REM}, creates the (false) Euclidean quotient of
\kbd{x} and~\kbd{y}, and puts (the address of) the remainder into~\kbd{*r}.
If \kbd{r} is equal to \kbd{NULL}, do not create the remainder, and if
\kbd{r} is equal to \kbd{ONLY\_REM}, create and output only the remainder.
The remainder is created after the quotient and can be disposed of
individually with a \kbd{cgiv(r)}.

\fun{GEN}{poldivres}{GEN x, GEN y, GEN *r} same as \key{gdivmod} but
specifically for polynomials \kbd{x} and~\kbd{y}.

\fun{GEN}{gdeuc}{GEN x, GEN y} creates the Euclidean quotient of the
polynomials \kbd{x} and~\kbd{y}.

\fun{GEN}{gdivround}{GEN x, GEN y} if \kbd{x} and \kbd{y} are integers,
returns the quotient $\kbd{x}/\kbd{y}$ of \kbd{x} and~\kbd{y}, rounded to
the nearest integer. If $\kbd{x}/\kbd{y}$ falls exactly halfway between
two consecutive integers, then it is rounded towards~$+\infty$ (as for
\key{round}). If \kbd{x} and \kbd{y} are not both integers, the result
is the same as that of \key{gdivent}.

\fun{GEN}{gmod[z]}{GEN x, GEN y[, GEN z]} yields the true remainder of \kbd{x}
modulo the integer or polynomial~\kbd{y}.

\fun{GEN}{gmodsg[z]}{long s, GEN x[, GEN z]} yields the true remainder of the
long \kbd{s} modulo the integer~\kbd{x}.

\fun{GEN}{gmodgs[z]}{GEN x, long s[, GEN z]} yields the true remainder of the
integer \kbd{x} modulo the long~\kbd{s}.

\fun{GEN}{gres}{GEN x, GEN y} creates the Euclidean remainder of the
polynomial \kbd{x} divided by the polynomial~\kbd{y}.

\fun{GEN}{ginvmod}{GEN x, GEN y} creates the inverse of \kbd{x} modulo \kbd{y}
when it exists.

\fun{GEN}{gpow}{GEN x, GEN y, long l} creates $\kbd{x}^{\kbd{y}}$. The
precision \kbd{l} is taken into account only if \kbd{y} is not an integer
and \kbd{x} is an exact object. If \kbd{y} is an integer, binary powering
is done. Otherwise, the result is $\exp(\kbd{y}*\log(\kbd{x}))$ computed
to precision~\kbd{l}.

\fun{GEN}{ggcd}{GEN x, GEN y} creates the GCD of \kbd{x} and~\kbd{y}.

\fun{GEN}{glcm}{GEN x, GEN y} creates the LCM of \kbd{x} and~\kbd{y}.

\fun{GEN}{subres}{GEN x, GEN y} creates the resultant of the polynomials
\kbd{x} and~\kbd{y} computed using the subresultant algorithm.

\fun{GEN}{gpowgs}{GEN x, long n} creates $\kbd{x}^{\kbd{n}}$ using
binary powering.

\fun{GEN}{gsubst}{GEN x, long v, GEN y} substitutes the object \kbd{y}
into~\kbd{x} for the variable number~\kbd{v}.

\fun{int}{gdivise}{GEN x, GEN y}  returns 1 (true) if \kbd{y} divides~\kbd{x},
0~otherwise.

\fun{GEN}{gbezout}{GEN x, GEN y, GEN *u, GEN *v} creates the GCD of \kbd{x}
and~\kbd{y}, and puts (the adresses of) objects $u$ and~$v$ such that
$u\kbd{x}+v\kbd{y}=\gcd(\kbd{x},\kbd{y})$ into \kbd{*u} and~\kbd{*v}.
\vfill\eject
