% $Id$
% Copyright (c) 2000  The PARI Group
%
% This file is part of the PARI/GP documentation
%
% Permission is granted to copy, distribute and/or modify this document
% under the terms of the GNU Free Documentation License
\appendix{Installation Guide for the UNIX Versions}

\def\tocwrite#1{}
\section{Required tools}

We assume that you have either an \kbd{ANSI C} or a \kbd{C++} compiler
available. If your machine does not have one (for example if you still use
\kbd{/bin/cc} in \kbd{SunOS 4.1.x}), we strongly suggest that you obtain the
\kbd{gcc/g++} compiler from the Free Software Foundation or by
anonymous~\kbd{ftp}. As for all GNU software mentioned afterwards, you can
find the most convenient site to fetch \kbd{gcc} at the address

\kbd{http://www.gnu.ai.mit.edu/order/ftp.html}

\noindent
You can certainly compile PARI with a different compiler, but the PARI
kernel takes advantage of some optimizations provided by \kbd{gcc} if it is
available. This results in about 20\% speedup on most architectures%
\footnote*{One notable exception is the native AIX C compiler on IBM
RS/6000 workstations, which generates fast code even without any special
help from the PARI kernel sources.}.

\subsec{Optional packages:} The following programs and libraries are useful
in conjunction with GP, but not mandatory. They're probably already installed
somewhere on your system (with the possible exception of \kbd{readline},
which we think is really worth a try). In any case, get them before
proceeding if you want the functionalities they provide. All of them are free
(though you ought to make a small donation to the FSF if you use (and like)
GNU wares).

  $\bullet$ GNU \kbd{readline} library. This provides line editing under GP,
an automatic context-dependent completion, and an editable history of
commands. Note that it is incompatible with SUN commandtools (yet another
reason to dump Suntools for X Windows). A recent readline (version number at
least 2.2) is preferred, but older versions should be usable.

  $\bullet$ GNU \kbd{gzip/gunzip/gzcat} package enables GP to read
compressed data.

  $\bullet$ GNU \kbd{emacs}. GP can be run in an Emacs buffer, with all the
obvious advantages if you are familiar with this editor. Note that
\kbd{readline} is still useful in this case since it provides a much better
automatic completion than is provided by Emacs GP-mode.

  $\bullet$ \kbd{perl} provides extended online help (full text from this
manual) about functions and concepts, which can be used under GP or
independently (\kbd{http://www.perl.com} will direct you to the nearest
\kbd{CPAN} archive site).

  $\bullet$ A colour-capable \kbd{xterm}, which enables GP to use different
(user configurable) colours for its output. All \kbd{xterm} programs which come
with current X11R6.3 distributions will satisfy this requirement. Under X11R6,
you can for instance use \kbd{color\_xterm} (get the latest version at
\kbd{http://www.clark.net/pub/dickey/xterm}).

\vfill\eject
\section{Compiling the library and the GP calculator}

\subsec{Basic configuration:} First, have a look at the \kbd{MACHINES} file
to see if anything funny applies to your architecture or operating system.
Then, type 

\kbd{./Configure}

\noindent in the toplevel directory. This will attempt to configure GP/PARI
without outside help. Note that if you want to install the end product in
some nonstandard place, you can use the \kbd{--prefix} option, as in

\kbd{./Configure --prefix=/an/exotic/directory}

\noindent (the default prefix is \kbd{/usr/local}). This phase extracts some
files and creates a directory \kbd{O$xxx$} where the object files and
executables will be built. The $xxx$ part depends on your architecture and
operating system, thus you can build GP for several different machines from
the same source tree (the builds are completely independent, so can be done
simultaneously).

\misctitle{Technical note:} The precise default destinations are as follows:
the \kbd{gp} binary, the scripts \kbd{gphelp} and \kbd{tex2mail} go to
\kbd{\$prefix/bin}. The pari library goes to \kbd{\$prefix/lib} and include
files to \kbd{\$prefix/include/pari}. As for architecture independant files,
the man pages go into \kbd{\$prefix/man} and the rest in various
subdirectories of \kbd{\$prefix/lib/pari}.

\noindent You can also supply a \kbd{--share-prefix} argument in which case
architecture independant files go to different directories than in the above
scheme: \kbd{\$share/pari} (galdata package), \kbd{\$share/man} (man pages)
and \kbd{\$share/doc/pari/} (everything else: sample GP scripts and C code,
documentation, emacs macros). For instance, to build a package for a Linux
distribution, you will want to use

\kbd{./Configure --prefix=/usr --share-prefix=/usr/share}

\noindent \kbd{Configure} will let the following environment variable
override the defaults if set:

\kbd{AS}: Assembler.

\kbd{CC}: C compiler.

\kbd{CFLAGS}: Flags for the C compiler.

\kbd{DLLD}: Dynamic library linker.

\noindent For instance, \kbd{Configure} avoids \kbd{gcc} on some
architectures due to various problems which may have been fixed in your
version of the compiler. You can try

\kbd{env CC=gcc Configure}

\noindent and compare the benches. Also, if you insist on using a \kbd{C++}
compiler and run into trouble with a recent \kbd{g++}, try to use
\kbd{g++ -fpermissive}.

\subsec{Troubleshooting and fine tuning:} Decide whether you agree with what
\kbd{Configure} printed on your screen (in particular the architecture,
compiler and optimization flags). If anything should have been found and was
not, consider that \kbd{Configure} failed and follow the instructions below.
Look especially for the \kbd{readline} and \kbd{X11} libraries, and the
\kbd{perl} and \kbd{gunzip} (or \kbd{zcat}) binaries.

In case the default \kbd{Configure} run fails miserably, try

\kbd{./Configure -a}

\noindent (interactive mode) and answer all the questions (there aren't that
many). Of course, \kbd{Configure} will still provide defaults for each answer
but if you accept them all, it will fail just the same, so be wary. In any
case, we would appreciate a bug report including the complete output from
\kbd{Configure} and the file \kbd{O$xxx$/dft.Config.in} that was produced in
the process.

\subsec{Problems related to readline:}
Even in interactive mode, you cannot directly tell \kbd{Configure}
where the \kbd{readline} library and include files are. If they are not in a
standard place, it won't find them. Nonetheless, it first searches the
distribution toplevel for a \kbd{readline} directory. Thus, if you just want
to give \kbd{readline} a try (as you probably should), you can get the source
and compile it there (you don't need to install it). You can also use this
feature together with a symbolic link, named \kbd{readline}, in the PARI
toplevel directory if you have compiled the readline library somewhere else,
without installing it to one of its standard locations. 

\misctitle{Linux:} Linux distributions have separate \kbd{readline} and
\kbd{readline-devel} packages. You need both of them installed to compile gp
with readline support. If only \kbd{readline} is installed, \kbd{Configure}
will complain. \kbd{Configure} may also complain about a missing
libncurses.so, in which case, you will have to install the
\kbd{ncurses-devel} package (some distributions let you install
\kbd{readline-devel} without \kbd{ncurses-devel}, which is a bug in their
package dependency handling).

\misctitle{Technical note:} Configure can build GP on different architectures
simultaneously from the same toplevel sources. Instead of the \kbd{readline}
link alluded above, you can create \kbd{readline-{\sl osname}-{\sl arch}},
using the same naming conventions as for the \kbd{O$xxx$} directory,
e.g \kbd{readline-linux-i686}.

\subsec{Debugging/profiling}: If you also want to debug the PARI library,

\kbd{Configure -g}

\noindent will create a directory \kbd{O$xxx$.dbg} containing a special
\kbd{Makefile} ensuring that the GP and PARI library built there will be
suitable for debugging (if your compiler doesn't use standard flags,
e.g.~\kbd{-g} you may have to tweak that \kbd{Makefile}). If you want to
profile GP or the library (using \kbd{gprof} for instance), 

\kbd{Configure -pg}

\noindent will create an \kbd{O$xxx$.prf} directory where a suitable version
of PARI can be built.

\subsec{Compilation and tests:} To compile the GP binary, simply type

\kbd{make gp}

\noindent in the distribution directory. If your \kbd{make} program supports
parallel make, you can speed up the process by going to the \kbd{O$xxx$}
directory that \kbd{Configure} created and doing a parallel make here (for
instance \kbd{make -j4} with GNU make).

\subsubsec{Testing}

To test the binary, type \kbd{make bench}. This will build a static
executable (the default, built by \kbd{make gp} is probably dynamic) and
run a series of comparative tests on those two. To test only the default
binary, use \kbd{make dobench} which starts the bench immediately.

The static binary should be slightly faster. In any case, this should not
take more than one minute (user time) on modern machines. See the file
\kbd{MACHINES} to get an idea of how much time comparable systems need (we
would appreciate a short note in the same format in case your system is not
listed and you nevertheless have a working GP executable).

If a \kbd{[BUG]} message shows up, something went wrong. Probably with the
installation procedure, but it may be a bug in the Pari system, in which
case we would appreciate a report (including the relevant \kbd{*.dif} file
in the \kbd{O$xxx$} directory and the file \kbd{dft.Config.in}).

\misctitle{Known problems:}

$\bullet$ \kbd{elliptic}: the test \kbd{cmcurve=ellinit([0,-3/4,0,-2,-1])}
may give results which differ slightly from the template (last decimal in a
few entries). This ultimately depends on the output of

\kbd{polroots(x\pow 3-3/4*x\pow 2-2*x-1)[1]}

\noindent at \kbd{\b{p}38}, which may be $2.0$ or $1.999\dots$ depending on
your hardware, libraries, compiler\dots Intel Pentiums running Linux often
trigger this \kbd{BUG} (unrelated to the infamous \kbd{fdiv} bug), which
can safely be ignored in any case: both results are correct given the
requested precision.

$\bullet$ \kbd{program}: the GP function \kbd{install} may not be available on
your platform, triggering an error message (``not yet available for this
architecture''). Have a look at the \kbd{MACHINES} files (the \kbd{dl}
column) to check if your system is known not to support it, or has never
been tested yet.

$\bullet$ If when running \kbd{gp-dyn}, you get a message of the form

\kbd{ld.so: warning: libpari.so.$xxx$ has older revision than expected $xxx$}

\noindent (possibly followed by more errors), you already have a dynamic PARI
library installed {\it and\/} a broken local configuration. Either remove the
old library or unset the \kbd{LD\_LIBRARY\_PATH} environment variable. Try to
disable this variable in any case if anything {\it very} wrong occurs with
the \kbd{gp-dyn} binary (e.g Illegal Instruction on startup). It doesn't
affect \kbd{gp-sta}.

\subsubsec{Some more testing} [{\sl Optional\/}]

You can test GP in compatibility mode with \kbd{make test-compat}. If you
want to test the graphic routines, use \kbd{make test-graphic}. You will
have to click on the mouse button after seeing each image (under X11).
There will be eight of them, probably shown twice (under X11, try to resize
at least one of them as a further test).

The \kbd{make bench} and \kbd{make test-compat} runs produce a Postscript
file \kbd{pari.ps} in \kbd{O$xxx$} which you can send to a Postscript
printer. The output should bear some similarity to the screen images.

\section{Installation} When everything looks fine, type

\kbd{make install}

\noindent (You may have to do this with superuser privileges, depending on
the target directories.) Beware that, if you chose the same installation
directory as before in the \kbd{Configure} process, this will wipe out any
files from version 1.39.15 and below that might already be there. Libraries
and executable files from newer versions (starting with version 1.900) are
not removed since they are only links to files bearing the version number
(beware of that as well: if you're an avid GP fan, don't forget to delete the
old pari libraries once in a while).

This installs in the directories chosen at \kbd{Configure} time the default
GP executable (probably \kbd{gp-dyn}) under the name \kbd{gp}, the default
PARI library (probably \kbd{libpari.so}), the necessary include files, the
manual pages, the documentation and help scripts and emacs macros.

To save on disk space, you can manually \kbd{gzip} some of the documentation
files if you wish~: \kbd{usersch*.tex} and all \kbd{dvi} files (assuming your
\kbd{xdvi} knows how to deal with compressed files); the online-help system
will handle it.

By default, if a dynamic library \kbd{libpari.so} could be built, the static
library \kbd{libpari.a} will not be created. If you want it as well, you can
use the target \kbd{make install-lib-sta}. You can install a statically
linked \kbd{gp} with the target \kbd{make install-bin-sta}. As a rule,
programs linked statically (with \kbd{libpari.a}) may be slightly faster
(about 5\% gain), but use much more disk space and take more time to compile.
They are also harder to upgrade: you will have to recompile them all instead
of just installing the new dynamic library. On the other hand, there's no
risk of breaking them by installing a new pari library.

\subsec{The Galois package:} The default \kbd{polgalois} function can only
compute Galois groups of polynomials of degree less or equal to 7. If you
want to handle polynomials of degree bigger than 7 (and less than 11), you
need to fetch a separate archive: \kbd{galdata.tgz} which can probably be
found at the same place where you got the main PARI archive, and on the
\kbd{megrez} ftp server in any case. Untar the archive in the \kbd{datadir}
directory which was chosen at \kbd{Configure} time (it's one of the last
messages on the screen if you did not run \kbd{Configure -a}). You can then
test the \kbd{polgalois} function with your favourite polynomials.

\subsec{The \kbd{GPRC} file:} Copy the file \kbd{misc/gprc.dft} (or
\kbd{gprc.dos} if you're using \kbd{GP.EXE}) to \kbd{\$HOME/.gprc}. Modify
it to your liking. For instance, if you're not using an ANSI terminal,
remove control characters from the \kbd{prompt} variable. You can also
enable colors. 

If desired, also copy/modify \kbd{misc/gpalias} somewhere and call it from
the \kbd{gprc} file (this provides some common shortcuts to lengthy names).
Finally, if you have superuser privileges and want to provide system-wide
defaults, you can copy your customized \kbd{.gprc} file to \kbd{/etc/gprc}.

In older versions, \kbd{gphelp} was hidden in pari lib directory and wasn't
meant to be used from the shell prompt, but not anymore. If gp complains it
can't find \kbd{gphelp}, check whether your \kbd{.gprc} (or the system-wide
\kbd{gprc}) does contain explicit paths. If so, correct them according to the
current \kbd{misc/gprc.dft}.

\section{Getting Started}

\subsec{Printable Documentation:} To print the user's guide, for which you'll
need a working (plain) \TeX\ installation; type

\kbd{make doc}

\noindent This will create, in two passes, a file \kbd{doc/users.dvi}
containing the manual with a table of contents and an index. You must then
send the \kbd{users.dvi} file to your favourite printer in the usual way,
probably via \kbd{dvips}. Also included are a short tutorial
(\kbd{doc/tutorial.dvi}) and a reference card (\kbd{doc/refcard.dvi}
and \kbd{doc/refcard.ps}) for GP.

\noindent If the \kbd{pdftex} package is part of your \TeX\ setup, you can
produce these documents in PDF format, which may be more convenient for
online browsing (the manual is complete with hyperlinks); type

\kbd{make docpdf}

\noindent All these documents are available online from PARI home page and on
the \kbd{megrez} ftp server.

\subsec{C programming:} Once all libraries and include files are installed,
you can link your C programs to the PARI library. A sample makefile
\kbd{examples/Makefile} is provided to illustrate the use of the various
libraries. Type \kbd{make all} in the \kbd{examples} directory to see how
they perform on the \kbd{mattrans.c} program, which is commented in the
manual.

\subsec{GP scripts:} Several complete sample GP programs are also given in
the \kbd{examples} directory, for example Shanks's SQUFOF factoring method,
the Pollard rho factoring method, the Lucas-Lehmer primality test for
Mersenne numbers and a simple general class group and fundamental unit
algorithm (much worse than the built-in \kbd{bnfinit}!). See the file
\kbd{examples/EXPLAIN} for some explanations.

\subsec{EMACS:} If you want to use \kbd{gp} under GNU Emacs, read the file
\kbd{emacs/pariemacs.txt}. If you are familiar with Emacs, we suggest that
you do~so.

\subsec{The PARI Community:} There are three mailing lists devoted to the
PARI/GP package (run courtesy of Dan Bernstein), and most feedback should be
directed to those. They are:

$\bullet$ \kbd{pari-announce}: to announce major version changes.
You can't write to this one, but you should probably subscribe.

$\bullet$ \kbd{pari-dev}: for everything related to the development
of PARI, including suggestions, technical questions, bug reports or patch
submissions.

 $\bullet$ \kbd{pari-users}: for everything else.

\noindent To subscribe, send empty messages respectively to

\centerline{\kbd{pari-announce-subscribe@list.cr.yp.to}}

\centerline{\kbd{pari-users-subscribe@list.cr.yp.to}}

\centerline{\kbd{pari-dev-subscribe@list.cr.yp.to}}

\noindent The PARI home page (maintained by Gerhard Niklasch) at the address

\centerline{\wwwsite}

\noindent maintains an archive of all discussions as well as a download area.
If don't want to subscribe to those lists, you can write to us at the address

\centerline{\kbd{pari@math.u-bordeaux.fr}}

\noindent At the very least, we will forward you mail to the lists above and
correct faulty behaviour, if necessary. But we cannot promise you will get an
individual answer.

If you have used PARI in the preparation of a paper, please cite it in the
following form (BibTeX format):

\def\@{@}
\bprog
@@manual{PARI2,
    organization = "{The PARI~Group}",
    title        = "{PARI/GP, Version @vers}",
    year         = 2000,
    address      = "Bordeaux",
    note         = "available from {\tt @wwwsite}"
}
@eprog
\smallskip

\noindent In any case, if you like this software, we would be indebted if you
could send us an email message giving us some information about yourself and
what you use PARI for.

\medskip
{\bf Good luck and enjoy!}
\vfill\eject
