% $Id$
% Copyright (c) 2000  The PARI Group
%
% This file is part of the PARI/\kbd{gp} documentation
%
% Permission is granted to copy, distribute and/or modify this document
% under the terms of the GNU Free Documentation License

% This should be compiled with plain TeX
\def\TITLE{A Tutorial for Pari-\kbd{gp}}
\input parimacro.tex
\ifPDF
  \input pdfmacs.tex
\fi
\def\maketitle#1{\currentlabel.\ #1}
\begintitle
\vskip2.5truecm
\centerline{\mine A Tutorial}
\vskip1.truecm
\centerline{\mine for}
\vskip1.truecm
\centerline{\mine PARI / \kbd{gp}}
\vskip1.truecm
\authors
\centerline{last updated September 17, 2002}
\centerline{(this document distributed with version \vers)}
\endtitle

\copyrightpage

This booklet is intended to be a guided tour and a tutorial to the \kbd{gp}
calculator. Many examples will be given, but each time a new function is
used, the reader should look at the appropriate section in the User's Manual
for detailed explanations. This chapter can be read independently, for
example to get acquainted with the possibilities of \kbd{gp} without having
to read the whole manual. But the reader will profit most from it by reading
it in conjunction with the reference manual.

\section{Greetings!}

So you are sitting in front of your workstation (or terminal, or PC,\dots),
and you type \kbd{gp} to get the program started. It says hello in its
particular manner, and then waits for you after its \kbd{prompt}, initially
\kbd{?} (or something like {\bf gp}~\kbd{>}).

Type \kbd{2 + 2}. What happens? Maybe not what you expect. First of all, of
course, you should tell \kbd{gp} that your input is finished, and this is
done by hitting the \kbd{Return} (or \kbd{Newline}) key, or the \kbd{Enter}
key on the Mac. If you do exactly this, you will get the expected answer.
However some of you may be used to other systems like Gap, Macsyma, Magma or
Maple. In this case, you will have subconsciously ended the line with a
semicolon ``\kbd{;}'' before hitting \kbd{Return}, since this is how it is
done on those systems. In that case, you will simply see \kbd{gp} answering
you with a smug expression, i.e.~a new prompt and no answer!  This is because
a semicolon at the end of a line tells \kbd{gp} not to print the result (it
is still stored in the result history). You will certainly want to use this
feature if the output is several pages long.

Try \kbd{27 * 37}. Wow! even multiplication works. Actually, maybe those
spaces are not necessary after all. Let's try \kbd{27*37}. Yup, seems to be
ok. We will still insert them in this document since it makes things easier
to read, but as \kbd{gp} does not care about them, you don't have to type
them all!

Now this session is getting lengthy, so the second thing one needs to learn
is to quit. Each system has its quit signal. In \kbd{gp}, you can use
\kbd{quit} or \b{q} (backslash q), the \kbd{q} being of course for quit.
Try it.

Now you've done it! You're out of \kbd{gp}, so how do you want to continue
studying this tutorial? Get back in please (see above).

Let's get to more serious stuff. I seem to remember that the decimal
expansion of $1/7$ has some interesting properties. Let's see what \kbd{gp}
has to say about this. Type \kbd{1 / 7}. What? This computer is making fun of
me, it just spits back to me my own input, that's not what I want!

Now stop complaining, and think a little. This system has been written mainly
for pure mathematicians (although anybody is welcome to use it). And
mathematically, $1/7$ is an element of the field $\Q$ of rational numbers, so
how else but $1/7$ can the computer give the answer to you? (well maybe
$2/14$ or $7^{-1}$, but why complicate matters?). Seriously, the basic point
here is that PARI, hence \kbd{gp}, will almost always try to give you a
result which is as precise as possible (we will see why ``almost'' later),
hence since here the result can be represented exactly, that's what it gives
you.

OK, but I still want the decimal expansion of $1/7$. No problem. Type one of
the following:
\bprog
1./ 7
1 / 7.
1./ 7.
1 / 7 + 0.@com etc \dots
@eprog
Immediately a number of decimals of this fraction appear (28 on most systems,
38 on the others), and the repeating pattern is $142857$. The reason is that
you have included in the operations numbers like \kbd{0.}, \kbd{1.} or \kbd{7.}
which are \emph{imprecise} real numbers, hence \kbd{gp} cannot give you an
exact result.

Why 28 / 38 decimals by the way? Well, it is the default initial precision.
This has been chosen so that the computations are very fast, and gives
already 12 decimals more accuracy than conventional double precision floating
point operations. The precise value depends on a technical reason: if your
machine supports 64-bit integers (the standard C library can handle integers
up to $2^{64}$), the default precision will be 38 decimals, and 28 otherwise.
As the latter is most probably the case, we will assume it henceforth.

Only large mainframes or supercomputers have 28 digits of precision in their
standard libraries, and that is their absolute limit. Not here of course. You
can extend the precision (almost) as much as you like as we will see in a
moment.

I'm getting bored, why don't we get on with some more exciting stuff?  Well,
try \kbd{exp(1)}. Presto, comes out the value of $e$ to 28 digits. Try
\kbd{log(exp(1))}. Well, we get a floating point number and not an exact $1$,
but pretty close! That's what you lose by working numerically.

Let's see, what could we try now. Hum, \kbd{pi}? The answer is not that
enlightening. \kbd{Pi}? Ah. This works better. But let's remember that
\kbd{gp} distinguishes between uppercase and lowercase letters. \kbd{pi} was
as meaningless to it as \kbd{stupid garbage} would have been: in both cases
\kbd{gp} will just create a variable with that funny unknown name you just
used. Try it! Note that it is actually equivalent to type
\kbd{stupidgarbage}: all spaces are suppressed from the input. In the
\kbd{27~*~37} example  it was not so conspicuous as we had an operator to
separate the two operands. This has important consequences for the writing of
\kbd{gp} scripts (more about this later).

By the way, you can ask \kbd{gp} about any identifier you think it might know
about: just type it, prepending a question mark ``\kbd{?}''. Try \kbd{?Pi}
and \kbd{?pi} for instance. On most systems, an extended online help should
be available: try doubling the question mark to check whether it's the case
on yours: \kbd{??Pi}. In fact the \kbd{gp} header already gave you that
information if it was the case, just before the copyright message. As well,
if it says something like ``\kbd{readline enabled}'' then you should have a
look at \secref{se:readline} in the User's Manual before you go on: it will
be much easier to type in examples and correct typos after you've done that.

Now try \kbd{exp(Pi * sqrt(163))}. Hmmm, we suspect that the last digit may
be wrong, can this really be an integer? This is the time to change
precision. Type \kbd{\b{p} 50}, then try \kbd{exp(Pi * sqrt(163))} again. We
were right to suspect that the last decimal was incorrect, since we get quite
a few nines in its place, but it is now convincingly clear that this is not
an integer. Maybe it's a bug in PARI, and the result is really an integer?
Type \kbd{sqr(log(\%) / Pi)} immediately after the preceding computation
(\kbd{\%} means the result of the last computed expression). More generally,
the results are numbered \kbd{\%1, \%2, \dots} \emph{including} the results
that you do not want to see printed by putting a semicolon at the end of the
line, and you can evidently use all these quantities in any further
computations. \kbd{sqr} is the square function (\kbd{sqr(x) = x * x}), not to
be confused with \kbd{sqrt} which is the square root function. The result
seems to be indistinguishable from $163$, hence it does not seem to be a bug.

In fact, it is known that $\exp(\pi*\sqrt{n})$ not only is not an integer or
a rational number, but is even a transcendental number when $n$ is a non-zero
rational number.

So \kbd{gp} is just a fancy calculator, able to give me more decimals than I
will ever need? Not so, \kbd{gp} is incredibly more powerful than an ordinary
calculator, independently of its arbitrary precision possibilities.

\noindent {\bf Additional comments} (you are supposed to skip this at first,
and come back later)

1) If you are a PARI old timer, you will notice pretty soon (or maybe you
have already?) that many many things changed between the older 1.39.xx
versions and this one. Conspicuously, most function names have been changed.
We sincerely think it is for the best since they are much more logical now
and the systematic prefixing is very convenient coupled with the automatic
completion mechanism: it is now easy to know what functions are available to
deal with, say, elliptic curves since they all share the prefix \kbd{ell}.

Of course, this is going to break all your nice old scripts. Well, you can
either change the compatibility level (typing \kbd{default(compatible, 3)}
will send you back to the stone-age behaviour of good ol' version 1.39.15),
or rewrite the scripts. We really advise you to do the latter if they are not
too long, since they can now be written much more cleanly than before
(especially with the new control statements: \kbd{break}, \kbd{next},
\kbd{return}), and besides it'll be as good a way as any to get used to the
new names. We \emph{might} provide an automatic transcriptor with future
versions.

To know how a specific function was changed, just type \kbd{whatnow({\rm
function})}.

2) It seems that the text implicitly says that as soon as an imprecise number
is entered, the result will be imprecise. Is this always true? There is a
unique exception: when you multiply an imprecise number by the exact number
0, you will get the exact 0. Compare \kbd{0 * 1.4} and \kbd{0.~*~1.4}.
\smallskip
%
3) Not only can the number of decimal places of real numbers be large, but
the number of digits of integers also. Try \kbd{100!}. It is never necessary
to tell \kbd{gp} in advance the size of the integers that it will encounter,
since this is adjusted automatically. On the other hand, for many
computations with real numbers, it is necessary to specify a default
precision (initially 28 digits).
\smallskip
%
4) Come back to 28 digits of precision (\kbd{\b{p} 28}), and type \kbd{exp(24
* Pi)}. As you can see the result is printed in exponential format. This is
because \kbd{gp} never wants you to believe that a result is correct when it
is not.

We are working with 28 digits of precision, but the integer part of
$\exp(24*\pi)$ has 33 decimal digits. Hence if \kbd{gp} had dutifully printed
out 33 digits, the last few digits would have been wrong. Hence \kbd{gp}
wants to print only 28 significant digits, but to do so it has to print in
exponential format.
\smallskip
%
5) There are two ways to avoid this. One is of course to increase the
precision to more than 33 decimals. Let's try it. To give it a wide margin,
we set the precision to 40 decimals. Then we recall our last result (\kbd{\%}
or \kbd{\%n} where \kbd{n} is the number of the result). What? We still have
an exponential format! Do you understand why?

Again let's try to see what's happening. The number you recalled had been
computed only to 28 decimals, and even if you set the precision to 1000
decimals, \kbd{gp} knows that your number has only 28 digits of accuracy but
an integral part with 33 digits. So you haven't improved things by increasing
the precision. Or have you? What if we retype \kbd{exp(24 * Pi)} now that we
have 40 digits? Try it. Now we do not have an exponential format, and we see
that at 28 decimals the last 6 digits would have been wrong if they had been
printed in fixed-point format. \smallskip
\smallskip
%
6) What if I forget what the current precision is and I don't feel like
counting all the decimals? Well, you can learn about \kbd{gp} internal
variables (and change them!) using \kbd{default}. Type
\kbd{default(realprecision)}, then \kbd{default(realprecision, 38)}. Huh? In
fact this last command is strictly equivalent to \kbd{\b{p} 38}! (admittedly
more cumbersome to type). There are more ``defaults'' than just \kbd{format}
and \kbd{realprecision}: type \kbd{default} by itself now, they are all
there.
\smallskip
%
7) Note that the \kbd{default} command reacts differently according to the
number of input arguments. This is not an uncommon behaviour for \kbd{gp}
functions. You can see this from the online help (or the complete description
in Chapter~3): any argument surrounded by braces \kbd{\obr\cbr} in the
function prototype is optional (which really means that a \emph{default}
argument will be supplied by \kbd{gp}). You can then check out from the text
what effect a given value will have, and in particular the default one.
\smallskip
%
8) \emph{Warning}. Try the following: starting in precision 28, type first
\kbd{default(format, "e0.50")}, then \kbd{exp(24 * Pi)}. Do you understand why
the result is so bad, and why there are lots of zeros at the end?  Convince
yourself by typing \kbd{log(exp(1))}. The moral is that the
\kbd{default(format,)} command changes only the output format, but \emph{not}
the default precision. On the other hand, the \b{p} command changes both the
precision and the output format.

\section{Warming up}

Another thing you better get used to pretty fast is error messages. Try
typing \kbd{1/0}. Couldn't be clearer. Taking again our universal example in
precision 28, type \kbd{floor(exp(24 * Pi))} (\kbd{floor} is the
mathematician's integer part, not to be confused with \kbd{truncate}, which
is the computer scientist's: \kbd{floor(-3.4)} is equal to $-4$ whereas
\kbd{truncate(-3.4)} is equal to $-3$).  You get a more cryptic error
message, which you would immediately understand if you had read the
additional comments of the preceding section. Since I told you not to read
them, the explanation is simply that \kbd{gp} is unable to compute the
integer part of \kbd{exp(24 * Pi)} given only 28 decimals of accuracy, since
it has 33 digits.

Some error messages are even much more cryptic than that and are sometimes
not so easy to understand (well, it's nothing compared to \TeX's error
messages!).

For instance, try \kbd{log(x)}. Not really clear, is it? It simply tells you
that \kbd{gp} does not understand what \kbd{log(x)} is (although it does know
the \kbd{log} function, as \kbd{?log} will readily tell us).

Now let's try \kbd{sqrt(-1)} to see what error message we get now. Haha!
\kbd{gp} even knows about complex numbers, so impossible to trick it that
way. Similarly, try typing \kbd{log(-2)}, \kbd{exp(I*Pi)}, \kbd{I\pow
I},\dots So we have a lot of real and complex analysis at our disposal (note
that there always is a specific branch of multivalued complex transcendental
functions which is taken, specified in the manual). Again, beware that
\kbd{I} and \kbd{i} are not the same thing. Compare \kbd{I\pow2} with
\kbd{i\pow2} for instance.

Just for fun, let's try \kbd{6*zeta(2) / Pi\pow2}. Pretty close, no?

\medskip
Now \kbd{gp} didn't seem to know what \kbd{log(x)} was, although it did know
how to compute numerical values of \kbd{log}. This is annoying. Maybe it
knows the exponential function? Let's give it a try. Type \kbd{exp(x)}.
What's this? If you had any experience with other computer algebra systems,
the answer should have simply been \kbd{exp(x)} again. But here the answer is
the Taylor expansion of the function around $\kbd{x}=0$, to 16 terms. 16 is
the default \kbd{seriesprecision}, which can be changed by typing \kbd{\b{ps}
$n$} or \kbd{default(seriesprecision, $n$)} where $n$ is the number of terms
that you want in your power series. Note the \kbd{O(x\pow16)} which ends the
series, and which is trademark of this type of object in \kbd{gp}. It is the
familiar ``big--oh'' notation of analysis.

You will thus automatically get the Taylor expansion of any function that can
be expanded around $0$, and incidentally this explains why we weren't
able to do anything with \kbd{log(x)} which is not defined at $0$. (In fact
\kbd{gp} knows about Laurent series, but \kbd{log(x)} is not meromorphic
either at $0$.) But if we try \kbd{log(1+x)}, then it works. But what if we
wanted the expansion around a point different from 0? Well, you're able to
change $x$ into $x-a$, aren't you? So for instance you can type
\kbd{log(x+2)} to have the expansion of \kbd{log} around $\kbd{x}=2$. As
exercises you can try
\bprog
cos(x)
cos(x)^2 + sin(x)^2
exp(cos(x))
gamma(1 + x)
exp(exp(x) - 1)
1 / tan(x) @eprog
\noindent for different values of \kbd{serieslength} (change it using \b{ps}
\var{newvalue}).

Let's try something else: type \kbd{(1 + x)\pow 3}. No \kbd{O(x)} here, since
the result is a polynomial.  Haha, but I have learnt that if you do not take
exponents which are integers greater or equal to 0, you obtain a power series
with an infinite number of non-zero terms. Let's try.  Type
\kbd{(1 + x)\pow (-3)} (the parentheses around \kbd{-3} are not necessary but
make things easier to read). Surprise! Contrary to what we expected, we don't
get a power series but a rational function. Again this is for the same reason
that \kbd{1 / 7} just gave you $1/7$: the result being exact, PARI doesn't see
any reason to make it non-exact.

But I still want that power series. To obtain it, you can do as in the $1/7$
example and type
\bprog
(1 + x)^(-3) + O(x^16)
(1 + O(x^16)) * (1 + x)^(-3)
(1 + x + O(x^16))^(-3)@com, etc \dots
@eprog
You can also use the series constructor which transforms any object into a
power series, using the current \kbd{seriesprecision}, and simply type
\bprog
Ser( (1 + x)^(-3) )
@eprog

Now try \kbd{(1 + x)\pow (1/2)}: we obtain a power series, since the
result is an object which PARI does not know how to represent exactly. (We
could teach PARI about algebraic functions, but then take \kbd{(1 + x)\pow Pi}
as another example.) This gives us still another solution to our preceding
exercise: we can type \kbd{(1 + x)\pow (-3.)}. Since \kbd{-3.} is not an exact
quantity, PARI has no means to know that we are dealing with a rational
function, and will instead give you the power series, this time with real
instead of integer coefficients.

Finally, if you want to be really fancy, you can type
\kbd{taylor((1 + x)\pow (-3), x)} (look at the entry for \kbd{taylor} for the
description of the syntax), but this is in fact almost never used.
\smallskip

To summarize, in this section we have seen that in addition to integers, real
numbers and rational numbers, PARI can handle complex numbers, polynomials,
power series, rational functions. A large number of functions exist which
handle these types, but in this tutorial we will only look at a few.

\noindent {\bf Additional comments} (as before, you are supposed to skip this
at first reading)

1) To be able to duplicate the following example, first type \b{y} to
suppress automatic simplification.

A complex number has a real part and an imaginary part (who would have
guessed?). However, beware that when the imaginary part is the exact integer
zero, it is not printed, but the complex number is not converted to a real
number. Hence it may \emph{look} like a real number without being one, and
this may cause some confusion in programs which expect real numbers. For
example, type \kbd{n = 3 + I - I}. The answer reads \kbd{3}, as expected. But
it is still a complex number. Check it with \kbd{type(n)}. Hence if you then
type \kbd{(1+x)\pow n}, instead of getting the polynomial
\kbd{1 + 3*x + 3*x\pow 2 + x\pow 3} as expected, you obtain a power series.
Worse, when you try to apply an arithmetic function, say the Euler totient
function (known as \kbd{eulerphi} to \kbd{gp}), you get a sententious error message
recalling you that ``arithmetic functions want integer arguments''. You would
have guessed yourself, but the message is difficult to understand since 3 looks
like a genuine integer! (Please read again if this is not clear. It is a
common source of incomprehension.)

Similarly, \kbd{3 + x - x} is not the integer 3 but a constant polynomial
(in the variable \kbd{x}), equal to $3 = 3x^0$.

If you want the final expression to be in the simplest form possible (for
example before applying an arithmetic function, or simply because things will
go faster afterwards), apply the function \kbd{simplify} to the result. In
fact, by default \kbd{gp} does this automatically at the end of a \kbd{gp} command. Note
that it does \emph{not} simplify in intermediate expressions. This default
can be switched off and on by the command \b{y}. This is why I asked you to
type this before starting.

2) As already stated, power series expansions are always implicitly around
$\kbd{x} = 0$. When we wanted them around $\kbd{x} = \kbd{a}$, we replaced
\kbd{x} by \kbd{z + a} in the function we wanted to expand. For complicated
functions, it may be simpler to use the substitution function \kbd{subst}.
For example, if \kbd{p~= 1 / (x\pow 4 + 3*x\pow 3 + 5*x\pow 2 - 6*x + 7)},
you may not want to retype this, replacing \kbd{x} by \kbd{z~+ a}, so you can
write \kbd{subst(p, x, z+a)} (look up the exact description of the
\kbd{subst} function).

Now try typing \kbd{p = 1 + x + x\pow 2 + O(x\pow 10)}, then
\kbd{subst(p, x, z+1)}. Do you understand why you get an error message?

3) The valuation at $\kbd{x} = 0$ for a power series \kbd{p} is obtained
as \kbd{valuation(p, x)}.

\section{The Remaining PARI Types}
Let's talk some more about the basic PARI types.

Type \kbd{p = x * exp(-x)}. As expected, you get the power series expansion
to 16 terms (if you have not changed the default). Now type
\kbd{pr = serreverse(p)}. You are asking here for the \emph{reversion} of the
power series \kbd{p}, in other words the inverse function. This is possible
only for power series whose first non-zero coefficient is that of $x^1$.  To
check the correctness of the result, you can type \kbd{subst(p, x, pr)} or
\kbd{ subst(pr, x, p)} and you should get back \kbd{x + O(x\pow 17)}.

Now the coefficients of \kbd{pr} obey a very simple formula. First, we would
like to multiply the coefficient of \kbd{x\pow n} by \kbd{n!} (in the case of
the exponential function, this would simplify things considerably!). The PARI
function \kbd{serlaplace} does just that. So type \kbd{ps = serlaplace(pr)}.
The coefficients now become integers, which can be immediately recognized by
inspection. The coefficient of $x^n$ is now equal to
$n^{n-1}$. In other words, we have
%
$$\kbd{pr} = \sum_{n\ge1}\dfrac{n^{n-1}}{n!} X^{n}.$$
%
Do you know how to prove this? (If you have never seen this, the proof is
difficult.)
\smallskip
%
Of course PARI knows about vectors (rows and columns are distinguished, even
though mathematically there is no difference) and matrices. Type for example
\kbd{[1,2,3,4]}. This gives the row vector whose coordinates are 1, 2, 3 and
4.  If you want a column vector, type \kbd{[1,2,3,4]\til}, the tilde meaning
of course transpose. You don't see much difference in the output, except for
the tilde at the end. However, now type \b{b}: lo and behold, the vector does
become a column. The \b{b} command is used mainly for this purpose.

Type \kbd{m = [a,b,c; d,e,f]}. You have just entered a matrix with 2 rows and
3 columns. Note that the matrix is entered by \emph{rows} and the rows are
separated by semicolons ``\kbd{;}''. The matrix is printed naturally in a
rectangle shape. If you want it printed horizontally just as you typed it,
type \b{a}, or if you want this type of printing to be the permanent default
type \kbd{default(output, 0)}. Type \kbd{default(output, 1)} if you want to
come back to the original output mode.

Now type \kbd{m[1,2]}, \kbd{m[1,]}, \kbd{m[,2]}. Are explanations necessary?
(In an expression such as \kbd{m[j,k]}, the \kbd{j} always refers to the
row number, and the \kbd{k} to the column number, and the first index is
always 1, never 0. This default cannot be changed.)

Even better, type \kbd{m[1,2] = 5; m} (the semicolon also allows us to put
several instructions on the same line. The final result will be the output of
the last statement on the line). Now type \kbd{m[1,] = [15,-17,8]}. No
problem. Finally type \kbd{m[,2] = [j,k]}. You have an error message since you
have typed a row vector, while \kbd{m[,2]} is a column vector. If you type
instead \kbd{m[,2] = [j,k]\til} it works. \smallskip
%
Type now \kbd{h = mathilbert(20)}. You get the so-called ``Hilbert matrix''
whose coefficient of row $i$ and column $j$ is equal to $(i+j-1)^{-1}$.
Incidentally, the matrix \kbd{h} takes too much room. If you don't want to
see it, simply type a semi-colon ``\kbd{;}'' at the end of the line
(\kbd{h = mathilbert(20);}). This is an example of a ``precomputed'' matrix,
built into PARI. There are only a few. We will see later an example of a much
more general construction.

What is interesting about Hilbert matrices is that first their inverses and
determinants can be computed explicitly (and the inverse has integer
coefficients), and second they are numerically very unstable, which make them
a severe test for linear algebra packages in numerical analysis.  Of course
with PARI, no such problem can occur: since the coefficients are given as
rational numbers, the computation will be done exactly, so there cannot be
any numerical error. Try it. Type \kbd{d~=~matdet(h)}. The result is a
rational number (of course) of numerator equal to 1 and denominator having
226 decimal digits. How do I know, by the way? I did not count! Instead,
simply type \kbd{1.* d}. The result is now in exponential format, and the
exponent gives us the answer. Another, more natural, way would be to simply
type \kbd{sizedigit(1/d)}.

Now type \kbd{hr = 1.* h;} (do not forget the semicolon, we don't want to see
all the junk!), then \kbd{dr = matdet(hr)}. You notice two things. First the
computation, although not instantaneous, is much faster than in the rational
case. The reason for this is that PARI is handling real numbers with 28
digits of accuracy, while in the rational case it is handling integers having
up to 226 decimal digits.

The second more important fact is that the result is terribly wrong. If you
compare with \kbd{1.$*$d} computed earlier, which is correct, you will see
that only 2 decimals agree! This catastrophic instability is as already
mentioned one of the characteristics of Hilbert matrices. In fact, the
situation is much worse than that. Type \kbd{norml2(1/h - 1/hr)} (the
function \kbd{norml2} gives the square of the $L^2$ norm, i.e.~the sum of the
squares of the coefficients). The result is larger than $10^{50}$, showing
that some coefficients of \kbd{1/hr} are wrong by as much as $10^{24}$ (the
largest error is in fact equal to $4.244 \cdot 10^{24}$ for the coefficient
of row 15 and column 15, which is a 28 digit integer).

To obtain the correct result after rounding for the inverse, we have to use a
default precision of 56 digits (try it).
\smallskip

Although vectors and matrices can be entered manually, by typing explicitly
their elements, very often the elements satisfy a simple law and one uses a
different syntax. For example, assume that you want a vector whose $i$-th
coordinate is equal to $i^2$. No problem, type for example
\kbd{vector(10,i, i\pow 2)} if you want a vector of length 10. Similarly, if
you type

\centerline{\kbd{matrix(5,5,i,j, 1/(i+j-1))}}

\noindent you will get the Hilbert matrix of order 5 (hence the
\kbd{mathilbert} function is redundant).  The \kbd{i} and \kbd{j} represent
dummy variables which are used to number the rows and columns respectively
(in the case of a vector only one is present of course). You must not forget,
in addition to the dimensions of the vector or matrix, to indicate explicitly
the names of these variables. You may omit the variables and the final
expression to get zero entries, as in \kbd{matrix(10,20)}.

\misctitle{Warning.} The letter \kbd{I} is reserved for the complex number
equal to the square root of $-1$. Hence it is forbidden to use it as a
variable. Try typing \kbd{vector(10,I, I\pow 2)}, the error message that you
get clearly indicates that \kbd{gp} does not consider \kbd{I} as a variable.
There are two other reserved variable names: \kbd{Pi} and \kbd{Euler}. All
function names are forbidden as well. On the other hand there is nothing
special about \kbd{i}, \kbd{pi} or \kbd{euler}.

When creating vectors or matrices, it is often useful to use boolean
operators and the \kbd{if()} statement (see the section on programming for
details on using this statement). Indeed, an \kbd{if} expression has a value,
which is of course equal to the evaluated part of the \kbd{if}. So for
example you can type
\bprog
matrix(8,8, i,j, if ((i-j)%2, 1, 0))
@eprog
\noindent to get a checkerboard matrix of \kbd{1} and \kbd{0}. Note however
that a vector or matrix must be \emph{created} first before being used. For
example, it is forbidden to write
\bprog
for (i = 1, 5, v[i] = 1/i)
@eprog
\noindent if the vector \kbd{v} has not been created beforehand (for example
by a \kbd{v = vector(5)} command).

\medskip The last PARI types which we have not yet played with are closely
linked to number theory (hence people not interested in number theory can
skip ahead).

The first is the type ``integer--modulo''. Let us see an example. Type
\kbd{n = 10\pow 15 + 3}. We want to know whether this number is prime or not. Of
course we could make use of the built-in facilities of PARI, but let us do
otherwise. We first trial divide by the built-in table of primes. We slightly
cheat here and use a variant of the function \kbd{factor} which does exactly
this. So type \kbd{factor(n, 200000)}. (The last argument tells \kbd{factor}
to trial divide up to the given bound and stop at this point. You can set it
to 0 to trial divide by the full set of built-in primes, which goes up to
$500000$ by default.)

The result is a 2 column matrix (as for all factoring functions), the first
column giving the primes and the second their exponents. Here we get a single
row, telling us that \kbd{n} is not divisible by any prime up to $200000$. We
could now trial divide further, or even cheat completely and call the PARI
function \kbd{factor} without the optional second argument, but before we do
this let us see how to get an answer ourselves.

By Fermat's little theorem, if $n$ is prime we must have $a^{n-1}\equiv 1
\pmod{n}$ for all $a$ not divisible by $n$. Hence we could try this with $a=2$
for example. But $2^{n-1}$ is a number with approximately $3\cdot10^{14}$
digits, hence impossible to write down, let alone to compute. But instead type
\kbd{a = Mod(2,n)}. This creates the number $2$ considered now as an element
of the ring $R = \Z/\kbd{n}\Z$. The elements of $R$, called integermods, can
always be represented by numbers smaller than \kbd{n}, hence very small.
Fermat's theorem can be rewritten
%
$\kbd{a}^{n-1} = \kbd{Mod(1,n)}$
%
in the ring $R$, and this can be computed very efficiently. (Elements of $R$
may be lifted back to $\Z$ with either \kbd{lift} or \kbd{centerlift}.) Type
\kbd{a\pow (n-1)}. The result is definitely \emph{not} equal to
\kbd{Mod(1,n)}, thus \emph{proving} that \kbd{n} is not a prime. If we had
obtained \kbd{Mod(1,n)} on the other hand, it would have given us a hint that
\kbd{n} is maybe prime, but never a proof. 

To find the factors is another story. One must use less naive techniques than
trial division (or be very patient). To finish this example, type
\kbd{fa = factor(n)} to see the factors. Since the smallest factor is 14902357,
you would have had to be very patient with trial division!

Note that, as is the case with most ``prime''-producing functions, the
``prime'' factors given by \kbd{factor} are only strong pseudoprimes, and not
\emph{proven} primes.  Use \kbd{isprime( fa[,1] )} to rigorously prove
primality of the factors. The latter command applies \kbd{isprime} to all
entries in the first column of \kbd{fa}, i.e to all pseudoprimes, and returns
the column vector of results (all equal to 1, so the pseudoprimes are indeed
primes). All arithmetic functions can be applied in this way to the entries
of a vector or matrix. In fact, it has been checked that the strong
pseudoprimes output by \kbd{factor} (Baillie-Pomerance-Selfridge-Wagstaff
pseudoprimes, without small divisors) are true primes at least up to
$10^{13}$, so the above example was OK after all.
\smallskip

The second specifically number-theoretic type is the $p$-adic numbers. I have
no room for definitions, so please skip ahead if you have no use for such
beasts. A $p$-adic number is entered as a rational or integer valued
expression to which is added \kbd{O(p\pow n)} (or simply \kbd{O(p)} if
$\kbd{n}=1$) where \kbd{p} is the prime and \kbd{n} the $p$-adic precision.
Note that you have to explicitly type in \kbd{3\pow 2} for instance, \kbd{9}
will not do. (Unless you want to cheat \kbd{gp} into believing that \kbd{9}
is prime, but you had better know what you are doing in this case.) Apart
from the usual arithmetic operations, you can apply a number of
transcendental functions. For example, type \kbd{n = 569 + O(7\pow 8)}, then
\kbd{s~=~sqrt(n)}, you obtain one of the square roots of \kbd{n} (if you want
to check, type \kbd{s\pow 2 - n}). Type now \kbd{s = log(n)}, then \kbd{e =
exp(s)}. If you know about $p$-adic logarithms, you will not be surprised
that \kbd{e} is not equal to \kbd{n}. Type \kbd{(n/e)\pow 6}: \kbd{e} is in
fact equal to \kbd{n} times the $(p-1)$-st root of unity \kbd{teichmuller(n)}.

Incidentally, if you want to get back the integer 569 from the $p$-adic
number \kbd{n}, type \kbd{lift(n)} or \kbd{truncate(n)}.
\smallskip

The third number-theoretic type is the type ``quadratic number''. This type
is specially tailored so that we can easily work in a quadratic extension of
a base field (usually $\Q$). It is a generalization of the type
``complex''. To start, we must specify which quadratic field we want to work
in. For this, we use the function \kbd{quadgen} applied to the
\emph{discriminant} \kbd{d} (as opposed to the radicand) of the quadratic
field. This returns a number (always printed as \kbd{w}) equal to
$(\kbd{d}+a) / 2$ where $a$ is equal to 0 or 1 according to whether \kbd{d} is
even or odd. The behavior of \kbd{quadgen} is a little special: although its
result is always printed as \kbd{w}, the variable \kbd{w} itself is not set
to that value. Hence it is necessary to write systematically
\kbd{w = quadgen(d)} using the variable name \kbd{w} (or \kbd{w1} etc. if you
have several quadratic fields), otherwise things will get confusing.

So type \kbd{w = quadgen(-163)}, then \kbd{charpoly(w)} which asks for the
characteristic polynomial of \kbd{w} (in the variable \kbd{x};
you can type \kbd{charpoly(w, y)} to directly express it in terms of some
other variable without resorting to the \kbd{subst} function). The result
shows what \kbd{w} will represent. You can also ask for \kbd{1.*w} to see
which root of the quadratic has been taken, but this is rarely necessary. We
can now play in the field $\Q(\sqrt{-163})$. Type for example
\kbd{w\pow 10}, \kbd{norm(3 + 4*w)}, \kbd{1 / (4+w)}. More interesting, type
\kbd{a = Mod(1,23) * w} then \kbd{b = a\pow 264}. This is a generalization of
Fermat's theorem to quadratic fields. If you do not want to see the modulus 23
all the time, type \kbd{lift(b)}.

Another example: type \kbd{p = x\pow 2 + w*x + 5*w + 7}, then \kbd{norm(p)}. We
thus obtain the quartic equation over $\Q$ corresponding to the relative
quadratic extension over $\Q(\kbd{w})$ defined by \kbd{p}.

On the other hand, if you type \kbd{wr  = sqrt(w\pow 2)}, do not expect to get
back \kbd{w}. Instead, you get the numerical value, the function \kbd{sqrt}
being considered as a ``transcendental'' function, even though it is
algebraic. Type \kbd{algdep(wr,2)} (this looks for algebraic relations
involving the powers of \kbd{w} up to degree 2). This is one way to get
\kbd{w} back. Similarly, type \kbd{algdep(sqrt(3*w + 5), 4)}. See the user's
manual for the function \kbd{algdep}.\smallskip

The fourth number-theoretic type is the type ``polynomial--modulo'', i.e.
polynomial modulo another polynomial. This type is used to work in general
algebraic extensions, for example elements of number fields (if the base
field is $\Q$), or elements of finite fields (if the base field is
$\Z/p\Z$ for a prime $p$, defined by an integermod). In a sense it is a
generalization of the type quadratic number. The syntax used is the same as
for integermods. For example, instead of typing \kbd{w = quadgen(-163)}, you
can type \kbd{w = Mod(x, x\pow 2 - x + 41)}. Then, exactly as in the
quadratic case, you can type \kbd{w\pow 10}, \kbd{norm(3 + 4*w)},
\kbd{1 / (4+w)}, \kbd{a = Mod(1,23)*w}, \kbd{b = a\pow 264}, obtaining of
course the same results (type \kbd{lift(\dots)} if you don't want to see the
polynomial \kbd{x\pow 2 - x + 41} repeated all the time). Of course, the basic
interest is that you can work in any degree, not only quadratic (of course,
even for quadratic moduli, the corresponding elementary operations will be
slower than with quadratic numbers).

There is however a slight difference in behavior. Keeping our polmod \kbd{w},
type \kbd{1.*w}. As you can see, the result is not the same. Type
\kbd{sqrt(w)}. Here, we obtain a vector with 2 components, the two components
being the principal branch of the square root of all the possible embeddings
of \kbd{w} in $\C$ (\emph{not} the two square roots). More generally, if
\kbd{w} was of degree $n$, we would get an $n$-component vector, and similarly
for transcendental functions.

We have at our disposal the usual arithmetic functions, plus a few others.
Type \kbd{a = Mod(x, x\pow 3 - x - 1)} defining a cubic extension. We can for
example ask for \kbd{b = a\pow 5}. Now assume we want to express \kbd{a}
as a polynomial in \kbd{b}. This is possible since \kbd{b} is also a
generator of the same field. No problem, type \kbd{modreverse(b)}. This gives
a new defining polynomial for the same field (i.e.~the characteristic
polynomial of \kbd{b}), and expresses \kbd{a} in terms of this new polmod,
i.e.~in terms of \kbd{a}. We will see this in much more detail in the number
field section.

\section{Elementary Arithmetic Functions}

Since PARI is aimed at number theorists, it is not surprising that there
exists a large number of arithmetic functions (see the list in the
corresponding section of the users manual). We have already seen several,
such as \kbd{factor}. Note that \kbd{factor} handles not only integers, but
also (univariate) polynomials. Type for example \kbd{factor(x\pow 15 - 1)}.
You can also ask to factor a polynomial modulo a prime $p$ (\kbd{factormod})
and even in a finite field which is not a prime field (\kbd{factorff}).

Evidently, you have functions for computing GCD's (\kbd{gcd}), extended GCD's
(\kbd{bezout}), solving the Chinese remainder theorem (\kbd{chinese}) and so
on.

In addition to the factoring facilities, you have a few functions related to
primality testing such as \kbd{isprime}, \kbd{ispseudoprime},
\kbd{precprime}, and \kbd{nextprime}. As previously mentioned, only strong
pseudoprimes are produced by the latter two (they pass the
\kbd{ispseudoprime} test); the more sophisticated primality tests in
\kbd{isprime} are not applied by default.

We also have the usual multiplicative arithmetic functions: the M\"obius $\mu$
function (\kbd{moebius}), the Euler $\phi$ function (\kbd{eulerphi}), the
$\omega$ and $\Omega$ functions (\kbd{omega} and \kbd{bigomega}), the
$\sigma_k$ functions (\kbd{sigma}), which compute sums of $k$-th powers of the
positive divisors of a given integer, etc\dots

You can compute continued fractions. For example, type \kbd{\b{p} 1000}, then
\kbd{contfrac(exp(1))}: you obtain the continued fraction of the base of
natural logarithms, which as you can see obeys a very simple pattern (can you
prove it?).

In many cases, one wants to perform some task only when an arithmetic
condition is satisfied. \kbd{gp} gives you the following functions: \kbd{isprime}
as mentioned above, \kbd{issquare}, \kbd{isfundamental} to test whether an
integer is a fundamental discriminant (i.e.~$1$ or the discriminant of the
ring of integers of a quadratic field), and the \kbd{forprime}, \kbd{fordiv}
and \kbd{sumdiv} loops. Assume for example that we want to compute the
product of all the divisors of a positive integer \kbd{n}. The easiest way is
to write

\centerline{\tt p=1; fordiv(n,d, p *= d); p }

\noindent
(there is a very simple formula for this product: find and prove it). The
notation \kbd{p *= d} (inherited from the C programming language) is just a
shorthand for \kbd{p = p * d}.

If we want to know the list of primes $p$ less than 1000 such that 2 is a
primitive root modulo $p$, one way would be to write:
\bprog
  forprime(p=3,1000, if (znprimroot(p) == 2, print(p)))
@eprog\noindent
%
Note that this assumes that \kbd{znprimroot} returns the smallest primitive
root, and this is indeed the case. Had we not known about this, we could
have written
\bprog
  forprime(p=3,1000, if (znorder(Mod(2,p)) == p-1, print(p)))
@eprog\noindent
%
(which is actually faster since we only compute the order of $2$ in $\Z/p\Z$,
instead of looking for a generator by trying successive elements whose orders
have to be computed as well.)

Functions related to quadratic fields, binary quadratic forms and general
number fields will be seen in the next sections.

\section{Performing Linear Algebra}
All the standard linear algebra programs are available of course, and many
more. In addition, linear algebra over $\Z$, i.e.~work on lattices, can also
be performed. Let us see how this works. First of all, a vector space (more
generally a module) will be given by a generating set of vectors (often a
basis) which will be represented as \emph{column} vectors. This set of vectors
will in turn be represented as a matrix: in PARI, we have chosen to consider
matrices as row vectors of column vectors. The base field (or ring) can be any
ring type PARI supports (except $p$-adics which are currently not correctly
handled by the linear algebra package). However, certain operations are
specifically written for a real or complex base field, while others are
written for $\Z$ as the base ring.

----- TO BE COMPLETED -----



\section{Using Transcendental Functions}

All the elementary transcendental functions and several higher transcendental
functions (gam\-ma function, incomplete gamma function, error function,
exponential integral, $K$-bessel functions, confluent hypergeometric functions,
Riemann $\zeta$ function, polylogarithms, Weber functions, theta functions)
are provided. More may be written if the need arises.

In this type of functions, the default precision plays an essential role.
In almost all cases transcendental functions work in the following way.
If the argument is exact, the result will be computed using the current
default precision. If the argument is not exact, the precision of the
argument is used for the computation. A note of warning however: even in this
case the \emph{printed} value will be the current real format (usually the
same as the default precision). In the present chapter we assume that your
machine works with 32-bit long integers. If it is not the case, we leave it
to you as a very good exercise to make the necessary modifications.

Let's assume that we have 28 decimals of default precision (this is what we
get automatically at the start of a \kbd{gp} session on 32-bit machines). Type
\kbd{e = exp(1)}. We get the number $e=2.718\dots$ to 28 decimals. Let us check
how many correct decimals we really have. The hard (but reasonable) way is to
proceed as follows. Change the precision to a substantially higher value, for
example by typing \kbd{\b{p} 50}. Then type \kbd{e}, then \kbd{exp(1)} once
again. This last value is the correct value of the mathematical constant $e$ to
50 decimals, while the variable \kbd{e} shows the value that was computed to 28
decimals. Clearly they coincide to exactly 29 significant digits.

A simpler way to see how many decimals we have, is to ask for \kbd{length(e)}
which indicates we have exactly $3$ mantissa words. Since
$3\ln(2^{32}) / \ln(10)\approx28.9$ we see that we have 28 or 29 significant
digits (on 32-bit machines).

\smallskip
Come back to 28 decimals (\kbd{\b{p} 28}). If we type \kbd{exp(1.)}
you can check that we also obtain 28 decimals. However, type
\kbd{f = exp(1 + 10.\pow(-30))}. Although the default precision is still 28,
you can check using one of the two methods above that we have in fact 59
significant digits! The reason is that \kbd{1 + 10.\pow(-30)} is computed
according to the PARI philosophy, i.e.~to the best possible precision. Since
\kbd{10.} has 29 significant digits and 1 has ``infinite'' precision, the
number \kbd{1 + 10.\pow(-30)} will have $59=29+30$ significant digits,
hence \kbd{f} also.

Now type \kbd{cos(10.\pow(-15))}. The result is printed as $1.0000\dots$, but
is of course not exactly equal to 1. Using \kbd{length(\%)}, we see that the
result has 7 mantissa words, giving us the possibility of having 67
correct significant digits. In fact (look in precision 100), only 60 are
correct. PARI gives you as much as it can, and since 6 mantissa words
would have given you only 57 digits, it uses 7. But why does it give so
precise a result? Well, it is the same reason as before. When $x$ is close
to 1, $\cos(x)$ is close to $1-x^2/2$, hence the precision is going to be
approximately the same as this quantity, which will be $1-0.5*10^{-30}$ where
$0.5*10^{-30}$ is considered with 28 significant digit accuracy, hence the
result will have approximately $28+30=58$ significant digits.

Unfortunately, this philosophy cannot go too far. For example, when you
type \kbd{cos(0)}, \kbd{gp} should give you exactly 1. Since it is reasonable for
a program to assume that a transcendental function never gives you an exact
result, \kbd{gp} gives you $1.000\dots$ to one more mantissa word than the current
precision.
\medskip
OK, now let's see some more transcendental functions at work. Type
\kbd{gamma(10)}. No problem (type \kbd{9!} to check). Type \kbd{gamma(100)}.
The number is now written in exponential format because the default
accuracy is too small to give the correct result (type \kbd{99!} to get the
complete answer).
To get the complete value, there are two solutions. The first and most natural
one is to increase the precision. Since \kbd{gamma(100)} has 156 decimal
digits, type \kbd{\b{p} 170} (to be on the safe side), then \kbd{gamma(100)}
once again. After some work, the printed result is this time perfectly
correct.

However, this is probably not the best way to proceed. Come back first to the
default precision (type \kbd{\b{p} 28}). As the gamma function increases
very rapidly, one usually uses its logarithm. Type \kbd{lngamma(100)}. This
time the result has a reasonable size, and is exactly equal to \kbd{log(99!)}.

Try \kbd{gamma(1/2 + 10*I)}. No problem, we have the complex gamma function.
Now type

\kbd{t = 1000; z = gamma(1 + I*t) * t\pow(-1/2) * exp(Pi/2*t)/sqrt(2*Pi)},

\noindent then \kbd{norm(z)}. We see that \kbd{norm(z)} is very close to 1,
in accordance with the complex Stirling formula. \smallskip
%
Let's play now with the Riemann zeta function. First turn on the timer (type
\kbd{\#}). Type \kbd{zeta(2)}, then \kbd{Pi\pow 2/6}. This seems correct. Type
\kbd{zeta(3)}. All this takes essentially no time at all. However, type
\kbd{zeta(3.)}. Although the result is the same, you will notice that the
time is substantially larger (if your machine is too fast to see the
difference, increase the precision!). This is because PARI uses special
formulas to compute \kbd{zeta(n)} when \kbd{n} is a (positive or negative)
integer.

Type \kbd{zeta(1 + I)}. This also works. Now for fun, let us compute in a
very naive way the first complex zero of \kbd{zeta}. We know that it is
of the form $1/2 + i*t$ with $t$ between 14 and 15. Thus, we can use the
following series of instructions. But instead of typing them directly, write
them into a file, say \kbd{zeta.gp}, then type \kbd{\b{r} zeta.gp} under
\kbd{gp} to read it in:
\bprog
{
  t1 = 1/2 + 14*I;
  t2 = 1/2 + 15*I; eps = 10^(-50);
  z1 = zeta(t1);
  until (norm(z2) < eps,
    z2 = zeta(t2);
    if (norm(z2) < norm(z1),
      t3 = t1; t1 = t2; t2 = t3; z1 = z2
    );
    t2 = (t1+t2) / 2.;
    print(t1 ": " z1)
  )
}
@eprog

Don't forget the braces: they tell \kbd{gp} that a sequence of instructions is going
to span many lines (another, less convenient, way would be to type \b{} at the
end of each line to tell the parser that the input was not yet finished).
By the way, you don't need to type in the suffix~\kbd{.gp} it will be
supplied by \kbd{gp}, if you forget it (the suffix is not mandatory either, it is
just more convenient to have all your \kbd{gp} scripts labeled in the same
distinctive way).

We thus obtain the first zero to 25 significant digits.
\medskip
%
As mentioned at the beginning of this tutorial, some transcendental functions
can also be applied to $p$-adic numbers. This is as good a time as any to
familiarize yourself with them. Type \kbd{a~=~exp(7 + O(7\pow 10))}, then
\kbd{log(a)}. All seems in order. Now type \kbd{b = log(5 + O(7\pow 10))},
then \kbd{exp(b)}. Is something wrong, since we don't recover the number we
started with? Absolutely not. Type
 \kbd{exp(b) * teichmuller(5 + O(7\pow 10))},
and we indeed recover our initial number. The function \kbd{teichmuller(x)}
is the Teichm\"uller character, and is characterized by the fact that it is
the unique \hbox{$(p-1)$-st} root of unity (here with $p=7$) which is
congruent to \kbd{x} modulo $p$, assuming that \kbd{x} is a $p$-adic
unit.\smallskip
%
Let us come back to real numbers for the moment. Type \kbd{agm(1,sqrt(2))}.
This gives the arithmetic-geometric mean of 1 and $\sqrt2$, and is the basic
method for computing (complete) elliptic integrals. In fact, type

\kbd{Pi/2 / intnum(t=0,Pi/2, 1 / sqrt(1 + sin(t)\pow 2))},

\noindent and the result is the same. The elementary transformation
\kbd{x = sin(t)} gives the mathematical equality
$$\int_0^1 \dfrac{dx}{\sqrt{1-x^4}} = \dfrac{\pi}{2\text{agm}(1,\sqrt2)}
\enspace,$$
which was one of Gauss's remarkable discoveries in his youth.

Now type \kbd{2 * agm(1,I) / (1+I)}. As you see, the complex AGM also works,
although one must be careful with its definition. The result found is
almost identical to the previous one. Exercise: do you see why?

Finally, type \kbd{agm(1, 1 + 7 + O(7\pow 10))}. So we also have $p$-adic
AGM. Note however that since the square root of a $p$-adic number is not
in general an element of the same $p$-adic field,
only certain $p$-adic AGMs can be computed. In addition,
when $p=2$, the congruence restriction is that \kbd{agm(a,b)} can be computed
only when \kbd{a/b} is congruent to 1 modulo $16$ (not 8 as could be
expected).\smallskip
%
Now type \kbd{?3}. This gives you the list of all transcendental functions.
Instead of continuing with more examples, we suggest that you experiment
yourself with the list of functions. In each case, try integer, real, complex
and $p$-adic arguments. You will notice that some have not been implemented
(or do not have a reasonable definition).

\section{Using Numerical Tools}

 Although not written to be a numerical analysis package, PARI can
nonetheless perform some numerical computations. We leave for a subsequent
section linear algebra and polynomial computations.

You of course know the formula $\pi = 4(1-\dfrac13+\dfrac15-\dfrac17+\cdots)$
which is deduced from the power series expansion of \kbd{atan(x)}. You also
know that $\pi$ cannot be computed from this formula, since the convergence
is so slow. Right? Wrong! Type \kbd{\b{p} 100} (just to make it more
interesting), then \kbd{4~*~sumalt(k=0, (-1)\pow k/(2*k + 1))}. In a split
second (admittedly more than simply typing \kbd{Pi}), we get $\pi$ to 100
significant digits (type \kbd{Pi} to check). In version 1.38, the method used
was a combination of a method due to Euler for accelerating alternating sums,
and a programming trick due to the Dutch mathematician van Wijngaarden (see
one of the Numerical Recipes books for an explanation). The method which we
presently use is considerably better, and is based on a combination of ideas of
F.~Villegas, D.~Zagier and H.~Cohen.

Similarly, try \kbd{sumpos(k=1, 1 / k\pow 2)}. Although once again the
convergence is slow, a similar trick allows to compute the sum when the terms
are positive (compare with the exact result \kbd{Pi\pow 2/6}). This is much
less impressive because quite a bit slower, but still useful.

Even better, \kbd{sumalt} can be used to sum divergent series! Type

\centerline{\tt zet(s) = sumalt(k=1, (-1)\pow(k-1) / k\pow s) / (1 - 2\pow(1-s))}

Then for positive values of \kbd{s} different from 1, \kbd{zet(s)} is equal
to \kbd{zeta(s)} and the series converges, albeit slowly (sumalt doesn't
care however). For negative \kbd{s}, the series diverges, but \kbd{zet(s)}
still gives the correct result! Try \kbd{zet(-1)}, \kbd{zet(-2)},
\kbd{zet(-1.5)}, and compare with the corresponding values of \kbd{zeta}.
You should not push the game too far: \kbd{zet(-100)}, for example,
gives a completely wrong answer.

Try \kbd{zet(I)}, and compare with \kbd{zeta(I)}. Even (some) complex values
work, although the sum is not alternating any more! Similarly, try
\kbd{sumalt(n=1, (-1)\pow n/(n+I))}.
\medskip
%
More traditional functions are the numerical integration functions.
Come back to \kbd{\b{p} 28} since these routines are very slow when working
with too many significant digits. Try \kbd{intnum(t=1,2, 1/t)}
and presto! you get 26 decimals of $\log(2)$. Look at Chapter 3 to see the
available integration functions.

With PARI, however, you can go further since complex types are allowed.
For example, assume that we want to know the location of the zeros of the
function $h(z)=e^z-z$. We use Cauchy's theorem, which tells us that the
number of zeros in a disk of radius $r$ centered around the origin is
equal to
$$\dfrac{1}{2i\pi}\int_{C_r}\dfrac{h'(z)}{h(z)}\,dz\enspace,$$
where $C_r$ is the circle of radius $r$ centered at the origin.
Hence type
\bprog
fun(z) =
{ local(u = exp(z));

  (u-1) / (u-z)
}
zero(r) = r/(2*Pi) * intnum(t=0, 2*Pi, real( fun(r*exp(I*t)) * exp(I*t) ))
@eprog
\noindent (Here \kbd{u} is a local variable to the function \kbd{f}: whenever
a function is called, \kbd{gp} fills its argument list with the actual arguments
given, and initializes the other declared parameters and local variables to
0. It will then restore their former values upon exit. If, however, we had
not declared \kbd{u} in the function prototype, it would be considered as a
global variable, whose value would be permanently changed. It is not
mandatory to declare in this way all the parameters you use, but beware of
side effects!)

The function \kbd{zero(r)} will then count the number of zeros: we simply
made the change of variable $z = r*\exp(i*t)$, and took the real part to
avoid integrating the imaginary part.

Now type \kbd{\b{p} 9} (otherwise the computation would take too long, and
anyway we know that the result is an integer), then \kbd{zero(1)},
\kbd{zero(1.5)}. The result tells us that there are no zeros inside the unit
disk, but that there are two (necessarily complex conjugate) whose modulus is
between $1$ and $1.5$. For the sake of completeness, let us compute them. Let
$z$ be such a zero, and write $z=x+iy$ with $x$ and $y$ real. Then the
equation $e^z-z=0$ implies, after elementary transformations, that
$e^{2x}=x^2+y^2$ and that $e^x\cos(y)=x$. Hence $y=\pm\sqrt{e^{2x}-x^2}$ and
hence $e^x\cos(\sqrt{e^{2x}-x^2})=x$. Therefore, type 
\bprog
fun(x) =
{ local(u = exp(x));

  u * cos(sqrt(u^2 - x^2)) - x
}
@eprog
\noindent Then \kbd{fun(0)} is positive while \kbd{fun(1)} is negative. Come
back to precision 28 and type
\bprog
  x0 = solve(x=0,1, fun(x))
  z = x0 + I*sqrt(exp(2*x0) - x0^2)
@eprog
\noindent which (together with its complex conjugate) is the required zero.
As a check, type \kbd{exp(z) - z}.

Of course you can integrate over contours which are more complicated than
circles, but you must perform yourself the changes of variable as we have
done above to reduce the integral to a number of integrals on line segments.
\smallskip
%
The example above also shows the use of the \kbd{solve} function. To use
\kbd{solve} on functions of a complex variable, it is necessary to reduce the
problem to a real one. For example, to find the first complex zero of the
Riemann zeta function as above, we could try typing

\kbd{solve(t=14,15, real( zeta(1/2 + I*t) ))},

\noindent but this would not work because the real part is positive for
\kbd{t=14} and \kbd{t=15}. As it happens, the imaginary part works. Type

\kbd{solve(t=14,15, imag( zeta(1/2 + I*t) ))},

\noindent and this now works. We could also narrow the search interval and
type for instance

\kbd{solve(t=14,14.2, real( zeta(1/2 + I*t) ))}

\noindent which would also work.

\section{Functions Related to Polynomials and Power Series}

First a word of warning to the unwary: it is essential to understand the
crucial difference between exact and inexact objects (see Section~1). This
is especially important in the case of polynomials. Let's immediately take
a plunge into these problems. Type

\centerline{\tt gcd(x\pow 2 - 1, x\pow 2 - 3*x + 2)}

\noindent The result is \kbd{x - 1} as expected. But now type

\centerline{\tt gcd(x\pow 2 - 1., x\pow2 - 3.*x + 2.)}

You are lucky, the result is almost correct except for a bizarre factor of
3 which comes from the way PARI does the computation. In any case, it is still
essentially a reasonable result. But now type
\kbd{gcd(x - Pi, x\pow 2 - 6*zeta(2))}.
Although this should be equal to \kbd{x - Pi}, PARI finds a
constant as a result. This is because the notion of GCD of non-exact
polynomials doesn't make much sense. However, type
\kbd{polresultant(x - Pi, x\pow 2 - 6*zeta(2))}.
The result is extremely close to zero, showing that indeed the GCD is
non-trivial, without telling us what it is. This being said, we will usually
use polynomials (and power series) with exact coefficients in our
examples.\smallskip

Set \kbd{pol = polcyclo(15)}. This gives the $15$-th cyclotomic polynomial,
which is of degree $\varphi(15)=8$. Now, type \kbd{r = polroots(pol)}. You
have the 8 complex roots of pol given to 28 significant digits. To see them
better, type \b{b}. As you see, they are given as pairs of complex conjugate
roots, in a random order. The only ordering done by the function
\kbd{polroots} concerns the real roots, which are given first, and in
increasing order.

The roots of \kbd{pol} are by definition the primitive $15$-th roots of unity.
To check this, simply type \kbd{rc = r\pow 15}. Why, we get an error message!
Well, fair enough, vectors cannot be multiplied (even less raised to a power)
that easily. However, type \kbd{rc = r\pow 15.} with a $.$ at the end. Now it
works, because powering to a non-integer (here real) exponent is a
transcendental function and hence is applied termwise. Note that the fact that
$15.$ is a real number which is representable exactly as an integer has
nothing to do with the problem.

We see that the components of the result are very close to 1. It is however
tedious to look at all these real and imaginary parts. It would be impossible
if we had many more. Let's do it automatically. Type
\kbd{rr = round( real(rc) )}, then \kbd{sqrt( norml2(rc - rr) )}. We see that
\kbd{rr} is indeed all 1's, and that the L2-norm of \kbd{rc - rr} is around
$3.10^{-28}$, reasonable enough when we work with 28 significant digits! Note
that the function \kbd{norml2}, contrary to what its name implies, does not
give the L2 norm but its square, hence we must take the square root. (Well,
this is not absolutely necessary in this case!).
%
\smallskip
Now type \kbd{pol = x\pow 5 + x\pow 4 + 2*x\pow 3 - 2*x\pow 2 - 4*x - 3},
then \kbd{factor(pol)}. The polynomial \kbd{pol} factors over $\Q$ (or $\Z$)
as a product of two factors. Type \kbd{fun(p)= factorpadic(pol,p,10)}. This
creates a function \kbd{fun(p)} which factors \kbd{pol} over $\Q_p$ to $p$-adic
precision 10. If now we type \kbd{factor(poldisc(pol))}, we learn that the
primes dividing the discriminant are $11$, $23$ and $37$. Type \kbd{fun(5)},
\kbd{fun(11)}, \kbd{fun(23)}, and \kbd{fun(37)} to see different splittings.

Similarly, we can type \kbd{lf(p)= lift(factormod(pol,p))}, and
\kbd{lf(2)}, \kbd{lf(11)}, \kbd{lf(23)} and \kbd{lf(37)} which show the
different factorizations, this time over $\F_p$. In fact, even better: type
successively
\bprog
T = ffinit(3,3, t)
pol2 = subst(GF27, t, x)
fq = factorff(pol2, 3, GF27)
centerlift( lift(fq) )
@eprog
%
$T$, which is actually \kbd{t\pow 3 + t\pow 2 + t + 2} (with integermod
coefficients), is defined above to be an irreducible polynomial of degree $3$
over $\F_3$. This factors the polynomial \kbd{pol2} over the finite field
$\F_3[t]/(T)$. This is of course a form of the field $\F_{27}$. We know that
Gal$(\F_{27}/\F_3)$ is cyclic of order 3 generated by the Frobenius
homomorphism $u\mapsto u^3$, and the roots that we have found give the action
of the powers of the Frobenius on \kbd{t}. (If you do not know what I am
talking about, please try some more examples, it's not so hard to figure
out.) We took pain above to factor a polynomial in the variable $x$
over a finite field defined by a polynomial in $t$, even though they were
apparently one and the same. There is a crucial rule in all routines
involving relative extensions: the variable associated to the base field is
required to have lower priority than the variables of polynomials whose
coefficients are taken in that base field. Have a look at the section on 
\emph{Variable priorities} in the user's manual (see ``The GP programming
language'').

Similarly, type \kbd{pol3 = x\pow 4 - 4*x\pow 2 + 16} and
\kbd{fn = factornf(pol3,t\pow 2 + 1)}, and we get the factorization of the
polynomial \kbd{pol3} over the number field defined by \kbd{t\pow 2 + 1},
i.e.~over $\Q(i)$. To see the result even better, type \kbd{lift(fn)},
remembering that \kbd{t} stands for the generator of the number field
(here equal to $i=\sqrt{-1}$).

Note that it is possible, although ill advised, to use the same variable
for the polynomial and the number field. You may for example type
\kbd{fn2 = factornf(pol3, x\pow 2 + 1)}, and the result is correct. However,
the PARI object thus created may give unreasonable results. For example,
if you type \kbd{lift(fn2)} in the example above, you will get a strange
object, with a symbol such as \kbd{x*x} typed. This is because PARI knows
that the dummy variable \kbd{x} is not the same as the explicit variable
\kbd{x}, but since it must print it when you lift, it has to do something.
\smallskip
%
To summarize, in addition to being able to factor integers, you can
factor polynomials over $\C$ and $\R$ (this is the function \kbd{polroots}),
over $\F_p$ (the function \kbd{factormod}, over $\F_{p^k}$ (the function
\kbd{factorff}), over $\Q_p$ (the function \kbd{factorpadic}), over $\Q$ or
$\Z$ (the function \kbd{factor}), and over number fields (the functions
\kbd{factornf} and \kbd{nffactor}). Note however that \kbd{factor} itself
will try to guess intelligently over which ring you want to factor: set
\kbd{pol = x\pow2+1}
and try to \kbd{factor} successively \kbd{pol}, \kbd{pol * 1.},
\kbd{pol * (1+0.*I)}, \kbd{pol * Mod(1,2)},
\kbd{pol * Mod(1,Mod(1,3)*(t\pow2+1))}.

 In the present version \vers{}, it is \emph{not} possible to factor over
other rings than the ones mentioned above, for example \kbd{gp} cannot factor
multivariate polynomials. Other functions related to factoring are
\kbd{padicappr}, \kbd{polrootsmod}, \kbd{polrootspadic}, \kbd{polsturm}. Play
with them a little.

Now let's type \kbd{polsym(pol3,20)}, where \kbd{pol3} is the same
polynomial as above. This gives the sum of the $k$-th powers of the roots
of \kbd{pol3} up to $k=20$, of course computed using Newton's formula and
not using \kbd{polroots}. You notice that every odd sum is zero (this is
trivial since the polynomial is even), but also that the signs follow a
regular pattern and that the  (non-zero) absolute values are powers of 2.
This is true: prove it, and more precisely find an explicit formula for the
$k$-th symmetric power not involving (non-rational) algebraic numbers.
\medskip

Now let's play a little with power series. We have already done so a little
at the beginning.  Type

\centerline{\tt 8*x + prod(n=1,39, if(n\%4, 1 - x\pow n, 1),
                           1 + O(x\pow 40))\pow 8}

  You obtain a power series which has apparently only even powers of \kbd{x}
appearing. This is surprising, but can be proved using the theory of modular
forms. Note that we have initialized the product to \kbd{1 + O(x\pow 40)} and
not simply to 1 otherwise the whole computation would have been done with
polynomials, and this would first have been slightly slower and also totally
useless since the coefficients of \kbd{x\pow 40} and above are irrelevant
anyhow if we stop the product at \kbd{n=39}.

While we are on the subject of modular forms (which, together with Taylor
series expansions of common functions, are another great source of power
series), type \kbd{\b{ps} 122} (which is a shortcut for
\kbd{default(seriesprecision, 122)}), then \kbd{d = x * eta(x)\pow 24}. This
gives the first 122 terms of the (Fourier) series expansion of the modular
discriminant function $\Delta$ of Ramanujan, its coefficients giving by
definition the Ramanujan $\tau$ function which has a number of marvelous
properties (look at any book on modular forms for explanations). We would like
to see its properties modulo 2. Type \kbd{d\%2}. Hmm, apparently PARI
doesn't like to
reduce coefficients of power series (or polynomials for that matter) directly.
Can we do it without writing a little program? No problem. Type instead
\kbd{lift(Mod(1,2) * d)} and now this works like a charm.

The pattern in the result is clear. Of course, it now remains to prove it
(again modular forms, see Antwerp III or your resident modular forms guru).
Similarly, type \kbd{centerlift(Mod(1,3) * d)}. This time the pattern is
less clear, but nonetheless there is one. Refer to Anwerp III again.

\section{Working with Elliptic Curves}

Now we are getting to more complicated objects. Just as with number fields
which we will meet later on, the first thing to do is to initialize them.
That's because a lot of data will be needed repeatedly, and it's much more
convenient to have it ready once and for all. Here, this is done with the
function \kbd{ellinit} (try to guess what we'll use for number fields\dots).

So type \kbd{e = ellinit([6,-3,9,-16,-14])}. This computes a number of things
about the elliptic curve defined by the affine equation
%
$$ y^2+6xy+9y = x^3-3x^2-16x-14\enspace. $$
%
It's not that clear what all these funny numbers mean, except that we
recognize the first few of them as the coefficients we just input. To
retrieve meaningful information from such complicated objects (and number
fields will be much worse), you are advised to use the so-called \emph{member
functions}. Type \kbd{?.} to get a complete list. Whenever \kbd{ell} appears
in the right hand side, we can apply the corresponding function to an object
output by \kbd{ellinit} (I'm sure you know how the other \kbd{init} functions
will be called now, don't you? Oh, by the way, there is no \kbd{clgpinit}
function).

  Let's try it. We see that the discriminant \kbd{e.disc} is equal to 37,
hence the conductor of the curve is 37. Of course in general it is not so
trivial. In fact, the equation of the curve is clearly not minimal, so type
\kbd{r = ellglobalred(e)}. The first component \kbd{r[1]} tells us that the
conductor is 37 as we already knew. The second component is a 4-component
vector which will allow us to get the minimal equation: simply type
\kbd{e = ellchangecurve(e, r[2])} and the new \kbd{e} is now our minimal
equation with corresponding data. You can for the moment ignore the third
component \kbd{r[3]} (for the impatient reader, this is the product of the
local Tamagawa numbers, $c_p$).

The new \kbd{e} tells us that the minimal equation is $y^2+y = x^3-x$.
Let us now play a little with points on \kbd{e}. Type \kbd{q = [0,0]}, which is
clearly on the curve (type \kbd{ellisoncurve(e, q)} to check). Well, \kbd{q}
may be a torsion point. Type \kbd{ellheight(e, q)}, which computes the
canonical Neron-Tate height of \kbd{q}. This is non-zero, hence \kbd{q} is
not torsion. To see this even better, type

\kbd{for(k = 1, 20, print(ellpow(e, q,k)))}

\noindent and we see the characteristic parabolic explosion of the size of
the points. As a further check, type
\kbd{ellheight(e, ellpow(e, q,20)) / ellheight(e, q)}. We indeed find
$400=20^2$ as it should be. You can also type \kbd{ellorder(e, q)} which
returns 0, telling you that \kbd{q} is non-torsion.

Notice how all those \kbd{ell}--prefixed functions take our elliptic curve as
a first argument? This will be true with number fields as well: whatever
object was initialized by an $ob$--\kbd{init} function will have to be used as
a first argument of all the $ob$--prefixed functions. Conversely, you won't be
able to use any such high-level function before you correctly initialize the
relevant object. \smallskip

Ok, let's try another curve. Type \kbd{e = ellinit([0,-1,1,0,0])}. This
corresponds to the equation $y^2+y = x^3-x^2$. Again from the discriminant
we see that the conductor is equal to 11, and if you type \kbd{ellglobalred(e)}
you will see that the equation  for \kbd{e} is minimal. Type \kbd{q = [0,0]}
which is clearly a point on \kbd{e}, and \kbd{ellheight(e, q)}. This time we
obtain a value which is very close to zero, hence \kbd{q} must be a torsion
point. Indeed, typing \kbd{for(k=1,5, print(ellpow(e, q,k)))} we see that
\kbd{q} is a point of order 5 (note that the point at infinity is represented
as \kbd{[0]}). More simply, you can type \kbd{ellorder(e, q)}.\smallskip

Let's try still another curve. Type \kbd{e = ellinit([0,0,1,-7,6])} to get
the curve $y^2+y=x^3-7x+6$. Typing \kbd{ellglobalred(e)} shows that this is a
minimal equation and that the conductor, equal to the discriminant, is 5077.
There are some trivial integral points on this curve, but let's try to be
more systematic.

First, let's study the torsion points. Typing \kbd{elltors(e)} shows that the
torsion subgroup is trivial, so we won't have to worry about torsion points.
Next, the member \kbd{e.roots} gives us the 3 roots of the minimal
equation over $\C$, i.e.~$Y^2=X^3-7X+25/4$ (set $(X,Y)=(x,y+1/2)$) so if
$(x,y)$ is a real point on the curve, $x$ must be at least equal to the
smallest root, i.e.~$x\ge-3$. Finally, if $(x,y)$ is on the curve, its
opposite is clearly $(x,-y-1)$. So we are going to use the \kbd{ellordinate}
function and type (for instance in \kbd{points.gp} which you can read in with
\kbd{\b{r} points} as we saw before)
\bprog
{
  v=[];
  for (x = -3, 1000,
    s = ellordinate(e,x);
    if (#s,            \\ @com if cardinality of \kbd{s} is non-zero
      v = concat(v, [[x,s[1]]])
    )
  ); v
}
@eprog

\noindent By the way, this is how you insert a comment in a script:
everything following a double backslash (up to the first newline character)
is ignored. If you want comments which span many lines, you can brace them
between \kbd{/* ... */} pairs. Everything in between will be ignored as well.
For instance as a header for the file \kbd{points.gp} you could insert the
following:
\bprog
/* Finds rational points on the elliptic curve e, using the naivest
 * algorithm I could think of right now (TO BE IMPROVED).
 * e should have rational coefficients.
 * TODO: Make that into a usable function.
 */
@eprog

(I hope you did not waste your time copying this nonsense, did you?)

We thus get a large number (18) of integral points. Together with their
opposites and the point at infinity, this makes a total of 37 integral
points, which is large for a curve having such a small conductor. So we
suspect (if we don't know already, since this curve is quite famous!) that
the rank of this curve must be high. Let's try and put some order into this
(note that we work only with the integral points, but in general rational
points should also be considered).

Type \kbd{hv = ellheight(e, v)}. This gives the vector of canonical heights.
Let us order the points according to their height. For this, type
\bprog
  iv = vecsort(hv,, 1);    \\@com indirect sorting
  hv = vecextract(hv, iv);
  v  = vecextract( v, iv);
@eprog
\noindent
It seems reasonable to take the numbers with smallest height as generators of
the Mordell-Weil group. Let's try the first 4: type

\kbd{m = ellheightmatrix(e, vecextract(v,[1,2,3,4])); matdet(m)}

Since the curve has no torsion, the determinant being close to zero implies
that the first four points are dependent. To find the dependency, it is
enough to find the kernel of the matrix \kbd{m}. So type \kbd{matker(m)}:
we indeed get a non-trivial kernel, and the coefficients are (close to)
integers as they should. Typing \kbd{elladd(e, v[1],v[3])} does indeed show
that it is equal to \kbd{v[4]}.

Taking any other four points, we would in fact always find a dependency.
Let's find them all. Type \kbd{vp = [v[1],v[2],v[3]]\til;
m = ellheightmatrix(e,vp);
matdet(m)}. This is now clearly non-zero so the first 3 points
are linearly independent, showing that the rank of the curve is at least
equal to 3 (it is in fact equal to 3, and \kbd{e} is the curve of smallest
conductor having rank 3). We would like to see whether the other points are
dependent. For this, we use the function \kbd{ellbil}. Indeed, if \kbd{Q} is
some point which is dependent on \kbd{v[1],v[2]} and \kbd{v[3]}, then
\kbd{matsolve(m, ellbil(e, vp,Q))} will by definition give the coefficients
of the dependence relation. If these coefficients are close to integers, then
there is a dependency, otherwise not.  This is much safer than using the
\kbd{matker} function. Thus, type

\centerline{\tt w = vector(18,k, matsolve(m, ellbil(e, vp,v[k])))}

 We ``see'' that the coefficients are all very close to integers, and we can
prove it by typing

\centerline{\tt wr = round(w); sqrt(norml2(w - wr))}

\noindent which gives an upper bound on the maximum distance to an integer.
Thus \kbd{wr} is the vector expressing all the components of \kbd{v} on its
first 3. We are thus led to strongly believe that the curve has rank exactly
3, and this can be proved to be the case.\smallskip

Let's now explore a few more elliptic curve related functions. Let us keep
our curve \kbd{e} of rank 3, and type
\bprog
v1 = [1,0]; v2 = [2,0];
z1 = ellpointtoz(e, v1)
z2 = ellpointtoz(e, v2)
@eprog

We thus get the complex parametrization of the curve. To add the points
\kbd{v1} and \kbd{v2}, we should of course type \kbd{elladd(e, v1,v2)},
but we can also type \kbd{ellztopoint(e, z1 + z2)} which of course has the
disadvantage of giving complex numbers, but illustrates how the group law on
\kbd{e} is obtained from the addition law on $\C$.

Type \kbd{f = x * Ser(ellan(e, 30))}. This gives a power series which
is the Fourier expansion of a modular form of weight 2 for $\Gamma_0(5077)$
(this has been proved directly, but also follows from Wiles' result since
\kbd{e} is semi-stable). In fact, to find the modular parametrization of
the curve, type \kbd{modul = elltaniyama(e)}, then
\kbd{u=modul[1]; v=modul[2];}. Type

\centerline{\tt (v\pow 2 + v) - (u\pow 3 - 7*u + 6)}

\noindent to see that this indeed parametrizes the curve.

Now type \kbd{x * u' / (2*v + 1)}, and we see that this is equal to the
modular form \kbd{f} found above (the quote \kbd{'} tells \kbd{gp} to take the
derivative of the expression with respect to its main variable). The
functions \kbd{u} and \kbd{v}, considered on the upper half plane (with
$x=e^{2i\pi\tau}$), are in fact modular \emph{functions} for $\Gamma_0(5077)$.
\smallskip

Finally, let us come back to the curve defined by typing
\kbd{e = ellinit([0,-1,1,0,0])} where we had seen that the point
\kbd{q = [0,0]} was of order 5. We know that the conductor of this curve is
equal to 11 (type \kbd{ellglobalred(e)}). We want the sign of the functional
equation. Type

\centerline{\kbd{elllseries(e, 1,-11,1.1)}, \quad then \quad
\kbd{elllseries(e, 1,-11,1)}.}

  Since the values are clearly different, the sign cannot be $-$. In fact
there is an algebraic algorithm which would allow to compute this sign, but
it has not yet been completely written, although in case of conductors prime
to 6 it is very simple.

Now type \kbd{ls = elllseries(e, 1,11,1)}, and just as a check
\kbd{elllseries(e, 1,11,1.1)}. The values agree (approximately) as they should,
and give the value of the L-function of \kbd{e} at 1. Now according to
the Birch and Swinnerton-Dyer conjecture (which is proved for this curve),
\kbd{ls} is given by the following formula (in this case):
%
\def\sha{\hbox{III}}
$$L(E,1)=\dfrac{\Omega\cdot c\cdot|\sha|}{|E_{\text{tors}}|^2}\enspace,$$
%
where $\Omega$ is the real period of $E$, $c$ is the global Tamagawa number,
product of the local $c_p$ for primes $p$ dividing the conductor, $|\sha|$ is
the order of the Tate-Shafarevich group, and $E_{\text{tors}}$ is the
torsion group of $E$.

Now we know many of these quantities: $\Omega$ is equal to \kbd{e.omega[1]}
(if there had been 3 real roots instead of 1 in \kbd{e.roots}, $\Omega$ would
be equal to \kbd{2 * e.omega[1]}). The Tamagawa number $c$ is given as the
last component of \kbd{ellglobalred(e)}, and here is equal to 1. We already
know that the torsion subgroup of $E$ contains a point of order 5, and typing
\kbd{torsell(e)} shows that it is of order exactly 5. Hence type
\kbd{ls * 25/e[15]}. This shows that $|\sha|$ must be equal to 1.

\section{Working in Quadratic Number Fields}

The simplest of all number fields outside $\Q$ are quadratic fields. Such
fields are characterized by their discriminant, and even better, any non-square
integer $D$ congruent to 0 or 1 modulo 4 is the discriminant of a specific
order in a quadratic field. We can check whether this order is maximal by
using the command \kbd{isfundamental(D)}. Elements of a quadratic field are
of the form $a+b\omega$, where $\omega$ is chosen as $\sqrt{D}/2$ if $D$ is
even and $(1+\sqrt{D})/2$ if $D$ is odd, and are represented in PARI by
quadratic numbers. To initialize working in a quadratic order, one should
start by the command \kbd{w = quadgen($D$)}.

This sets \kbd{w} equal to $\omega$ as above, and is printed \kbd{w}. Note
however that if several different quadratic orders are used, a printed \kbd{w}
may have several different meanings. For example if you type

\centerline{\tt w1 = quadgen(-23); w2 = quadgen(-15);}

Then ask for the value of \kbd{w1} and \kbd{w2}, both will be printed as
\kbd{w}, but of course they are not equal. Hence beware when dealing with
several quadratic orders at once. \smallskip
%
In addition to elements of a quadratic order, we also want to be able to
handle ideals of such orders. In the quadratic case, it is equivalent to
handling binary quadratic forms, and this has been chosen in PARI. For
negative discriminants, quadratic forms are triples $(a,b,c)$ representing
the form $ax^2+bxy+cy^2$. Such a form will be printed as, and can be created
by, \kbd{Qfb($a$,$b$,$c$)}.

Such forms can be multiplied, divided, powered as many PARI objects using
the usual operations, and they can also be reduced using the function
\kbd{qfbred} (it is not the purpose of this tutorial to explain what all
these things mean). In addition, Shanks's NUCOMP algorithm has been
implemented (functions \kbd{qfbnucomp} and \kbd{qfbnupow}), and this is
usually a little faster.

Finally, you have at your disposal the functions \kbd{qfbclassno} which
(\emph{usually}) gives the class number, the function \kbd{qfbhclassno}
which gives the Hurwitz class number, and the much more sophisticated
\kbd{quadclassunit} function which gives the class number and class group
structure.

Let us see examples of all this at work.

Type \kbd{qfbclassno(-10007)}. \kbd{gp} tells us that the result is 77. However,
you may have noticed in the explanation above that the result is only
\emph{usually} correct. This is because the implementers of the algorithm
have been lazy and have not put the complete Shanks algorithm into PARI,
causing it to fail in certain very rare cases. In practice, it is almost
always correct, and the much more powerful \kbd{quadclassunit} program, which
\emph{is} complete (at least for fundamental discriminants) can give
confirmation (but now, under the Generalized Riemann Hypothesis!!!).

So we may be a little suspicious of this class number. Let us check it.
First, we need to find a quadratic form of discriminant $-10007$. Since this
discriminant is congruent to 1 modulo 8, we know that there is an ideal of
norm equal to 2, i.e.~a binary quadratic form $(a,b,c)$ with $a=2$. To
compute it we type \kbd{f = qfbprimeform(-10007, 2)}. OK, now we have a form.
If the class number is correct, the very least is that this form raised to
the power 77 should equal the identity. Let's check this. Type \kbd{f\pow 77}.
We get a form starting with 1, i.e.~the identity, so this test is OK. Raising
\kbd{f} to the powers 11 and 7 does not give the identity, thus we now know
that the order of \kbd{f} is exactly 77, hence the class number is a multiple
of 77. But how can we be sure that it is exactly 77 and not a proper multiple?
Well, type
\bprog
  sqrt(10007)/Pi * prodeuler(p=2,500, 1./(1 - kronecker(-10007,p)/p))
@eprog
%
This is nothing else than an approximation to the Dirichlet class number
formula. The function \kbd{kronecker} is the Kronecker symbol, in this case
simply the Legendre symbol. Note also that we have written \kbd{1./(1 - \dots)}
with a dot after the first 1. Otherwise, PARI may want to compute the whole
thing as a rational number, which would be terribly long and useless. In fact
PARI does no such thing in this particular case (\kbd{prodeuler} is always
computed as a real number), but you never know. Better safe than sorry!

We find 77.77, pretty close to 77, so things seem in order. Explicit bounds
on the prime limit to be used in the Euler product can be given which make
the above reasoning rigorous.

Let us try the same thing with $D=-3299$. \kbd{qfbclassno} and the Euler
product convince us that the class number must be 27. However, we get stuck
when we try to prove this in the simple-minded way above. Indeed, we type
\kbd{f = qfbprimeform(-3299, 3)} (here 2 is not the norm of a prime ideal but
3 is), and we see that \kbd{f} raised to the power 9 is equal to the identity.
This is the case for any other quadratic form we choose. So we suspect that the
class group is not cyclic. Indeed, if we list all 9 distinct powers of \kbd{f},
we see that \kbd{qfbprimeform(-3299, 5)} is not on the list (although its cube
is as it must). This implies that the class group is probably equal to a
product of a cyclic group of order 9 by a cyclic group of order 3. The Euler
product plus explicit bounds prove this.

Another way to check it is to use the \kbd{quadclassunit} function by typing
for example

\centerline{\tt quadclassunit(-3299)}

Note that this function cheats a little and could still give a wrong answer,
even assuming GRH (we could get a subgroup and not the whole class group).
If we want to use proven bounds under GRH, we have to type

\centerline{\tt quadclassunit(-3299,,[1,6])}

The double comma \kbd{,,} is not a typo, it means we omit an optional second
argument (we would use it to compute the narrow class group, which would be
the same here of course). As we want to use the optional \emph{third}
argument, we have to indicate to \kbd{gp} we skipped this one.

Now, if we believe in GRH, the class group is as we thought (see Chapter 3
for a complete description of this function).

  Note that using the even more general function \kbd{bnfinit} (which handles
general number fields and gives much more complicated results), we could
\emph{certify} this result (remove the GRH assumption). Let's do it, type
\bprog
bnf = bnfinit(x^2 + 3299); bnfcertify(bnf)
@eprog

  A non-zero result (here 1) means that everything is ok. Good, but what did
we certify after all? Let's have a look at this \kbd{bnf} (just type it!).
Enlightening, isn't it? Recall that the \kbd{init} functions (we've already
seen \kbd{ellinit}) store all kind of technical information which you
certainly don't care about, but which will be put to good use by some higher
level functions. That's why \kbd{bnfcertify} could not be used on the output
of \kbd{quadclassunit}: it needs much more data.

  To extract sensible information from such complicated objects, you must use
one of the many \emph{member functions} (remember: \kbd{?.} to get a complete
list). In this case \kbd{bnf.clgp} which extracts the class group structure.
This is much better. Type \kbd{\%.no} to check that this leading 27 is indeed
what we think it is and not some stupid technical parameter. Note that
\kbd{bnf.clgp.no} would work just as well, or even \kbd{bnf.no}!

As a last check, we can request a relative equation for the Hilbert class
field of $\Q(\sqrt{-3299})$: type \kbd{quadhilbert(-3299)}. It is indeed of
degree 27 so everything fits together.

\medskip
%
Working in real quadratic fields instead of complex ones, i.e.~with $D>0$, is
not very different.

The same \kbd{quadgen} function is used to create elements. Ideals are again
represented by binary quadratic forms $(a,b,c)$, this time indefinite. However,
the Archimedean valuations of the number field start to come into play (as
is clear if one considers ideles instead of ideals), hence in fact quadratic
forms with positive discriminant will be represented as a quadruplet
$(a,b,c,d)$ where the quadratic form itself is $ax^2+bxy+cy^2$ with $a$,
$b$ and $c$ integral, and $d$ is the Archimedean component, a real number.
For people familiar with the notion, $d$ represents a ``distance'' as defined
by Shanks and Lenstra.

To create such forms, one uses the same function as for definite ones, but
you can add a fourth (optional) argument to initialize the distance:

\centerline{\tt Qfb($a$, $b$, $c$, $d$)}

If the discriminant of $(a,b,c)$ is negative, $d$ is silently
discarded. If you omit it, this component is set to \kbd{0.} (i.e.~a real zero
to the current precision).

Again these forms can be multiplied, divided, powered, and they can be
reduced using the function \kbd{qfbred}. This function is in fact a
succession of elementary reduction steps corresponding essentially to a
continued fraction expansion, and a single one of these steps can be achieved
by adding an (optional) flag to the arguments of using this function. Since
handling the fourth component $d$ usually involves computing logarithms, the
same flag may be used to ignore the fourth component. Finally, it is
sometimes useful to operate on forms of positive discriminant without
performing any reduction (this is useless in the negative case), the
functions \kbd{qfbcompraw} and \kbd{qfbpowraw} do exactly that.

Again, the function \kbd{qfbprimeform} gives a prime form, but the form which
is given corresponds to an ideal of prime norm which is usually not reduced.
If desired, it can be reduced using \kbd{qfbred}.

Finally, you still have at your disposal the function \kbd{qfbclassno} which
gives the class number (this time \emph{guaranteed} correct),
\kbd{quadregulator} which gives the regulator, and the much more sophisticated
\kbd{quadclassunit} giving the class group's structure and its generators,
as well as the regulator. The \kbd{qfbclassno} and \kbd{quadregulator}
functions use an algorithm which is $O(\sqrt D)$, hence become very slow for
discriminants of more than 10 digits. \kbd{quadclassunit} can be used on a
much larger range.

Let us see examples of all this at work and learn some little known number
theory at the same time. First of all, type
\kbd{d = 3 * 3299; qfbclassno(d)}. We see that the class number is 3 (we know
in advance that it must be divisible by 3 from the \kbd{d = -3299} case above
and Scholz's theorem). Let us create a form by typing
\kbd{f = qfbred(qfbprimeform(d,2), 2)} (the last 2 tells \kbd{qfbred} to
ignore the archimedean component). This gives us a prime ideal of norm
equal to 2. Is this ideal principal? Well, one way to check this, which is
not the most efficient but will suffice for now, is to look at the complete
cycle of reduced forms equivalent to \kbd{f}. Type
\bprog
 g = f; for(i=1,20, g = qfbred(g, 3); print(g))
@eprog\noindent
(this time the 3 means to do a single reduction step, still not using
Shanks's distance). We see that we come back to the form \kbd{f} without
having the principal form (starting with $\pm1$) in the cycle, so the ideal
corresponding to \kbd{f} is not principal.

Since the class number is equal to 3, we know however that \kbd{f\pow 3} will
be a principal ideal $\alpha\Z_K$. How do we find $\alpha$? For this, type
\kbd{f3 = qfbpowraw(f, 3)}. This computes the cube of \kbd{f}, without
reducing it. Hence it corresponds to an ideal of norm equal to $8=2^3$, so we
already know that the norm of $\alpha$ is equal to $\pm8$. We need more
information, and this will be given by the fourth component of the form.
Reduce your form until you reach the unit form (you will have to type
\kbd{qfbred(\%,~1)} exactly 6 times).

Extract the Archimedean component by typing \kbd{c = component(\%, 4)}. By
definition of this distance, we know that
$${\alpha\over{\sigma(\alpha)}}=\pm e^{2c},$$
where $\sigma$ denotes real conjugation in our quadratic field. Thus, if we
type

\centerline{\tt a = sqrt(8 * exp(2*c))}

\noindent and then \kbd{sa = 8 / a}, we know that up to sign, \kbd{a} and
\kbd{sa} are numerical approximations of $\alpha$ and $\sigma(\alpha)$. Of
course, $\alpha$ can always be chosen to be positive, and a quick numerical
check shows that the difference of \kbd{a} and \kbd{sa} is close to an
integer, and not the sum, so that in fact the norm of $\alpha$ is equal to
$-8$ and the numerical approximation to $\sigma(\alpha)$ is \kbd{$-$sa}. Thus
we type

\centerline{\tt p = x\pow 2 - round(a-sa)*x - 8}

\noindent and this is the characteristic polynomial of $\alpha$. We can check
that the discriminant of this polynomial is a square multiple of \kbd{d}, so
$\alpha$ is indeed in our field. More precisely, solving for $\alpha$ and
using the numerical approximation that we have to resolve the sign ambiguity in
the square root, we get explicitly $\alpha=(15221+153\sqrt d)/2$. Note that
this can also be done automatically using the functions \kbd{polred} and
\kbd{modreverse}, as we will see later in the general number field case, or by
solving a system of 2 linear equations in 2 variables.

\noindent{\bf Exercise:} now that we have $\alpha$ explicitly, check that it
is indeed a generator of the ideal corresponding to the form \kbd{f3}.

\medskip Let us now play a little with cycles. Set \kbd{D = 10\pow 7 + 1},
then type

\centerline{\tt quadclassunit(D,,[1,6])}

We get as a result a 5-component vector, which tells us that (under GRH) the
class number is equal to 1, and the regulator is approximately
equal to $2641.5$. If you want to certify this, use \kbd{qfbclassno} and
\kbd{quadregulator}, \emph{not} \kbd{bnfinit} and \kbd{bnfcertify}, which will
take an absurdly long time (well, about 5 minutes if you are careful and set
the initial precision correctly). Indeed \kbd{bnfcertify} needs the fundamental
unit which is so large that \kbd{bnfinit} will have a (relatively) hard time
computing it: you will need about $R/\log(10)\approx 1147$ digits of precision!
On the other hand, you can try \kbd{quadunit(D)}. Impressive, isn't it? (you
can check that its logarithm is indeed equal to the regulator).

Now just as an example, let's assume that we want the regulator to 500
decimals, say (without cheating and computing the fundamental unit exactly
first!). I claim that by simply knowing the crude approximation above, this
can be computed with no effort.

This time, we want to start with the unit form. Since \kbd{D} is odd, we can
type:

\centerline{\tt u = qfbred(Qfb(1,1,(1 - D)/4), 2)}

We use the function \kbd{qfbred} with no distance since we want the initial
distance to be equal to~0.

Now we type  \kbd{f = qfbred(u, 1)}. This is the first form encountered along
the principal cycle. For the moment, keep the precision low, for example the
initial default precision. The distance from the identity of \kbd{f} is
around 4.253. Very crudely, since we want a distance of $2641.5$, this should
be encountered approximately at $2641.5/4.253=621$ times the distance of
\kbd{f}. Hence, as a first try, we type \kbd{f\pow 621}. Oops, we overshot,
since the distance is now $3173.02$. Now we can refine our initial estimate and
believe that we should be close to the correct distance if we raise \kbd{f} to
the power $621*2641.5/3173$ which is close to $517$. Now if we compute
\kbd{f\pow 517} we hit the principal form right on the dot. Note that this is
not a lucky accident: we will always land extremely close to the correct target
using this method, and usually at most one reduction correction step is
necessary. Of course, only the distance component can tell us where we are
along the cycle.

Up to now, we have only worked to low precision. The goal was to obtain this
unknown integer $517$. Note that this number has absolutely no mathematical
significance: indeed the notion of reduction of a form with positive
discriminant is not well defined since there are usually many reduced forms
equivalent to a given form. However, when PARI makes its computations, the
specific order and reductions that it performs are dictated entirely by the
coefficients of the quadratic form itself, and not by the distance component,
hence the precision used has no effect.

Hence we now start again by setting the precision to (for example) 500,
we retype the definition of \kbd{u} (why is this necessary?), and then
\kbd{f = qfbred(u, 1)} and finally \kbd{f\pow 517}. Of course we know in
advance that we land on the unit form, and the fourth component gives us the
regulator to 500 decimal places with no effort at all.

In a similar way, we could obtain the so-called \emph{compact representation}
of the fundamental unit itself, or $p$-adic regulators. I leave this as
exercises for the interested reader.

You can try the \kbd{quadhilbert} function on that field but, since the class
number is $1$, the result won't be that exciting. If you try it on our
preceding example ($3*3299$) it should take about five minutes (time for a
coffee break?).

\section{Working in General Number Fields}

Note for the present release: this section is a little obsolete since many
new powerful functions are available now. This needs to be rewritten
entirely. \smallskip

The situation here is of course more difficult. First of all, remembering
what we did with elliptic curves, we need to initialize it (with something
more serious than \kbd{quadgen}). For example assume that we want to work in
the number field $K$ defined by one of the roots of the equation
$x^4+24x^2+585x+1791=0$. This is done by typing
\bprog
T = x^4 + 24*x^2 + 585*x + 1791
nf = nfinit(T)
@eprog

We get quite a complicated object but, thanks to member functions, we don't
need to know anything about its internal structure (which is dutifully
described in Chapter~3). If you type \kbd{nf.pol}, you will get the
polynomial \kbd{T} which you just input. \kbd{nf.sign} yields the signature
$(r_1,r_2)$ of the field, \kbd{nf.disc} the field discriminant, \kbd{nf.zk}
an integral basis, etc\dots.

The integral basis is expressed in terms of a generic root \kbd{x} of \kbd{T}
and we notice it's very far from being a power integral basis, which is a
little strange for such a small field. Hum, let's check that: \kbd{poldisc(T)}?
Ooops, small wonder we had such denominators, the index is, well, type
\kbd{sqrt(\% / nf.disc)}. That's $3087$, we don't want to work with such
a badly skewed polynomial!

  So, type \kbd{P = polred(T)}. We see from the third component that the
polynomial $x^4-x^3-21x^2+17x+133$ defines the same field with much smaller
coefficients, so type \kbd{A = P[3]}. The \kbd{polred} function gives a
(usually) simpler polynomial, and also sometimes some information on the
existence of subfields. For example in this case, the second component of
\kbd{polred} tells us that the field defined by $x^2-x+1=0$, i.e.~the field
generated by the cube roots of unity, is a subfield of our number field $K$.
Note this is given as incidental information and that the list of subfields
found in this way is usually far from complete. To get the complete list, you
will have to use the function \kbd{nfsubfields} (we'll do that later on).

  Type \kbd{poldisc(A)}, this is much better, but maybe not optimal yet
(the index is still $7$). Type \kbd{polredabs(A)} (the \kbd{abs} stands for
absolute). Since it seems that we won't get anything better, we'll stick with
\kbd{A} (note however that \kbd{polredabs} finds a smallest generating
polynomial with respect to a bizarre norm which ensures that the index will
be small, but not necessarily minimal). In fact, had you typed
\kbd{nfinit(T, 3)}, \kbd{nfinit} would first have tried to find a good
polynomial defining the same field (i.e.~one with small index) before
proceeding.

  It's not too late, let's redefine our number field: \kbd{NF = nfinit(nf, 3)}.
The output is a two-component vector. The first component is the new
\kbd{nf} (type \kbd{nf = NF[1];}). If you type \kbd{nf.pol}, you notice that \kbd{gp}
indeed replaced your bad polynomial \kbd{T} by a much better one, which
happens to be \kbd{A} (small wonder, \kbd{nfinit} internally called
\kbd{polredabs}!). The second component enables you to switch conveniently to
our new polynomial.

Namely, call $\theta$ a root of our initial polynomial \kbd{T}, and $\alpha$
a root of the one that \kbd{polred} has found, namely \kbd{A}. These are
algebraic numbers, and as already mentioned are represented as polmods. For
example, in our special case $\theta$ is equal to the polmod

\centerline{\tt Mod(x, x\pow 4 + 24*x\pow 2 + 585*x + 1791)}

\noindent while $\alpha$ is equal to the polmod

\centerline{\tt Mod(x, x\pow 4 - x\pow 3 - 21*x\pow 2 + 17*x + 133)}

\noindent Here of course we are considering only the algebraic aspect, and
hence ignore completely \emph{which} root $\theta$ or $\alpha$ is chosen.

Now probably you may have a number of elements of your number field which are
expressed as polmods with respect to your old polynomial, i.e.~as
polynomials in $\theta$. Since we are now going to work with $\alpha$
instead, it is necessary to convert these numbers to a representation using
$\alpha$. This is what the second component of \kbd{NF} is for: type
\kbd{NF[2]}, you get

\centerline{\tt Mod(x\pow 2 + x - 11, x\pow 4 - x\pow 3 - 21*x\pow 2 +
17*x + 133)}

\noindent meaning that $\theta = \alpha^2+\alpha-11$, and hence the conversion
from a polynomial in $\theta$ to one in $\alpha$ is easy, using \kbd{subst}
(we could get this polynomial from \kbd{polred} as well, try
\kbd{polred(T, 2)}). If we want to do the reverse, i.e.~go back from a
representation in $\alpha$ to a representation in $\theta$, we use the
function \kbd{modreverse} on this polynomial \kbd{NF[2]}. Try it. The result
has a big denominator (147) essentially because our initial polynomial
\kbd{T} was so bad. By the way to get the 147, you should type
\kbd{denominator(content(NF[2]))}. Trying \kbd{denominator} by itself would not
work since the denominator of a polynomial is defined to be 1 (and its
numerator is itself). The reason for this ``surprising'' behaviour is that we
think of a polynomial as a special case of a rational function. \smallskip

From now on, we completely forget about \kbd{T}, and use only the polynomial
\kbd{A} defining $\alpha$, and the components of the vector \kbd{nf} which
gives information on our number field $K$. Type

\centerline{\tt u = Mod(x\pow 3 - 5*x\pow 2 - 8*x + 56, A) / 7}

This is an element in $K$. There are three essentially equivalent
representations for number field elements: polmod, polynomial, and column
vector giving a decomposition in the integral basis \kbd{nf.zk} (\emph{not} on
the power basis $(1,x,x^2,\dots)$). All three are equally valid when the
number field is understood (is given as first argument to the function).
You will be able to use any one of them as long as the function you call
requires an \kbd{nf} argument as well. However, most PARI functions will
return elements as column vectors.

  It's a very important feature of number theoretic functions that, although
they may have a preferred format for input, they will accept a wealth of
other different formats. We already saw this for \kbd{nfinit} which
accepts either a polynomial or an \kbd{nf}. It will be true for ideals,
ideles, congruence subgroups, etc.

  Ok, let's stick with elements for the time being. How does one go from one
representation to the other? Between polynomials and polmods, it's easy:
\kbd{lift} and \kbd{Mod} will do the job. Next, from polmods/polynomials to
column vectors: type \kbd{v = nfalgtobasis(nf, u)}. So $\kbd{u} = \alpha^3-
\alpha^2 - \alpha + 8$, right? Wrong! The coordinates of \kbd{u} are given
with respect to the \emph{integral basis}, not the power basis
$(1,\alpha,\alpha^2,\alpha^3)$ (and they don't coincide, type \kbd{nf.zk} if
you forgot what the integral basis looked like). As a polynomial in $\alpha$,
we simply have $\kbd{u} = {1\over7}\alpha^3 -
{5\over7}\alpha^2-{8\over7}\alpha+8$, which is trivially deduced from the
original polmod representation!

Of course \kbd{v = nfalgtobasis(nf, lift(u))} would work equally well. Indeed
we don't need the polmod information since \kbd{nf} already provides the
defining polynomial. To go back to polmod representation, use
\kbd{nfbasistoalg(nf, v)}. Notice that \kbd{u} is in fact an integer since
\kbd{v} has integer coordinates (try \kbd{denominator(v) == 1}, which is of
course overkill here, but not so in a program).

Let's try this out. We may for instance compute \kbd{u\pow 3}. Try it. Or, we
can type \kbd{1/u}. Better yet, if we want to know the norm from $K$ to $\Q$
of \kbd{u}, we type \kbd{norm(u)} (what else?). \kbd{trace(u)} works as well.
Notice that none of this would work on polynomials or column vectors since
you don't have the opportunity to supply \kbd{nf}! But we could use
\kbd{nfeltpow(nf,u,3)}, \kbd{nfeltdiv(nf,1,u)} (or \kbd{nfeltpow(nf,u,-1)})
which would work whatever representation was chosen. There is no
\kbd{nfeltnorm} function (\kbd{nfelttrace} does not exist either), but we can
easily write one:
\bprog
nfeltnorm(nf,u) =
{
  local(t);
  t = type(u);
  if (t != "t_POLMOD",
    if (t == "t_COL",
      u = nfbasistoalg(nf, u)
    ,
      u = Mod(u, nf.pol)
    )
  );
  norm(u)
}
@eprog

Notice that this is certainly not foolproof (try it with complex or quadratic
arguments!), but we could refine it if the need arose. In fact there was no
need for this function, since you can consider (\kbd{u}) as a principal
ideal, and just type \kbd{idealnorm(nf,u)} whatever the chosen representation
for \kbd{u}. We'll talk about ideals later on.

  If we want all the symmetric functions of \kbd{u} and not only the norm, we
type \kbd{charpoly(u)} (we could write \kbd{charpoly(u, y)} to tell \kbd{gp} to
use the variable \kbd{y} for the characteristic polynomial). Note that this
gives the characteristic polynomial of \kbd{u}, and not in general the
minimal polynomial. Exercises: how does one (easily) find the minimal
polynomial from this? Find a simpler expression for \kbd{u}.\smallskip

  OK, now let's work on the field itself. The \kbd{nfinit} command already
gave us some information. The field is totally complex (its signature
\kbd{nf.sign} is $[0,2]$), its discriminant \kbd{nf.disc} is $D=18981$ and
$(1,\alpha, \alpha^2, {1\over7}\alpha^3+{2\over7}\alpha^2+{6\over7}\alpha)$
is an integral basis (\kbd{nf.zk}). The Galois group of its Galois closure
can be obtained by typing \kbd{polgalois(A)}. The answer ($[8,-1,1]$) shows
that it is equal to $D_4$, the dihedral group with 8 elements, i.e.~the group
of symmetries of a square. \smallskip

This implies that the field is ``partially Galois'', i.e.~that there exists
at least one non-trivial field isomorphism which fixes $K$ (exactly one in
this case). To find out which it is, we use the function \kbd{nfgaloisconj}.
This uses the LLL algorithm to find linear relations. So type
\kbd{nfgaloisconj(nf)}. The result tells us that, apart from the trivial
automorphism, the map $\alpha \mapsto (-\alpha^3+5\alpha^2+\alpha-49)/7$ (in
the third component) is a field automorphism. Indeed, if we type
\kbd{s = Mod(\%[3], A); charpoly(s)}, we obtain the polynomial \kbd{A} once
again. \smallskip

The fixed field of this automorphism is going to be the only non-trivial
subfield of $K$. I seem to recall that \kbd{polred} told us this was the
third cyclotomic field. Let's check this: type \kbd{nfsubfields(nf)}. Indeed,
there's a quadratic subfield, but it's given by \kbd{Q = x\pow 2 + 22*x + 133
} and I don't recognize it. Now \kbd{polred(Q)} proves that this subfield is
indeed the field generated by a cube root of unity. Let's check that \kbd{s}
is of order 2: \kbd{subst(lift(s), x, s)}. Yup, it is. Let's express it as a
matrix:
\bprog
{
  v = [;]; b = nf.zk;
  for (i=1, 4,
    v = concat(v, nfalgtobasis(nf, nfgaloisapply(nf, s, b[i])))
  )
}
@eprog

\kbd{v} gives the action of \kbd{s} on the integral basis. Let's check
\kbd{v\pow2}. That's the identity all right. \kbd{k = matker(v-1)} is indeed
two-dimensional, and \kbd{z = nfbasistoalg(nf, k[,2])} generates the
quadratic subfield. Notice that 1, \kbd{z} and \kbd{u} are $\Q$-linearly
dependent (and in fact $\Z$-linearly as well). Exercise: how would you check
these two assertions in general? (Answer: \kbd{concat}, then respectively
\kbd{matrank} or \kbd{matkerint} (or \kbd{qflll})). \kbd{z = charpoly(z)},
\kbd{z = gcd(z,z')} and \kbd{polred(z)} tell us that we found back the same
subfield again (as we ought to!).

As a final check, type \kbd{nfrootsof1(nf)}. Again we find that $K$ contains
a cube root of unity, since the torsion subgroup of its unit group
is of order 6. And the given generator happens to be equal to \kbd{u} (so if
you did not do the above exercise as you should have, you now know the answer
anyway).
\smallskip

\noindent{\bf Additional comment} (you're not supposed to skip this anymore,
but do as you wish):

Before working with ideals, let us note one more thing. The main part of the
work of \kbd{polred} or \kbd{nfinit} is to compute an integral basis, i.e.~a
$\Z$-basis of the maximal order $\Z_K$ of $K$. For a large polynomial, this
implies factoring the discriminant of the polynomial, which is very often out
of the question. There are two ways in which the situation may be improved:

1) First, it is often the case that the polynomial that one considers is of
quite a special type, giving some information on the discriminant. For
example, one may know in advance that the discriminant is a square. Hence we
can ``help'' PARI by giving it that information. More precisely, using the
extra information that we have we may be able to factor the discriminant of
the polynomial. We can then use the function \kbd{addprimes} to inform
PARI of this factorization. Namely, add the primes which are known to
divide the discriminant; this will save PARI some work. Do it only for big
primes (bigger than \kbd{primelimit}, whose value you can get using
\kbd{default})~--- it will be useless otherwise.

2) The second way in which the situation may be improved is that often we do
not need the complete information on the maximal order, but only require that
the order be $p$-maximal for a certain number of primes $p$ (but then, we
may not be able to use the functions which require a genuine \kbd{nf}). The
function \kbd{nfbasis} specifically computes the integral basis and is not
much quicker than \kbd{nfinit} so is not very useful in its standard use. But
you can provide a factorization of the discriminant as an optional third
argument. And here we can cheat, and give on purpose an incomplete
factorization involving only the primes we want. For example coming
back to our initial polynomial $T$, the discriminant of the polynomial is
$3^7\cdot7^6\cdot19\cdot37$. If we only want a $7$-maximal order, we simply
type

\centerline{\tt nfbasis(T, ,[7,6; 1537461,1])}

\noindent and the factors of 1537461 will not be looked at! (of course in
this example it would be stupid to cheat, but if the discriminant has 2000
digits, this can be a handy trick).\medskip

We now would like to work with ideals (and even with ideles) and not only
with elements. An ideal can be represented in many different ways. First, an
element of the field (in any of the various guises seen above) will be
considered as a principal ideal. Then the standard representation is a
square matrix giving the Hermite Normal Form of a $\Z$-basis of the ideal
expressed on the integral basis. Standard means that most ideal related
functions will use this representation for their output. Note that, as
mentioned before, we always represent elements on the integral basis and not
on a power basis.

Prime ideals can be represented in a special form as well (see the
description of \kbd{idealprimedec} in Chapter~3) and all ideal-related
functions will accept them. On the other hand, the function \kbd{idealtwoelt}
can be used to find a two-element $\Z_K$-basis of a given ideal (as $a\Z_K +
b\Z_K$, where $a$ and $b$ belong to $K$), but this is \emph{not} a valid
representation for an ideal under \kbd{gp}, and most functions will choke on it (or
worse, take it for something else and output a completely meaningless
result). To be able to use such an ideal, you will first have to convert it
to HNF form.

Whereas it's very easy to go to HNF form (use \kbd{idealhnf(nf,id)} for valid
ideals, or \kbd{idealhnf(nf,a,b)} for a two-element representation as above),
it's a much more complicated problem to check whether an ideal is principal
and find a generator. In fact an \kbd{nf} does not contain enough data for
this particular task. We'll need a Big Number Field (or \kbd{bnf}) for that
(in particular, we need the class group and fundamental units). More on this
later.\smallskip

 An ``idele'' will be represented as a 2-element vector, the first element
being the corresponding ideal (in any valid form), which summarizes the
non-archimedean information, and the second element a vector of real and
complex numbers with $r_1+r_2$ components, the first $r_1$ being real, the
remaining ones complex. In fact, to avoid certain ambiguities, the first
$r_1$ components are allowed to have an imaginary part which is a multiple of
$\pi$. These $r_1+r_2$ components correspond to the Archimedean places of the
number field $K$. \medskip

Let us keep our number field $K$ as above, and hence the vector \kbd{nf}. Type

\centerline{\tt P = idealprimedec(nf,7)}

This gives the decomposition of the prime number 7 into prime ideals. We have
chosen 7 because it is the index of $\Z[\theta]$ in $\Z_K$, hence is the most
difficult case to treat. The index is given in \kbd{nf[4]} and cannot be
accessed through member functions since it's rarely needed. It is in any case
trivial to compute as \kbd{sqrtint(poldisc(nf.pol) / nf.disc)}.

The result is a vector with 4 components, showing that 7 is totally split in
the field $K$ into four prime ideals of norm 7 (you can check:
\kbd{idealnorm(nf,P[1])}). Let us take one of these ideals, say the first, so
type \kbd{pr = P[1]}. We would like to have the Hermite Normal Form of this
ideal. No problem: since ideal multiplication always gives the result in HNF,
we will simply multiply our prime ideal by $\Z_K$, which is represented by the
identity matrix. So type

\centerline{\tt idealmul(nf, matid(4), pr)}

\noindent or in fact simply

\centerline{\tt idealmul(nf, 1, pr)}

\noindent or even simpler yet

\centerline{\tt idealhnf(nf,pr)}

\noindent and we have the desired HNF. Let's now perform ideal operations.
For example type

\centerline{\tt idealmul(nf, pr, idealmul(nf, pr,pr))}

\noindent or more simply

\centerline{\tt pr3 = idealpow(nf, pr,3)}

\noindent to get the cube of the ideal \kbd{pr}. Since the norm of this ideal
is equal to $343=7^3$, to check that it is really the cube of \kbd{pr} and
not of other ideals above 7, we can type

\centerline{\tt for(i=1,4, print(idealval(nf, pr3,P[i])))}

\noindent and we see that the valuation at \kbd{pr} is equal to 3, while the
others are equal to zero. We could see this as well from
\kbd{idealfactor(nf, pr3)}.

Let us now ``idelize'' \kbd{pr3} by typing \kbd{id3 = [pr3, [0,0]]}.
(We need $r_1+r_2=2$ components for the second vector.) Then type
\kbd{r1 = idealred(nf, id3)}. We get a new ideal which is equivalent to the
first (modulo the principal ideals). The Archimedean component is non-trivial
and gives the distance from the reduced ideal to the original one (see H. Cohen
\emph{A Course in Computational Algebraic Number Theory}, GTM {\bf 138} for
details, especially Sections 5.8.4 and 6.5). Now, just for fun type

\centerline{\tt r = r1; for(i=1,3, r = idealred(nf,r,[1,5]); print(r))}

We see that the third \kbd{r} is equal to the initial \kbd{r1}. This means
that we have found a unit in our field, and it is easy to extract this unit
given the Archimedean information (clearly it would be impossible without).
We first form the difference of the Archimedean contributions of \kbd{r} and
\kbd{r1} by typing

\centerline{\tt arch = r[2] - r1[2]; l1 = arch[1]; l2 = arch[2];}

From this, we obtain the logarithmic embedding of the unit by typing

\centerline{\tt l = real(l1 + l2) / 4;
                v1 = [l1,l2,conj(l1),conj(l2)]\til / 2 - [l,l,l,l]\til; }

This 4-component vector contains by definition the logarithms of the
four complex embeddings of the unit. Since the matrix \kbd{nf[5][1]}
contains the values of the $r_1+r_2$ embeddings of the elements of the
integral basis, we can obtain the representation of the unit on the
integral basis by typing
\bprog
{
  m1 = nf[5][1];
  m = matrix(4,4,j,k,
    if (j<=2,
      m1[j,k]
    ,
      conj(m1[j-2, k])
    )
  );
  v = exp(v1);
  au = matsolve(m,v);
  vu = round(real(au))
}
@eprog

Then \kbd{vu} is the representation of the unit on the integral basis.
The closeness of the approximation of \kbd{au} to \kbd{vu} gives us
confidence that we have made no numerical mistake. To be sure that \kbd{vu}
represents a unit, type \kbd{u = nfbasistoalg(nf,vu)}, then typing
\kbd{norm(u)} we see that it is equal to 1 so \kbd{u} is a unit.

There is of course no reason for \kbd{u} to be a fundamental unit. Let us see
if it is a square. Type \kbd{f1 = factor(subst(charpoly(u,x), x, x\pow 2))}.
We see that the characteristic polynomial of \kbd{u} where \kbd{x} is
replaced by \kbd{x\pow 2} is a product of 2 polynomials of degree 4, hence
\kbd{u} is a square (Exercise: why?).

We now want to find the square root of \kbd{u}. We can again use the
\kbd{matsolve} function as above. For this we need to take the square
root of each element of the vector \kbd{v}, and hence there are
sign ambiguities. Let's do it anyway. Type \kbd{v = sqrt(v)}. We see that
\kbd{v[1]} and \kbd{v[3]} are conjugates, as well as \kbd{v[2]} and \kbd{v[4]},
so for the moment the signs seem OK. Now try \kbd{au = matsolve(m,v)}. The
numbers obtained are clearly not integers, hence the last remaining sign
change must be performed. Type \kbd{v[1] = -v[1]; v[3] = -v[3]} (they must
stay conjugate) and then again \kbd{au = matsolve(m,v)}. This time the
components are close to integers, so we are done (after typing
\kbd{vu = round(real(au))} as before).

Anyway, we find that a square root \kbd{u2} of \kbd{-u} is represented by
the vector \kbd{vu=[-4,1,1,-1]\til} on the integral basis, and this is in
fact a fundamental unit.\medskip

The function \kbd{polred} gives us another method to find \kbd{u2} as
follows: type \kbd{q = polred(f1[1,1], 2)}. We recognize the polynomial
\kbd{A} as the component \kbd{q[3,2]}. To obtain the square root of our unit
we then simply type \kbd{up2 = modreverse(Mod(q[3,1], f1[1,1]))}
(Exercise: why?). We find that \kbd{up2} is represented by the vector
\kbd{[-3,-1,0,0]\til} on the integral basis, which is not the result that we
have found before nor its opposite. Where is the error? (Please think about
this before reading on. There is a mathematical subtlety hidden here.)

Have you solved the problem? Good! The problem occurs because as mentioned
before (but you may not have noticed since it is not stressed in standard
textbooks) although the number field $K$ is not Galois over $\Q$, there does
exist a non-trivial automorphism, and we have found it above by using the
function \kbd{nfgaloisconj}. Indeed, if we apply this automorphism to
\kbd{up2} (by typing \kbd{nfgaloisapply(nf,s,up2)} where \kbd{s} is the
non-trivial component of \kbd{nfgaloisconj(nf)} computed above), we find the
opposite of \kbd{u2}, which is OK. \smallskip

Still another method which avoids all sign ambiguities and automorphism
problems is as follows. Type \kbd{r = f1[1,1] \% (x\pow 2 - u)} to find the
remainder of the characteristic polynomial of \kbd{u2} divided by
\kbd{x\pow 2 - u}. This will be a polynomial of degree 1 in \kbd{x} (with
polmod coefficients) and we know that \kbd{u2}, being a root of both
polynomials, will be the root of \kbd{r}, hence can be obtained by typing
\kbd{u2 = -coeff(r,0) / coeff(r,1)}. Indeed, we immediately find the correct
result with no trial and error.\smallskip

  Still another method to find the square root of \kbd{u} is to use
\kbd{nffactor(nf,y\pow 2 + u)}. Except that this won't work as is since the
main variable of the polynomial to be factored must have \emph{higher}
priority than the number field variable. This won't be possible here since
\kbd{nf} was defined using the variable \kbd{x} which has the highest possible
priority. So we need to substitute variables around (using \kbd{subst}). I
leave you to work out the details.\smallskip

Now ideals can be used in a wide variety of formats. We have already seen
HNF-representations, ideles and prime ideals. We can also use algebraic
numbers. For example type \kbd{al = Mod(x\pow 2 - 9, A)}, then
\kbd{ideleprincipal(nf,al)}. We obtain the idele corresponding to \kbd{al}
(see the manual for the exact description). However it is usually not
necessary to compute this explicitly since this is done automatically inside
PARI. For example, you can type

\centerline{\tt for(i=1,4, print(i ": " idealval(nf,al,P[i])))}

We see that the valuation is non-zero (equal to 1) at the prime ideals
\kbd{P[2]} and \kbd{P[3]}. In addition, typing \kbd{norm(al)} shows that
\kbd{al} is of norm $49=7^2$ (\kbd{idealnorm(nf,al)} gives the same result of
course, and would be more generic). Let's check this differently. Type
\kbd{P23 = idealmul(nf,P[2],P[3])} and then \kbd{idealhnf(nf,al)}. We see that
the results are the same, hence the product of the two prime ideals \kbd{P[2]}
and \kbd{P[3]} is equal to the principal ideal generated by \kbd{al}. There is
still something to be done with this example as we shall see below after we
introduce Big Number Fields (which will trivialize the examples we have just
seen).

Essentially all functions that you would want on ideals are available.
We mention here the complete list, referring to Chapter 3 for detailed
explanations:

\kbd{idealadd}, \kbd{idealaddtoone}, \kbd{idealappr}, \kbd{idealchinese},
\kbd{idealcoprime}, \kbd{idealdiv}, \kbd{idealfactor}, \kbd{idealhnf},
\kbd{idealintersect}, \kbd{idealinv}, \kbd{ideallist}, \kbd{ideallog},
\kbd{idealmin}, \kbd{idealmul}, \kbd{idealnorm}, \kbd{idealpow},
\kbd{idealprimedec}, \kbd{idealprincipal}, \kbd{idealred},
\kbd{idealstar}, \kbd{idealtwoelt}, \kbd{idealval}, \kbd{ideleprincipal},
\kbd{nfisideal}.

We suggest you play with these functions to get a feel for the algebraic
number theory package. Remember simply that when a matrix (usually in Hermite
normal form) is output, it is always a $\Z$-basis of the result expressed on
the \emph{integral basis} \kbd{nf.zk} of the number field, which is usually
\emph{not} a power basis. \medskip

Apart from the above functions you have at your disposal the very powerful
functions \kbd{bnfclassunit} which is of the same type as \kbd{quadclassunit}
seen above, but for general number fields, and hence much slower. See Chapter
3 for a detailed explanation of its use.

First type \kbd{setrand(1)}: this resets the random seed (to make sure we get
the exact same results). Now type \kbd{m = bnfclassunit(A)}, where \kbd{A} is
the same polynomial as before. After some work, we get a matrix with one
column, which does not look too horrible (this is because much of the really
useful information has been discarded, this is not an \kbd{init} function).
Let's extract the column with \kbd{v = m[,1]}.

Then \kbd{v} is a vector with 10 components. We immediately recognize the first
component as being the polynomial \kbd{A} once again. In fact member
functions are still available for \kbd{m}, even though it was not created
by an \kbd{init} function. So, let's try them: \kbd{m.pol} gives \kbd{A},
\kbd{m.sign}, \kbd{m.disc}, \kbd{m.zk}, ok nothing really exciting. But new
ones are available now: \kbd{m.no} tells us the class number is 4,
\kbd{m.cyc} that it's cyclic (of order 4 but that we already knew),
\kbd{m.gen} that it's generated by a prime ideal above seven (in HNF form). If
you play a little bit with the \kbd{idealhnf} function you see that this ideal
is \kbd{P[4]}.

 The regulator \kbd{m.reg} is equal to $3.794\dots$. \kbd{m.tu} tells us that
the roots of unity in $K$ are exactly the sixth roots of 1 and gives a
primitive root. Finally \kbd{m.fu} gives us a fundamental unit (which must be
linked to the unit \kbd{u2} found above since the unit rank is 1).

To find the relation (without trial and error, because in this case it is
quite easy to see it directly!), let us use the logarithmic embeddings. Type
\kbd{uf = m.fu[1]}, \kbd{uu= m.tu[2]} to get the generators of the unit
group, then
\bprog
cu2 = log(conjvec(u2));
cuf = log(conjvec(Mod(uf,A)));
cuu = log(conjvec(Mod(uu,A)));
@eprog\noindent
to get the (complex) logarithmic embeddings. Then type

\kbd{lindep(real([cu2[1], cuf[1], cuu[1]]))}

\noindent to find a linear dependence. Unfortunately, the result $[0,0,1]$ is
to be expected since \kbd{uu} is a root of unity, hence the components of
\kbd{cuu} are pure imaginary. Hence we must not take the real part. However,
in that case the logs are defined only up to a multiple of $2i\pi$. Hence we
type

\kbd{lindep([cu2[1], cuf[1], cuu[1], 2*I*Pi])}

The result $[1,-1,-3,0]$ shows that \kbd{u2} is probably equal to
\kbd{uf * uu\pow 3}, which is clear since \kbd{uu} is a sixth root of unity,
and which can be checked directly. This works quite nicely, but we'll see
later on a much better method.

{\bf Note:} since the fundamental unit obtained depends on the random
seed, you could have obtained a different unit from the one given here, had
you not reset the random seed before the computation. This was the purpose
of the initial \kbd{setrand} instruction (which was otherwise completely
unnecessary of course). \medskip

If you followed closely what component of \kbd{m} the various member
functions were giving (after possibly some cosmetic change), you must have
noticed that the seventh and tenth components were not given. You may ignore
the last component of \kbd{v} (it gives the accuracy to which the fundamental
unit was computed).  The seventh is a technical check number, which must be
close to 1 (between $1/\sqrt2$ and $\sqrt2$ at the very least). This tells
us, under LOTS of assumptions (notably the Generalized Riemann Hypothesis),
that our result is correct (i.e.~neither the class group nor the regulator are
smaller). We'll see shortly how to remove all those assumptions: we need much
more data than was output here. This is a little wasteful since the missing
data \emph{were} computed, then discarded to provide a human-readable output.
Thus you will usually not use this function, but its \kbd{init} variant:
\kbd{bnfinit}.
\smallskip

So type: \kbd{setrand(1); bnf = bnfinit(A);}

This performs exactly the same computations as \kbd{bnfclassunit}, but keeps
much more information. In particular, you don't \emph{want} to see the result,
whence the semicolon (if you really want to, you can have a look, it's only
about three screenful). All the member functions that we used previously
still apply to this \kbd{bnf} object. In fact, \kbd{bnfinit} even recomputed
the information provided by \kbd{nfinit} (we could have typed
\kbd{bnfinit(NF)} to avoid the waste), and it's included in the \kbd{bnf}
structure: \kbd{bnf.nf} should be exactly identical to \kbd{NF}. Thus, all
functions which took an \kbd{nf} as first argument, will equally accept a
\kbd{bnf} (and a \kbd{bnr} as well which contains even more data).

We are now ready to perform more sophisticated operations in the class group.
First and foremost, we can now certify the result: type
\kbd{bnfcertify(bnf)}. The output (\kbd{1} if all went well) is not that
impressive, but it means that we now know the class group and fundamental
units unconditionally (in particular, the GRH assumption could be removed)!
In this case, the certification process takes a very short time, and you
might wonder why it was not built in as a final check in the \kbd{bnfinit}
function. The answer is that as the regulator gets bigger this process gets
increasingly difficult, and becomes soon impractical, while \kbd{bnfinit}
still happily spits out results. So it makes sense to dissociate the two: you
can always make the check afterwards, if the result is interesting enough
(and looking at the tentative regulator, you know in advance whether the
certification can possibly succeed: if \kbd{bnf.reg} is about 2000, don't
waste your time).

Ok, now that we feel safe about the \kbd{bnf} output, let's do some real
work. For example, let us take again our prime ideal \kbd{pr} above 7. Since
we know that the class group is of order 4, we deduce that \kbd{pr} raised to
the fourth power must be principal. Type \kbd{pr4 = idealpow(nf, pr, 4)} then
\kbd{vis = bnfisprincipal(bnf,pr4)}. The function \kbd{bnfisprincipal} now
uses all the information contained in \kbd{bnf} and tells us that indeed
\kbd{pr4} is principal. More precisely, the first component, \kbd{[0]}, gives
us the factorization of the ideal in the class group. Here, \kbd{[0]} means
that it is up to equivalence equal to the 0-th power of the generator given
in \kbd{bnf.gen}, in other words that it is a principal ideal.

The second component gives us the algebraic number $\alpha$ such that
\kbd{pr4$=\alpha\Z_K$}, where $\Z_K$ is the ring of integers of our number
field, $\alpha$ being as usual expressed on the integral basis. To get $\alpha$
as an algebraic number, we type \kbd{alpha = nfbasistoalg(bnf,vis[2])} (note
that we can use a \kbd{bnf} with all the \kbd{nf} functions; but not the other
way round, of course).

Let us check that the result is correct: first, type \kbd{norm(alpha)}
(\kbd{idealnorm(nf, vis[2])} would have worked directly). It is indeed equal to
$7^4 = 2401$, which is the norm of \kbd{pr4} (it could also have been equal to
$-2401$). This is only a first check. The complete check is obtained by
computing the HNF of the principal ideal generated by \kbd{alpha}. To do this,
type \kbd{idealhnf(nf,alpha)}.

The result that we obtain is identical to \kbd{pr4}, thus showing that
\kbd{alpha} is correct (not that there was any doubt!). You may ignore the
third component of \kbd{vis} which just tells us that the accuracy to which
\kbd{alpha} was computed was amply sufficient.

But \kbd{bnfisprincipal} also gives us information for non-principal ideals.
For example, type

\kbd{vit = bnfisprincipal(bnf, pr)}.

The component \kbd{vit[1]} is now equal to $[3]$, and tells us that \kbd{pr}
is ideal-equivalent to the cube of the generator \kbd{g} given by
\kbd{bnfinit}. Of course we already knew this since the product of \kbd{P[2]}
and \kbd{P[3]} was principal, as well as the product of all the \kbd{P[$i$]}
(generated by 7), and we noticed that \kbd{P[4]} was of order 4 when we looked
at \kbd{bnf.gen}.

The second component \kbd{vit[2]} gives us $\alpha$ on the integral basis
such that \kbd{pr$=\alpha \kbd{g}^3$}. Note that if you \emph{don't} want
this $\alpha$, which may be large and whose computation may take some time,
you can just add a flag (1 in this case, see the online help) to the
arguments of the \kbd{bnfisprincipal} function, so that it only returns the
position of \kbd{pr} in the class group. \smallskip

Let us now take the example of the principal ideal \kbd{P23} that we have
seen above. We know that \kbd{P23} is principal, but of course we have
forgotten the generator that we had found, so we need another one. For this,
we type \kbd{pp = bnfisprincipal(bnf,P23)}. The first component of the result
is \kbd{[0]}, telling us that the ideal is indeed principal, as we knew
already. But the second component gives us the components of a generator of
\kbd{P23} on the integral basis. Now we remember suddenly that we already had
a generator \kbd{al} for the same ideal. Hence type
\kbd{u3=nfeltdiv(nf,al2,al)}. This must be a unit. It's already obviously an
integer and \kbd{idealnorm(nf,\%)} tells us that \kbd{u3} is indeed a unit.
You can again find out what unit it is as we did above. However, as we
mentioned, this is not really the best method.

To find the unit, we use explicitly the third component of the vector
\kbd{bnf} given by \kbd{bnfinit}. This contains an $(r+1)\times r$ complex
matrix whose columns represent the complex logarithmic embedding of the
fundamental units. Here $r=r_1+r_2-1$ is the unit rank. We first compute
the component of \kbd{u3} on the torsion-free part of the group of units
by proceeding as follows. Type

\kbd{me = concat(bnf[3],[2,2]\til)}.

Indeed, this is a variant of the regulator matrix and is more practical to
use since it is more symmetric and avoids suppressing one row arbitrarily.
Now type

\kbd{cu3 = ideleprincipal(nf,u3)[2]\til}

to get the complex logarithmic embedding of \kbd{u3} (as a column vector). We
could of course also have computed this logarithmic embeddings directly using
\kbd{conjvec} as we did above, but then we must take care of the factors of 1
and 2 occurring.

Then type \kbd{xc = matsolve(real(me), real(cu3))}

Whatever field we are in, if \kbd{u3} is a unit this \emph{must} end with a 0
(approximate of course) because of the ``spurious'' vector $[2,2]$, and the
other components (here only one) give the exponents on the fundamental units.
Here the only other component is the first, with a coefficient of $1$ (we
could type \kbd{round(xc)} to tidy up the result). So we know that \kbd{u} is
equal to \kbd{uf} multiplied by a root of unity.

To find this root of unity, we type \kbd{xd = cu3 - me*xc} then
\kbd{xu = ideleprincipal(nf,uu)[2]} and finally \kbd{xd[1] / xu[1]}. We find
$3$ as a result, so finally our unit \kbd{u3} must be equal to
\kbd{uu\pow 3 * uf} itself, which is the case.

Of course, you don't need to do all that: just type \kbd{bnfisunit(bnf,u3)}.
Like the \kbd{bnfisprincipal} function, this gives us the decomposition of
some object (here a unit) on the precomputed generators (here \kbd{bnf.tufu})
of some abelian group of finite type (here the units of $K$). The result
\kbd{[1,Mod(3,6)]} tells us that \kbd{u3} is equal to \kbd{uu\pow 3 * uf} as
before.\smallskip

Another famous so-called \emph{discrete logarithm} problem can be easily
solved with PARI, namely the one associated to the invertible elements modulo
an ideal: $(\Z_K / I)^*$. Just use \kbd{idealstar} (this is an \kbd{init}
function) and \kbd{ideallog}.

----- TO BE COMPLETED -----

\section{\kbd{gp} Programming}

----- TO BE WRITTEN -----

\section{Plotting}

PARI supports a multitude of high and low-level graphing functions, on a
variety of output devices: a special purpose window under the \kbd{X
Windows} system, a \kbd{PostScript} file ready for the printer, or a
\kbd{gnuplot} output device (only the first two are available by default).
These functions use a multitude of flags, which are mostly power-of-2. To
simplify understanding we first give these flags symbolic names.
\bprog
/* Generic flags: */
parametric = 1;  no_x_axis =  8;  points       = 64;
recursive  = 2;  no_y_axis = 16;  points_lines = 128;
norescale  = 4;  no_frame  = 32;  splines      = 256; 

/* Relative positioning of graphic objects: */
nw       = 0;  se       = 4;  relative = 1;
sw       = 2;  ne       = 6;

/* String positioning: */
/* V */ bottom  =  0;   /* H */  left   = 0;   /* Fine tuning */ hgap = 16;
        vcenter =  4;            center = 1;                     vgap = 32;
        top     =  8;            right  = 2;
@eprog
We also decrease drastically the default precision.
\bprog
\p 9
@eprog
This is very important, since plotting involves calculation of functions at
a huge number of points, and a relative precision of 28 significant digits
is an obvious overkill: the output device resolution certainly won't reach
$10^{28} \times 10^{28}$ pixels!

Start with something really simple:
\bprog
ploth(X = -2, 2, sin(X^7))
@eprog

You can see the limitations of the ``straightforward'' mode of plotting:
while the first several cycles of \kbd{sin} reach $-1$ and $1$, the cycles
which are closer to the left and right border do not. This is understandable,
since PARI is calculating $\sin(X^7)$ at many (evenly spaced) points, but
these points have no direct relationship to the ``interesting'' points on
the graph of this function.  No value close enough to the maxima and minima
are calculated, which leads to wrong turning points of the graph.

There is a way to fix this: one can ask PARI to use variable step which
smaller at the points where the graph of the function is more curved:
\bprog
  ploth(X = -2, 2, sin(X^7),recursive)
@eprog

\noindent The precision near the edges of the graph is much better now.
However, the recursive plotting (named so since PARI subdivides intervals
until the graph becomes almost straight) has its own pitfalls.  Try
\bprog
  ploth(X = -2, 2, sin(X*7), recursive)
@eprog

\noindent Note that the graph looks correct far away, but it has a straight
interval near the origin, and some sharp corners as well.  This happens
because the graph is symmetric with respect to the origin, thus the middle 3
points calculated during the initial subdivision of $[-2,2]$ are exactly on
the same line.  To PARI this indicates that no further subdivision is needed,
and it plots the graph on this subinterval as a straight line.

There are many ways to circumvent this.  Say, one can make the right limit
2.1.  Or one can ask PARI for an initial subdivision into 16 points instead
of default 15:
\bprog
  ploth(X = -2, 2, sin(X*7), recursive, 16)
@eprog

All these arrangements break the symmetry of the initial subdivision, thus
make the problem go away.  Eventually PARI will be able to better detect such
pathological cases, but currently some manual intervention may be required.

Function \kbd{ploth} has some additional enhancements which allow graphing
in situations when the calculation of the function takes a lot of time.  Let
us plot $\zeta({1\over 2} + it)$:
\bprog
  ploth(t = 100, 110, real(zeta(0.5+I*t)), /*empty*/, 1000)
@eprog

\noindent This can take quite some time.  (1000 is close to the default for
many plotting devices, we want to specify it explicitly so that the result
do not depend on the output device.)  Try the recursive plot:
\bprog
  ploth(t = 100, 110, real(zeta(0.5+I*t)), recursive)
@eprog

It takes approximately the same time.  Now try specifying fewer points,
but make PARI approximate the data by a smooth curve:
\bprog
ploth(t = 100, 110, real(zeta(0.5+I*t)), splines, 100)
@eprog

This takes much less time, and the output is practically the same.  How to
compare these two outputs?  We will see it shortly.  Right now let us
plot both real and complex parts of $\zeta$ on the same graph:
\bprog
  f(t) = z=zeta(0.5+I*t); [real(z),imag(z)]
  ploth(t = 100, 110, f(t), , 1000)
@eprog

Note how one half of the roots of the real and imaginary parts coincide.
Why did we define a function \kbd{f(t)}?  To avoid calculation of
$\zeta({1\over2} + it)$ twice for the same value of t.  Similarly, we can
plot parametric graphs:
\bprog
  ploth(t = 100, 110, f(t), parametric, 1000)
@eprog

Again, one can speed up the calculation with
\bprog
  ploth(t = 100, 110, f(t), parametric+splines, 100)
@eprog

If your plotting device supports it, you may ask PARI to show the points
in which it calculated your function:
\bprog
  ploth(t = 100, 110, f(t), parametric+splines+points_lines, 100)
@eprog

As you can see, the points are very dense on the graph.  To see some crude
graph, one can even decrease the number of points to 30.  However, if you
decrease the number of points to 20, you can see that the approximation to
the graph now misses zero.  Using splines, one can create reasonable graphs
for larger values of t, say with
\bprog
  ploth(t = 10000, 10004, f(t), parametric+splines+points_lines, 50)
@eprog

How can we compare two graphs of the same function plotted by different
methods?  Documentation shows that \kbd{ploth} does not provide any direct
method to do so.  However, it is possible, and even not very complicated.

The solution comes from the other direction.  PARI has a power mix of high
level plotting function with low level plotting functions, and these functions
can be combined together to obtain many different effects.  Return back
to the graph of $\sin(X^7)$.  Suppose we want to create an additional 
rectangular frame around our graph.  No problem!

First, all low-level graphing work takes place in some virtual drawing
boards (numbered from 0 to 15), called ``rectangles'' (or ``rectwindows'').
So we create an empty ``rectangle'' and name it rectangle 2 (any
number between 0 and 15 would do):
\bprog
plotinit(2)
plotscale(2, 0,1, 0,1)
@eprog
This creates a rectwindow whose size exactly fits the size of the output
device, and makes the coordinate system inside it go from 0 to 1 (for both
$x$ and $y$). Create a rectangular frame along the boundary of this rectangle:
\bprog
plotmove(2, 0,0)
plotbox(2, 1,1)
@eprog
Suppose we want to draw the graph inside a subrectangle of this with upper
and left margins of $0.10$ (so 10\% of the full rectwindow width), and
lower and top margins of $0.02$, just to make it more interesting. That
makes it an $0.88 \times 0.88$ subrectangle; so we create another rectangle
(call it 3) of linear size 0.88 of the size of the initial rectangle and
graph the function in there:
\bprog
plotinit(3, 0.88, 0.88, relative)
plotrecth(3, X = -2, 2, sin(X^7), recursive)
@eprog
(nothing is output yet, these commands only fills the virtual drawing
boards with PARI graphic objects). Finally, output rectangles 2 and 3 on
the same plot, with the required offsets (counted from upper-left corner):
\bprog
plotdraw([2, 0,0,  3, 0.1,0.02], relative)
@eprog
\noindent The output misses something comparing to the output of
\kbd{ploth}: there are no coordinates of the corners of the internal
rectangle.  If your output device supports mouse operations (only
\kbd{gnuplot} does), you can find coordinates of particular points of the
graph, but it is nice to have something printed on a hard copy too.

However, it is easy to put $x$- and $y$-limits on the graph.  In the
coordinate system of the rectangle 2 the corners are $(0.1,0.1)$,
$(0.1,0.98)$, $(0.98,0.1)$, $(0.98,0.98)$.  We can mark lower $x$-limit by
doing
\bprog
plotmove(2, 0.1,0.1)
plotstring(2, "-2.000", left+top+vgap)
@eprog\noindent
Computing the minimal and maximal $y$-coordinates might be trickier, since
in principle we do not know the range in advance (though for $\sin X^7$ it
is quite easy to guess!). Fortunately, \kbd{plotrecth} returns the $x$- and
$y$-limits.

Here is the complete program:
\bprog
plotinit(3, 0.88, 0.88, relative)
lims = plotrecth(3, X = -2, 2, sin(X^7), recursive)
\p 3          \\ @com $3$ significant digits for the bounding box are enough
limits = vector(4,i, Str(lims[i]))
plotinit(2);      plotscale(2, 0,1, 0,1)
plotmove(2, 0,0); plotbox(2, 1,1)
plotmove(2, 0.1,0.1);
plotstring(2, limits[1], left+top+vgap)
plotstring(2, limits[3], bottom+vgap+right+hgap)
plotmove(2, 0.98,0.1); plotstring(2, limits[2], right+top+vgap)
plotmove(2, 0.1,0.98); plotstring(2, limits[4], right+hgap+top)
plotdraw([2, 0,0,  3, 0.1,0.02], relative)
@eprog

We started with a trivial requirement: have an additional frame around
the graph, and it took some effort to do so.  But at least it was possible,
and PARI did the hardest part: creating the actual graph.
Now do a different thing: plot together the ``exact'' graph of 
$\zeta({1/2}+it)$ together with one obtained from splines approximation.
We can emit these graphs into two rectangles, say 0 and 1, then put these
two rectangles together on one plot.  Or we can emit these graphs into one
rectangle 0.

However, a problem arises: note how we
introduced a coordinate system in rectangle 2 of the above example, but we
did not introduce a coordinate system in rectangle 3.  Plotting a
graph into rectangle 3 automatically created a coordinate system
inside this rectangle (you could see this coordinate system in action
if your output device supports mouse operations).  If we use two different
methods of graphing, the bounding boxes of the graphs will not be exactly
the same, thus outputting the rectangles may be tricky.  Thus during
the second plotting we ask \kbd{plotrecth} to use the coordinate system of
the first plotting.  Let us add another plotting with fewer
points too:
\bprog
plotinit(0, 0.9,0.9, relative)
plotrecth(0, t=100,110, f(t), parametric, 300)
plotrecth(0, t=100,110, f(t), parametric+splines+points_lines+norescale, 30);
plotrecth(0, t=100,110, f(t), parametric+splines+points_lines+norescale, 20);
plotdraw([0, 0.05,0.05], relative)
@eprog

This achieves what we wanted: we may compare different ways to plot a graph,
but the picture is confusing: which graph is what, and why there are multiple
boxes around the graph?  At least with some output devices one can control
how the output curves look like, so we can use this to distinguish different
graphs.  And the mystery of multiple boxes is also not that hard to solve:
they are bounding boxes for calculated points on each graph.  We can disable
output of bounding boxes with appropriate options for \kbd{plotrect}.
With these frills the script becomes:
\bprog
plotinit(0, 0.9,0.9, relative)
plotpointtype(-1, 0)                \\@com set color of graph points
plotpointsize(0, 0.4)               \\@com use tiny markers (if available)
plotrecth(0, t=100,110, f(t), parametric+points, 300)
plotpointsize(0, 1)                 \\@com normal-size markers
plotlinetype(-1, 1)                 \\@com set color of graph lines
plotpointtype(-1, 1)                \\@com set color of graph points
curve_only = norescale + no_frame + no_x_axis + no_y_axis
plotrecth(0, t=100,110,f(t), parametric+splines+points_lines+curve_only, 30);
plotlinetype(-1, 2)                 \\@com set color of graph lines
plotpointtype(-1, 2)                \\@com set color of graph points
plotrecth(0, t=100,110,f(t), parametric+splines+points_lines+curve_only, 20);
plotdraw([0, 0.05,0.05], relative)
@eprog

\noindent Plotting axes on the second and third graph would not hurt, but
is not needed either, so we omit them.  One can see that the discrepancies
between the exact graph and one based on 30 points exist, but are pretty
small.  On the other hand, decreasing the number of points to 20 makes
quite a noticeable difference.

Keep in mind that \kbd{plotlinetype}, \kbd{plotpointtype},
\kbd{plotpointsize} may do nothing on some terminals.  

What if we
want to create a high-resolution hard copy of the plot?  There may be several
possible solutions.  First, the display output device may allow a
high-resolution hard copy itself.  Say, PM display (with gnuplot output on
OS/2) pretends that its resolution is $19500\times 12500$, thus the data
PARI sends to it are already high-resolution, and printing is available
through the menu bar.  Alternatively, with gnuplot output one can switch
the output plotting device to many different hard copy devices:
\kbd{plotfile("plot.tex")}, \kbd{plotterm("texdraw")}.
After this all the plotting will go into file {\tt plot.tex} with whatever
output conventions gnuplot format {\tt texdraw} provides.  To switch output
back to normal, one needs to restore the initial plotting terminal, and
restore the initial output file by doing \kbd{plotfile("-")}.

One can combine PARI programming capabilities to produce multiple plots:
\bprog
plotfile("manypl1.gif")       \\@com Avoid switching STDOUT to binary mode
plotterm("gif=300,200")
wpoints = plothsizes()[1]     \\@com $300 \times 200$ is advice only
{
  for( k=1,6,
    plotfile("manypl" k ".gif");
    ploth(x = -1, 3, sin(x^k), , wpoints)
  )
}
@eprog

\noindent This plots 6 graphs of $\sin x^k$, $k=1\dots 6$ into 
$300\times 200$ GIF files {\tt manypl1.gif}\dots {\tt manypl6.gif}.

Additionally, one can ask PARI to output a plot into a PS file: just
use the command \kbd{psdraw} instead of \kbd{plotdraw} in the above examples
(or \kbd{psploth} instead of \kbd{ploth}).  See \kbd{psfile} argument
to \kbd{default} for how to change the output file for this operation.  Keep
in mind that the precision of PARI PS output will be the same as for the
current output device.

Now suppose we want to join many different small graphs into one picture.
We cannot use one rectangle for all the output as we did in the example
with $\zeta({1/2}+it)$, since the graphs should go into different places.
Rectangles are a scarce commodity in PARI, since only 16 of them are
user-accessible.  Does it mean that we cannot have more than 16 graphs on
one picture?  Thanks to an additional operation of PARI plotting engine,
there is no such restrictions.  This operation is \kbd{plotcopy}.

The following script puts 4 different graphs on one plot using 2 rectangles
only, \kbd{A} and \kbd{T}:
\bprog
A = 2;   \\@com accumulator
T = 3;   \\@com temporary target
plotinit(A);         plotscale(A, 0, 1, 0, 1)

plotinit(T, 0.42, 0.42, relative);
plotrecth(T, x= -5, 5, sin(x), recursive)
plotcopy(T, 2, 0.05, 0.05, relative + nw)

plotmove(A, 0.05 + 0.42/2, 1 - 0.05/2)
plotstring(A,"Graph", center + vcenter)

plotinit(T, 0.42, 0.42, relative);
plotrecth(T, x= -5, 5, [sin(x),cos(2*x)], 0)
plotcopy(T, 2, 0.05, 0.05, relative + ne)

plotmove(A, 1 - 0.05 - 0.42/2, 1 - 0.05/2)
plotstring(A,"Multigraph", center + vcenter)

plotinit(T, 0.42, 0.42, relative);
plotrecth(T, x= -5, 5, [sin(3*x), cos(2*x)], parametric)
plotcopy(T, 2, 0.05, 0.05, relative + sw)

plotmove(A, 0.05 + 0.42/2, 0.5)
plotstring(A,"Parametric", center + vcenter)

plotinit(T, 0.42, 0.42, relative);
plotrecth(T, x= -5, 5, [sin(x), cos(x), sin(3*x),cos(2*x)], parametric)
plotcopy(T, 2, 0.05, 0.05, relative + se)

plotmove(A, 1 - 0.05 - 0.42/2, 0.5)
plotstring(A,"Multiparametric", center + vcenter)

plotlinetype(A, 3)
plotmove(A, 0, 0)
plotbox(A, 1, 1)

plotdraw([A, 0, 0])
\\ psdraw([A, 0, 0], relative)          \\ @com if hard copy is needed
@eprog

The rectangle \kbd{A} plays the role of accumulator, rectangle \kbd{T} is
used as a target to \kbd{plotrecth} only.  Immediately after plotting into 
rectangle \kbd{T} the contents is copied to accumulator.  Let us explain
numbers which appear in this example: we want to create 4 internal rectangles
with a gap 0.06 between them, and the outside gap 0.05 (of the size of the
plot).  This leads to the size 0.42 for each rectangle.  We also
put captions above each graph, centered in the middle of each gap.  There
is no need to have a special rectangle for captions: they go into the
accumulator too.

To simplify positioning of the rectangles, the above example uses relative
``geographic'' notation: the last argument of \kbd{plotcopy} specifies the
corner of the graph (say, northwest) one counts offset from. (Positive
offsets go into the rectangle.)

To demonstrate yet another useful plotting function, design a program to
plot Taylor polynomials for a $\sin x$ about 0.  For simplicity, make the
program good for any function, but assume that a function is odd, so only
odd-numbered Taylor series about 0 should be plotted.  Start with defining
some useful shortcuts
\bprog
xlim = 13;  ordlim = 25;  f(x) = sin(x);
default(seriesprecision,ordlim)
farray(t) = vector((ordlim+1)/2, k, truncate( f(1.*t + O(t^(2*k+1)) )))
FARRAY = farray('t);  \\@com\kbd{'t} to make sure \kbd{t} is a free variable
@eprog

\noindent \kbd{farray(x)} returns a vector of Taylor polynomials for
$f(x)$, which we store in \kbd{FARRAY}.  We want to plot $f(x)$ into a
rectangle, then make the rectangle which is 1.2 times higher, and plot
Taylor polynomials into the larger rectangle.  Assume that the larger
rectangle takes 0.9 of the final plot.

First of all, we need to measure the height of the smaller rectangle:
\bprog
plotinit(3, 0.9, 0.9/1.2, 1);
curve_only = no_x_axis+no_y_axis+no_frame;
lims = plotrecth(3, x= -xlim, xlim, f(x), recursive+curve_only,16);
h = lims[4] - lims[3];
@eprog

\noindent Next step is to create a larger rectangle, and plot the Taylor
polynomials into the larger rectangle:
\bprog
plotinit(4, 0.9,0.9, relative);
plotscale(4, lims[1], lims[2], lims[3] - h/10, lims[4] + h/10)
plotrecth(4, x = -xlim, xlim, subst(FARRAY,t,x), norescale);
@eprog

Here comes the central command of this example:

\kbd{plotclip(4)}

\noindent What does it do?  The command \kbd{plotrecth(\dots, norescale)}
scales the graphs according to coordinate system in the 
rectangle, but it does not pay any other attention to the size of
the rectangle.  Since \kbd{xlim} is 13, the Taylor polynomials take 
very large values in the interval \kbd{-xlim...xlim}.  In particular,
significant part of the graphs is going to be \emph{outside} of the rectangle.
\kbd{plotclip} removes the parts of the plottings which fall off the
rectangle boundary
\bprog
plotinit(2)
plotscale(2, 0.0, 1.0, 0.0, 1.0)
plotmove(2,0.5,0.975)
plotstring(2,"Multiple Taylor Approximations",center+vcenter)
plotdraw([2, 0, 0,  3, 0.05, 0.05 + 0.9/12,  4, 0.05, 0.05], relative)
@eprog

These commands draw a caption, and combine 3 rectangles (one with the
caption, one with the graph of the function, and one with graph of Taylor
polynomials) together.

This finishes our short survey of PARI plotting functions, but let us add
some remarks.  First of all, for a typical output device the picture is
composed of small colored squares (pixels) (as a very large checkerboard).
Each output rectangle is in fact a union of such squares.  Each drop
of paint in the rectangle will color a whole square it is in.  Since the
rectangle has a coordinate system, sometimes it is important to know how
this coordinate system is positioned with respect to the boundaries of
these squares.

The command \kbd{plotscale} describes a range of $x$ and $y$ in the
rectangle.  The limit values of $x$ and $y$ in the coordinate system are
coordinates \emph{of the centers} of corner squares.  In particular,
if ranges of $x$ and $y$ are $[0,1]$, then the segment which connects (0,0)
with (0,1) goes along the \emph{middle} of the left column of the rectangle.
In particular, if we made tiny errors in calculation of endpoints of this
segment, this will not change which squares the segment intersects, thus
the resulting picture will be the same.  (It is important to know such details
since many calculations in PARI are approximate.)

Another consideration is that all examples we did in this section were
using relative sizes and positions for the rectangles.  This is nice, since
different output devices will have very similar pictures, while we
did not need to care about particular resolution of the output device.
On the other hand,
using relative positions does not guarantee that the pictures will be
similar.  Why?  Even if two output devices have the same resolution,
the picture may be different.  The devices may use fonts of different
size, or may have a different ``unit of length''.

The information about the device in PARI is encoded in 6 numbers: resolution,
size of a character cell of the font, and unit of length, all separately
for horizontal and vertical direction.  These sizes are expressed as
numbers of pixels.  To inspect these numbers one may use the function
\kbd{plothsizes}.  The ``units of length'' are currently used to calculate
right and top gaps near graph rectangle of \kbd{ploth}, and gaps for
\kbd{plotstring}.  Left and bottom gaps near graph rectangle are calculate
using both units of length, and sizes of character boxes (so that there
is enough place to print limits of the graphs).

What does it show?  Using relative sizes during plotting produces
\var{approximately} the same plotting on different devices, but does not
ensure that the plottings ``look the same''.  Moreover, ``looking the
same'' is not a desirable target, ``looking tuned for the environment''
will be much better.  If you want to produce such fine-tuned plottings,
you need to abandon a relative-size model, and do your plottings in
pixel units.  To do this one removes flag \kbd{relative} from the above
examples, which will make size and offset arguments interpreted this way.
After querying sizes with \kbd{plothsizes} one can fine-tune sizes and
locations of subrectangles to the details of an arbitrary plotting
device.

To check how good your fine-tuning is, you may test your graphs with a
medium-resolution plotting (as many display output devices are), and
with a low-resolution plotting (say, with \kbd{plotterm("dumb")} of gnuplot).
\vfill\eject\bye
