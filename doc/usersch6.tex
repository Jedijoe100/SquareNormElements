% $Id: usersch6.tex 10212 2008-05-30 23:21:35Z kb $
% Copyright (c) 2000  The PARI Group
%
% This file is part of the PARI/GP documentation
%
% Permission is granted to copy, distribute and/or modify this document
% under the terms of the GNU General Public License
\chapter{Technical Reference Guide for Algebraic Number Theory}

\section{General Number Fields}

\subsec{Number field types}

None of the following routines thoroughly check their intput: they
distinguish between \emph{bona fide} structures as output by PARI routines,
but designing perverse data will easily fool them. To give an example, a
square matrix will be interpreted as an ideal even though the $\Z$-module
generated by its columns may not be an $\Z_K$-module (i.e. the expensive
\kbd{nfisideal} routine will \emph{not} be called).

\fun{long}{nftyp}{GEN x}. Returns the type of number field structure stored in
\kbd{x}, \tet{typ_NF}, \tet{typ_BNF}, or \tet{typ_BNR}. Other answers
are possible, meaning \kbd{x} is not a number field structure.

\fun{GEN}{get_nf}{GEN x, long *t}. Extract an \var{nf} structure from
\kbd{x} if possible and return it, otherwise return \kbd{NULL}. Sets
\kbd{t} to the \kbd{nftyp} of \kbd{x} in any case.

\fun{GEN}{get_bnf}{GEN x, long *t}. Extract a \kbd{bnf} structure from
\kbd{x} if possible and return it, otherwise return \kbd{NULL}. Sets
\kbd{t} to the \kbd{nftyp} of \kbd{x} in any case.

\fun{GEN}{get_nfpol}{GEN x, GEN *nf} try to extract and \var{nf} structure
from \kbd{x}, and sets \kbd{*nf} to \kbd{NULL} (failure) or to the \var{nf}.
Returns the (monic, integral) polynomial defining the field.

\fun{GEN}{get_bnfpol}{GEN x, GEN *bnf, GEN *nf} try to extract a \var{bnf}
and an \var{nf} structure from \kbd{x}, and sets \kbd{*bnf}
and \kbd{*nf} to \kbd{NULL} (failure) or to the corresponding structure.
Returns the (monic, integral) polynomial defining the field.

\fun{GEN}{checknf}{GEN x} if an \var{nf} structure can be extracted from
\kbd{x}, return it; otherwise raise an exception. The more general
\kbd{get\_nf} is often more flexible.

\fun{GEN}{checkbnf}{GEN x} if an \var{bnf} structure can be extracted from
\kbd{x}, return it; otherwise raise an exception. The more general
\kbd{get\_bnf} is often more flexible.

\fun{void}{checkbnr}{GEN bnr} Raise an exception if the argument
is not a \var{bnr} structure.

\fun{void}{checkbnrgen}{GEN bnr} Raise an exception if the argument is not a
\var{bnr} structure, complete with explicit generators for the ray class group.

\fun{void}{checkrnf}{GEN rnf} Raise an exception if the argument is not an
\var{rnf} structure.

\fun{void}{checkbid}{GEN bid} Raise an exception if the argument is not a
\var{bid} structure.

\fun{GEN}{checkgal}{GEN x} if a \var{galoisinit} structure can be extracted
from \kbd{x}, return it; otherwise raise an exception.

\fun{void}{checksqmat}{GEN x, long N} check whether \kbd{x} is a square matrix
of dimension \kbd{N}. May be used to check for ideals if \kbd{N} is the field
degree.

\fun{void}{checkprid}{GEN pr} Raise an exception if the argument is not a
prime ideal structure.

\fun{GEN}{get_prid}{GEN ideal} return the underlying prime ideal structure
if one can be extracted from \kbd{ideal} (ideal or extended ideal), and
return \kbd{NULL} otherwise.

\fun{void}{checkmodpr}{GEN modpr} Raise an exception if the argument is not a
 prime ideal structure.

\fun{GEN}{checknfelt_mod}{GEN nf, GEN x, const char *s} given an \var{nf}
structure \kbd{nf} and a \typ{POLMOD} \kbd{x}, return the associated
polynomial representative (shallow) if \kbd{x} and \kbd{nf} are compatible.
Raise an eception otherwise. Set $s$ to the name of the caller for a
meaningful error message.

\fun{void}{check_ZKmodule}{GEN x, const char *s} check whether $x$ looks like
$\Z_K$-module (a pair $[A,I]$, where $A$ is a matrix and $I$ is a list of
ideals; $A$ has as many columns as $I$ has elements. Otherwise
raises an exception. Sset $s$ to the name of the caller for a
meaningful error message.

\fun{long}{idealtyp}{GEN *ideal, GEN *fa} The input is \kbd{ideal}, a pointer
to an ideal (or extended ideal), which is usually modified, \kbd{fa} being
set as a side-effect. Returns the type of the underlying ideal among
\tet{id_PRINCIPAL} (a number field element), \tet{id_PRIME} (a prime ideal)
\tet{id_MAT} (an ideal in matrix form).

If \kbd{ideal} pointed to an ideal, set \kbd{fa} to \kbd{NULL}, and
possibly simplify \kbd{ideal} (for instance the zero ideal is replaced by
\kbd{gen\_0}). If it pointed to an extended ideal, replace
\kbd{ideal} by the underlying ideal and set \kbd{fa} to the factorization
matrix component.

\subsec{Extracting info from a \kbd{nf} structure}

These function expect a true \var{nf} argument associated to a number field
$K = \Q[x]/(T)$, e.g.~a \var{bnf} will not work. Let $n = [K:\Q]$ be the
field degree.

\fun{GEN}{nf_get_pol}{GEN nf} returns the polynomial $T$ (monic, in $\Z[x]$).

\fun{long}{nf_get_r1}{GEN nf} returns the number of real places $r_1$.

\fun{long}{nf_get_r2}{GEN nf} returns the number of complex places $r_2$.

\fun{void}{nf_get_sign}{GEN nf, long *r1, long *r2} sets $r_1$ and $r_2$
to the number of real and complex places respectively. Note that
$r_1+2r_2$ is the field degree.

\fun{long}{nf_get_degree}{GEN nf} returns the number field degree, $n = r_1 +
2r_2$.

\fun{GEN}{nf_get_disc}{GEN nf} returns the field discriminant.

\fun{GEN}{nf_get_index}{GEN nf} returns the index of $T$, i.e. the index of
the order generated by the power basis $(1,x,\ldots,x^{n-1})$ in the
maximal order of $K$.

\fun{GEN}{nf_get_zk}{GEN nf} returns a basis $(w_1,w_2,\ldots,w_n)$ for the
maximal order of $K$. Those are polynonials in $\Q[x]$ of degree $<n$; it is
guaranteed that $w_1 = 1$.

\fun{GEN}{nf_get_roots}{GEN nf} returns the $r_1$ real roots of the polynomial
defining the number fields: first the $r_1$ real roots (as \typ{REAL}s), then
the $r_2$ representatives of the pairs of complex conjugates.

\fun{GEN}{nf_get_allroots}{GEN nf} returns all the complex roots of $T$:
first the $r_1$ real roots (as \typ{REAL}s), then the $r_2$ pairs of complex
conjugates.

\fun{GEN}{nf_get_M}{GEN nf} returns the $(r_1+r_2)\times n$ matrix $M$
giving the embeddings of $K$: $M[i,j]$ contains $w_j(\alpha_i)$, where 
$\alpha_i$ is the $i$-th element of \kbd{nf\_get\_roots(nf)}. In particular,
if $v$ is an $n$-th dimensional \typ{COL} representing the element
$\sum_{i=1}^n v[i] w_i$ of $K$, then \kbd{RgM\_RgC\_mul(M,v)} represents the
embeddings of $v$.

\fun{GEN}{nf_get_G}{GEN nf} returns a $n\times n$ real matrix $G$ such that
$Gv \cdot Gv = T_2(v)$, where $v$ is an $n$-th dimensional \typ{COL}
representing the element $\sum_{i=1}^n v[i] w_i$ of $K$ and $T_2$ is the
standard Euclidean form on $K\otimes \R$, i.e.~$T_2(v)
= \sum_{\sigma} |\sigma(v)|^2$, where $\sigma$ runs through all $n$ complex
embeddings of $K$.

\fun{GEN}{nf_get_Tr}{GEN nf} returns the matrix of the Trace quadratic form
on the basis $(w_1,\ldots,w_n)$: its $(i,j)$ entry is $\text{Tr} w_i w_j$.

\fun{GEN}{nf_get_TrInv}{GEN nf} returns the primitive part of the inverse of
the above Trace matrix.

\fun{long}{nf_get_prec}{GEN nf} returns the precision (in words) to which the
\var{nf} was computed.

\subsec{Extracting info from a \kbd{bnf} structure}

These functions expect a true \var{bnf} argument, e.g.~a \var{bnr} will not
work.

\subsec{Extracting info from an \kbd{rnf} structure}

These functions expect a true \var{rnf} argument.

\fun{long}{rnf_get_degree}{GEN rnf} returns the \emph{relative} degree of the
extension.

\subsec{Increasing accuracy}

\fun{GEN}{nfnewprec}{GEN x, long prec}. Raise an exception if \kbd{x}
is not a number field structure (\var{nf}, \var{bnf} or \var{bnr}).
Otherwise, sets its accuracy to \kbd{prec} and return the new structure.
This is mostly useful with \kbd{prec} larger than the accuracy to
which \kbd{x} was computed, but it is also possible to decrease the accuracy
of \kbd{x} (truncating relevant components, which may speed up later
computations). This routine may modify the original \kbd{x} (see below).

This routine is straighforward for \var{nf} structures, but for the
other ones, it requires all principal ideals corresponding to the \var{bnf}
relations in algebraic form (they are originally only available via floating
point approximations). This in turn requires many calls to
\tet{bnfisprincipal0}, which is often slow, and may fail if the initial
accuracy was too low. In this case, the routine will not actually fail but
recomputes a \var{bnf} from scratch!

Since this process may be very expensive, the corresponding data is cached
(as a \emph{clone}) in the \emph{original} \kbd{x} so that later precision
increases become very fast. In particular, the copy returned by
\kbd{nfnewprec} also contains this additional data.

\fun{GEN}{bnfnewprec}{GEN x, long prec}. As \kbd{nfnewprec}, but extracts
a \var{bnf} structure form \kbd{x} before increasing its accuracy, and
returns only the latter.

\fun{GEN}{bnrnewprec}{GEN x, long prec}. As \kbd{nfnewprec}, but extracts a
\var{bnr} structure form \kbd{x} before increasing its accuracy, and
returns only the latter.

\fun{GEN}{nfnewprec_shallow}{GEN nf, long prec}

\fun{GEN}{bnfnewprec_shallow}{GEN bnf, long prec}

\fun{GEN}{bnrnewprec_shallow}{GEN bnr, long prec} Shallow functions
underlying the above, except that the first argument must now have the
corresponding number field type. I.e. one cannot call
\kbd{nfnewprec\_shallow(nf, prec)} if \kbd{nf} is actually a \var{bnf}.

\subsec{Number field arithmetic}
The number field $K = \Q[X]/(T)$ is represented by an \kbd{nf} (or \kbd{bnf}
or \kbd{bnr} structure). An algebraic number belonging to $K$ is given as

\item a \typ{INT}, \typ{FRAC} or \typ{POL} (implicitly modulo $T$), or

\item a \typ{POLMOD} (modulo $T$), or

\item a \typ{COL}~\kbd{v} of dimension $N = [K:\Q]$, representing
the element in terms of the computed integral basis $(e_i)$, as
\bprog
  sum(i = 1, N, v[i] * nf.zk[i])
@eprog
The preferred forms are \typ{INT} and \typ{COL} of \typ{INT}. Routines can
handle denominators but it is much more efficient to remove  denominators
first (\tet{Q_remove_denom}) and take them into account at the end.

\misctitle{Safe routines.} The following routines do not assume that their
\kbd{nf} argument is a true \var{nf} (it can be any number field type, e.g.~a
\var{bnf}), and accept number field elements in all the above forms. They
return their result in \typ{COL} form.

\fun{GEN}{nfadd}{GEN nf, GEN x, GEN y} return $x+y$

\fun{GEN}{nfdiv}{GEN nf, GEN x, GEN y} return $x / y$

\fun{GEN}{nfinv}{GEN nf, GEN x} returns $x^{-1}$

\fun{GEN}{nfmul}{GEN nf,GEN x,GEN y} returns $xy$.

\fun{GEN}{nfpow}{GEN nf,GEN x,GEN k} returns $x^k$, $k$ is in $\Z$.

\fun{GEN}{nfpow_u}{GEN nf,GEN x, ulong k} returns $x^k$, $k\geq 0$.

\fun{GEN}{nfsqr}{GEN nf,GEN x} returns $x^2$

\fun{long}{nfval}{GEN nf, GEN x, GEN pr} returns the valuation of $x$ at the
maximal ideal $\goth{p}$ associated to the \var{prid} \kbd{pr}.
Returns \kbd{LONG\_MAX} is $x$ is $0$.

The following three functions implement trivially functionnalities akin to
Euclidean division, for which we currently have no real use. Of course, even if
the number field is actually Euclidean, these do not in general implement a
true Euclidean division.

\fun{GEN}{nfdiveuc}{GEN nf, GEN a, GEN b} returns the algebraic integer
closest to $x / y$. Functionnally identical to \kbd{ground( nfdiv(nf,x,y) )}.

\fun{GEN}{nfdivrem}{GEN nf, GEN a, GEN b} returns the vector $[q,r]$, where
to
\bprog
  q = nfdiveuc(nf, a,b);
  r = nfadd(nf, a,nfmul(nf,q,gneg(b)));    \\ or r = nfmod(nf,a,b);
@eprog

\fun{GEN}{nfmod}{GEN nf, GEN a, GEN b} returns $r$ such that
\bprog
  q = nfdiveuc(nf, a,b);
  r = nfadd(nf, a, nfmul(nf,q, gneg(b)));
@eprog

\fun{GEN}{nf_to_scalar_or_basis}{GEN nf, GEN x} let $x$ be a number field
element. If it is a rational scalar, i.e.~can be represented by a \typ{INT}
or \typ{FRAC}, return the latter. Otherwise returns its basis representation
(\tet{nfalgtobasis}). Shallow function.

\fun{GEN}{nf_to_scalar_or_alg}{GEN nf, GEN x} let $x$ be a number field
element. If it is a rational scalar, i.e.~can be represented by a \typ{INT}
or \typ{FRAC}, return the latter. Otherwise returns its lifted \typ{POLMOD}
representation (lifted \tet{nfbasistoalg}). Shallow function.

\fun{GEN}{RgX_to_nfX}{GEN nf, GEN x} let $x$ be a \typ{POL} whose coefficients
are number field elements; apply \kbd{nf\_to\_scalar\_or\_basis} to each
coefficient and return the resulting new polynomial. Shallow function.

\fun{GEN}{RgM_to_nfM}{GEN nf, GEN x} let $x$ be a \typ{MAT} whose coefficients
are number field elements; apply \kbd{nf\_to\_scalar\_or\_basis} to each
coefficient and return the resulting new matrix. Shallow function.

\fun{GEN}{RgC_to_nfC}{GEN nf, GEN x} let $x$ be a \typ{COL} or
\typ{VEC} whose coefficients
are number field elements; apply \kbd{nf\_to\_scalar\_or\_basis} to each
coefficient and return the resulting new \typ{COL}. Shallow function.

\misctitle{Unsafe routines.} The following routines assume that their \kbd{nf}
argument is a true \var{nf} (e.g.~a \var{bnf} is not allowed) and their
argument are restricted in various ways, see the precise description below.

\fun{GEN}{nfinvmodideal}{GEN nf, GEN x, GEN A} given an algebraic integer
$x$ and a non-zero integral ideal $A$ in HNF, returns a $y$ such that
$xy \equiv 1$ modulo $A$.

\fun{GEN}{nfpowmodideal}{GEN nf, GEN x, GEN n, GEN ideal} given an algebraic
integer $x$, an integer $n$, and a non-zero integral ideal $A$ in HNF,
returns an algebraic integer congruent to $x^n$ modulo $A$.

\fun{GEN}{nfmuli}{GEN nf, GEN x, GEN y} returns $x\times y$ assuming
that both $x$ and $y$ are either \typ{INT}s or \kbd{ZV}s of the correct
dimension.

\fun{GEN}{nfsqri}{GEN nf, GEN x} returns $x^2$ assuming that $x$ is a \typ{INT}
or a \kbd{ZV} of the correct dimension.

\fun{GEN}{nfC_nf_mul}{GEN nf, GEN v, GEN x} given a \typ{VEC} or \typ{COL}
$v$ of elements of $K$ in \typ{INT}, \typ{FRAC} or \typ{COL} form, multiply
it by the element $x$ (arbitrary form). This is faster than multiplying
coordinatewise since pre-computations related to $x$ (computing the
multiplication table) are done only once. The components of the result
are in most cases \typ{COL}s but are allowed to be \typ{INT}s or \typ{FRAC}s.

\fun{GEN}{zk_multable}{GEN nf, GEN x} given a \kbd{ZC} $x$ (implicitly
representing an algebraic integer), returns the \kbd{ZM} giving the
multiplication table by $x$. Shallow function (the first column of the result
points to the same data as $x$).

\fun{GEN}{zk_scalar_or_multable}{GEN nf, GEN x} given a \typ{INT} or \kbd{ZC}
$x$, returns a \typ{INT} equal to $x$ if the latter is a scalar
(\typ{INT} or \kbd{ZV\_isscalar}$(x)$ is 1) and
\kbd{zk\_multable}$(\var{nf},x)$ otherwise. Shallow function.


The following routines implement multiplication in a commutative $R$-algebra,
generated by $(e_1 = 1,\dots, e_n)$, and given by a multiplication table $M$:
elements in the algebra are $n$-dimensional \typ{COL}s, and the matrix
$M$ is such that for all $1\leq i,j\leq n$, its column with index $(i-1)n +
j$, say $(c_k)$, gives $e_i\cdot e_j = \sum c_k e_k$. It is assumed that
$e_1$ is the neutral element for the multiplication (a convenient
optimization, true in practice for all multiplications we needed to implement).
If $x$ has any other type than \typ{COL} where an algebra element is
expected, it is understood as $x e_1$.

\fun{GEN}{multable}{GEN M, GEN x} given a column vector $x$, representing
the quantity $\sum_{i=1}^N x_i e_i$, returns the multiplication table by $x$.
Shallow function.

\fun{GEN}{ei_multable}{GEN M, long i} returns the multiplication table
by the $i$-th basis element $e_i$. Shallow function.

\fun{GEN}{tablemul}{GEN M, GEN x, GEN y} returns $x\cdot y$.

\fun{GEN}{tablesqr}{GEN M, GEN x} returns $x^2$.

\fun{GEN}{tablemul_ei}{GEN M, GEN x, long i} returns $x\cdot e_i$.

\fun{GEN}{tablemul_ei_ej}{GEN M, long i, long j} returns $e_i\cdot e_j$.

\fun{GEN}{tablemulvec}{GEN M, GEN x, GEN v} given a vector $v$ of elements
in the algebra, returns the $x\cdot v[i]$.

\subsec{Elements in factored form}

Computational algebraic theory performs extensively linear
algebra on $\Z$-modules with a natural multiplicative structure ($K^*$,
fractional ideals in $K$, $\Z_K^*$, ideal class group), thereby raising
elements to horrendously large powers. A seemingly inocuous elementary linear
algebra operation like $C_i\leftarrow C_i - 10000 C_1$ involves raising
entries in $C_1$ to the $10000$-th power. Understandably, it is often more
efficient to keep elements in factored form rather than expand every such
expression. A \emph{factorization matrix} (or \tev{famat}) is a two column
matrix, the first column containing \emph{elements} (arbitrary objects which
may be repeated in the column), and the second one contains \emph{exponents}
(\typ{INT}s, allowed to be 0). By abuse of notation, the empty matrix
\kbd{cgetg(1, t\_MAT)} is recognized as the trivial factorization (no
element, no exponent).

Even though we think of a \var{famat} with columns $g$ and $e$
as one meaningful object when fully expanded as $\prod g[i]^{e[i]}$,
\var{famat}s are basically about concatenating information to keep track of
linear algebra: the objects stored in a \var{famat} need not be
operation-compatible, they will not even be compared to each other (with one
exception: \tet{famat_reduce}). Multiplying two \var{famat}s just
concatenates their elements and exponents columns. In a context where a
\var{famat} is expected, an object $x$ which is not of type \typ{MAT} will be
treated as the factorization $x^1$. The following functions all return
\var{famat}s:

\fun{GEN}{famat_mul}{GEN f, GEN g} $f$, $g$ are \var{famat},
or objects whose type is \emph{not} \typ{MAT} (understood as $f^1$ or $g^1$).
Returns $fg$. The empty factorization is the neutral element for \var{famat}
multiplication.

\fun{GEN}{famat_mul_shallow}{GEN f, GEN g} $f$, $g$ are \var{famat}, returns
$fg$. Shallow function.

\fun{GEN}{famat_pow}{GEN f, GEN n} $n$ is a \typ{INT}. If $f$ is a \typ{MAT},
assume it is a \var{famat} and return $f^n$ (multiplies the exponent column
by $n$). Otherwise, understand it as an element and returns the $1$-line
\var{famat} $f^n$.

\fun{GEN}{famat_sqr}{GEN f} returns $f^2$.

\fun{GEN}{famat_inv}{GEN f} returns $f^{-1}$.

\fun{GEN}{to_famat}{GEN x, GEN k} given an element $x$ and an exponent
$k$, returns the \var{famat} $x^k$.

\fun{GEN}{to_famat_shallow}{GEN x, GEN k} same, as a shallow function.

Note that it is trivial to break up a \var{famat} into its two constituent
columns: \kbd{gel(f,1)} and \kbd{gel(f,2)} are the elements and exponents
respectively. Conversely, \kbd{mkmat2} builds a (shallow) \var{famat} from
two \typ{COL}s of the same length.

The last two functions makes an assumption about the elements: they must be
regular algebraic numbers (not \var{famat}s) over a given number field:

\fun{GEN}{famat_reduce}{GEN f} given a \var{famat} $f$, returns a \var{famat}
$g$ without repeated elements or 0 exponents, such that the expanded forms
of $f$ and $g$ would be equal.

\fun{GEN}{famat_to_nf}{GEN nf, GEN f} You normally never want to do this !
This is a simplified form of \tet{nffactorback}, where we do not check the
user input for consistency.

The description of \tet{famat_to_nf} says that you do not want to use this
function. Then how do we recover genuine number field elements? Well, in
most cases, we do not need to: most of the functions useful in this
context accept \var{famat}s as inputs, for instance \tet{nfsign},
\tet{nfsign_arch}, \tet{ideallog} and \tet{bnfisunit}. Otherwise, we can
generally make good use of a quotient operation (modulo a fixed conductor,
modulo $\ell$-th powers); see the end of \secref{se:Ideal_reduction}.

\misctitle{Caveat.} receiving a \var{famat} input, \kbd{bnfisunit} assumes that
it is an algebraic integer, since this is expensive to check, and normally
easy to ensure from the user's side; don't feed it ridiculous inputs.

\subsec{Ideal arithmetic}

\misctitle{Conversion to HNF.}

\fun{GEN}{idealhnf}{GEN nf, GEN x} returns the HNF of the ideal defined by $x$:
$x$ may be an algebraic  number (defining a principal ideal),  a maximal ideal
(as given by \tet{idealprimedec} or  \tet{idealfactor}), or a matrix whose
columns give generators for the  ideal. This  last format is complicated,  but
useful to reduce general modules to the canonical form once in a while:

\item if strictly less than $N = [K:Q]$ generators are given,  $x$ is the
$\Z_K$-module they generate,

\item if $N$ or more are given,  it is assumed that they form a $\Z$-basis
(that the matrix has maximal rank $N$).  This acts as \tet{mathnf} since the
$\Z_K$-module structure is (taken for granted hence) not taken into account
in this case.

Extended ideals are also accepted, their principal part being discarded.

\fun{GEN}{idealhnf0}{GEN nf, GEN x, GEN y} returns the HNF of the ideal
generated by the two algebraic numbers $x$ and $y$.

The following low-level function underly the above two: they all assume
that \kbd{nf} is a true \var{nf} and perform no type checks:

\fun{GEN}{idealhnf_principal}{GEN nf, GEN x}
returns the ideal generated by the algebraic number $x$.

\fun{GEN}{idealhnf_shallow}{GEN nf, GEN x} is \tet{idealhnf} except that the
result may not be suitable for \kbd{gerepile}: if $x$ is already in HNF, we
return $x$, not a copy !

\fun{GEN}{idealhnf_two}{GEN nf, GEN v} assuming $a = v[1]$ is a non-zero
\typ{INT} and $b = v[2]$ is an algebraic integer, possibly given in regular
representation by a \typ{MAT} (the multiplication table by $b$, see
\tet{zk_multable}), returns the HNF of $a\Z_K+b\Z_K$.

\misctitle{Operations}

The basic ideal routines accept all \kbd{nf}s (\var{nf}, \var{bnf},
\var{bnr}) and ideals in any form, including extended ideals, and return
ideals in HNF, or an extended ideal when that makes sense:

\fun{GEN}{idealadd}{GEN nf, GEN x, GEN y} returns $x+y$.

\fun{GEN}{idealdiv}{GEN nf, GEN x, GEN y} returns $x/y$. Returns an extended
ideal if $x$ or $y$ is an extended ideal.

\fun{GEN}{idealmul}{GEN nf, GEN x, GEN y} returns $xy$.
Returns an extended ideal if $x$ or $y$ is an extended ideal.

\fun{GEN}{idealsqr}{GEN nf, GEN x} returns $x^2$.
Returns an extended ideal if $x$ is an extended ideal.

\fun{GEN}{idealinv}{GEN nf, GEN x} returns $x^{-1}$.
Returns an extended ideal if $x$ is an extended ideal.

\fun{GEN}{idealpow}{GEN nf, GEN x, GEN n} returns $x^n$.
Returns an extended ideal if $x$ is an extended ideal.

\fun{GEN}{idealpows}{GEN nf, GEN ideal, long n} returns $x^n$.
Returns an extended ideal if $x$ is an extended ideal.

\fun{GEN}{idealmulred}{GEN nf, GEN x, GEN y} returns an extended ideal equal
to $xy$.

\fun{GEN}{idealpowred}{GEN nf, GEN x, GEN n} returns an extended ideal equal
to $x^n$.

More specialized routines suffer from various restrictions:

\fun{GEN}{idealdivexact}{GEN nf, GEN x, GEN y} returns $x/y$, assuming that
the quotient is an integral ideal. Much faster than \tet{idealdiv} when the
norm of the quotient is small compared to $Nx$. Strips the principal parts
if either $x$ or $y$ is an extended ideal.

\fun{GEN}{idealdivpowprime}{GEN nf, GEN x, GEN pr, GEN n} returns $x
\goth{p}^{-n}$, assuming $x$ is an ideal in HNF, and \kbd{pr}
a \var{prid} associated to $\goth{p}$. Not suitable for \tet{gerepileupto}
since it retunrs $x$ when $n = 0$.

\fun{GEN}{idealmulpowprime}{GEN nf, GEN x, GEN pr, GEN n} returns $x
\goth{p}^{n}$, assuming $x$ is an ideal in HNF, and \kbd{pr} a \var{prid}
associated to $\goth{p}$. Not suitable for \tet{gerepileupto} since it
retunrs $x$ when $n = 0$.

\fun{GEN}{idealprodprime}{GEN nf, GEN P} given a list $P$ of prime ideals
in \var{prid} form, return their product.

\fun{GEN}{idealmul_HNF}{GEN nf, GEN x, GEN y} returns $xy$, assuming
than \kbd{nf} is a true \var{nf}, $x$ is an integral ideal in HNF and $y$
is an integral ideal in HNF or precompiled form (see below).
For maximal speed, the second ideal $y$ may be given in precompiled form $y =
[a,b]$, where $a$ is a non-zero \typ{INT} and $b$ is an algebraic integer in
regular representation (a \typ{MAT} giving the multiplication table by the
fixed element): very useful when many ideals $x$ are going to be multiplied by
the same ideal $y$. This essentially reduces each ideal multiplication to
an $N\times N$ matrix multiplication followed by a $N\times 2N$ modular
HNF reduction (modulo $xy\cap \Z$).

\misctitle{Approximation.}

\fun{GEN}{idealaddtoone}{GEN nf, GEN A, GEN B} given to coprime integer ideals
$A$, $B$, returns $[a,b]$ with $a\in A$, $b\in B$, such that $a + b = 1$
The result is reduced mod $AB$, so $a$, $b$ will be small.

\fun{GEN}{idealaddtoone_i}{GEN nf, GEN A, GEN B} as \tet{idealaddtoone} except
that \kbd{nf} must be a true \var{nf}, and only $a$ is returned.

\fun{GEN}{hnfmerge_get_1}{GEN A, GEN B} given two square upper HNF integral
matrices $A$, $B$ of the same dimension $n > 0$, return $a$ in the image of
$A$ such that $1-a$ is in the image of $B$. (By abuse of notation we denote
$1$ the column vector $[1,0,\dots,0]$.) If such an $a$ does not exist, return
\kbd{NULL}. This is the function underlying \tet{idealaddtoone}.

\fun{GEN}{idealaddmultoone}{GEN nf, GEN v} given a list of $n$ (globally)
coprime integer ideals $(v[i])$ returns an $n$-dimensional vector $a$ such that
$a[i]\in v[i]$ and $\sum a[i] = 1$. If $[K:\Q] = N$, this routine computes
the HNF reduction (with $Gl_{nN}(\Z)$ base change) of an $N\times nN$ matrix;
so it is well worth pruning "useless" ideals from the list (as long as the
ideals remain globally coprime).

\fun{GEN}{idealappr}{GEN nf, GEN x} given a fractional ideal $x$, returns
an algebraic number $\alpha$ such that $v(x) = v(\alpha)$ for all valuations
such that $v(x) > 0$, and $v(\alpha) \geq 0$ at all others.

\fun{GEN}{idealapprfact}{GEN nf, GEN fx} same as \tet{idealappr}, $x$ being
given in factored form, as after \kbd{fx = idealfactor(nf,x)}, except that we
allow $0$ exponents in the factorization. Returns an algebraic number
$\alpha$ such that $v(x) = v(\alpha)$ for all valuations associated to the
prime ideal decomposition of $x$, and $v(\alpha) \geq 0$ at all others.

\fun{GEN}{idealcoprime}{GEN nf, GEN x, GEN y}. Given 2 integral ideals $x$ and
$y$, returns an algebraic number $\alpha$ such that
$\alpha x$ is an integral ideal coprime to $y$.

\fun{GEN}{idealcoprimefact}{GEN nf, GEN x, GEN fy} same as
\tet{idealcoprime}, except that $y$ is given in factored form, as from
\tet{idealfactor}.

\fun{GEN}{idealchinese}{GEN nf, GEN x, GEN y} $x$ being a prime ideal
factorization (i.e.~a 2 by 2 matrix whose first column contain prime ideals,
and the second column integral exponents), $y$ a vector of elements in
$\var{nf}$ indexed by the ideals in $x$, computes an element $b$ such that
$v_\wp(b - y_\wp) \geq v_\wp(x)$ for all prime ideals in $x$ and $v_\wp(b)\geq
0$ for all other $\wp$.

\subsec{Maximal ideals}

The PARI structure associated to maximal ideals is a \tev{prid} (for
\emph{pr}ime \emph{id}eal), usually produced by \tet{idealprimedec}
and \tet{idealfactor}. In this section, we describe the format; other sections
will deal with their daily use.

A \var{prid} associated to a maximal ideal $\goth{p}$ stores the following
data: the underlying rational prime $p$, the ramification degree $e\geq 1$,
the residue field degree $f\geq 1$, a $p$-uniformizer $\pi$ with valuation
$1$ at $\goth{p}$ and valuation $0$ at all other primes dividing $p$ and
a rescaled ``anti-uniformizer'' $\tau$ used to compute valuations. This
$\tau$ is an algebraic integer such that $\tau/p$ has valuation $-1$ at
$\goth{p}$ and valuation $0$ at all other primes dividing $p$; in particular,
the valuation of $x\in\Z_K$ is positive if and only if the algebraic integer
$x\tau$ is divisible by $p$ (easy to check for elements in \typ{COL} form).

The following functions are shallow and return directly components of the
\var{prid} \kbd{pr}:

\fun{GEN}{pr_get_p}{GEN pr} returns $p$. Shallow function.

\fun{GEN}{pr_get_gen}{GEN pr} returns $\pi$. Shallow function.

\fun{long}{pr_get_e}{GEN pr} returns $e$.

\fun{long}{pr_get_f}{GEN pr} returns $f$.

\fun{GEN}{pr_get_tau}{GEN pr} returns $\tau$. Shallow function.

\fun{int}{pr_is_inert}{GEN pr} returns $1$ is $p$ is inert, $0$ otherwise.

\fun{GEN}{pr_norm}{GEN pr} returns the norm $p^f$ of the maximal ideal.

\subsec{Reducing modulo maximal ideals}

\fun{GEN}{nfmodprinit}{GEN nf, GEN pr} returns an abstract \kbd{modpr}
structure, associated to reduction modulo the maximal ideal \kbd{pr}, in
\kbd{idealprimedec} format. From this data we can quickly project any
\kbd{pr}-integral number field element to the residue field. This function is
almost useless in library mode, we rather use:

\fun{GEN}{nf_to_Fq_init}{GEN nf, GEN *ppr, GEN *pT, GEN *pp} concrete
version of \kbd{nfmodprinit}: \kbd{nf} and \kbd{*ppr} are the inputs, the
return value is a \kbd{modpr} and \kbd{*ppr}, \kbd{*pT} and \kbd{*pp} are set
as side effects.

The input \kbd{*ppr} is either a maximal ideal or already a \kbd{modpr} (in
which case it is replaced by the underlying maximal ideal). The residue field
is realized as $\F_p[X]/(T)$ for some monic $T\in\F_p[X]$, and we set
\kbd{*pT} to $T$ and \kbd{*pp} to $p$. Set $T = \kbd{NULL}$ if the prime has
degree $1$ and the residue field is $\F_p$.

In short, this receives (or initializes) a \kbd{modpr} structure, and
extracts from it $T$, $p$ and $\goth{p}$.

\fun{GEN}{nf_to_Fq}{GEN nf, GEN x, GEN modpr} returns an \kbd{Fq} congruent
to $x$ modulo the maximal ideal associated to \kbd{modpr}. The output is
canonical: all elements in a given residue class are represented by the same
\kbd{Fq}.

\fun{GEN}{Fq_to_nf}{GEN x, GEN modpr} returns an \kbd{nf} element lifting
the residue field element $x$, either a \typ{INT} or an algebraic integer
in \kbd{algtobasis} format.

\fun{GEN}{zkmodprinit}{GEN nf, GEN pr} as \tet{nfmodprinit}, but we assume we
will only reduce algebraic integers, hence do not initialize data allowing to
remove denominators. More precisely, we can in fact still handle an $x$ whose
rational denominator is not $0$ in the residue field (i.e. if the valuation
of $x$ is non-negative at all primes dividing $p$).

\fun{GEN}{zk_to_Fq_init}{GEN nf, GEN *pr, GEN *T, GEN *p} as
\kbd{nf\_to\_Fq\_init}, able to reduce only $p$-integral elements.

\fun{GEN}{zk_to_Fq}{GEN x, GEN modpr} as \kbd{nf\_to\_Fq}, for
a $p$-integral $x$.

\fun{GEN}{nfM_to_FqM}{GEN M, GEN nf,GEN modpr} reduces a matrix
of \kbd{nf} elements to the residue field; returns an \kbd{FqM}.

\fun{GEN}{FqM_to_nfM}{GEN M, GEN modpr} lifts an \kbd{FqM} to a matrix of
\kbd{nf} elements.

\fun{GEN}{nfV_to_FqV}{GEN A, GEN nf,GEN modpr} reduces a vector
of \kbd{nf} elements to the residue field; returns an \kbd{FqV}
with the same type as \kbd{A} (\typ{VEC} or \typ{COL}).

\fun{GEN}{FqV_to_nfV}{GEN A, GEN modpr} lifts an \kbd{FqV} to a vector of
\kbd{nf} elements (same type as \kbd{A}).

\fun{GEN}{nfX_to_FqX}{GEN Q, GEN nf,GEN modpr} reduces a polynomial
with \kbd{nf} coefficients to the residue field; returns an \kbd{FqX}.

\fun{GEN}{FqX_to_nfX}{GEN Q, GEN modpr} lifts an \kbd{FqX} to a polynomial
with coefficients in \kbd{nf}.

\subsec{Signatures}

``Signs'' of the real embeddings of number field element are represented in
additive notation, using the standard identification $(\Z/2\Z, +) \to
(\{-1,1\},\times)$, $s\mapsto (-1)^s$.

With respect to a fixed \kbd{nf} structure, a selection of real places (a
divisor at infinity) is normally given as a \typ{VECSMALL} of indices of the
roots \kbd{nf.roots} of the defining polynomial for the number field. For
compatibility reasons, in particular under GP, the (obsolete) \kbd{vec01}
form is also accepted: a \typ{VEC} with \kbd{gen\_0} or \kbd{gen\_1} entries.

The following internal functions go back and forth between the two
representations for the archimedean part of divisors (GP: $0/1$ vectors,
library: list of indices):

\fun{GEN}{vec01_to_indices}{GEN v} given a \typ{VEC} $v$ with \typ{INT} entries
equal to $0$ or $1$, return as a \typ{VECSMALL} the list of indices $i$
such that $v[i] = 1$. If $v$ is already a \typ{VECSMALL}, return it
(not suitable for \kbd{gerepile} in this case).

\fun{GEN}{indices_to_vec01}{GEN p, long n} return the $0/1$ vector of length
$n$ with ones exactly at the positions $p[1], p[2], \ldots$


\fun{GEN}{nfsign}{GEN nf,GEN x} $x$ being a number field element and \kbd{nf}
any form of number field, return the $0-1$-vector giving the signs of the
$r_1$ real embeddings of $x$, as a \typ{VECSMALL}. Linear algebra functions
like \tet{Flv_add_inplace} then allow keeping track of signs in series of
multiplications.

If $x$ is a \typ{VEC} of number field elements, return the matrix whose
colums are the signs of the $x[i]$.

\fun{GEN}{nfsign_arch}{GEN nf,GEN x,GEN arch} \kbd{arch} being a list of
distinct real places, either in \kbd{vec01} (\typ{VEC} with \kbd{gen\_0} or
\kbd{gen\_1} entries) or \kbd{indices} (\typ{VECSMALL}) form (see
\tet{vec01_to_indices}), returns the signs of $x$ at the corresponding
places. This is the low-level function underlying \kbd{nfsign}.

\fun{GEN}{nfsign_units}{GEN bnf, GEN archp, int add_tu}
\kbd{archp} being a divisor at infinity in \kbd{indices} form
(or \kbd{NULL} for the divisor including all real places), return the signs
at \kbd{archp} of a system of fundamental units for the field, in the same
order as \kbd{bnf.tufu} if \kbd{add\_tu} is set; and in the same order as
\kbd{bnf.fu} otherwise.

\fun{GEN}{nfsign_from_logarch}{GEN L, GEN invpi, GEN archp} given $L$
the vector of the $\log \sigma(x)$, where $\sigma$ runs through the (real
or complex) embeddings of some number field, \kbd{invpi} being
a floating point approximation to $1/\pi$, and \kbd{archp} being a divisor
at infinity in \kbd{indices} form, return the signs of $x$
at the corresponding places. This is the low-level function underlying
\kbd{nfsign\_units}; the latter is actually a trivial wrapper
\kbd{bnf} structures include the $\log \sigma(x)$ for a system of fundamental
units of the field.

\fun{GEN}{set_sign_mod_divisor}{GEN nf, GEN x, GEN y, GEN module, GEN sarch}
let $f = f_0f_\infty$ be the divisor represented by \kbd{module}, $x$, $y$ two
number field elements. Returns $yt$ with $t = 1 \text{mod}^* f$ such that $x$
and $ty$ have the same signs at $f_\infty$; if $x =
\kbd{NULL}$, make $ty$ totally positive at $f_\infty$. \kbd{sarch} is the
output of \kbd{nfarchstar(nf, f0, finf)}.

\fun{GEN}{nfarchstar}{GEN nf, GEN f0, GEN finf} for a divisor $f =
f_0f_\infty$ represented by the integral ideal \kbd{f0} in HNF and
the \kbd{finf} in \kbd{indices} form, returns $(\Z_K/f_\infty)^*$ in a form
suitable for computations mod $f$. More precisely, returns
$[c, g, M]$, where $c = [2,\ldots, 2]$ gives the cyclic structure of that
group ($\#f_\infty$ copies of $\Z/2\Z$), $g$ a minimal system of independant
generators, which are furthermore congruent to $1$ mod $f_0$ (no condition if
$f_0 = \Z_K$), and $M$ is the matrix of signs of the $g[i]$ at $f_\infty$,
which is square and invertible over $\F_2$.

\subsec{Maximal order and discriminant}

A number field $K = \Q[X]/(T)$ is defined by a monic $T\in\Z[X]$. The
low-level function computing a maximal order is

\fun{void}{nfmaxord}{nfmaxord_t *S, GEN T, long flag, GEN fa}, where
the polynomial $T$ is as above.

The structure \tet{nfmaxord_t} is initialized by the call; it has the
following fields:
\bprog
  GEN dT, dK; /* discriminants of T and K */
  GEN index; /* index of power basis in maximal order */
  GEN dTP, dTE; /* factorization of |dT|, primes / exponents */
  GEN dKP, dKE; /* factorization of |dK|, primes / exponents */
  GEN basis; /* Z-basis for maximal order */
@eprog\noindent The exponent vectors are \typ{VECSMALL}. The primes
in \kbd{dTP} and \kbd{dKP} are pseudoprimes, not proven primes.

The \kbd{flag} is an or-ed combination of the binary flags:

\tet{nf_PARTIALFACT}: do not try to fully factor \kbd{dT} and only look for
primes less than \kbd{primelimit}. In that case, the elements in \kbd{dTP}
and \kbd{dKP} need not all be primes. But the resulting \kbd{dK},
\kbd{index} and \kbd{basis} are correct provided there exists no prime $p >
\kbd{primelimit}$ with $p^2$ divides the field discriminant \kbd{dK}.

\tet{nf_ROUND2}: use the ROUND2 algorithm instead of the default ROUND4
(do not use that, it is slower).

If \kbd{fa} is not \kbd{NULL}, it is assumed to be the factorisation of
the absolute value of the discriminant of $T$. It is not mandatory that all
entries in the first column be primes; this is useful if only a local  integral
basis  for  some small set of places is desired: only factors with exponents
greater or equal to $2$ will be considered.

\fun{GEN}{indexpartial}{GEN T, GEN dT} $T$ a monic separable \kbd{ZX},
\kbd{dT} is either \kbd{NULL} (no information) or a multiple of the
discriminant of $T$. Let $K = \Q[X]/(T)$ and $\Z_K$ its maximal order.
Returns a multiple of the exponent of the quotient group $\Z_K/(\Z[X]/(T))$.
In other word, a \emph{denominator} $d$ such that $d x\in\Z[X]/(T)$ for all
$x\in\Z_K$.

\subsec{Computing in the class group}

We compute with arbitrary ideal representatives (in any of the various
formats seen above), and call

\fun{GEN}{bnfisprincipal0}{GEN bnf, GEN x, long flag}. The \kbd{bnf}
structure already contains informations about the class group in the form
$\oplus_{i=1}^n (\Z/d_i\Z) g_i$ for canonical integers $d_i$
(with $d_n\mid\dots\mid d_1$ all $> 1$) and essentially random generators
$g_i$, which are ideals in HNF. We normally do not need the value of the
$g_i$, only that they are fixed once and for all and that any (non-zero)
fractional ideal $x$ can be expressed uniquely as $x = (t)\prod_{i=1}^n
g_i^{e_i}$, where $0 \leq e_i < d_i$, and $(t)$ is some principal ideal.
Computing $e$ is straightforward, but $t$ may be very expensive to obtain
explicitly. The routine returns (possibly partial) information about the pair
$[e,t]$, depending on \kbd{flag}, which is an or-ed combination of the
following symbolic flags:

\item \tet{nf_GEN} tries to compute $t$.
Returns $[e,t]$, with $t$ an empty vector if the computation failed. This
flag is normally useless in non-trivial situations since the next two serve
analogous purposes in more efficient ways.

\item \tet{nf_GENMAT} tries to compute $t$ in factored form, which is
much more efficient than \kbd{nf\_GEN} if the class group is moderately
large; imagine a small ideal $x = (t)g^{10000}$: the norm of $t$ has $10000$
as many digits as the norm of $g$; do we want to see it as a vector
of huge meaningless integers? The idea is to compute $e$ first, which is
easy, then compute $(t)$ as $x \prod g_i^{-e_i}$ using successive
\tet{idealmulred}, where the ideal reduction extracts small principal ideals
along the way, eventually raised to large powers because of the binary
exponentiation technique; the point is to keep this principal part in
factored \emph{unexpanded} form. Returns $[e,t]$, with $t$ an empty vector if
the computation failed; this should be exceedingly rare, unless the initial
accuracy to which \kbd{bnf} was computed was ridiculously low (and then
\kbd{bnfinit} should not have succeeded either). Setting/unsetting
\kbd{nf\_GEN} has no effect when this flag is set.

\item \tet{nf_GEN_IF_PRINCIPAL} tries to compute $t$ \emph{only} if the
ideal is principal ($e = 0$). Returns \kbd{gen\_0} if the ideal is not
principal. Setting/unsetting \kbd{nf\_GEN} has no effect when this flag is
set, but setting/unsetting \kbd{nf\_GENMAT} is possible.

\item \tet{nf_FORCE} in the above, insist on computing $t$, even if it
requires recomputing a \kbd{bnf} from scratch. This is a last resort, and
normally the accuracy of a \kbd{bnf} can be increased without trouble, but it
may be that some algebraic information simply cannot be recovered from what
we have: see \tet{bnfnewprec}. It should be very rare, though.

In simple cases where you do not care about $t$, you may use

\fun{GEN}{isprincipal}{GEN bnf, GEN x}, which is a shortcut for
\kbd{bnfisprincipal0(bnf, x, 0)}.

The following low-level functions are often more useful:

\fun{GEN}{isprincipalfact}{GEN bnf, GEN C, GEN L, GEN f, long flag} is
about the same as \kbd{bnfisprincipal0} applied to $C \prod L[i]^{f[i]}$,
where the $L[i]$ are ideals, the $f[i]$ integers and $C$ is either an ideal
or \kbd{NULL} (omitted). Make sure to include \tet{nf_GENMAT} in \kbd{flag}!

\fun{GEN}{isprincipalfact_or_fail}{GEN bnf, GEN C, GEN L, GEN f} is
for delicate cases, where we must be more clever than \kbd{nf\_FORCE}
(it is used when trying to increase the accuracy of a \var{bnf}, for
instance). If performs
\bprog
  isprincipalfact(bnf,C, L, f, nf_GENMAT);
@eprog\noindent
but if it fails to compute $t$, it just returns a \typ{INT}, which is the
estimated precision (in words, as usual) that would have been sufficient to
complete the computation. The point is that \kbd{nf\_FORCE} does exactly this
internally, but goes on increasing the accuracy of the \kbd{bnf}, then
discarding it, which is a major inefficiency if you indend to compute lots of
discrete logs and have selected a precision which is just too low.
(It is sometimes not so bad since most of the really expensive data is cached
in \kbd{bnf} anyway, if all goes well.)  With this function, the \emph{caller}
may decide to increase the accuracy using \tet{bnfnewprec} (and keep the
resulting \kbd{bnf}!), or avoid the computation altogether. In any case the
decision can be taken at the place where it is most likely to be correct.

\subsec{Ideal reduction}\label{se:Ideal_reduction}

Given an ideal $x$ this means finding a ``simpler'' ideal in the same ideal
class. The public GP function is of course available

\fun{GEN}{idealred0}{GEN nf, GEN x, GEN v} finds a small $a\in x$ and returns
the primitive part of $x/(a)$ (which is an ideal in HNF). What ``small'' means
depends on the parameter $v$, see the GP description.

\fun{GEN}{idealred}{GEN nf, GEN x} is shortcut for \kbd{idealred0(nf,x,NULL)},
since $v$ is better left omitted in practice.

These two functions remain a little complicated to use: in order not to lose
information $x$ must be an extended ideal, otherwise the value of $a$ is lost.
There is a subtelty here: the principal ideal $(a)$ is easy to recover, but
$a$ itself is an instance of the principal ideal problem which is very
difficult given only an \var{nf} (once a \var{bnf} structure is available,
\tet{bnfisprincipal0} will recover it).

\fun{GEN}{idealred_elt0}{GEN nf, GEN x, GEN v} analog to \kbd{idealred0},
returning only the algebraic number $a$ (instead of a rational multiple
of the quotient $x/(a)$). This function is called within \kbd{idealred0}.
\kbd{nf} must be a true \var{nf} (not a \var{bnf} for instance) and the
ideal $x$ must be integral, in HNF.

\fun{GEN}{idealred_elt}{GEN nf, GEN x} shortcut for
\kbd{idealred\_elt0(nf,x,NULL)}.

\fun{GEN}{idealmoddivisor}{GEN bnr, GEN x} A proof-of-concept implementation,
useless in practice. If \kbd{bnr} is associated to some modulus $f$, returns a
``small'' ideal in the same class as $x$ in the ray class group modulo $f$.
The reason why this is useless is that using extended ideals with principal
part in a computation, there is a simple way to reduce them: simply reduce
the generator of the principal part in $(\Z_K/f)^*$.

\fun{GEN}{famat_to_nf_moddivisor}{GEN nf, GEN g, GEN e, GEN bid}
given a true \var{nf} associated to a number field $K$, a \var{bid} structure
associated to a modulus $f$, and an algebraic number in factored form $\prod
g[i]^{e[i]}$, such that $(g[i],f) = 1$ for all $i$, returns a small element in
$\Z_K$ congruent to it mod $f$. Note that if $f$ contains places at infinity,
this includes sign conditions at the specified places.

A simpler case when the conductor has no place at infinity:

\fun{GEN}{famat_to_nf_modideal_coprime}{GEN nf, GEN g, GEN e, GEN f, GEN expo}
as above except that the ideal $f$ is now integral in HNF (no need for a full
\var{bid}), and we pass the exponent of the group $(\Z_K/f)^*$ as \kbd{expo};
any multiple will also do, at the expense of efficiency. Of course if a
\var{bid} for $f$ is available, if is easy to extract $f$ and the exact value
of \kbd{expo} from it (the latter is the first elementary divisor in the
group structure). A useful trick: if you set \kbd{expo} to \emph{any}
positive integer, the result is correct up to \kbd{expo}-th powers, hence
exact if \kbd{expo} is a multiple of the exponent; this is useful when trying
to decide whether an element is a square in a residue field for instance!
(take \kbd{expo}$ = 2$).

What to do when the $g[i]$ are not coprime to $f$, but only $\prod
g[i]^{e[i]}$ is ? Then the situation is more complicated, and we advise to
solve it one prime divisor of $f$ at a time. Let $v$ the valuation
associated to a maximal ideal \kbd{pr} and assume $v(f) = k > 0$:

\fun{GEN}{famat_makecoprime}{GEN nf, GEN g, GEN e, GEN pr, GEN prk, GEN expo}
returns an element in $(\Z_K/\kbd{pr}^k)^*$ congruent to the product
$\prod g[i]^{e[i]}$, assumed to be globally coprime to $f$. As above,
\kbd{expo} is any positive multiple of the exponent of $(\Z_K/\kbd{pr}^k)^*$,
for instance $(Nv-1)p^{k-1}$, if $p$ is the underlying rational prime. You
may use other values of \kbd{expo} (see the useful trick in
\tet{famat_to_nf_modideal_coprime}).

\subsec{Class field theory}

Under GP, a class-field theoretic description of a number field is given by a
triple $A$, $B$, $C$, where the defining set $[A,B,C]$ can have any of the
following forms: $[\var{bnr}]$, $[\var{bnr},\var{subgroup}]$,
$[\var{bnf},\var{modulus}]$, $[\var{bnf},\var{modulus},\var{subgroup}]$.
You can still use directly all of (\kbd{libpari}'s routines implementing) GP's
functions as described in Chapter~3, but they are often awkward in the context
of \kbd{libpari} programming. In particular, it does not make much sense to
always input a triple $A,B,C$ because of the fringe
$[\var{bnf},\var{modulus},\var{subgroup}]$. The first routine to call, is
thus

\fun{GEN}{Buchray}{GEN bnf, GEN mod, long flag} initializes a \var{bnr}
structure from \kbd{bnf} and modulus \kbd{mod}. \kbd{flag} is an or-ed
combination of \kbd{nf\_GEN} (include generators) and \kbd{nf\_INIT} (if
omitted, do not return a \var{bnr}, only the ray class group as an abelian
group). In fact, a single value of \kbd{flag} actually makes sense:
\kbd{nf\_GEN | nf\_INIT} to initialize a proper \var{bnr}: removing
\kbd{nf\_GEN} saves very little time, but the corresponding crippled
\var{bnr} structure will raise errors in most class field theoretic
functions. Possibly also 0 to quickly compute the ray class group structure;
\tet{bnrclassno} is faster if we only need the \emph{order} of the ray class
group.

Now we have a proper \var{bnr} encoding a \kbd{bnf} and a modulus, we no longer
need the $[\var{bnf},\var{modulus}]$ and
$[\var{bnf},\var{modulus},\var{subgroup}]$ forms, which would internally call
\tet{Buchray} anyway. Recall that a subgroup $H$ is given by a matrix in HNF,
whose column express generators of $H$ on the fixed generators of the ray class
group that stored in our \var{bnr}. You may also code the trivial subgroup by
\kbd{NULL}.

\fun{GEN}{bnrconductor}{GEN bnr, GEN H, long flag} see the documentation of
the GP function.

\fun{long}{bnrisconductor}{GEN bnr, GEN H} returns 1 is the class field
defined by the subgroup $H$ (of the ray class group mod $f$ coded in \kbd{bnr})
has conductor $f$. Returns 0 otherwise.

\fun{GEN}{bnrdisc}{GEN bnr, GEN H, long flag} returns the discriminant and
signature of the class field defined by \kbd{bnr} and $H$. See the description
of the GP function for details. \fl\ is an or-ed combination of the flags
\tet{rnf_REL} (output relative data) and \tet{rnf_COND} (return 0 unless the
modulus is the conductor).

\fun{GEN}{bnrsurjection}{GEN BNR, GEN bnr} \kbd{BNR} and \kbd{bnr}
defined over the same field $K$, for moduli $F$ and $f$ with
$F\mid f$, returns the matrix of the canonical surjection
$\text{Cl}_K(F)\to \text{Cl}_K(f)$ (giving the image of the fixed ray class
group generators of \kbd{BNR} in terms of the ones in \kbd{bnr}).

\fun{GEN}{ABC_to_bnr}{GEN A, GEN B, GEN C, GEN *H, int addgen} This is a
quick conversion function designed to go from the too general (inefficient)
$A$, $B$, $C$ form to the preferred \var{bnr}, $H$ form for class fields.
Given $A$, $B$, $C$ as explained above (omitted entries coded by \kbd{NULL}),
return the associated \var{bnr}, and set $H$ to the associated subgroup. If
\kbd{addgen} is $1$, make sure that if the \var{bnr} needed to be computed,
then it contains generators.

\subsec{Relative equations, Galois conjugates}

\fun{GEN}{rnfequationall}{GEN A, GEN B, long *pk, GEN *pLPRS} $A$ is either an
\var{nf} type (corresponding to a number field $K$) or an irreducible \kbd{ZX}
defining a number field $K$. $B$ is an irreducible polynomial in $K[X]$.
Returns an absolute equation $C$ (over $\Q$) for the number field $K[X]/(B)$.
$C$ is the characteristic polynomial of $b + k a$ for some roots $a$ of $A$
and $b$ of $B$, and $k$ is a small rational integer. Set \kbd{*pk} to $k$.

If \kbd{pLPRS} is not \kbd{NULL} set it to $[h_0, h_1]$, $h_i\in \Q[X]$,
where $h_0+h_1 Y$ is the last non-constant polynomial in the pseudo-Euclidean
remainder sequence associated to $A(Y)$ and $B(X-kY)$, leading to $C =
\text{Res}_Y(A(Y), B(Y-kX))$. In particular $a := -h_0/h_1$ is a root of $A$
in $\Q[X]/(C)$, and $X - ka$ is a root of $B$.

\fun{GEN}{rnf_fix_pol}{GEN T, GEN B, int lift} check whether $B$ is a
polynomials with coefficients in the number field defined by the absolute
equation $T(y) = 0$, where $T$ is a \kbd{ZX} and returned a cleaned up version of
$B$. This means that $B$ is a \typ{POL} whose coefficients are \typ{INT},
\typ{FRAC}, \typ{POL} in the variable $y$ with rational coefficients, or
\typ{POLMOD} modulo $T$ which lift to a rational \typ{POL} as above. The cleanup
consists in the following improvements:

\item \typ{POL} coefficients are reduced modulo $T$.

\item \typ{POL} and \typ{POLMOD} coefficients belonging to the base field are
converted to rationals.

\item if \kbd{lift} is non-zero, lift all \typ{POLMOD}, otherwise convert all
\typ{POL} to \typ{POLMOD}s modulo $T$.

\noindent For instance, \kbd{rnfequationall} applies \tet{rnf_fix_pol} to its
argument $B$ (with \kbd{lift} equal to $1$).

\fun{long}{numberofconjugates}{GEN T, long pinit} returns a quick
multiple for the number of  $\Q$-automorphism of the (integral, monic)
\typ{POL} $T$, from modular factorizations, starting from prime \kbd{pinit}
(you can set it to $2$). This upper bounds often coincides with the
actual number of conjugates. Of course, you should use \tet{nfgaloisconj}
to be sure.

\subsec{Miscellaneous routines}

\fun{GEN}{bnfisintnormabs}{GEN bnf, GEN a} as \tet{bnfisintnorm}, but returns a
complete system of solutions modulo units of the absolute norm equation
$|\Norm(x)| = |a|$. As fast as \kbd{bnfisintnorm}, but solves simulaneously
the two equations $\Norm(x) = \pm a$.

\subsec{Obsolete routines}

Still provided for backward compatibility, but should not be used in new
programs. They will eventually disappear.

\fun{GEN}{zidealstar}{GEN nf, GEN x} short for \kbd{Idealstar(nf,x,nf\_GEN)}

\fun{GEN}{zidealstarinit}{GEN nf, GEN x}
short for \kbd{Idealstar(nf,x,nf\_INIT)}

\fun{GEN}{zidealstarinitgen}{GEN nf, GEN x}
short for \kbd{Idealstar(nf,x,nf\_GEN|nf\_INIT)}

\fun{GEN}{buchimag}{GEN D, GEN c1, GEN c2, GEN gCO} short for
\bprog
  Buchquad(D,gtodouble(c1),gtodouble(c2), /*ignored*/0)
@eprog

\fun{GEN}{buchreal}{GEN D, GEN gsens, GEN c1, GEN c2, GEN RELSUP, long prec}
short for
\bprog
Buchquad(D,gtodouble(c1),gtodouble(c2), prec)
@eprog

The following use a naming scheme which is error-prone and not easily
extendable; besides, they compute generators as per \kbd{nf\_GEN} and
not \kbd{nf\_GENMAT}. Don't use them:

\fun{GEN}{isprincipalforce}{GEN bnf,GEN x}

\fun{GEN}{isprincipalgen}{GEN bnf, GEN x}

\fun{GEN}{isprincipalgenforce}{GEN bnf, GEN x}

\fun{GEN}{isprincipalraygen}{GEN bnr, GEN x}, use \tet{bnrisprincipal}.

\noindent Variants on \kbd{polred}: use \kbd{polredabs0}. You almost
certainly want to include the \tet{nf_PARTIALFACT} flag.

\fun{GEN}{factoredpolred}{GEN x, GEN fa}

\fun{GEN}{factoredpolred2}{GEN x, GEN fa}

\fun{GEN}{polred2}{GEN x}

\fun{GEN}{smallpolred}{GEN x}

\fun{GEN}{smallpolred2}{GEN x}, use \tet{Polred}.

\fun{GEN}{polredabs}{GEN x}

\fun{GEN}{polredabs2}{GEN x}

\fun{GEN}{polredabsall}{GEN x, long flun}

\noindent Superseded by \tet{bnrdisc}:

\fun{GEN}{discrayabs}{GEN bnr,GEN subgroup}

\fun{GEN}{discrayabscond}{GEN bnr,GEN subgroup}

\fun{GEN}{discrayrel}{GEN bnr,GEN subgroup}

\fun{GEN}{discrayrelcond}{GEN bnr,GEN subgroup}

\noindent Superseded by \tet{bnrdisclist0}:

\fun{GEN}{discrayabslist}{GEN bnf,GEN listes}

\fun{GEN}{discrayabslistarch}{GEN bnf, GEN arch, long bound}

\fun{GEN}{discrayabslistlong}{GEN bnf, long bound}

\section{Quadratic number fields and quadratic forms}

\subsec{Checks}

\fun{void}{check_quaddisc}{GEN x, long *s, long *mod4, const char *f}
checks whether the \kbd{GEN} $x$ is a quadratic discriminant (\typ{INT},
not a square, congruent to $0,1$ modulo $4$), and raise an exception
otherwise. Set \kbd{*s} to the sign of $x$ and \kbd{*mod4} to $x$ modulo
$4$ (0 or 1).

\fun{void}{check_quaddisc_real}{GEN x, long *mod4, const char *f} as
\tet{check_quaddisc}; check that \kbd{signe(x)} is positive.

\fun{void}{check_quaddisc_imag}{GEN x, long *mod4, const char *f} as
\tet{check_quaddisc}; check that \kbd{signe(x)} is negative.

\subsec{\typ{QFI}, \typ{QFR}}

\fun{GEN}{qfi}{GEN x, GEN y, GEN z} creates the \typ{QFI} $(x,y,z)$.

\fun{GEN}{qfr}{GEN x, GEN y, GEN z, GEN d} creates the \typ{QFR} $(x,y,z)$
with distance component $d$.

\fun{GEN}{qfr_1}{GEN q} given a \typ{QFR} $q$, return the unit form $q^0$.

\fun{GEN}{qfi_1}{GEN q} given a \typ{QFI} $q$, return the unit form $q^0$.

\subsubsec{Composition}

\fun{GEN}{qficomp}{GEN x, GEN y} compose the two \typ{QFI} $x$ and $y$,
then reduce the result. This is the same as \kbd{gmul(x,y)}.

\fun{GEN}{qfrcomp}{GEN x, GEN y} compose the two \typ{QFR} $x$ and $y$,
then reduce the result. This is the same as \kbd{gmul(x,y)}.

\fun{GEN}{qfisqr}{GEN x} as \kbd{qficomp(x,y)}.

\fun{GEN}{qfrsqr}{GEN x} as \kbd{qfrcomp(x,y)}.

\noindent Same as above, \emph{without} reducing the result:

\fun{GEN}{qficompraw}{GEN x, GEN y}

\fun{GEN}{qfrcompraw}{GEN x, GEN y}

\fun{GEN}{qfisqrraw}{GEN x}

\fun{GEN}{qfrsqrraw}{GEN x}

\fun{GEN}{qfbcompraw}{GEN x, GEN y} compose two \typ{QFI}s or two \typ{QFR}s,
without reduce the result.

\subsubsec{Powering}

\fun{GEN}{powgi}{GEN x, GEN n} computes $x^n$ (will work for many more types
than \typ{QFI} and \typ{QFR}, of course). Reduce the result.

\fun{GEN}{qfrpow}{GEN x, GEN n} computes $x^n$ for a \typ{QFR} $x$, reducing
along the way. If the distance component is initially $0$, leave it alone;
otherwise update it.

\fun{GEN}{qfbpowraw}{GEN x, long n} compute $x^n$ (pure composition, no
reduction), for a \typ{QFI} or \typ{QFR} $x$.

\fun{GEN}{qfipowraw}{GEN x, long n} as \tet{qfbpowraw}, for a \typ{QFI} $x$.

\fun{GEN}{qfrpowraw}{GEN x, long n} as \tet{qfbpowraw}, for a \typ{QFR} $x$.

\subsubsec{Solve, Cornacchia}

The following functions underly \tet{qfbsolve}; $p$ denotes a prime number.

\fun{GEN}{qfisolvep}{GEN Q, GEN p} solves $Q(x,y) = p$ over the integers, for
a \typ{QFI} $Q$. Return \kbd{gen\_0} if there are no solutions.

\fun{GEN}{qfrsolvep}{GEN Q, GEN p} solves $Q(x,y) = p$ over the integers, for
a \typ{QFR} $Q$. Return \kbd{gen\_0} if there are no solutions.

\fun{long}{cornacchia}{GEN d, GEN p, GEN *px, GEN *py} solves
$x^2+ dy^2 = p$ over the integers, where $d > 0$. Return $1$ if there is a
solution (and store it in \kbd{*x} and \kbd{*y}), $0$ otherwise.

\fun{long}{cornacchia2}{GEN d, GEN p, GEN *px, GEN *py} as \kbd{cornacchia},
for the equation $x^2 + dy^2 = 4p$.

\subsubsec{Prime forms}

\fun{GEN}{primeform_u}{GEN x, ulong p} \typ{QFI} whose first coefficient
is the prime $p$.

\fun{GEN}{primeform}{GEN x, GEN p, long prec}

\subsec{Efficient real quadratic forms} Unfortunately, \typ{QFR}s
are very inefficient, and are only provided for backward compatibility.

\item they do not contain needed quantities, which are thus constantly
recomputed (the discriminant $D$, $\sqrt{D}$ and its integer part),

\item the distance component is stored in logarithmic form, which involves
computing one extra logarithm per operation. It is much more efficient
to store its exponential, computed from ordinary multiplications and
divisions (taking exponent overflow into account), and compute its logarithm
at the very end.

Internally, we have two representations for real quadratic forms:

\item \tet{qfr3}, a container $[a,b,c]$ with at least 3 entries: the three
coefficients; the idea is to ignore the distance component.

\item \tet{qfr5}, a container with at least 5 entries $[a,b,c,e,d]$: the
three coefficients a \typ{REAL} $d$ and a \typ{INT} $e$ coding the distance
component $2^{Ne} d$, in exponential form, for some large fixed $N$.

It is a feature that \kbd{qfr3} and \kbd{qfr5} have no specified length or
type. It implies that a \kbd{qfr5} or \typ{QFR} will do whenever a \kbd{qfr3}
is expected. Routines using these objects all require a global context,
provided by a \kbd{struct qfr\_data *}:
\bprog
  struct qfr_data {
    GEN D;        /* discriminant, t_INT   */
    GEN sqrtD;    /* sqrt(D), t_REAL       */
    GEN isqrtD;   /* floor(sqrt(D)), t_INT */
  };
@eprog
\fun{void}{qfr_data_init}{GEN D, long prec, struct qfr_data *S}
given a discriminant $D > 0$, initialize $S$ for computations at precision
\kbd{prec} ($\sqrt{D}$ is computed to that initial accuracy).

\noindent All functions below are shallow, and not stack clean.

\fun{GEN}{qfr3_comp}{GEN x, GEN y, struct qfr_data *S} compose two
\kbd{qfr3}, reducing the result.

\fun{GEN}{qfr3_pow}{GEN x, GEN n, struct qfr_data *S} compute $x^n$, reducing
along the way.

\fun{GEN}{qfr3_red}{GEN x, struct qfr_data *S} reduce $x$.

\fun{GEN}{qfr3_rho}{GEN x, struct qfr_data *S} perform one reduction step;
\kbd{qfr3\_red} just performs reduction steps until we hit a reduced form.

\fun{GEN}{qfr3_to_qfr}{GEN x, GEN d} recover an ordinary \typ{QFR} from the
\kbd{qfr3} $x$, adding distance component $d$.

Before we explain \kbd{qfr5}, recall that it corresponds to an ideal, that
reduction corresponds to multiplying by a principal ideal, and that the
distance component is a clever way to keep track of these principal ideals.
More precisely, reduction consists in a number of reduction steps,
going from the form $(a,b,c)$ to $\rho(a,b,c) = (c, -b \mod 2c, *)$;
the distance component is multiplied by (a floating point approximation to)
$(b + \sqrt{D}) / (b - \sqrt{D})$.

\fun{GEN}{qfr5_comp}{GEN x, GEN y, struct qfr_data *S} compose two
\kbd{qfr5}, reducing the result, and updating the distance component.

\fun{GEN}{qfr5_pow}{GEN x, GEN n, struct qfr_data *S} compute $x^n$, reducing
along the way.

\fun{GEN}{qfr5_red}{GEN x, struct qfr_data *S} reduce $x$.

\fun{GEN}{qfr5_rho}{GEN x, struct qfr_data *S} perform one reduction step.

\fun{GEN}{qfr5_dist}{GEN e, GEN d, long prec} decode the distance component
from exponential (\kbd{qfr5}-specific) to logarithmic form (as in a
\typ{QFR}).

\fun{GEN}{qfr_to_qfr5}{GEN x, long prec} convert a \typ{QFR} to a
\kbd{qfr5} with initial trivial distance component ($= 1$).

\fun{GEN}{qfr5_to_qfr}{GEN x, GEN d}, assume $x$ is a \kbd{qfr5} and
$d$ was the original distance component of some \typ{QFR} that we converted
using \tet{qfr_to_qfr5} to perform efficiently a number of operations.
Convert $x$ to a \typ{QFR} with the correct (logarithmic) distance component.

\section{Linear algebra over $\Z$}
\subsec{Hermite and Smith Normal Forms}

\fun{GEN}{ZM_hnf}{GEN x} returns the upper triangular Hermite Normal Form of the
\kbd{ZM} $x$ (removing $0$ columns), using the \tet{ZM_hnfall} algorithm. If you
want the true HNF, use \kbd{ZM\_hnfall(x, NULL, 0)}.

\fun{GEN}{ZM_hnfmod}{GEN x, GEN d} returns the HNF of the \kbd{ZM} $x$
(removing $0$ columns), assuming the \typ{INT} $d$ is a multiple of the
determinant of $x$. This is usually faster than \tet{ZM_hnf} (and uses less
memory) if the dimension is large, $> 50$ say.

\fun{GEN}{ZM_hnfmodid}{GEN x, GEN d} returns the HNF of the matrix $(x \mid d
\text{Id})$ (removing $0$ columns), for a \kbd{ZM} $x$ and a \typ{INT} $d$.

\fun{GEN}{ZM_hnfmodall}{GEN x, GEN d, long flag} low-level function underlying the
\kbd{ZM\_hnfmod} variants. If \kbd{flag} is $0$, calls \kbd{ZM\_hnfmod(x,d)};
\kbd{flag} is an or-ed conbination of:

\item \tet{hnf_MODID} call \kbd{ZM\_hnfmodid} instead of \kbd{ZM\_hnfmod},

\item \tet{hnf_PART} return as soon as we obtain an upper triangular matrix,
saving time. The pivots are non-negative and give the diagonal of the true HNF,
but the entries to the right of the pivots need not be reduced, i.e.~they may be
large or negative.

\item \tet{hnf_CENTER} returns the centered HNF, where the entries to the right of
a pivot $p$ are centered residues in $[-p/2, p/2[$, hence smallest possible in
absolute value, but possibly negative.

\fun{GEN}{ZM_hnfall}{GEN x, GEN *U, long remove} returns the upper triangular HNF
$H$ of the \kbd{ZM} $x$; if $U$ is not \kbd{NULL}, set if to the matrix $U$ such
that $x U = H$. If $\kbd{remove} = 0$, $H$ is the true HNF, including $0$ columns;
if $\kbd{remove} = 1$, delete the $0$ columns from $H$ but do not update $U$
accordingly (so that the integer kernel may still be recovered): we no longer have
$x U = H$; if $\kbd{remove} = 2$, remove $0$ columns from $H$ and update $U$ so
that $x U = H$. The matrix $U$ is square and invertible unless $\kbd{remove} = 2$.

This routine uses a naive algorithm which is potentially exponential in the
dimension (due to coefficient explosion) but is fast in practice, although it
may require lots of memory. The base change matrix $U$ may be very large,
when the kernel is large.

\fun{GEN}{ZM_hnfperm}{GEN A, GEN *ptU, GEN *ptperm} returns the hnf
$H = P A U$ of the matrix $P A$, where $P$ is a suitable permutation matrix,
and $U\in \text{Gl}_n(\Z)$. $P$ is chosen so as to (heuristically) minimize the
size of $U$; in this respect it is less efficient than \kbd{ZM\_hnflll}
but usually faster. Set \kbd{*ptU} to $U$ and \kbd{*pterm} to a \typ{VECSMALL}
representing the row permutation associated to $P = (\delta_{i,\kbd{perm}[i]}$.
If \kbd{ptU} is set to \kbd{NULL}, $U$ is not computed, saving some time;
although useless, setting \kbd{ptperm} to \kbd{NULL} is also allowed.

\fun{GEN}{ZM_hnflll}{GEN x, GEN *U, int remove} returns the HNF $H$ of the
\kbd{ZM} $x$; if $U$ is not \kbd{NULL}, set if to the matrix $U$ such that $x
U = H$. The meaning of \kbd{remove} is the same as in \tet{ZM_hnfall}.

This routine uses the \idx{LLL} variant of Havas, Majewski and Mathews, which is
polynomial time, but rather slow in practice because it uses an exact LLL
over the integers instead of a floating point variant; it uses polynomial
space but lots of memory is needed for large dimensions, say larger than 300.
On the other hand, the base change matrix $U$ is essentially optimally small
with respect to the $L_2$ norm.

\fun{GEN}{ZM_hnfcenter}{GEN M}. Given a \kbd{ZM} in HNF $M$, update it in
place so that non-diagonal entries belong to a system of \emph{centered}
residues. Not suitable for gerepile.

Some direct applications: the following routines apply to upper triangular
integral matrices; in practice, these come from HNF algorithms.

\fun{GEN}{hnf_divscale}{GEN A, GEN B,GEN t} $A$ an upper triangular \kbd{ZM},
$B$ a \kbd{ZM}, $t$ an integer, such that $C := tA^{-1}B$ is integral.
Return $C$.

\fun{GEN}{hnf_solve}{GEN A, GEN B} $A$ an upper triangular \kbd{ZM},
$B$ a \kbd{ZM} or \kbd{ZC}. Return $A^{-1}B$ if it is integral, and \kbd{NULL}
if it is not.

\fun{GEN}{hnf_invimage}{GEN A, GEN b} $A$ an upper triangular \kbd{ZM},
$b$ a \kbd{ZC}.  Return $A^{-1}B$ if it is integral, and \kbd{NULL} if it is
not.

\fun{int}{hnfdivide}{GEN A, GEN B} $A$ and $B$ are two upper triangular
\kbd{ZM}. Return $1$ if $A^{-1} B$ is integral, and $0$ otherwise.

\misctitle{Smith Normal Form}

\fun{GEN}{ZM_snf}{GEN x} returns the Smith Normal Form (vector of
elementary divisors) of the \kbd{ZM} $x$.

\fun{GEN}{ZM_snfall}{GEN x, GEN *U, GEN *V} returns
\kbd{ZM\_smith(x)} and sets $U$ and $V$ to unimodular matrices such that $U\,
x\, V = D$ (diagonal matrix of elementary divisors). Either (or both) $U$ or
$V$ may be \kbd{NULL} in which case the corresponding matrix is not computed.

\fun{GEN}{ZM_snfall_i}{GEN x, GEN *U, GEN *V, int returnvec} same as
\kbd{ZM\_snfall}, except that, depending on the value of \kbd{returnvec}, we
either return a diagonal matrix (as in \kbd{ZM\_snfall}, \kbd{returnvec} is 0)
or a vector of elementary divisors (as in \kbd{ZM\_snf}, \kbd{returnvec} is 1) .

\fun{void}{ZM_snfclean}{GEN d, GEN U, GEN V} assuming $d$, $U$, $V$ come
from \kbd{d = ZM\_snfall(x, \&U, \&V)}, where $U$ or $V$ may be \kbd{NULL},
cleans up the output in place. This means that elementary divisors equal to 1
are deleted and $U$, $V$ are updated. The output is not suitable for
\kbd{gerepileupto}.

\fun{GEN}{ZM_snf_group}{GEN H, GEN *U, GEN *Uinv} this function computes data
to go back and forth between an abelian group (of finite type) given by
generators and relations, and its canonical SNF form. Given an abstract
abelian group with generators $g = (g_1,\dots,g_n)$ and a vector
$X=(x_i)\in\Z^n$, we write $g X$ for the group element $\sum_i x_i g_i$;
analogously if $M$ is an $n\times r$ integer matrix $g M$ is a vector
containing $r$ group elements. The group neutral element is $0$; by abuse of
notation, we still write $0$ for a vector of group elements all equal to the
neutral element. The input is a full relation matrix $H$ among the
generators, i.e. a \kbd{ZM} (not necessarily square) such that $gX = 0$ for
some $X\in\Z^n$ if and only if $X$ is in the integer image of $H$, so that
the abelian group is isomorphic to $\Z^n/\text{Im} H$. \emph{The routine
assumes that $H$ is in HNF;} replace it by its HNF if it is not the case. (Of
course this defines the same group.)

Let $G$ a minimal system of generators in SNF for our abstract group:
if the $d_i$ are the elementary divisors ($\dots \mid d_2\mid d_1$), each
$G_i$ has either infinite order ($d_i = 0$) or order $d_i > 1$. Let $D$
the matrix with diagonal $(d_i)$, then
$$G D = 0,\quad G = g U_{\text{inv}},\quad g = G U,$$
for some integer matrices $U$ and $U_{\text{inv}}$. Note that these are not
even square in general; even if square, there is no guarantee that these are
unimodular: they are chosen to have minimal entries given the known relations
in the group and only satisfy $D \mid (U U_{\text{inv}} - \text{Id})$ and $H
\mid (U_{\text{inv}}U - \text{Id})$.

The function returns the vector of elementary divisors $(d_i)$; if \kbd{U} is
not \kbd{NULL}, it is set to $U$; if \kbd{Uinv} is not \kbd{NULL} it is
set to $U_{\text{inv}}$. The function is not memory clean.


The following 3 routines underly the various \tet{matrixqz} variants.
In all case the $m\times n$ \typ{MAT} $x$ is assumed to have rational
(\typ{INT} and \typ{FRAC}) coefficients

\fun{GEN}{QM_ImQ_hnf}{GEN x} returns an HNF basis for
$\text{Im}_\Q x \cap \Z^n$.

\fun{GEN}{QM_ImZ_hnf}{GEN x} returns an HNF basis for
$\text{Im}_\Z x \cap \Z^n$.

\fun{GEN}{QM_minors_coprime}{GEN x, GEN D}, assumes $m\geq n$, and returns
a matrix in $M_{m,n}(\Z)$ with the same $\Q$-image as $x$, such that
the GCD of all $n\times n$ minors is coprime to $D$; if $D$ is \kbd{NULL},
we want the GCD to be $1$.
\smallskip

The following routines are simple wrappers around the above ones and are
normally useless in library mode:

\fun{GEN}{hnf}{GEN x} checks whether $x$ is a \kbd{ZM}, then calls \tet{ZM_hnf}.
Normally useless in library mode.

\fun{GEN}{hnfmod}{GEN x, GEN d} checks whether $x$ is a \kbd{ZM}, then calls
\tet{ZM_hnfmod}. Normally useless in library mode.

\fun{GEN}{hnfmodid}{GEN x,GEN d} checks whether $x$ is a \kbd{ZM}, then calls
\tet{ZM_hnfmodid}. Normally useless in library mode.

\fun{GEN}{hnfall}{GEN x} calls
\kbd{ZM\_hnfall(x, \&U, 1)} and returns $[H, U]$. Normally useless in library
mode.

\fun{GEN}{hnflll}{GEN x} calls \kbd{ZM\_hnflll(x, \&U, 1)} and returns $[H,
U]$. Normally useless in library mode.

\fun{GEN}{hnfperm}{GEN x} calls \kbd{ZM\_hnfperm(x, \&U, \&P)} and returns
$[H, U, P]$. Normally useless in library mode.

\fun{GEN}{smith}{GEN x} checks whether $x$ is a \kbd{ZM}, then calls
\kbd{ZM\_smith}. Normally useless in library mode.

\fun{GEN}{smithall}{GEN x} checks whether $x$ is a \kbd{ZM}, then calls
\kbd{ZM\_smithall(x, \&U, \&V)} and returns $[U,V,D]$. Normally useless in
library mode.

\noindent Some related functions over $K[X]$, $K$ a field:

\fun{GEN}{gsmith}{GEN A} the input matrix must be square, returns the
elementary divisors.

\fun{GEN}{gsmithall}{GEN A} the input matrix must be square, returns the
$[U,V,D]$, $D$ diagonal, such that $UAV = D$.

\fun{GEN}{smithclean}{GEN z} cleanup the output of \kbd{smithall} or
\kbd{gsmithall} (delete elementary divisors equal to $1$, updating base
change matrices).

\subsec{The LLL algorithm}\sidx{LLL}

The basic GP functions and their immediate variants are normally not very
useful in library mode. We briefly list them here for completeness, see the
documentation of \kbd{qflll} and \kbd{qflllgram} for details:

\item \fun{GEN}{qflll0}{GEN x, long flag}

\fun{GEN}{lll}{GEN x} \fl = 0

\fun{GEN}{lllint}{GEN x} \fl = 1

\fun{GEN}{lllkerim}{GEN x} \fl = 4

\fun{GEN}{lllkerimgen}{GEN x} \fl = 5

\fun{GEN}{lllgen}{GEN x} \fl = 8

\item \fun{GEN}{qflllgram0}{GEN x, long flag}

\fun{GEN}{lllgram}{GEN x} \fl = 0

\fun{GEN}{lllgramint}{GEN x} \fl = 1

\fun{GEN}{lllgramkerim}{GEN x} \fl = 4

\fun{GEN}{lllgramkerimgen}{GEN x} \fl = 5

\fun{GEN}{lllgramgen}{GEN x} \fl = 8

\smallskip

The basic workhorse underlying all integral and floating point LLLs is

\fun{GEN}{ZM_lll}{GEN x, double D, long flag}, where $x$ is a \kbd{ZM};
$D \in ]1/4,1[$ is the Lov\'{a}sz constant determining the frequency of
swaps during the algorithm: a larger values means better guarantees for the
basis (in principle smaller basis vectors) but slower runtimes (suggested
value: $D = 0.99$).

\misctitle{Important:} This function does not collect garbage and its output
is not suitable for either \kbd{gerepile} or \kbd{gerepileupto}. We expect
the caller to do something simple with the output (e.g. matrix
multiplication), then collect garbage immediately.

\noindent\kbd{flag} is an or-ed combination of the following flags:

\item  \tet{LLL_GRAM}. If set, the input matrix $x$ is the Gram matrix ${}^t
v v$ of some lattice vectors $v$.

\item  \tet{LLL_INPLACE}. If unset, we return the base change matrix $U$,
otherwise the transformed matrix $x U$ or ${}^t U x U$ (\kbd{LLL\_GRAM}).
Implies \tet{LLL_IM} (see below).

\item  \tet{LLL_KEEP_FIRST}. The first vector in the output basis is the same
one as was originally input. Provided this is a shortest non-zero vector of
the lattice, the output basis is still LLL-reduced. This is used to reduce
maximal orders of number fields with respect to the $T_2$ quadratic form, to
ensure that the first vector in the output basis corresponds to $1$ (which is
a shortest vector).

The last three flags are mutually exclusive, either 0 or a single one must be
set:

\item  \tet{LLL_KER} If set, only return a kernel basis $K$ (not LLL-reduced).

\item  \tet{LLL_IM} If set, only return an LLL-reduced lattice basis $T$.
(This is implied by \tet{LLL_INPLACE}).

\item  \tet{LLL_ALL} If set, returns a 2-component vector $[K, T]$
corresponding to both kernel and image.


\fun{GEN}{lllfp}{GEN x, double D, long flag} is a variant for matrices
with inexact entries: $x$ is a matrix with real coefficients (types
\typ{INT}, \typ{FRAC} and \typ{REAL}), $D$ and $\fl$ are as in \tet{ZM_lll}.
The matrix is rescaled, rounded to nearest integers, then fed to
\kbd{ZM\_lll}. The flag \kbd{LLL\_INPLACE} is still accepted but less useful
(it returns an LLL-reduced basis associated to rounded input, instead of an
exact base change matrix).

\fun{GEN}{ZM_lll_norms}{GEN x, double D, long flag, GEN *ptB} slightly more
general version of \kbd{ZM\_lll}, setting \kbd{*ptB} to a vector containing
the squared norms of the Gram-Schmidt vectors $(b_i^*)$ associated to the
output basis $(b_i)$, $b_i^* = b_i + \sum_{j < i} \mu_{i,j} b_j^*$.


\fun{GEN}{lllintpartial_inplace}{GEN x} given a \kbd{ZM} $x$ of maximal rank,
returns a partially reduced basis $(b_i)$ for the space spanned by the
columns of $x$: $|b_i \pm b_j| \geq |b_i|$ for any two distinct basis vectors
$b_i$, $b_j$. This is faster than the LLL algorithm, but produces much larger
bases.

\fun{GEN}{lllintpartial}{GEN x} as \kbd{lllintpartial\_inplace}, but returns
the base change matrix $U$ from the canonical basis to the $b_i$, i.e. $x U$
is the output of \kbd{lllintpartial\_inplace}.

\subsec{Reduction modulo matrices}

\fun{GEN}{ZC_hnfremdiv}{GEN x, GEN y, GEN *Q} assuming $y$ is an
invertible \kbd{ZM} in HNF and $x$ is a \kbd{ZC}, returns the \kbd{ZC} $R$
equal to $x$ mod $y$ (whose $i$-th entry belongs to $[-y_{i,i}/2, y_{i,i}/2[$).
Stack clean \emph{unless} $x$ is already reduced (in which case, returns $x$
itself, not a copy). If $Q$ is not \kbd{NULL}, set it to the \kbd{ZC} such that
$x = yQ + R$.

\fun{GEN}{ZM_hnfremdiv}{GEN x, GEN y, GEN *Q} reduce
each column of the \kbd{ZM} $x$ using \kbd{ZC\_hnfremdiv}. If $Q$ is not
\kbd{NULL}, set it to the \kbd{ZM} such that $x = yQ + R$.

\fun{GEN}{ZC_hnfrem}{GEN x, GEN y} alias for \kbd{ZC\_hnfremdiv(x,y,NULL)}.

\fun{GEN}{ZM_hnfrem}{GEN x, GEN y} alias for \kbd{ZM\_hnfremdiv(x,y,NULL)}.

Besises the \emph{hnfrem} functions, which were specific to integral input,
we also have:

\fun{GEN}{reducemodinvertible}{GEN x, GEN y} $y$ is an invertible matrix
and $x$ a \typ{COL} or \typ{MAT} of compatible dimension.
Returns $x - y\lfloor y^{-1}x \rceil$, which has small entries and differs
from $x$ by an integral linear combination of the columns of $y$. Suitable
for \kbd{gerepileupto}, but does not collect garbage.

\fun{GEN}{closemodinvertible}{GEN x, GEN y} returns $x -
\kbd{reducemodinvertible}(x,y)$, i.e. an integral linear comination of
the columns of $y$, which is close to $x$.

\fun{GEN}{reducemodlll}{GEN x,GEN y} LLL-reduce the \kbd{ZM} $y$ and call
\kbd{reducemodinvertible} to find a small representative of $x$ mod $y \Z^n$.
Suitable for \kbd{gerepileupto}, but does not collect garbage.

\newpage
